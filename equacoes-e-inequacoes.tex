\documentclass[brazil, notheorems, 10pt]{beamer}
%
%   Arquivo de Configuração dos Slides
%


%
%   Pacotes utilizados
%

% Codificação dos caracteres em formato universal.
\usepackage[utf8]{inputenc}
\usepackage[T1]{fontenc}

% Traduz o texto gerados pelo LaTeX para português. ex.: Capítulo, Seção, Conteúdo.
\usepackage[brazil]{babel}

% Pacotes para ambientes matemáticos
\usepackage{amsmath}
\usepackage{amsthm}
\usepackage{amssymb}

% Diversas funções para o uso das aspas.
\usepackage{csquotes}

% Outros pacotes
\usepackage{hyperref}
\usepackage{tikz}
\usepackage{yfonts}
\usepackage{colortbl}
\usepackage{ragged2e}
\usepackage{helvet}
\usepackage{verbatim}


%
%   Tema
%

% Copyright 2007 by Till Tantau
%
% This file may be distributed and/or modified
%
% 1. under the LaTeX Project Public License and/or
% 2. under the GNU Public License.
%
% See the file doc/licenses/LICENSE for more details.


% Common packages


\usepackage{times}
 \mode<article> {
	\usepackage{times}
	\usepackage{mathptmx}
	\usepackage[left=1.5cm,right=6cm,top=1.5cm,bottom=3cm]{geometry}
}

\usepackage{hyperref}
\usepackage[T1]{fontenc}
\usepackage{amsmath,amssymb}
\usepackage{tikz}
\usepackage{colortbl}
\usepackage{yfonts}
\usepackage{colortbl}
\usepackage{translator} % comment this, if not available
\usepackage{ragged2e} % justifying
% Or whatever. Note that the encoding and the font should match. If T1
% does not look nice, try deleting the line with the fontenc.
\usepackage{helvet}
\usepackage{verbatim}


%\usepackage{lipsum}
%\usepackage{enumitem}


\usetheme[
%%% options passed to the outer theme
%    hidetitle,           % hide the (short) title in the sidebar
%    hideauthor,          % hide the (short) author in the sidebar
%    hideinstitute,       % hide the (short) institute in the bottom of the sidebar
%    shownavsym,          % show the navigation symbols
%    width=2cm,           % width of the sidebar (default is 2 cm)
%    hideothersubsections,% hide all subsections but the subsections in the current section
%    hideallsubsections,  % hide all subsections
		right               % right of left position of sidebar (default is right)
%%% options passed to the color theme
%    lightheaderbg,       % use a light header background
	]{AAUsidebar}

% If you want to change the colors of the various elements in the theme, edit and uncomment the following lines
% Change the bar and sidebar colors:
%\setbeamercolor{AAUsidebar}{fg=red!20,bg=red}
%\setbeamercolor{sidebar}{bg=red!20}
% Change the color of the structural elements:
%\setbeamercolor{structure}{fg=red}
% Change the frame title text color:
%\setbeamercolor{frametitle}{fg=blue}
% Change the normal text color background:
%\setbeamercolor{normal text}{bg=gray!10}
% Highlight the text in the sidebar
\usecolortheme{rose,sidebartab}
% ... and you can of course change a lot more - see the beamer user manual.

% colored hyperlinks
\newcommand{\chref}[2]{%
	\href{#1}{{\usebeamercolor[bg]{AAUsidebar}#2}}%
}



% specify a logo on the titlepage (you can specify additional logos an include them in
% institute command below
\pgfdeclareimage[height=1cm]{titlepagelogo}{theme/figures/ufrn2} % placed on the title page
\pgfdeclareimage[height=1cm]{titlepagelogo2}{theme/figures/imd} % placed on the title page
\titlegraphic{% is placed on the bottom of the title page
	\pgfuseimage{titlepagelogo}
	\hspace{1cm}\pgfuseimage{titlepagelogo2}
}


% Article version layout settings

\mode<article>

\makeatletter
\def\@listI{\leftmargin\leftmargini
	\parsep 0pt
	\topsep 5\p@   \@plus3\p@ \@minus5\p@
	\itemsep0pt}
\let\@listi=\@listI


\setbeamertemplate{frametitle}{\paragraph*{\insertframetitle\
		\ \small\insertframesubtitle}\ \par
}
\setbeamertemplate{frame end}{%
	\marginpar{\scriptsize\hbox to 1cm{\sffamily%
			\hfill\strut\insertshortlecture.\insertframenumber}\hrule height .2pt}}
\setlength{\marginparwidth}{1cm}
\setlength{\marginparsep}{4.5cm}

\def\@maketitle{\makechapter}

\def\makechapter{
	\newpage
	\null
	\vskip 2em%
	{%
		\parindent=0pt
		\raggedright
		\sffamily
		\vskip8pt
		\includegraphics[width=\linewidth]{theme/figures/imd.png}\par\vskip2em
		{\fontsize{36pt}{36pt}\selectfont Aula \insertshortlecture \par\vskip2pt}
		{\fontsize{24pt}{28pt}\selectfont \color{blue!50!black} \@title\par\vskip4pt}
		%{\Large\selectfont \color{blue!50!black} \insertsubtitle\par}
		\vskip10pt

		\normalsize\selectfont [Notas de Aula]
		Disciplina: \emph{\lecturename \ (\semestre)} \par\vskip1.5em
		\nomedoautor\hskip1em Email: \ \emaildoautor
	}
	\par
	\vskip 1.5em%
}

\let\origstartsection=\@startsection
\def\@startsection#1#2#3#4#5#6{%
	\origstartsection{#1}{#2}{#3}{#4}{#5}{#6\normalfont\sffamily\color{blue!50!black}\selectfont}}

\makeatother

\mode
<all>




% Typesetting Listings

\usepackage{listings}
\lstset{language=Java}

\alt<presentation>
{\lstset{%
	basicstyle=\footnotesize\ttfamily,
	commentstyle=\slshape\color{green!50!black},
	keywordstyle=\bfseries\color{blue!50!black},
	identifierstyle=\color{blue},
	stringstyle=\color{orange},
	escapechar=\#,
	emphstyle=\color{red}}
}
{
	\lstset{%
		basicstyle=\ttfamily,
		keywordstyle=\bfseries,
		commentstyle=\itshape,
		escapechar=\#,
		emphstyle=\bfseries\color{red}
	}
}



% Common theorem-like environments
%\usepackage{amsthm}

\setbeamertemplate{theorems}[numbered]

%
%	New useful definitions:
%

\newbox\mytempbox
\newdimen\mytempdimen

\newcommand\includegraphicscopyright[3][]{%
	\leavevmode\vbox{\vskip3pt\raggedright\setbox\mytempbox=\hbox{\includegraphics[#1]{#2}}%
		\mytempdimen=\wd\mytempbox\box\mytempbox\par\vskip1pt%
		\fontsize{3}{3.5}\selectfont{\color{black!25}{\vbox{\hsize=\mytempdimen#3}}}\vskip3pt%
}}

\newenvironment{colortabular}[1]{\medskip\rowcolors[]{1}{blue!20}{blue!10}\tabular{#1}\rowcolor{blue!40}}{\endtabular\medskip}

\def\equad{\leavevmode\hbox{}\quad}

\newenvironment{greencolortabular}[1]
{\medskip\rowcolors[]{1}{green!50!black!20}{green!50!black!10}%
	\tabular{#1}\rowcolor{green!50!black!40}}%
{\endtabular\medskip}

%\setbeamertemplate{theorem begin}{{ \inserttheoremheadfont
%\inserttheoremname \inserttheoremnumber
%\ifx\inserttheoremaddition\empty\else\ (\inserttheoremaddition)\fi%
%\inserttheorempunctuation }} \setbeamertemplate{theorem end}{}

\newcommand{\vu}{\vec{u}}
\newcommand{\vv}{\vec{v}}
\newcommand{\vi}{\vec{i}}
\newcommand{\vj}{\vec{j}}
\newcommand{\vk}{\vec{k}}
\newcommand{\vw}{\vec{w}}
\newcommand\segmento[2]{\overline{#1#2}}
\def\colc#1{\left[#1\right]}



%
%   Macros
%

\usepackage{macros/macros}


%
%   Ambientes
%

\theoremstyle{plain}
\newtheorem{teorema}{Teorema}

\theoremstyle{definition}
\newtheorem{definicao}[teorema]{Definição}
%\newtheorem{exercicio}{Exercício}

\theoremstyle{remark}
\newtheorem{obs}[teorema]{Observação}
\newtheorem{observacao}[teorema]{Observação}
\newtheorem{corolario}[teorema]{Corolário}
\newtheorem{exemplo}[teorema]{Exemplo}
\newtheorem{lema}[teorema]{Lema}
\newtheorem{proposicao}[teorema]{Proposição}

\newcounter{exercicios}
\newenvironment{exercicio}{\stepcounter{exercicios} \textbf{\arabic{exercicios}}.}{}

% compatibilidade
\newcommand{\Ex}[1]{\begin{exercicio}#1\end{exercicio}}

%
%   Definições e comandos auxiliares do preâmbulo
%

\newcommand{\capitulo}[1]{\lecture[#1]{Capítulo}}
\newcommand{\aula}[1]{\subtitle{#1}}
\newcommand{\autor}{Igor Oliveira}
\newcommand{\email}{\href{mailto:matematicaelementar@imd.ufrn.br}{\texttt{matematicaelementar@imd.ufrn.br}}}
\newcommand{\disciplina}{Matemática Elementar}
\newcommand{\codigo}{IMD1001}

\title{\disciplina}
\date{\today}
\author[\autor]
{
    \autor\\
    \email
}

\def\lecturename{\codigo

\disciplina}

\institute[
	UFRN\\
	Natal-RN
]
{
	Instituto Metrópole Digital\\
	Universidade Federal do Rio Grande do Norte\\
	Natal-RN

}

% compatibilidade
\newcommand{\vu}{\vec{u}}
\newcommand{\vv}{\vec{v}}
\newcommand{\vi}{\vec{i}}
\newcommand{\vj}{\vec{j}}
\newcommand{\vk}{\vec{k}}
\newcommand{\vw}{\vec{w}}
\newcommand{\segmento}[2]{\overline{#1#2}}
\def\colc#1{\left[#1\right]}
\newcommand{\negacao}{\sim}

\justifying


%lecture[number of class]{type}
\lecture[3]{Capítulo}

\def\lecturename{IMD1001


Matemática Elementar}
\def\semestre{2018.2}
\def\nomedoautor{Igor Oliveira}
\def\emaildoautor{\href{mailto:igoroliveira@imd.ufrn.br}{{\tt igoroliveira@imd.ufrn.br}}}

\title[\lecturename]% optional, use only with long paper titles
{Matemática Elementar}

\subtitle{Equações e Inequações}  % could also be a conference name

\date{\today}

\author[Igor Oliveira] % optional, use only with lots of authors
{
	\nomedoautor\\
	\emaildoautor
}
% - Give the names in the same order as they appear in the paper.
% - Use the \inst{?} command only if the authors have different
%   affiliation. See the beamer manual for an example

\institute[
%  {\includegraphics[scale=0.2]{aau_segl}}\\ %insert a company, department or university logo
	UFRN\\
	Natal-RN
] % optional - is placed in the bottom of the sidebar on every slide
{% is placed on the title page
	Instituto Metrópole Digital\\
	Universidade Federal do Rio Grande do Norte\\
	Natal-RN

	%there must be an empty line above this line - otherwise some unwanted space is added between the university and the country (I do not know why;( )
}


\begin{document}

\mode<article>
{
\begin{frame} % the plain option removes the sidebar and header from the title page
	\maketitle
\end{frame}

% the titlepage
\begin{frame}[plain,noframenumbering] % the plain option removes the sidebar and header from the title page
	\titlepage
\end{frame}
}

\mode<presentation>
{
% the titlepage
{\imagemfundo
\begin{frame}[plain,noframenumbering] % the plain option removes the sidebar and header from the title page
	\titlepage
\end{frame}}
}

% TOC
\begin{frame}{Índice}{}
\tableofcontents
\end{frame}
%%%%%%%%%%%%%%%%

\section{Introdução}

\frame { \frametitle{Apresentação da Aula}

Como você responderia se te perguntassem: Qual o número cujo dobro
somado com sua quinta parte é igual a 121?

Você já viu alguma brincadeira do tipo?\\
\begin{enumerate}
	\item Escolha um número;
	\item Multiplique esse número por 6;
	\item Some 12;
	\item Divida por 3;
	\item Subtraia o dobro do número que você escolheu; \pause
	\item O resultado é igual a 4.
\end{enumerate}
}

%------------------------------------------------------------------------------------------------------------

\section{Equação do 1º grau}
\frame { \frametitle{Equação do 1º grau}
\begin{Def}
Uma \sub{equação do primeiro grau} na variável $x$ é uma expressão
da forma $$ax+b=0,$$ onde $a,b \in \R$, $a \neq 0$ e $x$ é um número
real a ser encontrado.
\end{Def}

\begin{Prop}[Propriedades]
Sejam $a, b, c \in \R$. Os seguintes valem:
\begin{enumerate}[i.]
	\item $a=b \implies a+c = b+c$;
	\item $a=b \implies ac = bc$.
\end{enumerate}
\end{Prop}
}

%------------------------------------------------------------------------------------------------------------

\begin{frame}
\frametitle{Equação do 1º grau} %\framesubtitle{Exemplos}
\begin{Exem}
Resolva a equação $5x - 3 = 6$.
\end{Exem}

\begin{Exem}
Escreva em forma de expressões cada passo da brincadeira da
Introdução:
\begin{enumerate}
	\item Escolha um número;
	\item Multiplique esse número por 6;
	\item Some 12;
	\item Divida por 3;
	\item Subtraia o dobro do número que você escolheu;
	\item O resultado é igual a 4.
\end{enumerate}
\end{Exem}

\end{frame}



%------------------------------------------------------------------------------------------------------------

\begin{frame}
\frametitle{Equação do 1º grau} %\framesubtitle{Exemplos}
\begin{block}{Observação}
Muito cuidado ao efetuar divisões em ambos os lados de uma equação
para não cometer o erro de dividir os lados por zero. Já vimos em
sala uma prova obviamente falsa que $1=2$, você lembra? Tente
fazê-la.
\end{block}

\end{frame}


%------------------------------------------------------------------------------------------------------------


\section{Atividade Online}
\begin{frame}
\frametitle{Atividade Online} %\framesubtitle{Exemplos}

\link{https://pt.khanacademy.org/math/cc-sixth-grade-math/cc-6th-equations-and-inequalities/cc-6th-super-yoga/e/model-with-one-step-equations-and-solve}
{Atividade Online 06 - Modelo com equações de primeiro grau e
resolução}



Veja o desempenho na Missão 7º ano -- Introdução às equações e
inequações

\end{frame}
%------------------------------------------------------------------------------------------------------------


\begin{frame}
\frametitle{Equação do 1º grau} %\framesubtitle{Exemplos}


\begin{Exem}
Se $x$ representa um dígito na base 10 na equação $$x11 + 11x + 1x1
= 777,$$ qual o valor de $x$?
\end{Exem}
\end{frame}



%------------------------------------------------------------------------------------------------------------

\begin{frame}
\frametitle{Equação do 1º grau} %\framesubtitle{Exemplos}

\begin{Exem}
Determine se é possível completar o preenchimento do tabuleiro
abaixo com os números naturais de 1 a 9, sem repetição, de modo que
a soma de qualquer linha seja igual à de qualquer coluna ou
diagonal.

\begin{center}
\begin{tabular}{|c|c|c|}
	\hline
	% after \\: \hline or \cline{col1-col2} \cline{col3-col4} ...
	1 &   & 6 \\ \hline
		&   &   \\ \hline
		& 9 &   \\
	\hline
\end{tabular}
\end{center}

\end{Exem}

Os tabuleiros preenchidos com essas propriedades são conhecidos como
\sub{quadrados mágicos}.
\end{frame}



%------------------------------------------------------------------------------------------------------------

\begin{frame}
\frametitle{Equação do 1º grau} %\framesubtitle{Exemplos}

\begin{Exem}
Imagine que você possui um fio de cobre extremamente longo, mas tão
longo que você consegue dar a volta na Terra com ele. Para
simplificar, considere que a Terra é uma bola redonda e que seu raio
é de exatamente 6.378.000 metros.

O fio com seus milhões de metros está ajustado à Terra, ficando bem
colado ao chão ao longo do Equador. Digamos, agora, que você
acrescente 1 metro ao fio e o molde de modo que ele forme um círculo
enorme, cujo raio é um pouco maior que o raio da Terra e tenha o
mesmo centro. Você acha que essa folga será de que tamanho?
\end{Exem}

\pause Já sabemos que a folga obtida aumentando o fio independe do
raio em consideração. Além desse problema, veja outras curiosidades
sobre o número $\pi$ no vídeo \link
{https://www.youtube.com/watch?v=evfc6bv6_lM}{O Pi existe} e tente
calculá-o em casa usando algum objeto redondo.


\end{frame}

%------------------------------------------------------------------------------------------------------------
\section{Equação do 2º grau}
\begin{frame}
\frametitle{Equação do 2º grau} %\framesubtitle{Exemplos}

\begin{Def}
A \sub{equação do segundo grau} com coeficientes $a$, $b$ e $c$ é
uma equação da forma $$ax^2 + bx + c = 0,$$ onde $a, b, c \in \R$,
$a \neq 0$ e $x$ é uma variável real a ser determinada.
\end{Def}

\begin{Exem}
Encontre as soluções de uma equação do segundo grau.
\end{Exem}

\end{frame}

%------------------------------------------------------------------------------------------------------------

\section{Atividade Online}
\begin{frame}
\frametitle{Atividade Online} %\framesubtitle{Exemplos}

\link{https://pt.khanacademy.org/math/algebra/quadratics/quadratics-square-root/e/quadratics-by-taking-square-roots-with-steps}
{Atividade Online 07 - Equações do segundo grau com cálculo de
raízes quadradas: com etapas}

\link{https://pt.khanacademy.org/math/algebra/quadratics/solving-quadratics-by-completing-the-square/e/completing_the_square_2}
{Atividade Online 08 - Método de completar quadrados}

Veja o desempenho na Missão Álgebra I -- Equações do segundo grau

\end{frame}
%------------------------------------------------------------------------------------------------------------

\begin{frame}
\frametitle{Equação do 2º grau} %\framesubtitle{Exemplos}

\begin{Def}
Chamamos de \sub{discriminante} da equação do segundo grau a
expressão $b^2-4ac$ e denotamos pela letra grega maiúscula $\Delta$
(lê-se delta).
\end{Def}

Em resumo:
\begin{itemize}
	\item Se $\Delta > 0$, existem duas soluções reais;
	\item Se $\Delta = 0 $, existe uma solução real ($x_1 = x_2 =
	-b/2a)$;
	\item Se $\Delta < 0$, não existe solução real.
\end{itemize}

\end{frame}



%------------------------------------------------------------------------------------------------------------

\begin{frame}
\frametitle{Equação do 2º grau} %\framesubtitle{Exemplos}

\begin{Exem}
Sabendo que $x$ é um número real que satisfaz $$x = 1 + \frac 1 {1 +
\frac 1 x},$$ determine os valores possíveis de $x$.
\end{Exem}
\pause
\begin{block}{Observação}
O número $\phi = \frac{\paren {1+\sqrt 5}}2$ é conhecido como razão
áurea, número de ouro, proporção divina, entre outras denominações.
Veja o episódio A Proporção Divina
\link{https://www.youtube.com/watch?v=mfL6-g5mQw4}{parte 01} e
\link{https://www.youtube.com/watch?v=xtsTXAwWF20&}{parte 02} do
programa português Isto É Matemática.
\end{block}

\end{frame}

%------------------------------------------------------------------------------------------------------------
\section{Atividade Online}
\begin{frame}
\frametitle{Atividade Online} %\framesubtitle{Exemplos}

\link{https://pt.khanacademy.org/math/algebra/quadratics/solving-quadratics-using-the-quadratic-formula/e/quadratic_equation}
{Atividade Online 09 - Fórmula de Bhaskara}


Veja o desempenho na Missão Álgebra I -- Equações do segundo grau

\end{frame}
%------------------------------------------------------------------------------------------------------------

\begin{frame}
\frametitle{Equação do 2º grau} %\framesubtitle{Exemplos}

\begin{Teo}
Os números $\alpha$ e $\beta$ são as raízes da equação do segundo
grau $$ax^2+ bx+c=0$$ se, e somente se, $$\alpha + \beta = \frac
{-b} a \; \text{ e } \; \alpha \beta = \frac c a.$$
\end{Teo}

\end{frame}


%------------------------------------------------------------------------------------------------------------
\section{Inequação do 1º grau}
\begin{frame}
\frametitle{Inequação do 1º grau} %\framesubtitle{Exemplos}

\begin{Def}
Uma \sub{inequação do primeiro grau} é uma relação de uma das formas
abaixo $$ax+b <0, \; \; ax+b>0,$$ $$ax+b \leq 0, \; \; ax+b \geq
0,$$ onde $a, b \in \R$ e $ a \neq 0$. Lemos os símbolos da seguinte
maneira: $<$ (menor que), $>$ (maior que), $\leq$ (menor ou igual
que) e $\geq$ (maior ou igual que).
\end{Def}

O \sub{conjunto solução} de uma inequação do primeiro grau é o
conjunto $\mathcal{S}$ de números reais que satisfazem a inequação,
isto é, o conjunto de números que, quando substituídos na inequação,
tornam a desigualdade verdadeira.

\end{frame}


%------------------------------------------------------------------------------------------------------------
\begin{frame}
\frametitle{Inequação do 1º grau} %\framesubtitle{Exemplos}

\begin{Prop}[Propriedades de inequações]
Sejam $a, b, c, d \in \R$; $n \in \N^*$. Valem:
\begin{enumerate}[i.]
	\item Invariância por adição de números reais: $a < b \implies a+c < b+c$;
	\item Invariância por multiplicação de números reais positivos:
	$a < b ; c>0 \implies a \cdot c < b \cdot c$;
	\item Mudança por multiplicação de números reais
	negativos: $a < b ; c<0 \implies a \cdot c > b \cdot c$;
	\item Se $a < b$, então $\frac 1 a > \frac 1 b$, para $a, b \neq
	0$;
	\item Se $a,b \geq 0$ e $c>0$, segue que $a < b \implies a^c < b^c$;
	\item Se $a,b < 0$ e $n$ par, segue que $a < b \implies a^n > b^n$;
	\item Se $a,b < 0$ e $n$ ímpar, segue que $a < b \implies a^n <
	b^n$;
	\item Se $a< b$ e $c< d$, então $a+c < b+d$;
	\item Para $a, b, c, d \in \R_+$. Se $a< b$ e $c< d$, então $ac < bd$.
\end{enumerate}
O resultado é análogo para os tipos $>$, $\leq$ ou $\geq$.
\end{Prop}


\end{frame}


%------------------------------------------------------------------------------------------------------------
\begin{frame}
\frametitle{Inequação do 1º grau} %\framesubtitle{Exemplos}

\begin{Exem}
Qual o conjunto solução da inequação $8x - 4 \geq 0$?
\end{Exem}

\begin{Exem}
Antes de fazer os cálculos, diga: qual dos números $a = 3456784
\cdot 3456786 + 3456785$ e $b = 3456785^2 - 3456788$ é maior?
\end{Exem}

\end{frame}



%------------------------------------------------------------------------------------------------------------
\section{Atividade Online}
\begin{frame}
\frametitle{Atividade Online} %\framesubtitle{Exemplos}

\link{https://pt.khanacademy.org/math/cc-sixth-grade-math/cc-6th-equations-and-inequalities/cc-6th-inequalities/e/inequalities-in-one-variable-1}
{Atividade Online 10 - Problemas com Inequações}


Veja o desempenho na Missão 7º Ano -- Introdução às Equações e
Inequações

\end{frame}
%------------------------------------------------------------------------------------------------------------
\section{Inequação do 2º grau}
\begin{frame}
\frametitle{Inequação do 2º grau} %\framesubtitle{Exemplos}

\begin{Def}
Uma \sub{inequação do segundo grau} é uma relação de uma das formas
abaixo
$$ax^2 +bx + c <0, \; \; ax^2 +bx + c>0,$$ $$ax^2 +bx + c \leq 0, \; \; ax^2 +bx + c \geq 0,$$ onde
$a, b, c \in \R$ e $ a \neq 0$.
\end{Def}


\end{frame}

%------------------------------------------------------------------------------------------------------------

\begin{frame}
\frametitle{Inequação do 2º grau} %\framesubtitle{Exemplos}

\begin{Exem}
Resolva as seguintes inequações: $x^2 -3x +2 > 0$; $x^2 -3x +2 \leq
0$.
\end{Exem}

\begin{Exem}
Prove que a soma de um número positivo com seu inverso é sempre
maior ou igual que 2.
\end{Exem}

\end{frame}
%------------------------------------------------------------------------------------------------------------
\section{Módulos}
\begin{frame}
\frametitle{Definição de Módulo} %\framesubtitle{Exemplos}

\begin{Def}
O \sub{módulo} (ou \sub{valor absoluto}) de um número real $x$,
denotado por $\modu x$, é definido por:
$$
\modu x =
\begin{cases}
x , & \text{se $x\geq 0$} \\
-x, & \text{se $x<0$}.
\end{cases}
$$
\end{Def}


\end{frame}

%------------------------------------------------------------------------------------------------------------
\begin{frame}
\frametitle{Equações Modulares} %\framesubtitle{Exemplos}

Para resolver equações modulares, usaremos dois métodos:
\begin{itemize}
	\item Eliminação do módulo pela definição;
	\item Partição em intervalos.
\end{itemize}

\begin{Exem}
Resolva as equações
\begin{enumerate}[(a)]
	\item  $\modu {2x-5} = 3$;
	\item $\modu{2x-3} = 1-3x$;
	\item $\modu{3-x} - \modu{x+1} = 4$.
\end{enumerate}

\end{Exem}


\end{frame}
%------------------------------------------------------------------------------------------------------------
\section{Atividade Online}
\begin{frame}
\frametitle{Atividade Online} %\framesubtitle{Exemplos}

\link{https://pt.khanacademy.org/math/algebra-home/alg-absolute-value/alg-absolute-value-equations/e/absolute_value_equations}
{Atividade Online 11 - Resolva Equações Modulares}


Veja o desempenho na Missão O Mundo da Matemática

\end{frame}
%------------------------------------------------------------------------------------------------------------
\begin{frame}
\frametitle{Inequações Modulares} %\framesubtitle{Exemplos}

Para solucionarmos inequações modulares, usaremos as propriedades a
seguir:
\begin{Prop}[Propriedades]
Sejam $x \in \R$, $a\in \R^{\ast}_+ $.
\begin{enumerate}[(i)]
	\item $\modu x \geq 0$;
	\item $\modu x < a \Leftrightarrow -a < x < a$;
	\item $\modu x > a \Leftrightarrow x > a$ ou $-x < -a$;
	\item $-\modu x \leq x \leq \modu x$.

	Os resultados (ii) e (iii) também são válidos para os casos com $\leq$ ou
	$\geq$.
\end{enumerate}

\end{Prop}

\end{frame}

%------------------------------------------------------------------------------------------------------------
\begin{frame}
\frametitle{Inequações Modulares} %\framesubtitle{Exemplos}
\begin{Exem}
Resolva as inequações
\begin{enumerate}[(a)]
	\item  $\modu {2x-5} < 3$;
	\item $\modu{2x-3} \geq 1-3x$;
	\item $\modu{3-x} - \modu{x+1} \leq 4$.
\end{enumerate}

\end{Exem}


\end{frame}

%------------------------------------------------------------------------------------------------------------
\section{Desigualdades clássicas}
\begin{frame}
\frametitle{Desigualdades clássicas} %\framesubtitle{Exemplos}

Para iniciar, apresentamos algumas desigualdades simples mas
famosas, válidas para quaisquer $a,b \in \R$:
\begin{itemize}
	\item $\modu a \geq 0$;
	\item $a^2 \geq 0$;
	\item $\modu {a+b} \leq \modu a + \modu b$ (desigualdade triangular).
\end{itemize}



\end{frame}
%------------------------------------------------------------------------------------------------------------
\begin{frame}
\frametitle{Desigualdades clássicas} %\framesubtitle{Exemplos}

\begin{Teo}
Para quaisquer $x, y \in \R$ vale
\begin{equation}
		xy \leq \frac {x^2 +y^2} 2.
\end{equation}
Além disso, a igualdade acontece se, e somente se, $x=y$.
\end{Teo}

Vejamos no quadro um experimento geométrico relacionado a essa
desigualdade.
\end{frame}
%------------------------------------------------------------------------------------------------------------
\begin{frame}
\frametitle{Desigualdades clássicas} %\framesubtitle{Exemplos}
\begin{Teo}
Para quaisquer $a, b \in \R_+$ vale
\begin{equation}
		\sqrt{ab} \leq \frac {a +b} 2.
\end{equation}
Além disso, a igualdade acontece se, e somente se, $a=b$.
\end{Teo}

\begin{Teo}[Desigualdade das médias aritmética e geométrica]
Para quaisquer $a_1, a_2, \dots , a_n \in \R_+$ vale
\begin{equation}
		\sqrt[n]{a_1\dots a_n} \leq \frac {a_1 + \dots + a_n} n.
\end{equation}
\end{Teo}

\end{frame}
%------------------------------------------------------------------------------------------------------------
\begin{frame}
\frametitle{Desigualdades clássicas} %\framesubtitle{Exemplos}
\begin{Teo}[Desigualdade das médias harmônica e geométrica]
Para quaisquer $a_1, a_2, \dots , a_n \in \R_+^*$ vale
\begin{equation}
		\frac n {\frac 1 {a_1} + \dots + \frac 1 {a_n}}  \leq \sqrt[n]{a_1\dots a_n}  .
\end{equation}
\end{Teo}

\begin{Teo}[Desigualdade de Cauchy-Schwarz]
Sejam $x_1, \dots , x_n, y_1, \dots y_n \in \R$, então vale:
\begin{equation}
		\modu{x_1y_1 + \dots + x_ny_n} \leq \sqrt{x^2_1+ \dots + x^2_n}
		\cdot \sqrt{y^2_1+ \dots + y^2_n}.
\end{equation}
Além disso, a igualdade só ocorre se existir um número real $\alpha$
tal que $x_1 = \alpha y_1$, ..., $x_n = \alpha y_n$.
\end{Teo}

\end{frame}
%------------------------------------------------------------------------------------------------------------
\begin{frame}
\frametitle{Aplicações} %\framesubtitle{Exemplos}
\begin{Exem}
Duas torres são amarradas por uma corda $APB$ que vai do topo $A$ da
primeira torre para um ponto $P$ no chão, entre as torres, e então
até o topo $B$ da segunda torre. Qual a posição do ponto $P$ que nos
dá o comprimento mínimo da corda a ser utilizada?
\end{Exem}
\end{frame}
%------------------------------------------------------------------------------------------------------------
\begin{frame}
\frametitle{Aplicações} %\framesubtitle{Exemplos}
\begin{Exem}
Prove que, num triângulo retângulo, a altura relativa à hipotenusa é
sempre menor ou igual que a metade da hipotenusa. Prove, ainda, que
a igualdade só ocorre quando o triângulo retângulo é isósceles.
\end{Exem}
\end{frame}
%------------------------------------------------------------------------------------------------------------
\begin{frame}
\frametitle{Aplicações} %\framesubtitle{Exemplos}
\begin{Exem}
Prove que, entre todos os triângulos retângulos de catetos $a$ e
$b$, e com hipotenusa $c$ fixada, o que tem maior soma dos catetos
$S = a+b$ é o triângulo isósceles.
\end{Exem}

\end{frame}
%------------------------------------------------------------------------------------------------------------

\section{Exercícios}
\begin{frame}
\frametitle{Exercícios} %\framesubtitle{Exemplos}

\Ex{ Descubra os valores de $x$ de modo que seja possível completar
o preenchimento do quadrado mágico abaixo:
\begin{center}
\begin{tabular}{|c|c|c|}
	\hline
	% after \\: \hline or \cline{col1-col2} \cline{col3-col4} ...
	 &   &  \\ \hline
		& $x$ &   \\ \hline
		&  &   \\
	\hline
\end{tabular}
\end{center}
}

\Ex{ Observe as multiplicações a seguir:
\begin{enumerate}[i.]
	\item $12.345.679 \cdot 18 = 222.222.222$
	\item $12.345.679 \cdot 27 = 333.333.333$
	\item $12.345.679 \cdot 54 = 666.666.666$
\end{enumerate}
Para obter 999.999.999 devemos multiplicar 12.345.679 por quanto?
 }




\end{frame}





%------------------------------------------------------------------------------------------------------------

\begin{frame}
\frametitle{Exercícios} %\framesubtitle{Exemplos}

\Ex{ Com algarismos $x$, $y$ e $z$ não todos nulos formam-se os
números de dois algarismos $xy$ e $yx$, cuja soma é o número de três
algarismos $zxz$. Quanto valem $x$, $y$ e $z$?}

\Ex{ Quantos são os números inteiros de 2 algarismos que são iguais
ao dobro do produto de seus algarismos?}

\Ex{ O número $-3$ é a raiz da equação $x^2 -7x -2c = 0$. Nessas
condições, determine o valor do coeficiente $c$. }

\Ex{ Dada as frações $$\frac{966666555557}{966666555558} \; \text{ e
} \; \frac{966666555558}{966666555559},$$ qual é a maior? }

\Ex{Nove cópias de certas notas custam menos de R\$ 10,00 e dez
cópias das mesmas notas (custando o mesmo preço cada uma) custam
mais de R\$ 11,00. Quanto custa uma cópia das notas? }

\end{frame}


%------------------------------------------------------------------------------------------------------------

\begin{frame}
\frametitle{Exercícios} %\framesubtitle{Exemplos}

\Ex{ Ache os valores de $x$ para os quais cada uma das seguintes
expressões é positiva:
\begin{enumerate}[a.]
	\item $$\frac x {x^2+9};$$
	\item $$\frac{x-3}{x+1};$$
	\item $$\frac{x^2-1}{x^2-3}.$$
\end{enumerate}
}

\Ex{ Sejam  $a$, $b$, $c$, $d > 0$ tais que $\frac a b < \frac c d$.
Mostre que $$\frac a b < \frac {a+c} {b+d} < \frac c d.$$ }


\end{frame}


%------------------------------------------------------------------------------------------------------------
\begin{frame}
\frametitle{Exercícios} %\framesubtitle{Exemplos}

\Ex{ Determine o conjunto solução de cada uma das equações ou
inequações modulares abaixo:
\begin{enumerate}[a.]
	\item $\modu {3x - 5}= 7;$
	\item $\modu {-x +8} = -1;$
	\item $\modu{x^2 - 1}= 3;$
	\item $\modu{x +1} + \modu { -3x +2} = 6;$
	\item $\modu{x-1} \cdot \modu{x+2} = 3;$
	\item $\modu{x-1}+ \modu{x+1}>2;$
	\item $\modu{x+1}- \modu{x-1}<-2.$
\end{enumerate}
}

\Ex{ Prove que $\modu{x\cdot y } = \modu x \cdot \modu y$ para todo
$x, y \in \R$. }

\Ex{Seja $x \in \R$. Mostre que:
\begin{enumerate}[a.]
		\item $\modu{x-5}<0,1 \implies \modu{2x - 10}< 0,2;$
		\item $\modu{x+3}<0,1 \implies \modu{-\frac 3 2 x + 3 -
		7,5}<0,15;$
	\item $\modu{x-2}< \sqrt{5} - 2 \implies \modu{x^2 - 4}< 1.$
\end{enumerate}
}

\end{frame}


%------------------------------------------------------------------------------------------------------------

\begin{frame}
\frametitle{Exercícios} %\framesubtitle{Exemplos}
\Ex{Provar que em todo triângulo a soma dos comprimentos das
medianas é menor que o perímetro do triângulo e maior que o
semiperímetro (metade do perímetro) dele. }

\Ex{ Prove que $a^4 +b^4 +c^4 \geq abc \paren{a+b+c}$. }

\Ex{ Sejam $a, b, c \in \R_+$. Prove que $$\paren {a+b} \paren {a+c}
\paren {b+c} \geq 8abc.$$ }

\Ex{ Sejam $a, b, c, d \in \R_+^*$. Prove que $$\paren {a+b+c+d}
\paren {\frac 1 a + \frac 1 b + \frac 1 c + \frac 1 d} \geq 16.$$ }

\Ex{A soma de três números positivos é 6. Prove que a soma de seus
quadrados não é menor que 12. }
\end{frame}

%------------------------------------------------------------------------------------------------------------

\section{Bibliografia}

\frame{
	\frametitle{Bibliografia}
	\begin{thebibliography}{99}
		\bibitem {label1}
		OLIVEIRA, Krerley I M; FERNÁNDEZ, Adán J C.
		\newblock \emph{Iniciação à Matemática: um Curso com Problemas e Soluções}.
		\newblock 2. ed. Rio de Janeiro: SBM, 2010.

		\bibitem {label2}
		OLIVEIRA, Krerley I M; FERNÁNDEZ, Adán J C.
		\newblock \emph{Estágio dos Alunos Bolsistas - OBMEP 2005 - 4. Equações, Inequações e Desigualdades}.
		\newblock Rio de Janeiro: SBM, 2006.

	\end{thebibliography}
}

\end{document}
