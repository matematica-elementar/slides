\documentclass[brazil, notheorems, 10pt]{beamer}
%
%   Arquivo de Configuração dos Slides
%


%
%   Pacotes utilizados
%

% Codificação dos caracteres em formato universal.
\usepackage[utf8]{inputenc}
\usepackage[T1]{fontenc}

% Traduz o texto gerados pelo LaTeX para português. ex.: Capítulo, Seção, Conteúdo.
\usepackage[brazil]{babel}

% Pacotes para ambientes matemáticos
\usepackage{amsmath}
\usepackage{amsthm}
\usepackage{amssymb}

% Diversas funções para o uso das aspas.
\usepackage{csquotes}

% Outros pacotes
\usepackage{hyperref}
\usepackage{tikz}
\usepackage{yfonts}
\usepackage{colortbl}
\usepackage{ragged2e}
\usepackage{helvet}
\usepackage{verbatim}


%
%   Tema
%

% Copyright 2007 by Till Tantau
%
% This file may be distributed and/or modified
%
% 1. under the LaTeX Project Public License and/or
% 2. under the GNU Public License.
%
% See the file doc/licenses/LICENSE for more details.


% Common packages


\usepackage{times}
 \mode<article> {
	\usepackage{times}
	\usepackage{mathptmx}
	\usepackage[left=1.5cm,right=6cm,top=1.5cm,bottom=3cm]{geometry}
}

\usepackage{hyperref}
\usepackage[T1]{fontenc}
\usepackage{amsmath,amssymb}
\usepackage{tikz}
\usepackage{colortbl}
\usepackage{yfonts}
\usepackage{colortbl}
\usepackage{translator} % comment this, if not available
\usepackage{ragged2e} % justifying
% Or whatever. Note that the encoding and the font should match. If T1
% does not look nice, try deleting the line with the fontenc.
\usepackage{helvet}
\usepackage{verbatim}


%\usepackage{lipsum}
%\usepackage{enumitem}


\usetheme[
%%% options passed to the outer theme
%    hidetitle,           % hide the (short) title in the sidebar
%    hideauthor,          % hide the (short) author in the sidebar
%    hideinstitute,       % hide the (short) institute in the bottom of the sidebar
%    shownavsym,          % show the navigation symbols
%    width=2cm,           % width of the sidebar (default is 2 cm)
%    hideothersubsections,% hide all subsections but the subsections in the current section
%    hideallsubsections,  % hide all subsections
		right               % right of left position of sidebar (default is right)
%%% options passed to the color theme
%    lightheaderbg,       % use a light header background
	]{AAUsidebar}

% If you want to change the colors of the various elements in the theme, edit and uncomment the following lines
% Change the bar and sidebar colors:
%\setbeamercolor{AAUsidebar}{fg=red!20,bg=red}
%\setbeamercolor{sidebar}{bg=red!20}
% Change the color of the structural elements:
%\setbeamercolor{structure}{fg=red}
% Change the frame title text color:
%\setbeamercolor{frametitle}{fg=blue}
% Change the normal text color background:
%\setbeamercolor{normal text}{bg=gray!10}
% Highlight the text in the sidebar
\usecolortheme{rose,sidebartab}
% ... and you can of course change a lot more - see the beamer user manual.

% colored hyperlinks
\newcommand{\chref}[2]{%
	\href{#1}{{\usebeamercolor[bg]{AAUsidebar}#2}}%
}



% specify a logo on the titlepage (you can specify additional logos an include them in
% institute command below
\pgfdeclareimage[height=1cm]{titlepagelogo}{theme/figures/ufrn2} % placed on the title page
\pgfdeclareimage[height=1cm]{titlepagelogo2}{theme/figures/imd} % placed on the title page
\titlegraphic{% is placed on the bottom of the title page
	\pgfuseimage{titlepagelogo}
	\hspace{1cm}\pgfuseimage{titlepagelogo2}
}


% Article version layout settings

\mode<article>

\makeatletter
\def\@listI{\leftmargin\leftmargini
	\parsep 0pt
	\topsep 5\p@   \@plus3\p@ \@minus5\p@
	\itemsep0pt}
\let\@listi=\@listI


\setbeamertemplate{frametitle}{\paragraph*{\insertframetitle\
		\ \small\insertframesubtitle}\ \par
}
\setbeamertemplate{frame end}{%
	\marginpar{\scriptsize\hbox to 1cm{\sffamily%
			\hfill\strut\insertshortlecture.\insertframenumber}\hrule height .2pt}}
\setlength{\marginparwidth}{1cm}
\setlength{\marginparsep}{4.5cm}

\def\@maketitle{\makechapter}

\def\makechapter{
	\newpage
	\null
	\vskip 2em%
	{%
		\parindent=0pt
		\raggedright
		\sffamily
		\vskip8pt
		\includegraphics[width=\linewidth]{theme/figures/imd.png}\par\vskip2em
		{\fontsize{36pt}{36pt}\selectfont Aula \insertshortlecture \par\vskip2pt}
		{\fontsize{24pt}{28pt}\selectfont \color{blue!50!black} \@title\par\vskip4pt}
		%{\Large\selectfont \color{blue!50!black} \insertsubtitle\par}
		\vskip10pt

		\normalsize\selectfont [Notas de Aula]
		Disciplina: \emph{\lecturename \ (\semestre)} \par\vskip1.5em
		\nomedoautor\hskip1em Email: \ \emaildoautor
	}
	\par
	\vskip 1.5em%
}

\let\origstartsection=\@startsection
\def\@startsection#1#2#3#4#5#6{%
	\origstartsection{#1}{#2}{#3}{#4}{#5}{#6\normalfont\sffamily\color{blue!50!black}\selectfont}}

\makeatother

\mode
<all>




% Typesetting Listings

\usepackage{listings}
\lstset{language=Java}

\alt<presentation>
{\lstset{%
	basicstyle=\footnotesize\ttfamily,
	commentstyle=\slshape\color{green!50!black},
	keywordstyle=\bfseries\color{blue!50!black},
	identifierstyle=\color{blue},
	stringstyle=\color{orange},
	escapechar=\#,
	emphstyle=\color{red}}
}
{
	\lstset{%
		basicstyle=\ttfamily,
		keywordstyle=\bfseries,
		commentstyle=\itshape,
		escapechar=\#,
		emphstyle=\bfseries\color{red}
	}
}



% Common theorem-like environments
%\usepackage{amsthm}

\setbeamertemplate{theorems}[numbered]

%
%	New useful definitions:
%

\newbox\mytempbox
\newdimen\mytempdimen

\newcommand\includegraphicscopyright[3][]{%
	\leavevmode\vbox{\vskip3pt\raggedright\setbox\mytempbox=\hbox{\includegraphics[#1]{#2}}%
		\mytempdimen=\wd\mytempbox\box\mytempbox\par\vskip1pt%
		\fontsize{3}{3.5}\selectfont{\color{black!25}{\vbox{\hsize=\mytempdimen#3}}}\vskip3pt%
}}

\newenvironment{colortabular}[1]{\medskip\rowcolors[]{1}{blue!20}{blue!10}\tabular{#1}\rowcolor{blue!40}}{\endtabular\medskip}

\def\equad{\leavevmode\hbox{}\quad}

\newenvironment{greencolortabular}[1]
{\medskip\rowcolors[]{1}{green!50!black!20}{green!50!black!10}%
	\tabular{#1}\rowcolor{green!50!black!40}}%
{\endtabular\medskip}

%\setbeamertemplate{theorem begin}{{ \inserttheoremheadfont
%\inserttheoremname \inserttheoremnumber
%\ifx\inserttheoremaddition\empty\else\ (\inserttheoremaddition)\fi%
%\inserttheorempunctuation }} \setbeamertemplate{theorem end}{}

\newcommand{\vu}{\vec{u}}
\newcommand{\vv}{\vec{v}}
\newcommand{\vi}{\vec{i}}
\newcommand{\vj}{\vec{j}}
\newcommand{\vk}{\vec{k}}
\newcommand{\vw}{\vec{w}}
\newcommand\segmento[2]{\overline{#1#2}}
\def\colc#1{\left[#1\right]}



%
%   Macros
%

\usepackage{macros/macros}


%
%   Ambientes
%

\theoremstyle{plain}
\newtheorem{teorema}{Teorema}

\theoremstyle{definition}
\newtheorem{definicao}[teorema]{Definição}
%\newtheorem{exercicio}{Exercício}

\theoremstyle{remark}
\newtheorem{obs}[teorema]{Observação}
\newtheorem{observacao}[teorema]{Observação}
\newtheorem{corolario}[teorema]{Corolário}
\newtheorem{exemplo}[teorema]{Exemplo}
\newtheorem{lema}[teorema]{Lema}
\newtheorem{proposicao}[teorema]{Proposição}

\newcounter{exercicios}
\newenvironment{exercicio}{\stepcounter{exercicios} \textbf{\arabic{exercicios}}.}{}

% compatibilidade
\newcommand{\Ex}[1]{\begin{exercicio}#1\end{exercicio}}

%
%   Definições e comandos auxiliares do preâmbulo
%

\newcommand{\capitulo}[1]{\lecture[#1]{Capítulo}}
\newcommand{\aula}[1]{\subtitle{#1}}
\newcommand{\autor}{Igor Oliveira}
\newcommand{\email}{\href{mailto:matematicaelementar@imd.ufrn.br}{\texttt{matematicaelementar@imd.ufrn.br}}}
\newcommand{\disciplina}{Matemática Elementar}
\newcommand{\codigo}{IMD1001}

\title{\disciplina}
\date{\today}
\author[\autor]
{
    \autor\\
    \email
}

\def\lecturename{\codigo

\disciplina}

\institute[
	UFRN\\
	Natal-RN
]
{
	Instituto Metrópole Digital\\
	Universidade Federal do Rio Grande do Norte\\
	Natal-RN

}

% compatibilidade
\newcommand{\vu}{\vec{u}}
\newcommand{\vv}{\vec{v}}
\newcommand{\vi}{\vec{i}}
\newcommand{\vj}{\vec{j}}
\newcommand{\vk}{\vec{k}}
\newcommand{\vw}{\vec{w}}
\newcommand{\segmento}[2]{\overline{#1#2}}
\def\colc#1{\left[#1\right]}
\newcommand{\negacao}{\sim}

\justifying


%lecture[number of class]{type}
\lecture[2]{Capítulo}

\def\lecturename{IMD1001


Matemática Elementar}
\def\semestre{2018.1}
\def\nomedoautor{Igor Oliveira}
\def\emaildoautor{\href{mailto:igoroliveira@imd.ufrn.br}{{\tt igoroliveira@imd.ufrn.br}}}

\title[\lecturename]% optional, use only with long paper titles
{Matemática Elementar}

\subtitle{Conjuntos Numéricos e Potenciação}  % could also be a conference name

\date{\today}

\author[Igor Oliveira] % optional, use only with lots of authors
{
	\nomedoautor\\
	\emaildoautor
}
% - Give the names in the same order as they appear in the paper.
% - Use the \inst{?} command only if the authors have different
%   affiliation. See the beamer manual for an example

\institute[
%  {\includegraphics[scale=0.2]{aau_segl}}\\ %insert a company, department or university logo
	UFRN\\
	Natal-RN
] % optional - is placed in the bottom of the sidebar on every slide
{% is placed on the title page
	Instituto Metrópole Digital\\
	Universidade Federal do Rio Grande do Norte\\
	Natal-RN

	%there must be an empty line above this line - otherwise some unwanted space is added between the university and the country (I do not know why;( )
}


\begin{document}

\mode<article>
{
\begin{frame} % the plain option removes the sidebar and header from the title page
	\maketitle
\end{frame}

% the titlepage
\begin{frame}[plain,noframenumbering] % the plain option removes the sidebar and header from the title page
	\titlepage
\end{frame}
}

\mode<presentation>
{
% the titlepage
{\imagemfundo
\begin{frame}[plain,noframenumbering] % the plain option removes the sidebar and header from the title page
	\titlepage
\end{frame}}
}

% TOC
\begin{frame}{Índice}{}
\tableofcontents
\end{frame}
%%%%%%%%%%%%%%%%

\section{Apresentação}

\frame
{
\frametitle{Apresentação da Aula}
\begin{block}{Motivação}
\justifying
		Os números têm grande importância na matemática; eles podem
		servir para contar ou medir coisas. Conhecer os conjuntos
		numéricos e suas operações é indispensável para trabalhar
		corretamente com os números.
 \end{block}
}


%------------------------------------------------------------------------------------------------------------

\begin{comment}
\section{Objetivos}
\frame
{
\frametitle{Objetivos}

\begin{itemize}
\item<1-> Conhecer os \alert{sinais analógicos} e digitais
\item<2-> Fazer uma comparação entre sistemas analógicos e digitais
\item<3-> Conhecer os elementos elétricos passivos : resistor, capacitor e indutor
\item<4-> Conhecer as Leis de Kirchhoff
\item<5-> Conhecer os transistores e os diodos
\item<6-> Analisar as principais aplicações de sistemas digitais.
\end{itemize}

}
\end{comment}

%------------------------------------------------------------------------------------------------------------

\section{Conjuntos Numéricos}
\frame { \frametitle{Naturais}
\begin{Def}
Ao conjunto $\N = \set {0, 1, 2, \dots , n, n+1, \dots}$ damos o
nome de \sub{conjunto dos números naturais}.
\end{Def}
\begin{itemize}
\item Denotamos $\N \setminus \set 0 = \set {1, 2, \dots , n, n+1,
\dots}$ por $\N ^\ast$.

\item Usamos o conjunto dos números naturais para contar coisas, como
casas, animais, etc.
\end{itemize}
}

%------------------------------------------------------------------------------------------------------------


\begin{frame}
\frametitle{Inteiros} %\framesubtitle{Exemplos}
\begin{Def}
Ao conjunto $\Z =\set {\dots , -m -1, -m, \dots, -1, 0, 1,  \dots ,
n, n+1, \dots}$ damos o nome de \sub{conjunto dos números inteiros}.
\end{Def}

\begin{block}{Notação}
$\Z^\ast = \Z \setminus \set 0$; \\
$\Z_+ = \N$ (Inteiros não negativos); \\
$\Z^\ast_+ =\N ^\ast$ (Inteiros positivos); \\
$\Z_- =\set {\dots , -m -1, -m, \dots, -1, 0}$ (Inteiros não
positivos); \\
$\Z_-^\ast =\Z_- \setminus \set 0$ (Inteiros negativos).
\end{block}
\end{frame}



%------------------------------------------------------------------------------------------------------------
\begin{frame}
\frametitle{Racionais} %\framesubtitle{Exemplos}
\begin{Def}
Ao conjunto $\Q = \set{\frac p q \tq p, q \in \Z \text{ e } q \neq
0}$ damos o nome de \sub{conjunto dos números racionais}.
\end{Def}

A representação decimal de um número racional é finita ou é uma
dízima periódica (infinita).
\begin{block}{Exercício}
Reescreva os números $0,6$; $1,37$; $0,222\dots$; $0,313131 \dots$ e
$1,123123123 \dots$ em forma de fração irredutível, ou seja, já
simplificada.
\end{block}
\end{frame}



%------------------------------------------------------------------------------------------------------------
\begin{frame}
\frametitle{Irracionais} %\framesubtitle{Exemplos}
\begin{Def}
O \sub{conjunto dos números irracionais} é constituído por todos os
números que possuem uma representação decimal infinita e não
periódica.
\end{Def}

\begin{Exem}
$\sqrt 2$, $e$ e $\pi$ são números irracionais. Provemos que $\sqrt
2 \notin \Q$.
\end{Exem}

Você sabia que existem infinitos ``maiores'' que outros? Qual
conjunto você diria que tem mais elementos: racionais ou
irracionais?
\end{frame}

%------------------------------------------------------------------------------------------------------------

\begin{frame}
\frametitle{Problema} %\framesubtitle{Exemplos}

O Grande Hotel Georg Cantor tinha uma infinidade de quartos,
numerados consecutivamente, um para cada número natural. Todos eram
igualmente confortáveis. Num fim de semana prolongado, o hotel
estava com seus quartos todos ocupados, quando chega um visitante. A
recepcionista vai logo dizendo: \\
-Sinto muito, mas não há vagas. \\
Ouvindo isto, o gerente interveio: \\
-Podemos abrigar o cavalheiro sim, senhora. \\
E ordena: \\
Transfira o hóspede do quarto 1 para o quarto 2, passe o do quarto 2
para o quarto 3 e assim por diante. Quem estiver no quarto $n$, mude
para o quarto $n+1$. Isto manterá todos alojados e deixará
disponível o quarto 1 para o recém chegado. Logo depois chegou um
ônibus com 30 passageiros, todos querendo hospedagem. Como deve
proceder a recepcionista para acomodar todos?
\\ Horas depois, chegou um trem com uma infinidade de
passageiros. Como proceder para acomodá-los?


\end{frame}

%------------------------------------------------------------------------------------------------------------
\section{Atividade Online}
\begin{frame}
\frametitle{Atividade Online} %\framesubtitle{Exemplos}

\href{https://pt.khanacademy.org/math/algebra/rational-and-irrational-numbers/modal/e/recognizing-rational-and-irrational-numbers}
{{\tt Atividade 02 - Classifique números: racionais e irracionais}}

\href{https://pt.khanacademy.org/math/algebra/rational-and-irrational-numbers/modal/e/recognizing-rational-and-irrational-expressions}
{{\tt Atividade 03 - Expressões racionais versus irracionais}}

Veja o desempenho na Missão Álgebra I - Números Racionais e
Irracionais


\end{frame}

%------------------------------------------------------------------------------------------------------------
\begin{frame}
\frametitle{Reais} %\framesubtitle{Exemplos}
\begin{Def}
À reunião de $\Q$ com o conjunto dos números irracionais, nomeamos
de \sub{conjunto dos números reais}. Denotamos por $\R$.
\end{Def}

\begin{itemize}
\item $\R \setminus \Q = \set {x \tq x \text{ é irracional}}$;
\item Usamos os números reais para medir algo. A cada número real
está associado um ponto na reta graduada e vice-versa.
\item Entre dois números reais distintos sempre há pelo menos um número racional e um
irracional.
\href{https://pt.khanacademy.org/math/algebra/rational-and-irrational-numbers/proofs-concerning-irrational-numbers/v/proof-that-there-is-an-irrational-number-between-any-two-rational-numbers}
{{\tt Este vídeo}} do Khan Academy mostra que entre dois racionais
distintos sempre há pelo menos um número irracional.
\item A igualdade $0,999\dots = 1 $ é verdadeira?
\end{itemize}
\end{frame}



%------------------------------------------------------------------------------------------------------------
\begin{frame}
\frametitle{Complexos} %\framesubtitle{Exemplos}
\begin{Def}
Chamamos $i = \sqrt {-1}$ de \sub{número imaginário}, e ao conjunto
$\C = \set{ a+bi \tq a,b \in \R}$ damos o nome de \sub{conjunto dos
números complexos}.
\end{Def}

Seja $a+bi \in \C$. Nomeamos o número $a-bi$ de \sub{conjugado} de
$a+bi$.

Temos a seguinte cadeia de inclusões próprias: $\N \subset \Z
\subset \Q \subset \R \subset \C$.
\end{frame}



%------------------------------------------------------------------------------------------------------------
\section{Operações}
\begin{frame}
\frametitle{Operações} %\framesubtitle{Exemplos}
Definimos duas operações básicas com os elementos dos conjuntos
numéricos: a adição e a multiplicação. A subtração e a divisão
provêm da adição e da multiplicação, respectivamente.
\begin{itemize}
	\item Adição
		\begin{itemize}
		\item Subtração: é a soma de números negativos;
		\end{itemize}
	\item Multiplicação
		\begin{itemize}
		\item Divisão: é a multiplicação de números da forma $\frac 1
		q$.
		\end{itemize}
\end{itemize}

Você está bem treinado nas operações com frações? Dê uma treinada
\href{https://pt.khanacademy.org/math/arithmetic-home/arith-review-fractions}
{{\tt aqui}} no Khan Academy!
\end{frame}



%------------------------------------------------------------------------------------------------------------
\section{Potenciação}
\begin{frame}
\frametitle{Potenciação} %\framesubtitle{Exemplos}
\begin{Def}
A \sub{potência} $n \in \N^\ast$ de um número real $a$ é definida
como sendo a multiplicação de $a$ por ele mesmo $n$ vezes, ou seja:
$$a^n = \underbrace{a \cdot a  \dots  a}_{n \text{
vezes}}.$$
\end{Def}

\begin{Def}
Quando $a \neq 0$, $a^0 = 1$. $0^0$ é uma indeterminação; \\
$a^{-n} = \frac{1}{a^n}$; \\
$a^{1/n} = \sqrt[n] a$, para $n> 0$.
\end{Def}

É importante ressaltar que é comum definir $0^0 =1$ dependendo da
abordagem que se quer com as potências. Saiba mais
\href{https://pt.wikipedia.org/wiki/Zero_elevado_a_zero}{{\tt
aqui}}.
\end{frame}



%------------------------------------------------------------------------------------------------------------
\begin{frame}
\frametitle{Potenciação} %\framesubtitle{Exemplos}
\begin{Prop}[Propriedades]
Sejam $a, b, n, m \in \R$ a menos que se diga o contrário.
\begin{enumerate}[i.]
	\item $a^m \cdot a^n = a^{m+n}$;
	\item $\frac {a^m}{a^n} = a^{m-n}$, $a \neq 0$;
	\item $\paren{a^m}^n = a^{m\cdot n}$;
	\item $a^{m^n} = a^{\overbrace{m \cdot m  \dots  m}^{n \text{
	vezes}}}$, $n \in \N^\ast$;
	\item $\paren{a \cdot b }^n= a^n \cdot b^n$;
	\item $\paren{\frac a b }^n = \frac {a^n} {b^n}$;
	\item $a^{m/n} = \sqrt[n]{a^m}$, $n \neq 0$.
\end{enumerate}
\end{Prop}

\end{frame}



%------------------------------------------------------------------------------------------------------------
\begin{frame}
\frametitle{Potenciação} %\framesubtitle{Exemplos}
\begin{Obse}
Seja $a \in \R$. Temos que $\sqrt{a^2} = \modu a$. Mais geralmente,
$\sqrt[n] {a^n} = \modu a$ para $n$ par. \\
É errado dizer que $\sqrt 4 = \pm 2$. O correto é $\sqrt 4 = 2$,
mesmo que escrevas $\sqrt 4 = \sqrt{\paren {-2}^2}$. \\
Tal erro é comum, e o fator de confusão é que responder o conjunto
solução da equação $x^2=4$ não é equivalente a responder qual a raiz
de $4$, e sim responder quais números que elevados ao quadrado são
iguais a $4$.
\end{Obse}
\end{frame}



%------------------------------------------------------------------------------------------------------------


\section{Atividade Online}
\begin{frame}
\frametitle{Atividade Online} %\framesubtitle{Exemplos}

\href{https://pt.khanacademy.org/math/algebra/rational-exponents-and-radicals/rational-exponents-and-the-properties-of-exponents/e/exponents_4}
{{\tt Atividade 04 - Propriedades da potenciação (expoentes
racionais)}}

\href{https://pt.khanacademy.org/math/algebra/rational-exponents-and-radicals/alg1-simplify-square-roots/e/multiplying_radicals}
{{\tt Atividade 05 - Simplifique raízes quadradas (variáveis)}}

\href{https://pt.khanacademy.org/math/algebra/rational-exponents-and-radicals/alg1-simplify-square-roots/e/adding_and_subtracting_radicals}
{{\tt Atividade 06 - Simplifique expressões de raiz quadrada}}



Veja o desempenho na Missão Álgebra I - Expressões com expoentes
fracionários e radicais

\end{frame}

%------------------------------------------------------------------------------------------------------------

\section{Exercícios}
\begin{frame}
\frametitle{Exercícios} %\framesubtitle{Exemplos}

\Ex{ Faça os testes do Khan Academy do assunto
\href{https://pt.khanacademy.org/math/arithmetic-home/arith-review-fractions/modal/test/practice-test}
{{\tt Frações}}. Ao final, revise os assuntos que você teve
problema. }

\Ex{ Faça o estudo completo (vídeos e exercícios) no Khan Academy do
\href{https://pt.khanacademy.org/math/algebra/rational-and-irrational-numbers}
{{\tt assunto}} Números racionais e irracionais. }

\Ex{ Faça o estudo completo (vídeos e exercícios) no Khan Academy do
\href{https://pt.khanacademy.org/math/algebra/rational-exponents-and-radicals}
{{\tt conteúdo}} Expressões com expoentes fracionários e radicais. }

\end{frame}





%------------------------------------------------------------------------------------------------------------


\begin{comment}
\subsection{Exemplo de uma figura}

\begin{frame}{Título do slide}
\framesubtitle{Sub-título do slide}
\transdissolve
	\includegraphicscopyright[width=6cm]{Figuras/teste.jpg}
	{Copyright by Sir Silva}
\end{frame}
\end{comment}

%------------------------------------------------------------------------------------------------------------

\section{Bibliografia}

\frame{
\frametitle{Bibliografia}

\begin{thebibliography}{99}

\bibitem {label1}
MEDEIROS, Valéria Z; CALDEIRA, André M; SILVA, Luiza M O; MACHADO,
Maria A S.
\newblock {\em Pré-Cálculo}.
\newblock 2. ed. Rio de Janeiro: Cengage Learning, 2009.

\bibitem {label2}
LIMA, Elon L; CARVALHO, Paulo César P; Wagner, Eduardo; MORGADO,
Augusto C.
\newblock {\em A Matemática do Ensino Médio. Vol. 1}.
\newblock 9. ed. Rio de Janeiro: SBM, 2006.

%\bibitem {label2}
%OLIVEIRA, Krerley I M; FERNÁNDEZ, Adán J C.
%\newblock {\em Iniciação à Matemática: um Curso com Problemas e Soluções}.
%\newblock 2. ed. São Paulo: Cengage Learning, 2010.


\end{thebibliography}
}


%------------------------------------------------------------------------------------------------------------


%{\aauwavesbg
%\begin{frame}[plain,noframenumbering]
%  \finalpage{Thank you for using this theme!}
%\end{frame}}
%%%%%%%%%%%%%%%%%

\end{document}
