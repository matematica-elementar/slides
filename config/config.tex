%
%   Arquivo de Configuração dos Slides
%


%
%   Pacotes utilizados
%

% Codificação dos caracteres em formato universal.
\usepackage[utf8]{inputenc}
\usepackage[T1]{fontenc}

% Traduz o texto gerados pelo LaTeX para português. ex.: Capítulo, Seção, Conteúdo.
\usepackage[brazil]{babel}

% Pacotes para ambientes matemáticos
\usepackage{amsmath}
\usepackage{amsthm}
\usepackage{amssymb}

% Diversas funções para o uso das aspas.
\usepackage{csquotes}

% Outros pacotes
\usepackage{hyperref}
\usepackage{tikz}
\usepackage{yfonts}
\usepackage{colortbl}
\usepackage{ragged2e}
\usepackage{helvet}
\usepackage{verbatim}


%
%   Tema
%

% Copyright 2007 by Till Tantau
%
% This file may be distributed and/or modified
%
% 1. under the LaTeX Project Public License and/or
% 2. under the GNU Public License.
%
% See the file doc/licenses/LICENSE for more details.


% Common packages


\usepackage{times}
 \mode<article> {
	\usepackage{times}
	\usepackage{mathptmx}
	\usepackage[left=1.5cm,right=6cm,top=1.5cm,bottom=3cm]{geometry}
}

\usepackage{hyperref}
\usepackage[T1]{fontenc}
\usepackage{amsmath,amssymb}
\usepackage{tikz}
\usepackage{colortbl}
\usepackage{yfonts}
\usepackage{colortbl}
\usepackage{translator} % comment this, if not available
\usepackage{ragged2e} % justifying
% Or whatever. Note that the encoding and the font should match. If T1
% does not look nice, try deleting the line with the fontenc.
\usepackage{helvet}
\usepackage{verbatim}


%\usepackage{lipsum}
%\usepackage{enumitem}


\usetheme[
%%% options passed to the outer theme
%    hidetitle,           % hide the (short) title in the sidebar
%    hideauthor,          % hide the (short) author in the sidebar
%    hideinstitute,       % hide the (short) institute in the bottom of the sidebar
%    shownavsym,          % show the navigation symbols
%    width=2cm,           % width of the sidebar (default is 2 cm)
%    hideothersubsections,% hide all subsections but the subsections in the current section
%    hideallsubsections,  % hide all subsections
		right               % right of left position of sidebar (default is right)
%%% options passed to the color theme
%    lightheaderbg,       % use a light header background
	]{AAUsidebar}

% If you want to change the colors of the various elements in the theme, edit and uncomment the following lines
% Change the bar and sidebar colors:
%\setbeamercolor{AAUsidebar}{fg=red!20,bg=red}
%\setbeamercolor{sidebar}{bg=red!20}
% Change the color of the structural elements:
%\setbeamercolor{structure}{fg=red}
% Change the frame title text color:
%\setbeamercolor{frametitle}{fg=blue}
% Change the normal text color background:
%\setbeamercolor{normal text}{bg=gray!10}
% Highlight the text in the sidebar
\usecolortheme{rose,sidebartab}
% ... and you can of course change a lot more - see the beamer user manual.

% colored hyperlinks
\newcommand{\chref}[2]{%
	\href{#1}{{\usebeamercolor[bg]{AAUsidebar}#2}}%
}



% specify a logo on the titlepage (you can specify additional logos an include them in
% institute command below
\pgfdeclareimage[height=1cm]{titlepagelogo}{theme/figures/ufrn2} % placed on the title page
\pgfdeclareimage[height=1cm]{titlepagelogo2}{theme/figures/imd} % placed on the title page
\titlegraphic{% is placed on the bottom of the title page
	\pgfuseimage{titlepagelogo}
	\hspace{1cm}\pgfuseimage{titlepagelogo2}
}


% Article version layout settings

\mode<article>

\makeatletter
\def\@listI{\leftmargin\leftmargini
	\parsep 0pt
	\topsep 5\p@   \@plus3\p@ \@minus5\p@
	\itemsep0pt}
\let\@listi=\@listI


\setbeamertemplate{frametitle}{\paragraph*{\insertframetitle\
		\ \small\insertframesubtitle}\ \par
}
\setbeamertemplate{frame end}{%
	\marginpar{\scriptsize\hbox to 1cm{\sffamily%
			\hfill\strut\insertshortlecture.\insertframenumber}\hrule height .2pt}}
\setlength{\marginparwidth}{1cm}
\setlength{\marginparsep}{4.5cm}

\def\@maketitle{\makechapter}

\def\makechapter{
	\newpage
	\null
	\vskip 2em%
	{%
		\parindent=0pt
		\raggedright
		\sffamily
		\vskip8pt
		\includegraphics[width=\linewidth]{theme/figures/imd.png}\par\vskip2em
		{\fontsize{36pt}{36pt}\selectfont Aula \insertshortlecture \par\vskip2pt}
		{\fontsize{24pt}{28pt}\selectfont \color{blue!50!black} \@title\par\vskip4pt}
		%{\Large\selectfont \color{blue!50!black} \insertsubtitle\par}
		\vskip10pt

		\normalsize\selectfont [Notas de Aula]
		Disciplina: \emph{\lecturename \ (\semestre)} \par\vskip1.5em
		\nomedoautor\hskip1em Email: \ \emaildoautor
	}
	\par
	\vskip 1.5em%
}

\let\origstartsection=\@startsection
\def\@startsection#1#2#3#4#5#6{%
	\origstartsection{#1}{#2}{#3}{#4}{#5}{#6\normalfont\sffamily\color{blue!50!black}\selectfont}}

\makeatother

\mode
<all>




% Typesetting Listings

\usepackage{listings}
\lstset{language=Java}

\alt<presentation>
{\lstset{%
	basicstyle=\footnotesize\ttfamily,
	commentstyle=\slshape\color{green!50!black},
	keywordstyle=\bfseries\color{blue!50!black},
	identifierstyle=\color{blue},
	stringstyle=\color{orange},
	escapechar=\#,
	emphstyle=\color{red}}
}
{
	\lstset{%
		basicstyle=\ttfamily,
		keywordstyle=\bfseries,
		commentstyle=\itshape,
		escapechar=\#,
		emphstyle=\bfseries\color{red}
	}
}



% Common theorem-like environments
%\usepackage{amsthm}

\setbeamertemplate{theorems}[numbered]

%
%	New useful definitions:
%

\newbox\mytempbox
\newdimen\mytempdimen

\newcommand\includegraphicscopyright[3][]{%
	\leavevmode\vbox{\vskip3pt\raggedright\setbox\mytempbox=\hbox{\includegraphics[#1]{#2}}%
		\mytempdimen=\wd\mytempbox\box\mytempbox\par\vskip1pt%
		\fontsize{3}{3.5}\selectfont{\color{black!25}{\vbox{\hsize=\mytempdimen#3}}}\vskip3pt%
}}

\newenvironment{colortabular}[1]{\medskip\rowcolors[]{1}{blue!20}{blue!10}\tabular{#1}\rowcolor{blue!40}}{\endtabular\medskip}

\def\equad{\leavevmode\hbox{}\quad}

\newenvironment{greencolortabular}[1]
{\medskip\rowcolors[]{1}{green!50!black!20}{green!50!black!10}%
	\tabular{#1}\rowcolor{green!50!black!40}}%
{\endtabular\medskip}

%\setbeamertemplate{theorem begin}{{ \inserttheoremheadfont
%\inserttheoremname \inserttheoremnumber
%\ifx\inserttheoremaddition\empty\else\ (\inserttheoremaddition)\fi%
%\inserttheorempunctuation }} \setbeamertemplate{theorem end}{}

\newcommand{\vu}{\vec{u}}
\newcommand{\vv}{\vec{v}}
\newcommand{\vi}{\vec{i}}
\newcommand{\vj}{\vec{j}}
\newcommand{\vk}{\vec{k}}
\newcommand{\vw}{\vec{w}}
\newcommand\segmento[2]{\overline{#1#2}}
\def\colc#1{\left[#1\right]}



%
%   Macros
%

\usepackage{macros/macros}


%
%   Ambientes
%

\theoremstyle{plain}
\newtheorem{teorema}{Teorema}

\theoremstyle{definition}
\newtheorem{definicao}[teorema]{Definição}
%\newtheorem{exercicio}{Exercício}

\theoremstyle{remark}
\newtheorem{obs}[teorema]{Observação}
\newtheorem{observacao}[teorema]{Observação}
\newtheorem{corolario}[teorema]{Corolário}
\newtheorem{exemplo}[teorema]{Exemplo}
\newtheorem{lema}[teorema]{Lema}
\newtheorem{proposicao}[teorema]{Proposição}

\newcounter{exercicios}
\newenvironment{exercicio}{\stepcounter{exercicios} \textbf{\arabic{exercicios}}.}{}

% compatibilidade
\newcommand{\Ex}[1]{\begin{exercicio}#1\end{exercicio}}

%
%   Definições e comandos auxiliares do preâmbulo
%

\newcommand{\capitulo}[1]{\lecture[#1]{Capítulo}}
\newcommand{\aula}[1]{\subtitle{#1}}
\newcommand{\autor}{Igor Oliveira}
\newcommand{\email}{\href{mailto:matematicaelementar@imd.ufrn.br}{\texttt{matematicaelementar@imd.ufrn.br}}}
\newcommand{\disciplina}{Matemática Elementar}
\newcommand{\codigo}{IMD1001}

\title{\disciplina}
\date{\today}
\author[\autor]
{
    \autor\\
    \email
}

\def\lecturename{\codigo

\disciplina}

\institute[
	UFRN\\
	Natal-RN
]
{
	Instituto Metrópole Digital\\
	Universidade Federal do Rio Grande do Norte\\
	Natal-RN

}

% compatibilidade
\newcommand{\vu}{\vec{u}}
\newcommand{\vv}{\vec{v}}
\newcommand{\vi}{\vec{i}}
\newcommand{\vj}{\vec{j}}
\newcommand{\vk}{\vec{k}}
\newcommand{\vw}{\vec{w}}
\newcommand{\segmento}[2]{\overline{#1#2}}
\def\colc#1{\left[#1\right]}
\newcommand{\negacao}{\sim}

\justifying