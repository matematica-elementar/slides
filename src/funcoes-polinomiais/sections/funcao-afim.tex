\section{Função Afim}
\begin{frame}
\frametitle{Função Afim} 

\begin{definicao}
Uma função real chama-se \sub{afim} quando existem constantes $a, b
\in \R$ tais que $f(x) = ax +b$ para todo $x\in \R$.
\end{definicao}

Em uma função afim na qual $f(x) = ax +b$, chamamos o valor $a$ de
\sub{taxa de variação} da função ou de \sub{taxa de crescimento}.\\
Vale observar que é muito comum chamá-lo de coeficiente angular.
Esse termo não é apropriado pois define-se coeficiente angular para
retas, e não para funções, mesmo que vejamos que o gráfico de uma
função afim seja uma reta. 



\end{frame}


%------------------------------------------------------------------------------------------------------------
\begin{frame}
\frametitle{Função Afim} 

\begin{exemplo}
    A função identidade $f: \R \to \R$, definida por $f(x) = x$ para
    todo $x \in \R$, é afim, assim como suas translações $g(x) = x+b$.
    São, ainda, casos particulares de funções afins as funções lineares
    $f(x) = ax$ e as funções constantes $f(x) = b$.
    \end{exemplo}\pause

\begin{exemplo}
O preço a se pagar por uma corrida de táxi é dado por uma função
afim $f: x \mapsto ax+b$, em que $x$ é a distância percorrida
(usualmente medida em quilômetros), o valor inicial $b$ é a chamada
\emph{bandeirada} e o coeficiente $a$ é o preço de cada quilômetro
rodado.
\end{exemplo}


\end{frame}


%------------------------------------------------------------------------------------------------------------
\begin{frame}
\frametitle{Função Afim} 

\begin{exemplo}
    A escala $N$ de temperaturas foi feita com base nas temperaturas
máxima e mínima em Nova Iguaçu. A correspondência com a escala
Celsius é a seguinte:
\begin{center}
\begin{tabular}{|r|r|}
	\hline
	% after \\: \hline or \cline{col1-col2} \cline{col3-col4} ...
	°N & °C \\ \hline
	0 & 18 \\ \hline
	100 & 43 \\
	\hline
\end{tabular}
\end{center}
Modele o problema com funções afim que transformem a temperatura em
°N em °C e vice-versa. Qual a relação entre essas duas funções? Em
que temperatura ferve a água na escala $N$?
\end{exemplo}

\end{frame}


%------------------------------------------------------------------------------------------------------------
\begin{frame}
\frametitle{Gráfico da Função Afim} 
\begin{proposicao}
O gráfico de uma função afim $f(x) = ax + b$ é uma reta.
\end{proposicao}\pause
Note que, para desenhar o gráfico de uma função afim, basta conhecer
dois pontos, pois uma reta é inteiramente determinada por dois
pontos.

\end{frame}


%------------------------------------------------------------------------------------------------------------
%\begin{frame}
%    \frametitle{Função Afim e PA} 
    
%    \begin{proposicao}
%    Seja  $f: \R \to \R$. Se $f$ é uma função afim e $\paren{x_1, x_2,
%    \dots , x_i, \dots}$ é uma PA, então a sequência formada pelos
%    pontos $y_i = f(x_i)$, $i \in \N^{\ast}$ é uma PA. Reciprocamente,
%    se $f$ for monótona e transformar qualquer PA $\paren{x_1, x_2,
%    \dots , x_i, \dots}$ numa PA com termo geral $y_i = f(x_i)$, $i \in
%    \N^{\ast}$, então $f$ é uma função afim.
%    \end{proposicao}
    
%    \end{frame}
    
%------------------------------------------------------------------------------------------------------------