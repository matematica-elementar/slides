\section{Função Linear}
\begin{frame} \frametitle{Função Linear}
\begin{definicao}
Chamamos de \sub{função linear} uma função real com lei de formação
$f(x) = ax$, com $a \in \R$.
\end{definicao}\pause

A função linear é usada para modelar problemas de proporcionalidade
direta. Quando duas grandezas são diretamente proporcionais, podemos
escrevê-las sob a lei de formação de uma função linear.

Note que, sabendo que uma função é linear, o valor de $a$ é igual a
$f(1)$.\pause

No caso das grandezas inversamente proporcionais, a função
matemática que modela tal problema é uma função $f: \R^{\ast} \to
\R^{\ast}$ tal que $f(x) = \frac a x$. Nesse caso, também temos a
particularidade de $f(1) = a \in \R^{\ast}$.



\end{frame}

%------------------------------------------------------------------------------------------------------------

\begin{frame}
\frametitle{Função Linear} 

\begin{teorema}[Teorema Fundamental da Proporcionalidade]
Seja $f: \R \to \R$ uma função crescente. As seguintes afirmações
são equivalentes:
\begin{enumerate}[(i)]
	\item  $f$ é linear;
	\item $f(x+y) = f(x) + f(y)$ para quaisquer $x, y \in \R$;
	\item $f(nx) = nf(x)$ para todo $n \in \Z$ e todo $x \in \R$.
\end{enumerate}
\end{teorema}
Nas hipóteses do Teorema, tem-se $a=f(1) > 0$. No caso de se supor
$f$ decrescente, vale um resultado análogo, com $a<0$.

\end{frame}

%------------------------------------------------------------------------------------------------------------

\begin{frame}
\frametitle{Função Linear} 

A importância desse Teorema está no seguinte fato: se queremos saber
se $f: \R \to \R$ é uma função linear, basta verificar duas coisas:
\begin{enumerate}[1ª:]
	\item $f$ deve ser crescente ou decrescente. (Estamos deixando de
	lado o caso trivial de $f$ ser identicamente nula);
	\item $f(nx) = n f(x)$ para todo $x \in \R$ e todo $n \in \Z$. No
	caso de $f: \R_+ \to \R_+$, basta verificar essa última condição
	para $n \in \N$.
\end{enumerate}

\end{frame}



%------------------------------------------------------------------------------------------------------------
\begin{frame}
\frametitle{Função Linear} 

\begin{exemplo}
O lado de um quadrado é proporcional à sua área? Em outras palavras,
essas duas grandezas podem ser relacionadas por meio de uma função
linear?
\end{exemplo}

\end{frame}



%------------------------------------------------------------------------------------------------------------