\section{Funções Polinomiais}
\begin{frame}
\frametitle{Polinômio} 

\begin{definicao}
Um \sub{polinômio} é uma expressão formal do tipo
$$p(X) = a_n X^n+ a_{n-1} X^{n-1} + \dots + a_1X + a_0,$$
onde $\paren{a_0, a_1 , \dots , a_n}$ é uma lista ordenada de
números reais e $X$ é um símbolo, chamado de \sub{indeterminada},
sendo $X^i$ uma abreviatura para $X\cdot X  \dots  X$ ($i$ fatores).
Ao maior número $n$ tal que $a_n \neq 0$ damos o nome de \sub{grau
do polinômio $p(X)$}.
\end{definicao}\pause

Dizemos que dois polinômios $p(X) = a_n X^n+ a_{n-1} X^{n-1} + \dots
+ a_1X + a_0$ e $q(X) = b_m X^m + b_{m-1} X^{m-1} + \dots + b_1X +
b_0$ são iguais quando $n=m$ e $a_i = b_i$ para todo $i \in \set{0,
1, \dots , n}$.

\end{frame}



%------------------------------------------------------------------------------------------------------------