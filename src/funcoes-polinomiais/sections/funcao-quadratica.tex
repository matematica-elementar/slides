\section{Função Quadrática}
\begin{frame}
\frametitle{Função Quadrática} 

\begin{definicao}
Uma função $f: \R \to \R$ chama-se \sub{quadrática} quando existem
números reais $a, b, c$, com $a \neq 0$, tais que $f(x) = ax^2 +bx
+c$ para todo $x \in \R$.
\end{definicao}


\end{frame}

%------------------------------------------------------------------------------------------------------------

\begin{frame}
\frametitle{Função Quadrática} 

\begin{proposicao}
Seja $f$ uma função quadrática da forma $f(x) = ax^2 + bx +c$, onde
$a >0$.\\
Então $f$ não é limitada superiormente e o ponto
$\paren{-\dfrac {b} {2a}, -\dfrac {\Delta} {4a}}$ é o mínimo absoluto
da função.\\
Caso tenhamos $a<0$, então $f$ não é limitada inferiormente e  o
ponto $\paren{-\dfrac {b} {2a}, -\dfrac {\Delta} {4a}}$ é o máximo
absoluto da função.
\end{proposicao}\pause

\begin{proposicao}
Seja $f$ uma função quadrática da forma $f(x) = ax^2 + bx +c$. Se
$f(x_1) = f(x_2)$ para $x_1 \neq x_2$, então $x_1$ e $x_2$ são
equidistantes de $-\dfrac{b} {2a}$, ou seja, $\dfrac{x_1 +x_2} 2 =
-\dfrac{b}{2a}$.
\end{proposicao}

\end{frame}

%------------------------------------------------------------------------------------------------------------

\begin{frame}
\frametitle{O Gráfico da Função Quadrática} 

\begin{exemplo}
O gráfico da função quadrática $f(x) = ax^2$ é uma parábola cujo
foco é $F = \paren{0, \frac 1 {4a}}$ e cuja diretriz é a reta
horizontal $y = -\frac{1}{4a}$. Ademais, o vértice da parábola é a
origem do plano cartesiano.
\end{exemplo}\pause

\begin{proposicao}
O gráfico de uma função quadrática $f(x) = ax^2 + bx + c$ é uma
parábola, tem a reta $x = -\frac {b}{2a}$ como eixo de simetria e o
ponto $\paren{-\frac {b} {2a}, -\frac {\Delta} {4a}}$ é o vértice da
parábola.
\end{proposicao}

\end{frame}

%------------------------------------------------------------------------------------------------------------