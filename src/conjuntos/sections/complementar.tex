\section{Complementar}


\begin{frame}
\frametitle{O Complementar de um Conjunto} %\framesubtitle{Exemplos}

A noção de complementar de um conjunto só faz sentido quando fixamos
um \sub{conjunto universo}, que denotaremos por $\U$. Uma vez fixado
$\U$, todos os elementos considerados pertencerão a $\U$ e todos os
conjuntos serão subconjuntos de $\U$. Por exemplo, na geometria
plana, $\U$ é o plano.

\begin{definicao}
Dado um conjunto $A$ (isto é, um subconjunto de $\U$), chama-se
\sub{complementar} de $A$ ao conjunto $A^C$ formado pelos elementos
de $\U$ que não pertencem a $A$.
\end{definicao}

\begin{exemplo}
Seja $\U$ o conjunto dos triângulos. Qual o complementar do conjunto
dos triângulos escalenos?
\end{exemplo}

\end{frame}
%------------------------------------------------------------------------------------------------------------
\begin{frame}
\frametitle{O Complementar de um Conjunto} %\framesubtitle{Exemplos}
\begin{proposicao}[Propriedades do complementar] \label{prop-comp}
Fixado um conjunto universo $\U$, sejam $A$ e $B$ conjuntos. Tem-se:
\begin{enumerate}[i.]
	\item $\U^C = \emptyset$ e $\emptyset^C = \U$;
	\item $\left( A^C \right)^C = A$ (Todo conjunto é complementar do seu complementar);
	\item Se $A \subset B$ então $B^C \subset A^C$ (se um conjunto
	está contido em outro, seu complementar contém o complementar desse outro).
\end{enumerate}
\end{proposicao}

Demonstração no quadro.


\end{frame}

%------------------------------------------------------------------------------------------------------------
\begin{frame}
\frametitle{O Complementar de um Conjunto} %\framesubtitle{Exemplos}

\begin{definicao}
A \sub{diferença} entre dois conjuntos $A$ e $B$ é definida por:
$$ B \setminus A = \set {x \tq x \in B \text { e } x \notin A}.$$
\end{definicao}

\begin{itemize}
	\item Em geral, não temos $B \setminus A = A \backslash B$. Pense em um contraexemplo a essa
	igualdade.
	\item Note que $A^C = \U \setminus A$.
\end{itemize}

\end{frame}