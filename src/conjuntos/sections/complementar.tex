\section{Complementar}


\begin{frame}
	\frametitle{O Complementar de um Conjunto}


	\begin{definicao}[Complementar]
		Dado um conjunto $A$ (isto é, um subconjunto de $\U$), chama-se \sub{complementar} de $A$ ao conjunto $A^C$ formado pelos elementos de $\U$ que não pertencem a $A$.
	\end{definicao}\pause

	\begin{exemplo}
		Seja $\U$ o conjunto dos triângulos. Qual o complementar do conjunto dos triângulos escalenos?
	\end{exemplo}
\end{frame}


\begin{frame}
	\frametitle{O Complementar de um Conjunto}

	\begin{proposicao}[Propriedades do complementar]
		\label{prop-comp}
		Fixado um conjunto universo $\U$, sejam $A$ e $B$ conjuntos. Tem-se:
		\begin{enumerate}[i.]
			\item $\complemento{\universo} = \vazio$ e $\complemento{\vazio} = \universo$;
			\item $\complemento{\complemento A} = A$ (Todo conjunto é complementar do seu complementar);
			\item Se $A \contido B$ então $B\complementar \contido A\complementar$ (se um conjunto está contido em outro, seu complementar contém o complementar desse outro);
			\item $A \uniao A\complementar = \universo$ e $A \inter A\complementar = \vazio$;
			\item \sub{Leis de DeMorgan}: $(A \uniao B)^C = A^C \inter B^C$ e $(A \inter B)^C = A^C \uniao B^C$.
		\end{enumerate}
	\end{proposicao}

\end{frame}


\begin{frame}
	\frametitle{O Complementar de um Conjunto}

	\begin{exemplo}
		Sejam $\universo = \conjunto{1,2,3,4,5}$, $A = \conjunto{1, 2, 3}$ e $B = \conjunto{2, 5}$. Calcule $(A \uniao B)\complementar$, $A\complementar \inter B\complementar$ e $A\complementar \uniao B\complementar$.
	\end{exemplo}

\end{frame}


\begin{frame}
	\frametitle{O Complementar de um Conjunto}

	\begin{definicao}[Diferença]
		A \sub{diferença} entre dois conjuntos $A$ e $B$ é definida por:
		\[
			B \menos A = \conjunto { x \taisque x \pertence B \e x \naopertence A}.
		\]
	\end{definicao}\pause

	\begin{itemize}
		\item Note que $A\complementar = \universo \menos A$ e $B \menos A = B \inter A\complementar$;
		\item Em geral, não temos $B \menos A = A \menos B$. O exemplo a seguir comprova essa afirmação.
	\end{itemize}\pause

	\begin{exemplo}
		Sejam $A = \conjunto{1, 2, 3}$ e $B = \conjunto{2, 5}$. Determine $A \menos B$ e $B \menos A$.
	\end{exemplo}\pause

	\begin{exemplo}
		Represente o conjunto $\left[ D \cap \left( C^C \cup A\right)\right] \cup  [B \setminus (A^C \cap C)]$ em uma expressão da forma mais simplificada possível.
	\end{exemplo}
\end{frame}