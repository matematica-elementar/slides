
\section{Lógica}
\begin{frame}
\frametitle{Conjuntos e Lógica} %\framesubtitle{Exemplos}

Em toda essa seção, considere $P$ e $Q$ propriedades aplicáveis aos
elementos de $\U$. Considere também $A = \set {x \tq x \text{ possui
} P}$ e $B= \set {x \tq x \text{ possui } Q}$.

\begin{itemize}
	\item \sub{Inclusão e implicação}: $A \subset B$ é equivalente a
	$P \implies Q$.
	\item \sub{Igualdade e bi-implicação}: $A=B$ é equivalente a $P
	\iff Q$.
\end{itemize}
\end{frame}
%------------------------------------------------------------------------------------------------------------
\begin{frame}
\frametitle{Conjuntos e Lógica} %\framesubtitle{Exemplos}

\begin{exemplo}
Analise as implicações abaixo:
\begin{align*}
x^2+1=0 & \implies \left(x^2 +1 \right) \left( x^2-1 \right) = 0
\cdot \left( x^2-1 \right) \\
& \implies x^4 - 1 = 0 \\
& \implies x^4 = 1 \\
& \implies x \in \set {-1, 1}
\end{align*}

Isso quer dizer que o conjunto solução de $x^2 +1 = 0$ é $\set{-1,
1}$?
\end{exemplo}

\end{frame}
%------------------------------------------------------------------------------------------------------------
\begin{frame}
\frametitle{Conjuntos e Lógica} %\framesubtitle{Exemplos}

\begin{itemize}
	\item \sub{Complementar e negação}: $A^C$ é equivalente a $\sim P$;
	\item Podemos combinar os itens (ii) e (iii) da Proposição
	\ref{prop-comp} (Propriedades do complementar) e obter que $$P
	\implies Q \text{ se, e somente se, }\sim Q \implies \sim P.$$
	Chamamos $\sim Q \implies \sim P$ de \sub{contrapositiva} de $P
	\implies Q$.
	\item Chamamos $Q \implies P$ de \sub{recíproca} de $P \implies
	Q$ e $P \land \sim Q$ de \sub{negação} de $P \implies Q$.
\end{itemize}
%Exemplos no Exercício \ref{exrec}.
\end{frame}
%------------------------------------------------------------------------------------------------------------
\begin{frame}
\frametitle{Conjuntos e Lógica} %\framesubtitle{Exemplos}

\begin{exemplo}
Observe as afirmações abaixo:
\begin{itemize}
	\item Todo número primo maior do que 2 é ímpar;
	\item Todo número par maior do que 2 é composto.
\end{itemize}

Essas afirmações dizem exatamente a mesma coisa, ou seja, exprimem a
mesma ideia, só que com diferentes termos. Podemos reescrevê-las na
forma de implicações vendo claramente que uma é a contrapositiva da
outra, todas sob a hipótese que  $n \in \N$, $n>2$:
\begin{align*}
n \text{ primo } & \implies n \text{ ímpar } \\
\sim \left( n \text{ ímpar } \right) & \implies \sim \left( n \text{
primo } \right) \\
n \text{ par } & \implies n \text{ composto }
\end{align*}
\end{exemplo}

\end{frame}
%------------------------------------------------------------------------------------------------------------

\begin{frame}
\frametitle{Conjuntos e Lógica} %\framesubtitle{Exemplos}

\begin{itemize}
	\item \sub{União e disjunção}: $A \cup B$ é equivalente a $P \lor
	Q$ ($P$ ou $Q$).
	\item \sub{Interseção e conjunção}: $A \cap B$ é equivalente a $P
	\land Q$ ($P$ e $Q$).
\end{itemize}

\begin{observacao}
O conectivo lógico \sub{ou} tem significado diferente do usado
normalmente no português. Na linguagem coloquial, usamos $P$
\sub{ou} $Q$ sem permitir que sejam as duas coisas ao mesmo tempo.
Analisem a seguinte história:

Um obstetra que também era matemático acabara de realizar um parto
quando o pai perguntou: ``é menino ou menina, doutor?''. E ele
respondeu: ``sim''.
\end{observacao}
\end{frame}
%------------------------------------------------------------------------------------------------------------

\begin{frame}
\frametitle{Conjuntos e Lógica} \framesubtitle{Resumo}
\begin{center}
\begin{tabular}{|c|c|}
	\hline
	% after \\: \hline or \cline{col1-col2} \cline{col3-col4} ...
	$A=B$ & $P \iff Q$ \\ \hline
	$A \subset B$ & $P \implies Q$ \\ \hline
	$A^C$ & $\sim P$ \\ \hline
	$A \cup B$ & $P \lor Q$ \\ \hline
	$A \cap B$ & $P \land Q$ \\
	\hline
\end{tabular}
\end{center}
\end{frame}
%------------------------------------------------------------------------------------------------------------

\begin{frame}
\frametitle{Conjuntos e Lógica} %\framesubtitle{Resumo}
Problema: A
polícia prende quatro homens, um dos quais cometeu um furto. Eles fazem
as seguintes declarações:
\begin{itemize}
	\item Arnaldo: Bernaldo fez o furto.
	\item Bernaldo: Cernaldo fez o furto.
	\item Dernaldo: eu não fiz o furto.
	\item Cernaldo: Bernaldo mente ao dizer que eu fiz o furto.
\end{itemize}
Se sabemos que só uma destas declarações é a verdadeira, quem é
culpado pelo furto?

\end{frame}
