\section{Lógica}


\begin{frame}
	\frametitle{Conjuntos e Lógica}
	
	Em toda essa seção, considere $P$ e $Q$ propriedades aplicáveis aos elementos de $\U$. Considere também $A = \conjunto { x \tq x \text{ possui } P}$ e $B = \conjunto { x \tq x \text{ possui } Q}$.

	\begin{itemize}
		\item \sub{Inclusão e implicação}: $A \contido B$ é equivalente a
		$P \implica Q$.
		\item \sub{Igualdade e bi-implicação}: $A=B$ é equivalente a $P
		\sse Q$.
	\end{itemize}
\end{frame}


\begin{frame}
	\frametitle{Conjuntos e Lógica} 

	\begin{exemplo}
		Analise as implicações abaixo:
		\begin{align*}
			x^2+1=0 & \implica (x^2+1)(x^2-1) = 0\vezes(x^2-1) \\
			        & \implica x^4-1 = 0 \\
			        & \implica x^4 = 1 \\
			        & \implica x \pertence \conjunto{-1, 1}
		\end{align*}

		Isso quer dizer que o conjunto solução de $x^2 +1 = 0$ é $\conjunto{-1, 1}$?
	\end{exemplo}
\end{frame}


\begin{frame}
	\frametitle{Conjuntos e Lógica} 

	\begin{itemize}
		\item \sub{Complementar e negação}: $A^C$ é equivalente a $\negacao P$;
		
		\item Podemos combinar os itens (ii) e (iii) da Proposição \ref{prop-comp} (Propriedades do complementar) e obter que
			$$P \implica Q \text{ se, e somente se, }\negacao Q \implica \negacao P.$$
		Chamamos $\negacao Q \implica \negacao P$ de \sub{contrapositiva} de $P \implica Q$.
		
		\item Chamamos $Q \implica P$ de \sub{recíproca} de $P \implica Q$ e $P \land \negacao Q$ de \sub{negação} de $P \implica Q$.
	\end{itemize}
\end{frame}


\begin{frame}
	\frametitle{Conjuntos e Lógica} 

	\begin{exemplo}
		Observe as afirmações abaixo:
		\begin{itemize}
			\item Todo número primo maior do que 2 é ímpar;
			\item Todo número par maior do que 2 é composto.
		\end{itemize}
		Essas afirmações dizem exatamente a mesma coisa, ou seja, exprimem a mesma ideia, só que com diferentes termos. Podemos reescrevê-las na forma de implicações vendo claramente que uma é a contrapositiva da outra, todas sob a hipótese que  $n \pertence \N$, $n>2$:
		\begin{align*}
		          n \text{ primo } & \implica n \text{ ímpar } \\
		\negacao(n \text{ ímpar }) & \implica \negacao ( n \text{ primo }) \\
		            n \text{ par } & \implica n \text{ composto }
		\end{align*}
	\end{exemplo}
\end{frame}


\begin{frame}
	\frametitle{Conjuntos e Lógica} 

	\begin{itemize}
		\item \sub{União e disjunção}: $A \uniao B$ é equivalente a $P \lor Q$ ($P$ ou $Q$).
		\item \sub{Interseção e conjunção}: $A \inter B$ é equivalente a $P \land Q$ ($P$ e $Q$).
	\end{itemize}

	\begin{observacao}
		O conectivo lógico \sub{ou} tem significado diferente do usado normalmente no português. Na linguagem coloquial, usamos $P$ \sub{ou} $Q$ sem permitir que sejam as duas coisas ao mesmo tempo. Analisem a seguinte história:

		Um obstetra que também era matemático acabara de realizar um parto quando o pai perguntou: ``é menino ou menina, doutor?''. E ele respondeu: ``sim''.
	\end{observacao}
\end{frame}


\begin{frame}
	\frametitle{Conjuntos e Lógica} \framesubtitle{Resumo}
	\begin{center}
		\begin{tabular}{|c|c|}
			\hline
			$A=B$          & $P \sse Q$     \\ \hline
			$A \contido B$ & $P \implica Q$ \\ \hline
			$A^C$          & $\negacao P$   \\ \hline
			$A \uniao B$   & $P \lor Q$     \\ \hline
			$A \inter B$   & $P \land Q$    \\
			\hline
		\end{tabular}
	\end{center}
\end{frame}


\begin{frame}
	\frametitle{Conjuntos e Lógica}
	
	Problema: A polícia prende quatro homens, um dos quais cometeu um furto. Eles fazem as seguintes declarações:
	\begin{itemize}
		\item Arnaldo: Bernaldo fez o furto.
		\item Bernaldo: Cernaldo fez o furto.
		\item Dernaldo: eu não fiz o furto.
		\item Cernaldo: Bernaldo mente ao dizer que eu fiz o furto.
	\end{itemize}
	Se sabemos que só uma destas declarações é a verdadeira, quem é culpado pelo furto?
\end{frame}
