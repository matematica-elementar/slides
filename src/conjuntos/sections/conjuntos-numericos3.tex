\section{Conjuntos Numéricos}
\begin{frame}
    \frametitle{Reais} 
    \begin{definicao}
    À reunião de $\Q$ com o conjunto dos números irracionais, nomeamos
    de \sub{conjunto dos números reais}. Denotamos por $\R$.
    \end{definicao}\pause
    
    \begin{itemize}
    \item $\R \setminus \Q = \set {x \tq x \text{ é irracional}}$;
    \item Usamos os números reais para medir algo. A cada número real
    está associado um ponto na reta graduada e vice-versa;
    \item Entre dois números reais distintos sempre há pelo menos um número racional e um
    irracional.
    \href{https://pt.khanacademy.org/math/algebra/rational-and-irrational-numbers/proofs-concerning-irrational-numbers/v/proof-that-there-is-an-irrational-number-between-any-two-rational-numbers}
    {{\tt Este vídeo}} do Khan Academy mostra que entre dois racionais
    distintos sempre há pelo menos um número irracional;
    \item A igualdade $0,999\dots = 1 $ é verdadeira?
    \end{itemize}
    \end{frame}
    
    
    
    %------------------------------------------------------------------------------------------------------------
    \begin{frame}
        \frametitle{Intervalos Reais} 
        \begin{definicao}
            Sejam $a, b \in \R$ tais que $a < b$. Definimos o \sub{intervalo aberto de $a$ a $b$}, denotado por $(a, b)$, como sendo o seguinte subconjunto de $\R$:
            $$ (a,b) = \{ x \in \R ; \  a < x < b \}.$$
            Definimos o \sub{intervalo fechado de $a$ a $b$}, denotado por $[a, b]$, como sendo o seguinte subconjunto de $\R$:
            $$ [a,b] = \{ x \in \R ; \  a \leq x \leq b \}.$$


        \end{definicao}
        \end{frame}

        %------------------------------------------------------------------------------------------------------------
    \begin{frame}
        \frametitle{Intervalos Reais} 
        Além dos intervalos da definição anterior, nas mesmas condições, temos os seguintes:
        
        \begin{itemize}
        \item $ (a,b] = \{ x \in \R ; \  a < x \leq b \}$;
        \item $ [a,b) = \{ x \in \R ; \  a \leq x < b \}$; \pause
        \item $ (a,+\infty) = \{ x \in \R ; \   x > a \}$;
        \item $ [a,+\infty) = \{ x \in \R ; \   x \geq a \}$;
        \item $ (-\infty,a) = \{ x \in \R ; \   x < a \}$;
        \item $ (-\infty,a] = \{ x \in \R ; \   x \leq a \}$;
        \item $ (-\infty, +\infty) = \R$.
        \end{itemize}
        \end{frame}
        
        
        
        %------------------------------------------------------------------------------------------------------------
    \begin{frame}
    \frametitle{Complexos} 
    \begin{definicao}
    Chamamos $i = \sqrt {-1}$ de \sub{número imaginário}, e ao conjunto
    $\C = \set{ a+bi \tq a,b \in \R}$ damos o nome de \sub{conjunto dos
    números complexos}.
    \end{definicao}\pause
    
    Seja $a+bi \in \C$. Nomeamos o número $a-bi$ de \sub{conjugado} de
    $a+bi$.
    
    Temos a seguinte cadeia de inclusões próprias: $\N \contidoproprio \Z
    \contidoproprio \Q \contidoproprio \R \contidoproprio \C$.
    \end{frame}
    
    
    
    %------------------------------------------------------------------------------------------------------------