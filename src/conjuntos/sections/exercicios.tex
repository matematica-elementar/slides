\section{Exercícios}


\begin{frame}
	\frametitle{Exercícios}

	\begin{exercicio}
		De que outras formas podemos representar o conjunto vazio utilizando as duas notações de definição de conjuntos que conhecemos?
	\end{exercicio}


	\begin{exercicio}
		Decida quais das afirmações a seguir estão corretas. Justifique suas respostas.
		\begin{enumerate}[a)]
			\item $\vazio \pertence \vazio$;
			\item $\vazio \contido  \vazio$;
			\item $\vazio \pertence \unitario{\vazio}$;
			\item $\vazio \contido  \unitario{\vazio}$.
		\end{enumerate}
	\end{exercicio}

	\begin{exercicio}
		Dê exemplos de conjuntos $A$, $B$ e $C$, justificando, que satisfaçam:
		\begin{enumerate}[a)]
			\item $A \contem B$;
			\item $A \naocontem B$;
			\item $A \contidoproprio B$;
			\item $A \contido B$, $B \naocontido C$ e $A \contido C$;
			\item $A \naocontido B$, $B \contido C$ e $A \naocontido C$.
		\end{enumerate}
	\end{exercicio}
\end{frame}


\begin{frame}
	\frametitle{Exercícios}
	
	\begin{exercicio}
		Considere $A = \{x \in \Z_+ \talque x<3 \}$. Calcule $\partes{A}$.
	\end{exercicio}

	\begin{exercicio}
		Dê exemplos de conjuntos $A$, $B$ e $C$, justificando com os cálculos, que satisfaçam:
		\begin{enumerate}[a)]
			\item $A \uniao ( B \inter C) \diferente (A \uniao B) \inter C$. Qual o conjunto que será sempre igual a $A \uniao ( B \inter C)$?
			\item $A \contido B$ mas $A\complementar \naocontido B\complementar$. Qual inclusão é sempre válida envolvendo $A\complementar$ e $B\complementar$?
			\item $A \contidoproprio B$;
			\item $(A \inter B)\complementar \diferente A\complementar \inter B\complementar$. Qual o conjunto que será sempre igual a $(A \inter B)\complementar$?
		\end{enumerate}
	\end{exercicio}

	\begin{exercicio}
		As igualdades abaixo acerca dos conjuntos $A$, $B$ e $C$ não são válidas geralmente. Em cada um dos itens, dê exemplos que ilustram esses fatos.
		\begin{enumerate}[a)]
			\item $A \diferenca ( B \inter C) = \paren{A \diferenca B} \inter \paren{A \diferenca C}$;
			\item $A \diferenca ( B \uniao C) = \paren{A \diferenca B} \uniao \paren{A \diferenca C}$.
		\end{enumerate}
	\end{exercicio}
\end{frame}


\begin{frame}
	\frametitle{Exercícios}
	
	\begin{exercicio}
		Sejam $A$, $B$ conjuntos quaisquer. Classifique como verdadeiro ou falso cada sentença abaixo. Justifique ou dê um contra-exemplo no caso da sentença ser falsa.
            \begin{enumerate}[a)]
                \item  $\left(A \setminus B\right) \contido B$;
                \item  $\left(A \setminus B\right) \contido \left(A \cup B\right)$.
            \end{enumerate}
	\end{exercicio}

	\begin{exercicio}
		Sejam $A$, $B$ e $C$ conjuntos tais que $A \uniao B \uniao C = \universo$. Em cada um dos itens, use propriedades para obter um conjunto igual a esses escrito somente com uniões de conjuntos.
		\begin{enumerate}[a)]
			\item $A \uniao \paren{ B \inter C\complementar} \uniao \paren{A\complementar \inter B\complementar \inter C\complementar} \uniao C$;
			\item $\colche{\paren{A\complementar \inter B \inter C} \uniao \paren{A \inter B \inter C}}\complementar$;
			\item $\colche {\paren{A \uniao B\complementar \uniao C} \uniao \paren{B \uniao C\complementar}}\complementar $;
			\item $\colche {\paren{A \uniao B\complementar \uniao C} \inter \paren{B \inter C\complementar}}\complementar $;
			\item $A \cup \left( B \setminus C \right) \cup \left[ A^C \setminus \left( B \cup C \right) \right] \cup C $.
		\end{enumerate}
	\end{exercicio}

	\begin{exercicio}
		Sejam $A$, $B$, $C$, $D$ conjuntos. Use propriedades do nosso material para obter um conjunto igual ao abaixo escrito somente com interseções de, no máximo, 3 conjuntos.
        $$[A^C \cap B \cap (C \cup D^C)^C] \cup [A \cap (B^C \cup C)^C \cap D].$$
	\end{exercicio}
\end{frame}


\begin{frame}
	\frametitle{Exercícios}
	
	\begin{exercicio}
		O \href{https://pt.wikipedia.org/wiki/Diagrama_de_Venn}{{\tt Diagrama de Venn}} para os conjuntos $X$, $Y$, $Z$ decompõe o plano em oito regiões. Numere essas regiões e exprima cada um dos conjuntos abaixo como reunião de algumas dessas regiões.  (Por exemplo: $X \inter Y = 1 \uniao 2$.)
		
		\begin{enumerate}[a.]
			\item $\left(X^C \uniao Y \right)^C$;
			\item $\left(X^C \uniao Y \right) \uniao Z^C$;
			\item $\left(X^C \inter Y \right) \uniao \left(X \inter Z^C \right)$;
			\item $\left(X \uniao Y \right)^C \inter Z$.
		\end{enumerate}
	\end{exercicio}

	\begin{exercicio}
		Exprimindo cada membro como reunião de regiões numeradas, verifique as igualdades:
		\begin{enumerate}[a.]
			\item $\left(X \uniao Y \right)\inter Z = \left(X \inter Z \right) \uniao \left(Y \inter Z
			\right)$;
			\item $X \uniao \left(Y \inter Z \right)^C = X \uniao Y^C \uniao Z^C$.
		\end{enumerate}
	\end{exercicio}
\end{frame}


\begin{frame}
	\frametitle{Exercícios}
	
	\begin{exercicio}
		Reduza as expressões abaixo a um intervalo ou a união de intervalos disjuntos.
		\begin{enumerate}[a.]
			\item $[12; 36) \uniao [-2;37]$;
			\item $(-\infty ; -2) \inter (-1; \dfrac {-1}2)$;
			\item $([4; 6] \uniao (5; 8]) \inter ([4; 6] \uniao [4; 9))$;
			\item $[2; + \infty)\complementar$;
			\item $\paren{[3;5] \uniao [6;9]}\complementar$;
			\item $\paren{[2;5]\complementar \inter (-\infty ; 6)}\complementar$;
			\item $\paren{(2;4] \inter (1;4)} \uniao (7;12]$;
			\item $[\dfrac \pi 6 ; 2 \pi] \uniao [\pi ; 3 \pi)$;
			\item $\left( \left[\dfrac 5 2; +\infty \right) \cap (- \infty; 3] \right) \cup \left( \left(- \infty ; \dfrac 5 2 \right) \cap [2. + \infty ) \right)$;
			\item $\left[ \big( [1; +\infty) \cup (-5; +\infty) \big) \cap \big( (- \infty ; 1) \cup ( - \infty ; -5) \big)\right]^C$;
			\item $( \pi ; + \infty) \diferenca \R$;
			\item $\R \diferenca (-\infty ; 3]$.
		\end{enumerate}
	\end{exercicio}
\end{frame}


\begin{frame}
	\frametitle{Exercícios}
	
	\begin{exercicio}
		Faça os testes do Khan Academy das unidades que envolvam frações em
\href{https://pt.khanacademy.org/math/arithmetic}
{{\tt Aritmética}}. Ao final, revise os assuntos que você teve
problema. 
	\end{exercicio}

	\begin{exercicio}
		Faça o estudo completo (vídeos e exercícios) no Khan Academy da unidade
\href{https://pt.khanacademy.org/math/algebra/x2f8bb11595b61c86:irrational-numbers}
{{\tt Números Irracionais}}.
	\end{exercicio}

	\begin{exercicio}
		Faça o estudo completo (vídeos e exercícios) no Khan Academy da unidade
\href{https://pt.khanacademy.org/math/algebra2/x2ec2f6f830c9fb89:exp}
{{\tt Radicais e Expoentes Racionais}}.
	\end{exercicio}
\end{frame}


%\begin{frame}
%	\frametitle{Exercícios}
%	
%	\Ex{
%		Considere as seguintes (aparentes) equivalências lógicas:
%		\begin{align*}
%			      & \sse x^2 - 2 \vezes 1 + 1 = 0 \\
%			      & \sse x^2 - 1 = 0 \\
%			      & \sse x = \maisoumenos 1
%		\end{align*}
%		Conclusão (?): $x = 1 \sse x = \maisoumenos 1$. Onde está o erro?
%	} 
%\end{frame}

