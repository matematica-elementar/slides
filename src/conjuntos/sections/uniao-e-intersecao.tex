\section{União e Interseção}


\begin{frame}
\frametitle{União e Interseção de Conjuntos} %\framesubtitle{Exemplos}

\begin{Def}
Dados os conjuntos $A$ e $B$:
\begin{enumerate}[i.]
	\item A \sub{união} $A \cup B$ é o conjunto formado pelos
	elementos que pertencem a pelo menos um dos conjuntos $A$ e $B$;
	\item A \sub{interseção} $A \cap B$ é o conjunto formado por elementos que pertencem a ambos $A$ e
	$B$.
\end{enumerate}
\end{Def}

\begin{Exem}
Sejam $A = \set{1, 2, 3}$ e $ B = \set{2,5}$. Determine $A \cup B$,
$A \cap B$, $A \setminus B$ e $B \setminus A$.
\end{Exem}

\end{frame}

%------------------------------------------------------------------------------------------------------------

\begin{frame}
\frametitle{União e Interseção de Conjuntos} %\framesubtitle{Exemplos}
\begin{Prop}[Propriedades da união e interseção] \label{propuniaoint}
Sejam $A$, $B$ e $C$ conjuntos. Tem-se:
\begin{enumerate}[i.]
	\item $A \subset \paren{A \cup B}$ e $\paren{A \cap B} \subset A$;
	\item \sub{União/interseção com o universo}: $\U \cup A = \U$ e $A \cap \U = A$;
	\item \sub{Comutatividade}: $A \cup B = B \cup A$ e $A \cap B = B \cap A$;
	\item \sub{Associatividade}: $\left(A \cup B \right) \cup C = A
	\cup \left( B \cup C \right)$ e $\left(A \cap B \right) \cap C = A
	\cap \left( B \cap C \right)$;

	\item \sub{Distributividade, de uma em relação à outra}: $A \cap
	\left( B \cup C \right) = \left(A \cap B \right) \cup \left( A \cap C
	\right)$ e $A \cup \left( B \cap C \right) = \left(A \cup B \right) \cap
	\left( A \cup C  \right)$;

	\item \sub{Leis de DeMorgan}: $\left( A \cup B \right)^C = A^C \cap
	B^C$ e $\left(A \cap B \right)^C = A^C \cup B^C$.

	\end{enumerate}
\end{Prop}

Demonstração no quadro.


\end{frame}