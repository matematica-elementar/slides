\section{Conjuntos Numéricos}
\begin{frame}
    \frametitle{Irracionais} 
    \begin{definicao}
    O \sub{conjunto dos números irracionais} é constituído por todos os
    números que possuem uma representação decimal infinita e não
    periódica.
    \end{definicao}\pause
    
    \begin{exemplo}
    $\sqrt 2$, $e$ e $\pi$ são números irracionais.
    \end{exemplo}
    
    Você sabia que existem infinitos ``maiores'' que outros? Qual
    conjunto você diria que tem mais elementos: racionais ou
    irracionais?
    \end{frame}
    
    %------------------------------------------------------------------------------------------------------------
    
    \begin{frame}
    \frametitle{Problema} 
    
    O Grande Hotel Georg Cantor tinha uma infinidade de quartos,
    numerados consecutivamente, um para cada número natural. Todos eram
    igualmente confortáveis. Num fim de semana prolongado, o hotel
    estava com seus quartos todos ocupados, quando chega um visitante. A
    recepcionista vai logo dizendo: \\
    -Sinto muito, mas não há vagas. \\
    Ouvindo isto, o gerente interveio: \\
    -Podemos abrigar o cavalheiro sim, senhora. \\
    E ordena:\pause \\ 
    Transfira o hóspede do quarto 1 para o quarto 2, passe o do quarto 2
    para o quarto 3 e assim por diante. Quem estiver no quarto $n$, mude
    para o quarto $n+1$. Isto manterá todos alojados e deixará
    disponível o quarto 1 para o recém chegado. \pause Logo depois chegou um
    ônibus com 30 passageiros, todos querendo hospedagem. Como deve
    proceder a recepcionista para acomodar todos? \pause
    \\ Horas depois, chegou um trem com uma infinidade de
    passageiros. Como proceder para acomodá-los?
    
    
    \end{frame}
    
    %------------------------------------------------------------------------------------------------------------