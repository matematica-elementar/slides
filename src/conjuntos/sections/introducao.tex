\section{Introdução}


\begin{frame}
    \frametitle{A Noção de Conjunto}
    
    \begin{itemize}
        \item<1->
        Um conjunto é \sub{definido} por seus elementos (e nada mais). Isso nos traz imediatamente que dois conjuntos são \sub{iguais} se, e somente se, possuem os mesmos elementos.

        \item<2->
        Dados um conjunto $A$ e um objeto qualquer $b$, há somente uma pergunta cabível para nós: $b$ é um elemento do conjunto $A$? Tal pergunta só admite \sub{sim} ou \sub{não} como resposta. Isso se dá porque, na Matemática, qualquer afirmação é \sub{verdadeira} ou é \sub{falsa}, sem possibilidade de uma terceira opção ou de ser as duas coisas ao mesmo tempo.
        \begin{itemize}
            \item<3-> O item anterior faz parecer que a Matemática é infalível se utilizada corretamente, mas ela não é. Gödel provou que todo sistema formal que inclua a aritmética é falho no sentido de que vai possuir verdades que não podem ser provadas -- os chamados paradoxos.  Antes de assistir ao vídeo \href{https://youtu.be/UI1xR_AECrU}{{\tt Este vídeo está mentindo}}, reflita se você vai acreditar nele ou não.
        \end{itemize}
    \end{itemize}
\end{frame}


\begin{frame}
    \frametitle{A Noção de Conjunto}
    %\framesubtitle{Exemplos}

    \begin{exemplo}
        Temos $V = \left\{a, e, i, o, u \right\}$ como sendo o conjunto das vogais.
    \end{exemplo}

    \begin{exemplo}
        O conjunto $PP$ dos números primos pares pode ser representado por $PP = \left\{ x \tq x \text{ é primo e par} \right\} = \left\{ 2 \right\}$. Nunca escreva $PP = \left\{ \text{números primos pares} \right\}$.
    \end{exemplo}
    
    Quando um elemento pertence a um determinado conjunto, usamos o símbolo $\in$, e, quando não pertence, usamos $\notin$.
	
    \begin{exemplo}
        Considere $PP$ e $V$ conforme definido anteriormente. Temos que $e \in V$ e $3 \notin PP$.
    \end{exemplo}
\end{frame}


\begin{frame}
    \frametitle{A Noção de Conjunto}
    %\framesubtitle{Exemplos}
    
    \begin{exemplo}
        Considere o conjunto $A = \set{\set{1, 2}, \set{2}, 1 }$. Observe que existem elementos em $A$ que são conjuntos. Além disso:
        \begin{enumerate}
            \item $\set{1, 2} \in A$;
            \item $\set{2} \in A$;
            \item $1 \in A$;
            \item $2 \notin A$.
        \end{enumerate}
    \end{exemplo}
\end{frame}


\begin{frame}
    \frametitle{A Noção de Conjunto}

    \begin{definicao}
        O conjunto que não possui elementos é chamado de \sub{conjunto vazio} e é representado por $\emptyset$.
    \end{definicao}

    \begin{exemplo}
        Quais outros conjuntos você conhece? Que tal pensar sobre o conjunto $ A = \set{ x \tq x \notin A} $?
    \end{exemplo}
\end{frame}
