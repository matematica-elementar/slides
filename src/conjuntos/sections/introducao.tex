\section{Introdução}


\begin{frame}
    \frametitle{A Noção de Conjunto}
    
    \begin{itemize}
        \item<1->
        Um conjunto é \sub{definido} por seus elementos (e nada mais). Isso nos traz imediatamente que dois conjuntos são \sub{iguais} se, e somente se, possuem os mesmos elementos.

        \item<2->
        Dados um conjunto $A$ e um objeto qualquer $b$, há somente uma pergunta cabível para nós: $b$ é um elemento do conjunto $A$? Tal pergunta só admite \sub{sim} ou \sub{não} como resposta. Isso se dá porque, na Matemática, qualquer afirmação é \sub{verdadeira} ou é \sub{falsa}, sem possibilidade de uma terceira opção ou de ser as duas coisas ao mesmo tempo.
        \begin{itemize}
            \item<3-> O item anterior faz parecer que a Matemática é infalível se utilizada corretamente, mas ela não é. Gödel provou que todo sistema formal que inclua a aritmética é falho no sentido de que vai possuir verdades que não podem ser provadas -- os chamados paradoxos.  Antes de assistir ao vídeo \href{https://youtu.be/UI1xR_AECrU}{{\tt Este vídeo está mentindo}}, reflita se você vai acreditar nele ou não.
        \end{itemize}
    \end{itemize}
\end{frame}


\begin{frame}
    \frametitle{A Noção de Conjunto}

    \begin{exemplo}
        Temos $V = \conjunto {a, e, i, o, u}$ como sendo o conjunto das vogais.
    \end{exemplo}\pause

    \begin{exemplo}
        O conjunto $P$ dos números primos pares pode ser representado por $P = \conjunto { x \taisque x \text{ é primo e par}} = \unitario{2}$. Nunca escreva $P = \conjunto{\text{números primos pares}}$.
    \end{exemplo}
\end{frame}