\section{Pertinência}


\begin{frame}
    \frametitle{A Relação de Pertinência}
    
    \begin{definicao}[Relação de Pertinência]
        \label{def:pertinencia}
    Dados um objeto $x$ e um conjunto $A$, se for o caso de $x$ ser um elemento de $A$, dizemos que $x$ \sub{pertence} a $A$. Para denotar esse fato, escrevemos $x \pertence A$.

    \label{def:naopertinencia}
    Quando $x$ não é um elemento de $A$ dizemos que $x$ \sub{não pertence} a $A$, o que denotamos por $x \naopertence A$.
    \end{definicao}\pause

	
    \begin{exemplo}
        Considere $P$ e $V$ conforme definido anteriormente. Temos que $e \pertence V$ e $3 \naopertence P$.
    \end{exemplo}
\end{frame}


\begin{frame}
    \frametitle{Relação de Pertinência}
    
    \begin{exemplo}
        Considere o conjunto $A = \conjunto{\conjunto{1, 2}, \unitario{2}, 1 }$. Observe que existem elementos em $A$ que são conjuntos. Além disso:
        \begin{enumerate}
            \item $\conjunto{1, 2} \pertence A$;
            \item $\unitario{2} \pertence A$;
            \item $1 \pertence A$;
            \item $2 \naopertence A$.
        \end{enumerate}
    \end{exemplo}
\end{frame}


\begin{frame}
    \frametitle{Conjunto Vazio}

    \begin{definicao}[O Conjunto Vazio]
        O conjunto que não possui elementos é chamado de \sub{conjunto vazio} e é representado por $\vazio$.
    \end{definicao}\pause

    \begin{exemplo}
        Quais outros conjuntos você conhece? Que tal pensar sobre o conjunto $ A = \conjunto{ x \tq x \naopertence A} $?
    \end{exemplo}
\end{frame}