\section{Inclusão}


\begin{frame}
\frametitle{A Relação de Inclusão} %\framesubtitle{Exemplos}

\begin{definicao}
Sejam $A$ e $B$ conjuntos. Se todo elemento de $A$ for também
elemento de $B$, diz-se que $A$ é um \sub{subconjunto} de $B$, que
$A$ \sub{está contido} em $B$, ou que $A$ é \sub{parte} de $B$. Para
indicar esse fato, usa-se a notação $A \subset B$.

\end{definicao}

Quando $A$ não é um subconjunto de $B$, escreve-se $A \not\subset
B$. Em outras palavras, existe pelo menos um elemento $a$ tal que $a
\in A$ e $a \notin B$.
\bigskip

Quando $A \subset B$, dizemos que $B$ \sub{contém} $A$ e escrevemos
$B \supset A$.


\end{frame}


%------------------------------------------------------------------------------------------------------------

\begin{frame}
\frametitle{A Relação de Inclusão} %\framesubtitle{Exemplos}

\begin{exemplo}
Sejam $T$ o conjunto de todos os triângulos e $P$ o conjunto dos
polígonos do plano. Todo triângulo é um polígono, logo $ T \subset
P$.
\end{exemplo}

\begin{exemplo}
Na Geometria, uma reta, um plano e o espaço são conjuntos. Seus
elementos são pontos.

Quando dizemos que uma reta $r$ está no plano $\Pi$, estamos
afirmando que $r$ está contida em $\Pi$ ou, equivalentemente, que
$r$ é um subconjunto de $\Pi$, pois todos os pontos que pertencem a
$r$ pertencem também a $\Pi$.

Nesse caso, deve-se escrever $ r \subset \Pi$. Porém, não é correto
dizer que $r$ pertence a $\Pi$, nem escrever $r \in \Pi$. Os
elementos do conjunto $\Pi$ são pontos e não retas.
\end{exemplo}

\end{frame}


%------------------------------------------------------------------------------------------------------------

\begin{frame}
\frametitle{A Relação de Inclusão} %\framesubtitle{Exemplos}

\begin{proposicao}[Inclusão universal do $\emptyset$]
Para todo conjunto $A$, vale $\emptyset \subset A$.
\end{proposicao}

\begin{definicao}
Dizemos que $A \neq \emptyset$ é um \sub{subconjunto próprio} de $B$
quando $A \subset B$  e $A \neq B$.
\end{definicao}



\end{frame}


%------------------------------------------------------------------------------------------------------------

\begin{frame}
\frametitle{A Relação de Inclusão} %\framesubtitle{Exemplos}
\begin{proposicao}[Propriedades da inclusão]
Sejam $A$, $B$ e $C$ conjuntos. Tem-se:
\begin{enumerate}[i.]
	\item \sub{Reflexividade}: $A \subset A$;
	\item \sub{Antissimetria}: Se $A \subset B$ e $B \subset A$,
	então $A = B$;
	\item \sub{Transitividade}: Se $A \subset B$ e $B \subset C$,
	então $A \subset C$.
\end{enumerate}
\end{proposicao}

Demonstração no quadro.


\end{frame}

%------------------------------------------------------------------------------------------------------------

\begin{frame}
\frametitle{A Relação de Inclusão} %\framesubtitle{Exemplos}

\begin{definicao}
Dado um conjunto $A$, chamamos de \sub{conjunto das partes} de $A$ o conjunto formado por todos
os seus subconjuntos, e denotamo-lo $\mathcal{P}(A)$.
\end{definicao}

\begin{exemplo}
Dado $A = \set {1, 2, 3}$, determine $\mathcal{P}(A)$.
\end{exemplo}



\end{frame}