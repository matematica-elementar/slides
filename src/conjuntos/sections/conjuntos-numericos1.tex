
\section{Conjuntos Numéricos}
\frame { \frametitle{Naturais}
\begin{definicao}
Ao conjunto $\N = \set {0, 1, 2, \dots , n, n+1, \dots}$ damos o
nome de \sub{conjunto dos números naturais}.
\end{definicao}\pause
\begin{itemize}
\item Denotamos $\N \setminus \set 0 = \set {1, 2, \dots , n, n+1,
\dots}$ por $\N ^\ast$.

\item Usamos o conjunto dos números naturais para contar coisas, como
casas, animais, etc.
\end{itemize}
}

%------------------------------------------------------------------------------------------------------------


\begin{frame}
\frametitle{Inteiros} 
\begin{definicao}
Ao conjunto $\Z =\set {\dots , -m -1, -m, \dots, -1, 0, 1,  \dots ,
n, n+1, \dots}$ damos o nome de \sub{conjunto dos números inteiros}.
\end{definicao}\pause

\begin{block}{Notação}
$\Z^\ast = \Z \setminus \set 0$; \\
$\Z_+ = \N$ (Inteiros não negativos); \\
$\Z^\ast_+ =\N ^\ast$ (Inteiros positivos); \\
$\Z_- =\set {\dots , -m -1, -m, \dots, -1, 0}$ (Inteiros não
positivos); \\
$\Z_-^\ast =\Z_- \setminus \set 0$ (Inteiros negativos).
\end{block}
\end{frame}



%------------------------------------------------------------------------------------------------------------
\begin{frame}
\frametitle{Racionais} 
\begin{definicao}
Ao conjunto $\Q = \set{\frac p q \tq p, q \in \Z \text{ e } q \neq
0}$ damos o nome de \sub{conjunto dos números racionais}.
\end{definicao}\pause

A representação decimal de um número racional é finita ou é uma
dízima periódica (infinita).
\begin{exemplo}
Reescreva as frações $\dfrac{12}{30}$ e $\dfrac 3 9 $ em forma de número decimal. Além disso, reescreva os números $0{,}6$; $1{,}37$; $0{,}222\dots$; $0{,}313131 \dots$ e $1{,}123123123 \dots$ em forma de fração irredutível, ou seja, já simplificada.
\end{exemplo}
\end{frame}



%------------------------------------------------------------------------------------------------------------
