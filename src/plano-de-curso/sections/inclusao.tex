\section{Inclusão}


\begin{frame}
	\frametitle{A Relação de Inclusão}

	\begin{definicao}
		Sejam $A$ e $B$ conjuntos. Se todo elemento de $A$ for também elemento de $B$, diz-se que $A$ é um \sub{subconjunto} de $B$, que $A$ \sub{está contido} em $B$, ou que $A$ é \sub{parte} de $B$. Para indicar esse fato, usa-se a notação $A \contido B$.
	\end{definicao}

	Quando $A$ não é um subconjunto de $B$, escreve-se $A \naocontido B$. Em outras palavras, existe pelo menos um elemento $a$ tal que $a \pertence A$ e $a \naopertence B$.

	Quando $A \contido B$, dizemos que $B$ \sub{contém} $A$ e escrevemos $B \contem A$.
\end{frame}


\begin{frame}
	\frametitle{A Relação de Inclusão} 

	\begin{exemplo}
		Sejam $T$ o conjunto de todos os triângulos e $P$ o conjunto dos polígonos do plano. Todo triângulo é um polígono, logo $ T \contido P$.
	\end{exemplo}

	\begin{exemplo}
		Na Geometria, uma reta, um plano e o espaço são conjuntos. Seus elementos são pontos.

		Quando dizemos que uma reta $r$ está no plano $\Pi$, estamos afirmando que $r$ está contida em $\Pi$ ou, equivalentemente, que $r$ é um subconjunto de $\Pi$, pois todos os pontos que pertencem a $r$ pertencem também a $\Pi$.

		Nesse caso, deve-se escrever $ r \contido \Pi$. Porém, não é correto dizer que $r$ pertence a $\Pi$, nem escrever $r \pertence \Pi$. Os elementos do conjunto $\Pi$ são pontos e não retas.
	\end{exemplo}
\end{frame}


\begin{frame}
	\frametitle{A Relação de Inclusão} 

	\begin{proposicao}[Inclusão universal do $\vazio$]
		Para todo conjunto $A$, vale $\vazio \contido A$.
	\end{proposicao}

	\begin{definicao}
		Dizemos que $A \diferente \vazio$ é um \sub{subconjunto próprio} de $B$ quando $A \contido B$  e $A \diferente B$.
	\end{definicao}
\end{frame}


\begin{frame}
	\frametitle{A Relação de Inclusão} 

	\begin{proposicao}[Propriedades da inclusão]
		Sejam $A$, $B$ e $C$ conjuntos. Tem-se:
		\begin{enumerate}[i.]
			\item \sub{Reflexividade}: $A \contido A$;
			\item \sub{Antissimetria}: Se $A \contido B$ e $B \contido A$, então $A = B$;
			\item \sub{Transitividade}: Se $A \contido B$ e $B \contido C$, então $A \contido C$.
		\end{enumerate}
	\end{proposicao}

	Demonstração no quadro.
\end{frame}


\begin{frame}
	\frametitle{A Relação de Inclusão} 

	\begin{definicao}
		Dado um conjunto $A$, chamamos de \sub{conjunto das partes} de $A$ o conjunto formado por todos os seus subconjuntos, e denotamo-lo $\partes{A}$.
	\end{definicao}

	\begin{exemplo}
		Dado $A = \conjunto {1, 2, 3}$, determine $\partes{A}$.
	\end{exemplo}
\end{frame}