\section{Recursos Didáticos}


\begin{frame}
    \frametitle{Recursos Didáticos}
    \framesubtitle{Sites e Apps}

    \begin{itemize}
        \item \link{https://matematica-elementar.github.io}{Site da disciplina} contendo versão atualizada das Notas de Aula, grade de horários da monitoria, entre outras coisas; \pause 
        \item \link{https://sigaa.ufrn.br/sigaa/public/home.jsf}{SIGAA} para comunicação e envio das atividades avaliativas; \pause
        \item Plataforma \link{https://pt.khanacademy.org/}{Khan Academy} para Atividades Online que servirão como presença; \pause
        \item Site \link{https://www.overleaf.com/}{Overleaf} para escrita em \LaTeX~  dos exercícios e das avaliações. Haverá uma atividade de extensão para introduzir os primeiros comandos usando essa ferramenta. Há um tutorial para auxiliar a inscrição \link{https://docs.google.com/document/d/17MZHJJPPGVE7ZnQY0we_jJb80PYepXxd3Nxwx3wsA9s/edit?usp=sharing}{aqui}; \pause
        \item Site \link{Piazza}{Piazza} para organizações de fóruns onde serão enviadas as dúvidas e os exercícios dos alunos; \pause
        \item Leitor de PDF \link{https://www.foxitsoftware.com/pt-br/downloads/}{Foxit} para ter acesso completo aos comentários nas avaliações.

    \end{itemize}
\end{frame}

\begin{frame}
    \frametitle{Recursos Didáticos}
    \framesubtitle{Google Apps}
    
    Será necessário uma \link{https://myaccount.google.com/intro?hl=pt-BR}{Conta Google} para ter acesso à:
        \begin{itemize}
            \item Pasta compartilhada no \link{https://www.google.com.br/drive/apps.html}{Google Drive} com vídeos da TV Elementar e das Lives; \pause
            \item \link{https://www.google.com/intl/pt-BR/calendar/about/}{Google Agenda} para ter acesso aos horários e links das atividades síncronas do curso; \pause
            \item \link{https://apps.google.com/meet/}{Google Meet} para as atividades síncronas.
        \end{itemize}
    
    \begin{observacao*}
        Todas as ferramentas foram testadas em celulares e funcionaram de maneira minimamente satisfatórias.
    \end{observacao*}
    

\end{frame}