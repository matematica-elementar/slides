\section{Avaliação}

\begin{frame}
	\frametitle{Avaliação geral da UFRN}

	Segundo o \link{https://sistemas.ufrn.br/download/sigaa/public/regulamento_dos_cursos_de_graduacao.pdf}{Regulamento dos cursos de graduação da UFRN}:
	
	\begin{itemize}
		\item A avaliação da disciplina é dividida em 3 unidades e na Reposição;
		\item A \emph{média parcial} do discente é calculada pela média aritmética das notas de cada unidade;
		\item A \emph{média final} do discente é calculada pela média aritmética das notas de cada unidade, substituindo a menor nota de unidade pela nota da avaliação de reposição;
		\item Em ME, o conteúdo da avaliação de reposição será referente ao conteúdo da unidade que obteve menor nota. Em outras disciplinas o docente pode escolher que seja o conteúdo de toda a disciplina.
	\end{itemize}

	
\end{frame}

\begin{frame}
	\frametitle{Avaliação geral da UFRN}
	\framesubtitle{Critérios de aprovação}

	\begin{itemize}
		\item O discente é aprovado se obter a frequência mínima de $75\%$ e cumprir um dos itens abaixo: 
		\begin{itemize}
			\item Obter média parcial maior ou igual a $6{,}0$ E obter notas maiores ou iguais a $4{,}0$ em TODAS as unidades. Nesse caso, fica dispensada a avaliação de reposição;
			\item Obter média final maior ou igual a $5{,}0$ DESDE QUE a nota da reposição maior ou igual a $4{,}0$.\pause
		\end{itemize}
		\item O discente não é aprovado se ocorrer um dos itens abaixo: 
		\begin{itemize}
			\item Não obter a frequência mínima de $75\%$, sendo dispensado da avaliação de reposição;
			\item Não obter média parcial maior ou igual a $3{,}0$, sendo dispensado da avaliação de reposição;
			\item Não obter média final maior ou igual a $5{,}0$ após a reposição;
			\item Não obter nota maior ou igual a $4{,}0$ na reposição.
		\end{itemize}
	\end{itemize}
OBS.: Nos casos em que a avaliação de reposição for dispensada, a média parcial do discente será a sua média final.
	
\end{frame}

\begin{frame}
	\frametitle{Avaliação em ME}

	Teremos 3 avaliações por unidade, sendo 2 provas escritas e 1 nota de atividades online, e a reposição. A divisão da pontuação e os respectivos conteúdos será a seguinte:
	\begin{itemize}
		\item Unidade 1
		\begin{itemize}
			\item Prova 1.1: Capítulo 1 - $3{,}0$ pontos;
			\item Prova 1.2: Capítulo 2 - $6{,}0$ pontos;
			\item Atividade Online 1 - $1{,}0$ ponto.
		\end{itemize} 
		\item Unidade 2
		\begin{itemize}
			\item Prova 2.1: Capítulo 3 - $3{,}0$ pontos;
			\item Prova 2.2: Capítulo 4 - $6{,}0$ pontos;
			\item Atividade Online 2 - $1{,}0$ ponto.
		\end{itemize}
		\item Unidade 3
		\begin{itemize}
			\item Prova 3.1: Capítulos 5 e 6 - $6{,}0$ pontos;
			\item Prova 1.2: Capítulo 7 - $3{,}0$ pontos;
			\item Atividade Online 3 - $1{,}0$ ponto.
		\end{itemize}
		\item Reposição: Conteúdo da unidade de menor nota - $10{,}0$ pontos.
	\end{itemize}
\end{frame}


\begin{frame}
	\frametitle{Avaliação em ME}
	\framesubtitle{Atividades online}

	\begin{itemize}
		\item As atividades online serão feitas através da plataforma \link{https://pt.khanacademy.org/}{Khan Academy};
		\item O cadastro na plataforma DEVE seguir a orientação do tutorial presente na página inicial da disciplina no SIGAA;
		\item A contabilização de pontos e o prazo das atividades também se encontra no tutorial.
	\end{itemize}
		
\end{frame}
