\section{Avaliação}


\begin{frame}
	\frametitle{Avaliação}

	Teremos 1 avaliação por unidade e a reposição. A divisão das unidades segundo os capítulos das Notas de Aula será a seguinte:
	\begin{itemize}
		\item Unidade 1: Capítulos 1 ao 4;
		\item Unidade 2: Capítulos 5 ao 7;
		\item Unidade 3: Capítulos 8 ao 10;
		\item Reposição: Quaisquer capítulos.
	\end{itemize}\pause

	As avaliações serão enviadas através de tarefas via SIGAA. Iniciarão às 18h do dia marcado no calendário e se encerrarão 38h depois.
\end{frame}


\begin{frame}
	\frametitle{Avaliação}
	\framesubtitle{Frequência}

	\begin{itemize}
		\item A frequência será calculada através da média do desempenho das atividades da Khan Academy (KA). Estão previstas, podendo mudar, 55 atividades distribuídas nos capítulos.
		\item Para não ser reprovado por falta, o discente precisa ter frequência de pelo menos $75\%$. Ou seja, seu desempenho médio nas atividades da KA devem ser de pelo menos $75\%$.
		\item As atividades de cada capítulo tem um prazo estipulado que pode ser consultado na própria KA ou no calendário. \pause
		\item A presença nos encontros síncronos NÃO contarão para fins de frequência na disciplina.
	\end{itemize}
		
\end{frame}
