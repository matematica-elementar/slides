\section{União e Interseção}


\begin{frame}
	\frametitle{União e Interseção de Conjuntos} 

	\begin{definicao}
		Dados os conjuntos $A$ e $B$:
		\begin{enumerate}[i.]
			\item A \sub{união} $A \uniao B$ é o conjunto formado pelos elementos que pertencem a pelo menos um dos conjuntos $A$ e $B$;
			\item A \sub{interseção} $A \inter B$ é o conjunto formado por elementos que pertencem a ambos $A$ e $B$.
		\end{enumerate}
	\end{definicao}

	\begin{exemplo}
		Sejam $A = \conjunto{1, 2, 3}$ e $B = \conjunto{2, 5}$. Determine $A \uniao B$, $A \inter B$, $A \menos B$ e $B \menos A$.
	\end{exemplo}
\end{frame}


\begin{frame}
	\frametitle{União e Interseção de Conjuntos} 
	
	\begin{proposicao}[Propriedades da união e interseção]
		\label{propuniaoint}
		Sejam $A$, $B$ e $C$ conjuntos. Tem-se:
		\begin{enumerate}[i.]
			\item $A \contido \paren{A \uniao B}$ e $\paren{A \inter B} \contido A$;
			\item \sub{União/interseção com o universo}: $\U \uniao A = \U$ e $A \inter \U = A$;
			\item \sub{Comutatividade}: $A \uniao B = B \uniao A$ e $A \inter B = B \inter A$;
			\item \sub{Associatividade}: $(A \uniao B) \uniao C = A \uniao (B \uniao C)$ e $(A \inter B) \inter C = A \inter (B \inter C)$;
			\item \sub{Distributividade, de uma em relação à outra}: $A \inter
			(B \uniao C) = (A \inter B) \uniao (A \inter C)$ e $A \uniao (B \inter C) = (A \uniao B) \inter (A \uniao C)$;
			\item \sub{Leis de DeMorgan}: $(A \uniao B)^C = A^C \inter B^C$ e $(A \inter B)^C = A^C \uniao B^C$.
		\end{enumerate}
	\end{proposicao}

	Demonstração no quadro.
\end{frame}