\section{Considerações Finais}


\begin{frame}
	\frametitle{Considerações Finais}
	
	\begin{itemize}
		\item Os alunos que possuam alguma Necessidade Educacional Específica (NEE), peço que me procurem em particular ao final da aula; \pause
		\item Avaliação de dispensa da disciplina.
	\end{itemize}

\end{frame}


\begin{frame}
	\frametitle{Considerações Finais}
	\framesubtitle{Mensagens de veteranos}

	\begin{itemize}
		\item Não subestimem conjuntos.\begin{itemize}
			\item Importante! No ensino medio a gente vê so a decoreba... O negocio é um pouco mais embaixo. Esqueçam tudo que vocês já viram pois ou viram errado ou não o bastante. (Pelo menos pra mim eu não sabia de quase nada de verdade)
		\end{itemize}
		\item Não negligenciem trigonometria. 
		Não desistam após a primeira nota baixa.
		Tonhão é difícil, mas não é o bicho que pregam(ou pregavam) principalmente o Tonhão do novo testamento.
		Façam todas as questões: tipo... de VERDADE. 
		Não guardem duvidas,mesmo que tenha que perguntar 3... 4x 
		Vai dar certo!
		\item Vez ou outra, inclusive, o assunto de conjuntos é muito retomado, como em FMC1, vendo conceitos até com recursão
	\end{itemize}

\end{frame}


\begin{frame}
	\frametitle{Considerações Finais}
	\framesubtitle{Mensagens de veteranos 2}

	\begin{itemize}
		\item Eu fui um dia a monitoria e foi o suficiente para encontrar um parceiro de estudos. Resolvíamos as questões em casa e trocávamos fotos das resoluções, discutíamos entre nós. Também tinha uns grupoes em que o pessoal que vivia na monitoria lançava umas dicas boas para a gente. Acho que formar uma rede de apoio mutuo eh crucial e principalmente ter um "parceiro" ajudou ambos. Acho que funciona melhor que grupões fica mais individualizado. Mas ambos precisam ter o mesmo objetivo e horarios etc etc
		\begin{itemize}
			\item Concordo muito com essa questão da rede de apoio, no início do ano passado a gente criou um grupo de estudos e entrou uma boa parcela de calouros. Mesmo que a maioria não interagisse, já foi o suficiente pra a galera que tinha dúvidas encontrar a galera q tinha entendido e estava disposta a explicar. Inclusive, esse grupo ainda está ativo e gerou uma comunidade de outros grupos destinados a outras matérias.
		\end{itemize}
	\end{itemize}

\end{frame}


\begin{frame}
	\frametitle{Considerações Finais}
	\framesubtitle{Mensagens de veteranos 3}

	\begin{itemize}
		\item Eu também daria um toque (não tão leve) sobre as monitorias do PET-CC que são ótimas, tem como ter uma pegada mais individualizada, mas tbm tem como marcar com uma galera e estudar junto.
		\item Acho que dicas de estudo.
		Falar pra n fazer resumos (é algo q se faz mt no ensino médio e eu acho q n faculdade de TI n faz mt sentido)
		\item Algo legal seria falar q quase todo mundo q reprova é pq desiste. Lembro disso no meu semestre, q só uma pessoa q foi pra prova final reprovou. Acho q pode motivar a glr a ficar até o fim
	\end{itemize}

\end{frame}


\begin{frame}
	\frametitle{Considerações Finais}
	\framesubtitle{Mensagens de veteranos 4}

	\begin{itemize}
		\item Acho que o ponto de virada pra muita gente é começar a praticar de verdade. É uma matemática diferente. Ficar observando as demonstrações dos outros achando que aprendeu não adianta muito. Fazer os exercícios, mesmo que estejam errados e levar para correção depois agregam muito na forma de pensar para resolver as questões. Acredito sim que a base de matemática faz diferença. Mas uma pessoa que estudou comigo basicamente o ensino médio inteiro reprovou várias vezes em ME. E olhe que ele era um dos melhores da turma
		\item Então diz pra eles n acreditarem mt no que veem nos grupos de calouros do zap tbm.
		\begin{itemize}
			\item 90\% das mensagens é só falação de besteira. Muita gente dá conselhos sem nem saber o que tá falando.
			\item Mta gente frustada c a vida fica botando medo sem necessidade
		\end{itemize}
	\end{itemize}

\end{frame}


\begin{frame}
	\frametitle{Considerações Finais}
	\framesubtitle{Mensagens de veteranos 5}

	\begin{itemize}
		\item Pensar/discutir questões em conjunto tem um potencial altíssimo de fixar o conteúdo. Se você for desses que estuda em grupo, faça isso em ME
		\item Estude pelo material dele, faça as questões do final do material de preferência enquanto ele dá o conteúdo, e *vá na monitoria* . Faz toda a diferença. Não tem erro.

		Eu ia na monitoria e ficava fazendo as questões, sempre que eu tinha dúvida perguntava pra um monitor.
		
		O povo vai fazer o maior terror sobre ele mas o macete é esse. 
		
		E faça as atividades que ele manda que já é um ponto garantido e ajuda mt a fixar o conteúdo.\pause
	\end{itemize}
	
	\vspace{1.8cm}

	\begin{center}
		\begin{large}
			OBRIGADO!
		\end{large}
	\end{center}

\end{frame}
