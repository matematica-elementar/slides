
\section{Exercícios}
\begin{frame}
\frametitle{Exercícios} 

\Ex{Em cada um dos itens abaixo, defina uma função com a lei de
formação dada (indicando domínio e contradomínio). Verifique se é
injetiva, sobrejetiva ou bijetiva, a função
\begin{enumerate}[a)]
	\item Que a cada ponto do plano cartesiano associa a distância desse ponto à origem do plano;
	\item Que a cada dois números naturais associa seu mdc;
%  \item Que a cada vetor do plano associa seu módulo;
%  \item Que a cada matriz $2 \times 2$ associa sua matriz
%  transposta;
%  \item Que a cada matriz $2 \times 2$ associa seu determinante;
	\item Que a cada polinômio (não nulo) com coeficientes reais
	associa seu grau;
	\item Que a cada figura plana fechada e limitada associa a sua
	área;
	\item Que a cada subconjunto de $\R$ associa seu complementar;
	\item Que a cada subconjunto finito de $\N$ associa seu número de
	elementos;
	\item Que a cada subconjunto não vazio de $\N$ associa seu menor
	elemento.
%	\item Que a cada função $f: \R \to \R$ associa seu valor no ponto $x_0 = 0$.
\end{enumerate}}



\end{frame}


%------------------------------------------------------------------------------------------------------------

\begin{frame}
	\frametitle{Exercícios} 
	\Ex{Considere a função $g: [0 ; 5] \to \R$ definida por: $$g(x) =
																	\begin{cases}
																	4x-x^2 & \ \text{ se } \ x< 3 \\
																	x-2 & \ \text{ se } \  x \geq 3 \\
																	\end{cases}.$$
	Determine as soluções de:
	\begin{enumerate}[a)]
		\item $g(x) = -1$;
		\item $g(x) = 0$;
		\item $g(x) = 3$;
		\item $g(x) = 4$;
		\item $g(x) < 3$;
		\item $g(x) \geq 3$. 
	\end{enumerate} }
	\end{frame}
	
	
	%------------------------------------------------------------------------------------------------------------
	


\begin{frame}
\frametitle{Exercícios} 

\Ex{Considere a  função $f:  \N^\ast  \to      \Z $ tal que $$f(n) =
\begin{dcases} \dfrac {-n} 2, &\text{ se $n$ é par} \\ \dfrac {n-1} 2, &\text{ se $n$ é
ímpar} \end{dcases}.$$ Mostre que $f$ é bijetiva. %O que você pode concluir com esse resultado?
}

%\Ex{Mostre que a função inversa de $f: X \to Y$, caso exista, é única, isto é, se existem $g_1 : Y \to X$ e $g_2 : Y \to X$ satisfazendo a Definição \ref{funinv}, então $g_1 = g_2$.\\
%\emph{Dica: } Lembre-se que duas funções são iguais se, e só se, possuem mesmos domínios, contradomínios e seus valores são iguais em todos os elementos do domínio. Assim, procure mostrar que $g_1 (y) = g_2 (y)$, para todo $y \in Y$.}

\Ex{
    Considere a função $f: \R^\ast \to \R^\ast_+$ tal que $f(x) = \dfrac{1}{1+x^2}$. Responda as seguintes perguntas apresentando as respectivas justificativas.
    \begin{enumerate}[a)]
        \item $f$ é injetiva?
	\item $f$ é sobrejetiva?
    \end{enumerate}
}



\end{frame}

%------------------------------------------------------------------------------------------------------------

\begin{frame}
\frametitle{Exercícios} 

\Ex{
	Seja $f:\R \to \R$ a função cuja lei de associação é dada abaixo:
        \begin{equation*}
                f(x)=\left\{
                    \begin{array}{ll}
                        x^{2} +3x,   & \text{ se  $x \geq 0$} \\
                        \dfrac{3}{2}x,          & \text{ se $x< 0$} \\
                    \end{array}\right. .
        \end{equation*}

       Mostre que $f$ é bijetiva.
}

\Ex{
	Considere a função $f : (0,1) \to \mathbb R$ tal que
$$f(x) =
\begin{dcases} 
  \dfrac 1 x - 2, & \text{ se $x \leq \dfrac 1 2$} \\
  2 - \dfrac 1 {1 - x} , & \text{ se $x > \dfrac 1 2$}
\end{dcases}.$$
Mostre que $f$ é bijetiva.
}

\Ex{
	Considere  $f: [3,5; +\infty)  \to [ -2,25 ; +\infty)$ tal que $f(x) = x^2 -7x + 10$. 
	 Prove que $f$ é bijetiva.
}

\end{frame}



%------------------------------------------------------------------------------------------------------------

\begin{frame}
\frametitle{Exercícios} 

\Ex{
	Considere as funções $f: \R \to \R_+$ tal que $f(x) = x^2+3$ e $g: (-\infty ; 5] \to \R$ tal que $g(x) = \sqrt{x^2 - 10x +27}$. Faça o que se pede:
            \begin{enumerate}[a)]
                \item  Calcule $(f \circ g)$ e $(g \circ f)$. Caso não seja possível, justifique;
                \item  Verifique a injetividade e a sobrejetividade de alguma das funções compostas que você calculou no item anterior.
            \end{enumerate}
}

\Ex{
	Considere as funções $f$, $g$ e $h$ definidas por: $f:  (- \infty , 0] \to [- 4, + \infty) $, tal que $f(x)= -x -4$; $g:  (- \infty , 0] \to  \R$, tal que $g(x) = \sqrt {-x}$; e $h : \R \to [- 4, + \infty )$, tal que $h(x) = x^2 - 4$. Quais dessas funções é sobrejetiva e quais não são? Alguma dessas funções é resultante da composição das outras?
}

\end{frame}



%------------------------------------------------------------------------------------------------------------

\begin{frame}
\frametitle{Exercícios} 

\Ex{
	Considere as funções reais $f: X \to Y$ e $g: Y \to Z$. Demonstre, ou refute com um contraexemplo, as afirmações abaixo:
\begin{enumerate}[a)]
\item Se $f$ e $g$ são injetivas, então $(g \circ f)$ é injetiva;
\item Se $(g \circ f)$ é injetiva então $f$ e $g$ são injetivas;
\item Se $f$ e $g$ são sobrejetivas, então $(g \circ f)$ é sobrejetiva;
\item Se $(g \circ f)$ é sobrejetiva então $f$ e g são sobrejetivas.
\end{enumerate}
}

\Ex{
	Faça uso de pelo menos um dos resultados anteriores para mostrar a injetividade das funções $f: [1, + \infty) \to (- \infty , 0]$, tal que $f(x)= -x+1$; $g:  [1, + \infty) \to  \R$, tal que $g(x) = x^2 - 2x -3$; e $h : (- \infty , 0] \to  \R$, tal que $h(x) = x^2 - 4$.
}





\end{frame}

%------------------------------------------------------------------------------------------------------------


