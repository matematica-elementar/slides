
\section{Exercícios}
\begin{frame}
\frametitle{Exercícios} 

\Ex{ Descubra os valores de $x$ de modo que seja possível completar
o preenchimento do quadrado mágico abaixo:
\begin{center}
\begin{tabular}{|c|c|c|}
	\hline
	% after \\: \hline or \cline{col1-col2} \cline{col3-col4} ...
	 &   &  \\ \hline
		& $x$ &   \\ \hline
		&  &   \\
	\hline
\end{tabular}
\end{center}
}

\Ex{ Observe as multiplicações a seguir:
\begin{enumerate}[i.]
	\item $12.345.679 \cdot 18 = 222.222.222$
	\item $12.345.679 \cdot 27 = 333.333.333$
	\item $12.345.679 \cdot 54 = 666.666.666$
\end{enumerate}

Para obter 999.999.999 devemos multiplicar 12.345.679 por \\ quanto?
 }




\end{frame}





%------------------------------------------------------------------------------------------------------------

\begin{frame}
\frametitle{Exercícios} 

\Ex{
	Eu tenho o dobro da idade que tu tinhas quando eu tinha a tua idade. Quando tu tiveres a minha idade, a soma das nossas idades será de 45 anos. Quais são as nossas idades?
}
	
\Ex{
	Passarinhos brincam em volta de uma velha árvore. Se dois passarinhos pousam em cada galho, um passarinho fica voando. Se todos os passarinhos pousam, com três em cada galho, um galho fica vazio. Quantos são os passarinhos?
}

\Ex{ O número $-3$ é a raiz da equação $x^2 -7x -2c = 0$. Nessas
condições, determine o valor do coeficiente $c$. }

\Ex{
	Determine o conjunto solução $S \contido \Q$ formado pelo(s) número(s) que, adicionado ao triplo de seu quadrado, resulta em 14.
}

\Ex{
	Determine o(s) valor(es) de $m \in \R$ tal(is) que a equação $mx^2 + (m+1)x + (m+1) = 0$ tenha somente uma raiz real.
}

\Ex{
	Calcule as dimensões de um retângulo de $20cm$ de perímetro e $22cm^2$ de área.
}

\end{frame}


%------------------------------------------------------------------------------------------------------------

\begin{frame}
\frametitle{Exercícios} 

\Ex{
	Sejam $\alpha_1$ e $\alpha_2$ as raízes da equação do 2° grau $ax^2 + bx + c = 0$. Calcule as seguintes expressões em função de $a$, $b$ e $c$:
	\begin{enumerate}[a)]
		\item $\dfrac{\alpha_1 + \alpha_2}2$;
		\item $\sqrt{\alpha_1 \cdot \alpha_2}$;
		\item $\sqrt {\alpha_1} + \sqrt {\alpha_2}$;
		\item $\sqrt[4] {\alpha_1} + \sqrt[4] {\alpha_2}$.
	\end{enumerate}
	\emph{Dica}: no item c), inicie calculando o quadrado da expressão.
}

\Ex{
	Resolva as equações abaixo:
	\begin{enumerate}[a)]
		\item $x^4-3x^2-4=0$;
		\item $x^6+4x^3+4=0$.
	\end{enumerate}
}

\Ex{
	(Colégio Naval - 1986) Uma equação biquadrada tem duas de suas raízes iguais a $\sqrt 2$ e $3$. Determine o valor do coeficiente do termo de 2° grau dessa equação.
}

\Ex{
	(EPCAR - 2002) Determine o produto das raízes da equação $7 + \sqrt{x^2 - 1} =x^2$.
}

\end{frame}


%------------------------------------------------------------------------------------------------------------

