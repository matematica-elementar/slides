\section{Equação do 2º grau}
\begin{frame}
    \frametitle{Equação do 2º grau} 
    
    \begin{teorema}
    Os números $\alpha$ e $\beta$ são as raízes da equação do segundo
    grau $$ax^2+ bx+c=0$$ se, e somente se, $$\alpha + \beta = \frac
    {-b} a \; \text{ e } \; \alpha \beta = \frac c a.$$
    \end{teorema}\pause

    \begin{exemplo}
        Paulo cercou uma região retangular de área $28m^2$ com $24m$ de corda. Encontre as dimensões dessa região.
    \end{exemplo}
    
    \end{frame}
    
    
%------------------------------------------------------------------------------------------------------------

    \begin{frame}
        \frametitle{Equação do 2º grau} 
        
        \begin{definicao}[Equação biquadrada]
            A \sub{equação biquadrada} com coeficientes $a$, $b$ e $c$ é uma equação da forma $$ax^4 + bx^2 + c = 0,$$ onde $a, b, c \in \R$, $a \neq 0$ e $x$ é uma variável real a ser determinada.
        \end{definicao}\pause

        \begin{exemplo}
        Resolva a equação $x^4 - 2x^2 +1 = 0$.
        \end{exemplo}

    \end{frame}
    
    
    %------------------------------------------------------------------------------------------------------------
    
    \begin{frame}
        \frametitle{Equação do 2º grau} 

        De modo geral, podemos resolver uma equação do tipo $$ax^{2k} +bx^k +c = 0,$$ onde $k \pertence \N$, fazendo $y=x^k$. Assim, para cada solução $y = \alpha$ encontrada, teremos as seguintes possibilidades:\pause
        \begin{itemize}
            \item uma única solução: $x = \sqrt[k]\alpha $ se $k$ é ímpar;\pause
            \item nenhuma solução: se $\alpha < 0$ e $k$ é par;\pause
            \item duas soluções: $x = \pm \sqrt[k]\alpha $ se $\alpha > 0$ e $k$ é par.
        \end{itemize}
        
    
    \end{frame}
        
        
%------------------------------------------------------------------------------------------------------------