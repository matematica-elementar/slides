\section{Equação do 1º grau}
\begin{frame}
    \frametitle{Equação do 1º grau} 
    
    
    \begin{exemplo}
    Se $x$ representa um dígito na base 10 na equação $$x11 + 11x + 1x1
    = 777,$$ qual o valor de $x$?
    \end{exemplo}
    \end{frame}
    
    
    
    %------------------------------------------------------------------------------------------------------------
    
    \begin{frame}
    \frametitle{Equação do 1º grau} 
    
    \begin{exemplo}
    Determine se é possível completar o preenchimento do tabuleiro
    abaixo com os números naturais de 1 a 9, sem repetição, de modo que
    a soma de qualquer linha seja igual à de qualquer coluna ou
    diagonal.
    
    \begin{center}
    \begin{tabular}{|c|c|c|}
        \hline
        % after \\: \hline or \cline{col1-col2} \cline{col3-col4} ...
        1 &   & 6 \\ \hline
            &   &   \\ \hline
            & 9 &   \\
        \hline
    \end{tabular}
    \end{center}
    
    \end{exemplo}
    
    Os tabuleiros preenchidos com essas propriedades são conhecidos como
    \sub{quadrados mágicos}.
    \end{frame}
    
    
    
    %------------------------------------------------------------------------------------------------------------
    
    \begin{frame}
    \frametitle{Equação do 1º grau} 
    
    \begin{exemplo}
    Imagine que você possui um fio de cobre extremamente longo, mas tão
    longo que você consegue dar a volta na Terra com ele. Para
    simplificar, considere que a Terra é uma bola redonda e que seu raio
    é de exatamente 6.378.000 metros.
    
    O fio com seus milhões de metros está ajustado à Terra, ficando bem
    colado ao chão ao longo do Equador. Digamos, agora, que você
    acrescente 1 metro ao fio e o molde de modo que ele forme um círculo
    enorme, cujo raio é um pouco maior que o raio da Terra e tenha o
    mesmo centro. Você acha que essa folga será de que tamanho?
    \end{exemplo}
    
    \pause Já sabemos que a folga obtida aumentando o fio independe do
    raio em consideração. Além desse problema, veja outras curiosidades
    sobre o número $\pi$ no vídeo \link
    {https://www.youtube.com/watch?v=evfc6bv6_lM}{O Pi existe} e tente
    calculá-lo em casa usando algum objeto redondo.
    
    
    \end{frame}
    
    %------------------------------------------------------------------------------------------------------------