\documentclass[10pt]{beamer}
%
%   Arquivo de Configuração dos Slides
%


%
%   Pacotes utilizados
%

% Codificação dos caracteres em formato universal.
\usepackage[utf8]{inputenc}
\usepackage[T1]{fontenc}

% Traduz o texto gerados pelo LaTeX para português. ex.: Capítulo, Seção, Conteúdo.
\usepackage[brazil]{babel}

% Pacotes para ambientes matemáticos
\usepackage{amsmath}
\usepackage{amsthm}
\usepackage{amssymb}

% Diversas funções para o uso das aspas.
\usepackage{csquotes}

% Outros pacotes
\usepackage{hyperref}
\usepackage{tikz}
\usepackage{yfonts}
\usepackage{colortbl}
\usepackage{ragged2e}
\usepackage{helvet}
\usepackage{verbatim}


%
%   Tema
%

% Copyright 2007 by Till Tantau
%
% This file may be distributed and/or modified
%
% 1. under the LaTeX Project Public License and/or
% 2. under the GNU Public License.
%
% See the file doc/licenses/LICENSE for more details.


% Common packages


\usepackage{times}
 \mode<article> {
	\usepackage{times}
	\usepackage{mathptmx}
	\usepackage[left=1.5cm,right=6cm,top=1.5cm,bottom=3cm]{geometry}
}

\usepackage{hyperref}
\usepackage[T1]{fontenc}
\usepackage{amsmath,amssymb}
\usepackage{tikz}
\usepackage{colortbl}
\usepackage{yfonts}
\usepackage{colortbl}
\usepackage{translator} % comment this, if not available
\usepackage{ragged2e} % justifying
% Or whatever. Note that the encoding and the font should match. If T1
% does not look nice, try deleting the line with the fontenc.
\usepackage{helvet}
\usepackage{verbatim}


%\usepackage{lipsum}
%\usepackage{enumitem}


\usetheme[
%%% options passed to the outer theme
%    hidetitle,           % hide the (short) title in the sidebar
%    hideauthor,          % hide the (short) author in the sidebar
%    hideinstitute,       % hide the (short) institute in the bottom of the sidebar
%    shownavsym,          % show the navigation symbols
%    width=2cm,           % width of the sidebar (default is 2 cm)
%    hideothersubsections,% hide all subsections but the subsections in the current section
%    hideallsubsections,  % hide all subsections
		right               % right of left position of sidebar (default is right)
%%% options passed to the color theme
%    lightheaderbg,       % use a light header background
	]{AAUsidebar}

% If you want to change the colors of the various elements in the theme, edit and uncomment the following lines
% Change the bar and sidebar colors:
%\setbeamercolor{AAUsidebar}{fg=red!20,bg=red}
%\setbeamercolor{sidebar}{bg=red!20}
% Change the color of the structural elements:
%\setbeamercolor{structure}{fg=red}
% Change the frame title text color:
%\setbeamercolor{frametitle}{fg=blue}
% Change the normal text color background:
%\setbeamercolor{normal text}{bg=gray!10}
% Highlight the text in the sidebar
\usecolortheme{rose,sidebartab}
% ... and you can of course change a lot more - see the beamer user manual.

% colored hyperlinks
\newcommand{\chref}[2]{%
	\href{#1}{{\usebeamercolor[bg]{AAUsidebar}#2}}%
}



% specify a logo on the titlepage (you can specify additional logos an include them in
% institute command below
\pgfdeclareimage[height=1cm]{titlepagelogo}{theme/figures/ufrn2} % placed on the title page
\pgfdeclareimage[height=1cm]{titlepagelogo2}{theme/figures/imd} % placed on the title page
\titlegraphic{% is placed on the bottom of the title page
	\pgfuseimage{titlepagelogo}
	\hspace{1cm}\pgfuseimage{titlepagelogo2}
}


% Article version layout settings

\mode<article>

\makeatletter
\def\@listI{\leftmargin\leftmargini
	\parsep 0pt
	\topsep 5\p@   \@plus3\p@ \@minus5\p@
	\itemsep0pt}
\let\@listi=\@listI


\setbeamertemplate{frametitle}{\paragraph*{\insertframetitle\
		\ \small\insertframesubtitle}\ \par
}
\setbeamertemplate{frame end}{%
	\marginpar{\scriptsize\hbox to 1cm{\sffamily%
			\hfill\strut\insertshortlecture.\insertframenumber}\hrule height .2pt}}
\setlength{\marginparwidth}{1cm}
\setlength{\marginparsep}{4.5cm}

\def\@maketitle{\makechapter}

\def\makechapter{
	\newpage
	\null
	\vskip 2em%
	{%
		\parindent=0pt
		\raggedright
		\sffamily
		\vskip8pt
		\includegraphics[width=\linewidth]{theme/figures/imd.png}\par\vskip2em
		{\fontsize{36pt}{36pt}\selectfont Aula \insertshortlecture \par\vskip2pt}
		{\fontsize{24pt}{28pt}\selectfont \color{blue!50!black} \@title\par\vskip4pt}
		%{\Large\selectfont \color{blue!50!black} \insertsubtitle\par}
		\vskip10pt

		\normalsize\selectfont [Notas de Aula]
		Disciplina: \emph{\lecturename \ (\semestre)} \par\vskip1.5em
		\nomedoautor\hskip1em Email: \ \emaildoautor
	}
	\par
	\vskip 1.5em%
}

\let\origstartsection=\@startsection
\def\@startsection#1#2#3#4#5#6{%
	\origstartsection{#1}{#2}{#3}{#4}{#5}{#6\normalfont\sffamily\color{blue!50!black}\selectfont}}

\makeatother

\mode
<all>




% Typesetting Listings

\usepackage{listings}
\lstset{language=Java}

\alt<presentation>
{\lstset{%
	basicstyle=\footnotesize\ttfamily,
	commentstyle=\slshape\color{green!50!black},
	keywordstyle=\bfseries\color{blue!50!black},
	identifierstyle=\color{blue},
	stringstyle=\color{orange},
	escapechar=\#,
	emphstyle=\color{red}}
}
{
	\lstset{%
		basicstyle=\ttfamily,
		keywordstyle=\bfseries,
		commentstyle=\itshape,
		escapechar=\#,
		emphstyle=\bfseries\color{red}
	}
}



% Common theorem-like environments
%\usepackage{amsthm}

\setbeamertemplate{theorems}[numbered]

%
%	New useful definitions:
%

\newbox\mytempbox
\newdimen\mytempdimen

\newcommand\includegraphicscopyright[3][]{%
	\leavevmode\vbox{\vskip3pt\raggedright\setbox\mytempbox=\hbox{\includegraphics[#1]{#2}}%
		\mytempdimen=\wd\mytempbox\box\mytempbox\par\vskip1pt%
		\fontsize{3}{3.5}\selectfont{\color{black!25}{\vbox{\hsize=\mytempdimen#3}}}\vskip3pt%
}}

\newenvironment{colortabular}[1]{\medskip\rowcolors[]{1}{blue!20}{blue!10}\tabular{#1}\rowcolor{blue!40}}{\endtabular\medskip}

\def\equad{\leavevmode\hbox{}\quad}

\newenvironment{greencolortabular}[1]
{\medskip\rowcolors[]{1}{green!50!black!20}{green!50!black!10}%
	\tabular{#1}\rowcolor{green!50!black!40}}%
{\endtabular\medskip}

%\setbeamertemplate{theorem begin}{{ \inserttheoremheadfont
%\inserttheoremname \inserttheoremnumber
%\ifx\inserttheoremaddition\empty\else\ (\inserttheoremaddition)\fi%
%\inserttheorempunctuation }} \setbeamertemplate{theorem end}{}

\newcommand{\vu}{\vec{u}}
\newcommand{\vv}{\vec{v}}
\newcommand{\vi}{\vec{i}}
\newcommand{\vj}{\vec{j}}
\newcommand{\vk}{\vec{k}}
\newcommand{\vw}{\vec{w}}
\newcommand\segmento[2]{\overline{#1#2}}
\def\colc#1{\left[#1\right]}



%
%   Macros
%

\usepackage{macros/macros}


%
%   Ambientes
%

\theoremstyle{plain}
\newtheorem{teorema}{Teorema}

\theoremstyle{definition}
\newtheorem{definicao}[teorema]{Definição}
%\newtheorem{exercicio}{Exercício}

\theoremstyle{remark}
\newtheorem{obs}[teorema]{Observação}
\newtheorem{observacao}[teorema]{Observação}
\newtheorem{corolario}[teorema]{Corolário}
\newtheorem{exemplo}[teorema]{Exemplo}
\newtheorem{lema}[teorema]{Lema}
\newtheorem{proposicao}[teorema]{Proposição}

\newcounter{exercicios}
\newenvironment{exercicio}{\stepcounter{exercicios} \textbf{\arabic{exercicios}}.}{}

% compatibilidade
\newcommand{\Ex}[1]{\begin{exercicio}#1\end{exercicio}}

%
%   Definições e comandos auxiliares do preâmbulo
%

\newcommand{\capitulo}[1]{\lecture[#1]{Capítulo}}
\newcommand{\aula}[1]{\subtitle{#1}}
\newcommand{\autor}{Igor Oliveira}
\newcommand{\email}{\href{mailto:matematicaelementar@imd.ufrn.br}{\texttt{matematicaelementar@imd.ufrn.br}}}
\newcommand{\disciplina}{Matemática Elementar}
\newcommand{\codigo}{IMD1001}

\title{\disciplina}
\date{\today}
\author[\autor]
{
    \autor\\
    \email
}

\def\lecturename{\codigo

\disciplina}

\institute[
	UFRN\\
	Natal-RN
]
{
	Instituto Metrópole Digital\\
	Universidade Federal do Rio Grande do Norte\\
	Natal-RN

}

% compatibilidade
\newcommand{\vu}{\vec{u}}
\newcommand{\vv}{\vec{v}}
\newcommand{\vi}{\vec{i}}
\newcommand{\vj}{\vec{j}}
\newcommand{\vk}{\vec{k}}
\newcommand{\vw}{\vec{w}}
\newcommand{\segmento}[2]{\overline{#1#2}}
\def\colc#1{\left[#1\right]}
\newcommand{\negacao}{\sim}

\justifying


\aula{Funções Reais e Gráficos}
\capitulo{3}


\begin{document}	

	%
	%	Capa
	%

	{\backgroundimage\begin{frame}[plain]
		\titlepage
	\end{frame}}


	%
	%	Sumário
	%

	\begin{frame}
		\frametitle{Índice}
		\tableofcontents
	\end{frame}


	%
	%	Seções
	%

	\section{Introdução}

\begin{frame}  \frametitle{Apresentação da Aula}

Considere as funções
$$\begin{array}{cccc}
p : & \R & \to     & \R_+ \\
		 &  x & \mapsto & x^2
\end{array}
\text{\ \ \  e \ \ \ }
\begin{array}{cccc}
q : & \R_+ & \to     & \R \\
		 &  x & \mapsto & \sqrt x
\end{array}.$$
As funções $p$ e $q$ são inversas uma da outra? \\ \pause Elas são
bijetivas? \\ \pause
Quais outras informações podemos dizer acerca dessas funções?



\end{frame}

	
\section{Definição de Função}
\begin{frame} \frametitle{O que É uma Função?}
\begin{definicao}
Sejam $X$ e $Y$ dois conjuntos quaisquer.\\
Uma \sub{função} é uma relação $f: X \to Y$ que, a cada elemento $x
\in X$, associa um e somente um elemento $y \in Y$.\\
Nesse caso:
\begin{enumerate}[(i)]
	\item Os conjuntos $X$ e $Y$ são chamados \sub{domínio} e
	\sub{contradomínio} de $f$, respectivamente;
	\item O conjunto $f\paren X = \set{y \in Y \tq \exists x \in X , f \paren x =
	y} \contido Y$ é chamado \sub{imagem} de $f$;
	\item Dado $x \in X$, o (único) elemento $y = f(x) \in Y$
	correspondente é chamado \sub{imagem} de $x$.
\end{enumerate}
\end{definicao}

\end{frame}

%------------------------------------------------------------------------------------------------------------

\begin{frame} \frametitle{O que É uma Função?} 

Dessa forma, uma função é um terno constituído por: \sub{domínio},
\sub{contradomínio} e \sub{lei de associação} (dos elementos do
domínio com os do contradomínio). Precisamos desses três elementos
para que uma função seja bem definida. Poderíamos definir
função da seguinte forma: \\
Para que uma relação $f: X \to Y$ seja
uma função, ela deve satisfazer a duas condições fundamentais:
\begin{enumerate}[(I)]
	\item Estar bem definida em todo elemento do domínio (existência);
	\item Não fazer corresponder mais de um elemento do contradomínio
	a cada elemento do domínio (unicidade).
\end{enumerate}

\end{frame}

%------------------------------------------------------------------------------------------------------------

\begin{frame}
\frametitle{O que É uma Função?} 
\begin{exemplo}
Considere as funções
$$\begin{array}{cccc}
p : & \R & \to     & \R_+ \\
		 &  x & \mapsto & x^2
\end{array}
\text{\ \ \  e \ \ \ }
\begin{array}{cccc}
q : & \R_+ & \to     & \R \\
		 &  x & \mapsto & \sqrt x
\end{array}.$$
Qual o domínio, contradomínio e a lei de associação de $p$ e $q$?
\end{exemplo}
\pause

\begin{exemplo}
Seja $\I{X} : X \to X $ uma função tal que $\I{X} \paren x = x$ para
todo $x \in X$. Chamamos $\I{X}$ de \sub{função identidade do
conjunto $X$}.
\end{exemplo}

\end{frame}


%------------------------------------------------------------------------------------------------------------

	\section{Atividade Online}
\begin{frame}
\frametitle{Atividade Online} 

\href{https://pt.khanacademy.org/math/algebra-home/alg-functions/alg-recognizing-functions-ddp/e/recognizing_functions}
{{\tt Atividade 17 - Como Reconhecer Funções a Partir de\\ Tabelas}}

\href{https://pt.khanacademy.org/math/algebra/x2f8bb11595b61c86:functions/x2f8bb11595b61c86:determining-the-domain-of-a-function/e/interpreting-domain}
{{\tt Atividade 18 - Problemas de Domínio de Funções}}


\end{frame}

	\section{Funções Compostas}
\begin{frame}
\frametitle{Composição de Funções} 

\begin{definicao}
Sejam $f: X \to Y$ e $g: U \to V$ duas funções, com $Y \contido U$.\\
A \sub{função composta de $g$ com $f$} é a função denotada por $g
\circ f$, com domínio em $X$ e contradomínio em $V$, que a cada
elemento $x \in X$ faz corresponder o elemento $v = \paren{g \circ
f}(x) = g(f(x)) \in V$. Isto é:
$$\begin{array}{cccccc}
g \circ f : & X & \to     & Y \contido U & \to & V \\
		 &  x & \mapsto & f(x) & \mapsto & g(f(x))
\end{array}.$$
\end{definicao}

\end{frame}

%------------------------------------------------------------------------------------------------------------

\begin{frame}
\frametitle{Composição de Funções} 

\begin{exemplo}
Seja $f: X \to Y$ uma função. Mostre que $f \circ \I{X} = f$ e $\I{Y}
\circ f = f$.
\end{exemplo}\pause

\begin{exemplo}
Qual função resulta da composição $p \circ q$?
\end{exemplo}
\end{frame}



%------------------------------------------------------------------------------------------------------------

\begin{frame}
\frametitle{Composição de Funções} 


\begin{proposicao}[Associatividade da composição de funções]
	Considere $f: X \to Y$, $g: U \to V$ e $h: A \to B$ funções, com $B \contido U$ e $V \contido X$. Vale a seguinte igualdade:
	$$f \composta (g \composta h) = (f \composta g) \composta h.$$
\end{proposicao}

\end{frame}

%------------------------------------------------------------------------------------------------------------
	\section{Atividade Online}
\begin{frame}
\frametitle{Atividade Online} 

\href{https://pt.khanacademy.org/math/algebra2/manipulating-functions/function-composition/e/compose-functions}
{{\tt Atividade 19 - Encontre Funções Compostas}}

\href{https://pt.khanacademy.org/math/algebra2/manipulating-functions/combining-and-composing-modeling-functions/e/modeling-with-composite-functions}
{{\tt Atividade 20 - Modele com Funções Compostas}}



\end{frame}

	\section{Função Inversa}
\begin{frame}
\frametitle{Função Inversa} 

\begin{definicao}\label{funinv}
Uma função $f: X \to Y$ é \sub{invertível} se existe uma função $g:
Y \to X$ tal que
\begin{enumerate}[(i)]
	\item $f \circ g = \I{Y}$;
	\item $g \circ f = \I{X}$.
\end{enumerate}
Nesse caso, a função $g$ é dita \sub{função inversa} de $f$ e
denotada por $g = f^{-1}$.
\end{definicao}

\end{frame}



%------------------------------------------------------------------------------------------------------------


\begin{frame}
\frametitle{Função Inversa} 

\begin{exemplo}
A função $q$ é inversa de $p$?
\end{exemplo}\pause
 Esse exemplo ilustra a importância de verificarmos as duas
 condições para que tenhamos uma função inversa.

\end{frame}

	
\section{Atividade Online}
\begin{frame}
\frametitle{Atividade Online} 

\href{https://pt.khanacademy.org/math/algebra/x2f8bb11595b61c86:irrational-numbers/x2f8bb11595b61c86:irrational-numbers-intro/e/recognizing-rational-and-irrational-numbers}
{{\tt Atividade 03 - Classifique Números: Racionais e \\ Irracionais}}

\href{https://pt.khanacademy.org/math/algebra/x2f8bb11595b61c86:irrational-numbers/x2f8bb11595b61c86:sums-and-products-of-rational-and-irrational-numbers/e/recognizing-rational-and-irrational-expressions}
{{\tt Atividade 04 - Expressões Racionais Versus Irracionais}}




\end{frame}


	\section{Injetividade e Sobrejetividade}
\begin{frame}
\frametitle{Funções Injetivas, Sobrejetivas e Bijetivas} 


\begin{definicao}
Considere uma função $f: X \to Y$.
\begin{enumerate}[(i)]
	\item $f$ é \sub{sobrejetiva} se, para todo $y \in Y$, existe $x
	\in X$ tal que $f(x) = y$;
	\item $f$ é \sub{injetiva} se $x_1, x_2 \in X, x_1 \neq x_2
	\implica f(x_1) \neq f(x_2)$;
	\item $f$ é \sub{bijetiva} se é sobrejetiva e injetiva.
\end{enumerate}
\end{definicao}\pause
Há, ainda, formas alternativas de enunciar as definições acima:
\begin{itemize}
	\item $f$ é \sub{sobrejetiva} se, e somente se, $f(X) = Y$;
	\item $f$ é \sub{injetiva} se, e somente se, $x_1, x_2 \in X, f(x_1) = f(x_2)
	\implica x_1 = x_2 $;
	\item $f$ é \sub{injetiva} se, e somente se, para todo $y \in
	f(X)$, existe um único $x \in X$ tal que $f(x) = y$;
	\item $f$ é \sub{bijetiva} se, e somente se, para todo $y \in Y$,
	existe um único $x \in X$ tal que $f(x) = y$.
\end{itemize}
\end{frame}



%------------------------------------------------------------------------------------------------------------

\begin{frame}
\frametitle{Funções Injetivas, Sobrejetivas e Bijetivas} 

\begin{exemplo}\label{pqbij}
As funções $p$ e $q$ são sobrejetivas, injetivas ou bijetivas?
\end{exemplo}
\end{frame}



%------------------------------------------------------------------------------------------------------------

\begin{frame}
\frametitle{Funções Injetivas, Sobrejetivas e Bijetivas} 

\begin{teorema}\label{teoinv}
Uma função $f: X \to Y$ é invertível se, e somente se, é bijetiva.
\end{teorema}\pause

\begin{exemplo}
Decorre do Teorema \ref{teoinv} e do Exemplo \ref{pqbij} que as
funções $p$ e $q$ não são invertíveis.
\end{exemplo}

\end{frame}


	\section{Atividade Online}
\begin{frame}
\frametitle{Atividade Online} 

\href{https://pt.khanacademy.org/math/algebra2/manipulating-functions/invertible-functions/e/inverse-domain-range}
{{\tt Atividade 22 - Determine se uma Função É Inversível}}

\href{https://pt.khanacademy.org/math/algebra2/manipulating-functions/invertible-functions/e/restrict-the-domains-of-functions}
{{\tt Atividade 23 - Restrinja os Domínios de Funções para\\ Torná-las
Inversíveis}}


\end{frame}
	\section{Fórmulas e Funções}
\begin{frame}
\frametitle{Fórmulas e Funções} 
É muito importante não pensar que uma função é uma fórmula.
Considere as funções
$$\begin{array}{cccc}
p_1 : & \R & \to     & \R \\
		 &  x & \mapsto & x^2
\end{array}
\text{\ \ \  e \ \ \ }
\begin{array}{cccc}
p_2 : & \R_+ & \to     & \R_+ \\
		 &  x & \mapsto &  x^2
\end{array}.$$
Essas funções são iguais? \\ \pause NÃO! Note que $p_2$ é bijetiva e
$p_1$ não é, mesmo tendo a mesma fórmula.

Além disso, funções podem ser definidas por mais de uma fórmula,
como na função $h :  \R  \to      \R$ tal que
		 $$h(x) =  \begin{cases}
						0, &  \ \text{ se } x \in \R \setminus \Q \\
						1, & \ \text{ se } x \in \Q
						\end{cases} .$$
\end{frame}


	\section{Atividade Online}
\begin{frame}
\frametitle{Atividade Online} 

\href{https://pt.khanacademy.org/math/algebra-home/alg-functions/alg-piecewise-functions/e/evaluating-piecewise-functions}
{{\tt Atividade 38 - Cálculo de Funções Definidas por Partes}}

%\href{https://pt.khanacademy.org/math/algebra-home/alg-functions/alg-piecewise-functions/e/evaluate-step-functions-from-their-graph}
%{{\tt Atividade 25 - Cálculo de Funções Escalonadas}}


\end{frame}
	\section{Exercícios}
\begin{frame}
\frametitle{Exercícios} 

    \setcounter{exercicios}{12}

	\begin{exercicio}
		Considere os pontos $A = (x_1, y_1)$ e $B = (x_2, y_2)$ distintos e pertencentes a um plano cartesiano. Responda o que se pede:
		\begin{enumerate}[a)]
			\item Qual as equações paramétricas da reta que passa por $A$ e $B$?
			\item Mostre que o ponto $M = \paren{\dfrac{x_1+x_2} 2 , \dfrac{y_1+y_2} 2 }$ pertence à reta que passa por $A$ e $B$;
			\item Mostre que  $d(A, M) = d(M,B)$ e conclua que $M$ é o ponto médio do segmento $AB$.
		\end{enumerate}
	\end{exercicio}

	\begin{exercicio}
		Mostre que $f : (- \infty ; -4] \to \R$, tal que $f(x) = -x^2 - 8x -12$, é uma função crescente.
	\end{exercicio}

	\begin{exercicio}
		Seja a função $f:[3;5]\to\reals$ tal que $f(x)=-x^2+4x-3$.
		\begin{enumerate}[a)]
		  \item Mostre que $f$ é decrescente.
		  \item $f$ possui máximo absoluto? Se sim, ocorre em qual ponto?
		  \item $f$ possui mínimo absoluto? Se sim, ocorre em qual ponto?
		\end{enumerate}
	  \end{exercicio}

	\end{frame}


	%------------------------------------------------------------------------------------------------------------
	
	\begin{frame}
	\frametitle{Exercícios} 

	  \begin{exercicio}
		  Considere a função $f: \reals_- \to \reals^\ast_+$ tal que $f(x) = \dfrac{1}{1+x^2}$. Responda as seguintes perguntas apresentando as respectivas justificativas.
		  \begin{enumerate}[a)]
		  \item $f$ é monótona? Se sim, de que tipo? Se não, $f$ possui algum intervalo de monotonicidade?
		  \item $f$ possui máximo absoluto?
		  \item $f$ possui mínimo absoluto?
		  \item $f$ é limitada?
		  \end{enumerate}
	  \end{exercicio}
	  
	  \begin{exercicio}
		Considere a função real $f$ tal que $f(x) = -x^2 +2x +8$.
	  \begin{enumerate}[a)]
	  \item Mostre que $f$ é crescente no intervalo $( - \infty , 1]$;
	  \item Mostre que $f$ é decrescente no intervalo $[1, + \infty )$;
	  \item Use os itens anteriores para concluir que $1 \in \mathbb R$ é um ponto de máximo absoluto de $f$.
	  \end{enumerate}  
	  \end{exercicio}

	\end{frame}


	%------------------------------------------------------------------------------------------------------------
	
	\begin{frame}
	\frametitle{Exercícios} 

	\Ex{
	Considere a função $f: (- \infty , 2] \to \R$ tal que $f(x) = |x - 2| + 3$.
	\begin{enumerate}[a)]
		\item  $f$ é monótona de que tipo?
		\item  Qual dos extremos absolutos $f$ não possui?
	\end{enumerate}
}

	\Ex{
		Considere as funções $f: \R \to \R_+$ tal que $f(x) = x^2+3$ e $g: (-\infty ; 5] \to \R$ tal que $g(x) = \sqrt{x^2 - 10x +27}$. Faça o que se pede:
            \begin{enumerate}[a)]
                \item Calcule $(f \circ g)$ e $(g \circ f)$. Caso não seja possível, justifique;
                \item  Verifique se alguma das funções compostas que você calculou no primeiro item é monótona (crescente ou decrescente);
                \item  Verifique se alguma das funções compostas que você calculou no primeiro item possui máximo ou mínimo absoluto (escolha só um).
            \end{enumerate}
	}

	\Ex{
		Verifique os exercícios do capítulo que tem as leis de formação das funções iguais. Considere como se fosse um só exercício e tente refazer cada item usando outra informação dada ou pedida pelo exercício.
	}

\end{frame}


%------------------------------------------------------------------------------------------------------------

\begin{frame}
\frametitle{Exercícios} 

    \Ex{Sejam $f: \R \to \R $. Determine se as afirmações abaixo são
verdadeiras ou falsas, justificando suas respostas. As funções que
forem usadas como contraexemplo podem ser exibidas somente com o
esboço de seu gráfico.
\begin{enumerate}[(a)]
	\item Se $f$ é limitada superiormente, então $f$ tem pelo menos um máximo absoluto;
	\item Se $f$ é limitada superiormente, então $f$ tem pelo menos um máximo local;
	\item Se $f$ tem um máximo local, então $f$ tem um máximo absoluto;
	\item Todo máximo local de $f$ é máximo absoluto;
	\item Todo máximo absoluto de $f$ é máximo local;
	\item Se $x_0$ é o ponto de extremo local de $f$, então é ponto de
	extremo local de $f^2$, onde $(f^2)(x) = f(x) \cdot f(x)$;
	\item Se $x_0$ é o ponto de extremo local de $f^2$, então é ponto de
	extremo local de $f$.
\end{enumerate}}
\end{frame}


%------------------------------------------------------------------------------------------------------------

\begin{frame}
\frametitle{Exercícios} 

\begin{exercicio}
	Seja $f: \N \to \R $ e $g: \R \to \N$. Determine se as afirmações abaixo são
	verdadeiras ou falsas, justificando suas respostas. As funções que
	forem usadas como contraexemplo podem ser exibidas somente com o
	esboço de seu gráfico.
	\begin{enumerate}[a)]
	  \item A função $g$ pode ser ilimitada inferiormente;
	  \item $f$ é limitada superiormente ou $f$ é limitada inferiormente.
	\end{enumerate}
	\end{exercicio}

\Ex{Sejam $f: \R \to \R $ e $g: \R \to \R$. Determine se as
afirmações abaixo são verdadeiras ou falsas, justificando suas
respostas. As funções que forem usadas como contraexemplo podem ser
exibidas somente com o esboço de seu gráfico.
\begin{enumerate}[(a)]
	\item Se $f$ e $g$ são crescentes, então a composta $f \circ g$ é uma função crescente;
	\item Se $f$ e $g$ são crescentes, então o produto $f\cdot g$ é
	uma função crescente, onde $(f \cdot g)(x) = f(x) \cdot g(x)$;
	\item Se $f$ é crescente em $A \contido \R$ e em $B \contido \R$, então $f$ é crescente em $A \uniao B \contido \R$.
\end{enumerate}}

\end{frame}


%------------------------------------------------------------------------------------------------------------

\begin{frame}
\frametitle{Exercícios} 

\begin{exercicio}
	Seja $f$ uma função real. 
	\begin{enumerate}[a)]
		\item Suponha que $f$ é constante. Mostre que $f$ é não crescente e não decrescente;
		\item Suponha que $f$ é não crescente e não decrescente. Mostre que $f$ é constante.
	\end{enumerate}
%	Mostre que uma função real é constante se, e somente se, é não decrescente e não crescente.
  \end{exercicio}
  
  \begin{exercicio}
	Sejam $f : \reals \to \reals$ e $A$ e $B$ intervalos reais tais que $A \inter B$ é um intervalo não
	degenerado, ou seja, que possui pelo menos dois números. Mostre que, se $f$ é crescente
	em $A$ e em $B$, então $f$ é crescente em $A\inter B$.
  \end{exercicio}

\Ex{Mostre que a função inversa de uma função crescente é também uma
função crescente. E a função inversa de uma função decrescente é
decrescente.}

\end{frame}


%------------------------------------------------------------------------------------------------------------

\begin{frame}
\frametitle{Exercícios} 

\Ex{
	Dizemos que uma função $f: \R \to \R$ é \emph{par} quando se tem $f(-t) = f(t)$ para todo $t \in \R$. Se for o caso de $f(-t) = -f(t)$ para todo $t \in \R$, dizemos que $f$ é \emph{ímpar}.

  Considere a função real $f: \R \to \R$. Demonstre, ou refute com um contraexemplo, as afirmações abaixo:
  \begin{enumerate}[a)]
    \item Se $f$ é par e $x_0 \in \R$ é um ponto de máximo absoluto, então $-x_0 \in \R$ é também um ponto de máximo absoluto;
    \item Se $f$ é ímpar e $x_0 \in \R$ é um ponto de mínimo absoluto, então $-x_0 \in \R$ é um ponto de máximo absoluto;
	\item Se $f$ é par e limitada superiormente, então $f$ é limitada inferiormente;
    \item Se $f$ é ímpar e limitada superiormente, então $f$ é limitada inferiormente.
  \end{enumerate}
}

\end{frame}

%------------------------------------------------------------------------------------------------------------
	\section{Bibliografia}


\begin{frame}
    \frametitle{Bibliografia}

    \begin{thebibliography}{99}
        \bibitem {label1}
        LIMA, Ronaldo F.
        \newblock \emph{Álgebra Linear Essencial}.
        \newblock Acesso em \link{https://www.ronaldofreiredelima.com}{www.ronaldofreiredelima.com}


       \bibitem {label2}
       BOLDRINI, José L. (et al). 
        \newblock \emph{Álgebra linear}.
        \newblock 3. ed. São Paulo, SP: Harbra, 1986.
    \end{thebibliography}
\end{frame}


\section{Gráfico de Função Real}
\begin{frame} \frametitle{Função Real}




\end{frame}

%------------------------------------------------------------------------------------------------------------

\begin{frame}
\frametitle{Gráfico de Função Real} 

\begin{definicao}
O \sub{gráfico} de uma função real é o seguinte subconjunto do plano
cartesiano $\R^2$: $$G(f) = \set{(x,y) \in \R^2 \tq x \in D , y =
f(x)}.$$
\end{definicao}
Em outras palavras, o gráfico de uma função $f$ é o lugar geométrico
dos pontos cujas coordenadas satisfazem sua lei de associação.

\end{frame}



%------------------------------------------------------------------------------------------------------------

\begin{frame}
\frametitle{Gráfico de Função Real} 

\begin{exemplo}
Esboce o gráfico da função real
$$\begin{array}{cccl}
f : & \R^\ast & \to     & \R \\
		 &  x & \mapsto & \begin{cases}
												+1,  &  \ \text{ se } x >0 \\
												-1, &  \ \text{ se } x <0
												\end{cases}
\end{array}.$$
\end{exemplo}

\end{frame}


%------------------------------------------------------------------------------------------------------------
\section{Gráficos e Transformações no Plano}
\begin{frame}
\frametitle{Translações de gráficos} 

\begin{exemplo}
Compare os gráficos das funções reais $f, g , h: \R \to \R$ tais que
$f(x) = \sen x$, \\ $g(x) = f(x) + 1 = \sen x +1$ , \\ $h(x)=
f(x+\frac {\pi} 2)= \sen (x+ \frac {\pi} 2)$.
\end{exemplo}\pause

Dessa forma, se a função real $g$ é tal que $g(x) = f(x+b) +a$,
então o gráfico de $g$ pode ser obtido, do gráfico de $f$, através
de uma translação horizontal determinada pelo parâmetro $b$, e uma
translação vertical determinada pelo parâmetro $a$. \pause
\begin{itemize}
	\item O translado vertical será:
				\begin{itemize}
					\item No sentido positivo do eixo $y$ (para cima), se
					$a>0$;
					\item No sentido negativo do eixo $y$ (para baixo), se
					$a<0$.
				\end{itemize} \pause
	\item O translado horizontal será:
				\begin{itemize}
					\item No sentido positivo do eixo $x$ (para a direita), se $b<0$;
					\item No sentido negativo do eixo $x$ (para a esquerda), se $b>0$.
				\end{itemize}
\end{itemize}


\end{frame}

%------------------------------------------------------------------------------------------------------------

\begin{frame}
\frametitle{Dilatações de gráficos} 

\begin{exemplo}
Compare os gráficos das funções reais $f, g , h: \R \to \R$ tais que
$f(x) = \sen x$, \\ $g(x) = \frac 1 2 \cdot f(x)  = \frac 1 2 \cdot \sen x $, \\
$h(x)= f(2 \cdot x)= \sen (2 \cdot x)$.
\end{exemplo}\pause

\begin{exemplo}
Compare os gráficos das funções reais $f, g , h: \R \to \R$ tais que
$f(x) = \sen x$, \\ $g(x) = -1 \cdot f(x)  = -1 \cdot \sen x $ , \\
$h(x)= f(-1 \cdot x)= \sen (-1 \cdot x)$.
\end{exemplo}

\end{frame}

%------------------------------------------------------------------------------------------------------------

\begin{frame}
\frametitle{Dilatações de gráficos} 

Dessa forma, se a função real $g$ é tal que $g(x) = c \cdot f(d
\cdot x)$, então o gráfico de $g$ pode ser obtido, do gráfico de
$f$, através de uma dilatação horizontal determinada pelo parâmetro
$d$, e uma dilatação vertical determinada pelo parâmetro $c$. Se o
parâmetro for negativo, haverá, também, uma reflexão. \pause
\begin{itemize}
	\item A dilatação vertical será:
				\begin{itemize}
					\item Um esticamento se $c>1$;
					\item Um encolhimento se $0<c<1$;
					\item Um esticamento composto com reflexão em relação ao eixo $x$ se $c<-1$;
					\item Um encolhimento composto com reflexão em relação ao eixo $x$ se
					$-1<c<0$.
				\end{itemize} \pause
	\item A dilatação horizontal será:
				\begin{itemize}
					\item Um encolhimento se $d>1$;
					\item Um esticamento se $0<d<1$;
					\item Um encolhimento composto com reflexão em relação ao eixo $y$ se $d<-1$;
					\item Um esticamento composto com reflexão em relação ao eixo $y$ se
					$-1<d<0$.
				\end{itemize}
\end{itemize}


\end{frame}

%------------------------------------------------------------------------------------------------------------
\section{Atividade Online}
\begin{frame}
\frametitle{Atividade Online} 

\href{https://pt.khanacademy.org/math/algebra2/manipulating-functions/shifting-functions/e/shift-functions}
{{\tt Atividade 10 - Deslocamento de Funções}}

\href{https://pt.khanacademy.org/math/algebra2/manipulating-functions/stretching-functions/e/shifting_and_reflecting_functions}
{{\tt Atividade 11 - Como Transformar Funções}}


Veja o desempenho na Missão Álgebra II.


\end{frame}

%------------------------------------------------------------------------------------------------------------


\section{Crescimento e Pontos de Extremo}
\begin{frame}
\frametitle{Funções Monótonas} 

\begin{definicao}\label{funcmon}
Seja $f: D \contido \R \to \R$ uma função. Dizemos que
\begin{enumerate}[(i)]
	\item $f$ é \sub{monótona (estritamente) crescente} se, para todos $x_1, x_2 \in D$,
	$$x_1 < x_2 \implica f(x_1) < f(x_2);$$
	\item $f$ é \sub{monótona não decrescente} se, para todos $x_1, x_2 \in D$,
	$$x_1 < x_2 \implica f(x_1) \leq f(x_2);$$
	\item $f$ é \sub{monótona (estritamente) decrescente} se, para todos $x_1, x_2 \in D$,
	$$x_1 < x_2 \implica f(x_1) > f(x_2);$$
	\item $f$ é \sub{monótona não crescente} se, para todos $x_1, x_2 \in D$,
	$$x_1 < x_2 \implica f(x_1) \geq f(x_2).$$%
%  \item $f$ é \sub{constante} se, para todo $x \in D$, temos $f(x) =
%  k \in \R$.
\end{enumerate}
\end{definicao}

\end{frame}



%------------------------------------------------------------------------------------------------------------
\begin{frame}
\frametitle{Funções Monótonas} 



Nas mesmas condições da Definição \ref{funcmon} , se $f(x) = k \in
\R$ para todo $x \in D$, dizemos que $f$ é \sub{constante}.\\ \pause
Se $I \contido D$ é um intervalo, definimos a monotonicidade de $f$
no intervalo $I$ de maneira análoga ao feito anteriormente. Por
exemplo: \\
$f$ é \sub{monótona (estritamente) crescente em $I$} se, para todos
$x_1, x_2 \in I$,
	$$x_1 < x_2 \implica f(x_1) < f(x_2).$$


\end{frame}



%------------------------------------------------------------------------------------------------------------

\begin{frame}
\frametitle{Funções Limitadas} 

\begin{definicao}
Seja $f: D \contido \R \to \R$ uma função.
\begin{enumerate}[(i)]
	\item $f$ é \sub{limitada superiormente} se existe $M \in \R$ tal
	que $f(x) \leq M$, para todo $x \in D$;
	\item $f$ é \sub{limitada inferiormente} se existe $M \in \R$ tal
	que $f(x) \geq M$, para todo $x \in D$;
	\item $x_0 \in D$ é um \sub{ponto de máximo absoluto} de $f$ se
	$f(x_0) \geq f(x)$, para todo $x \in D$;
	\item $x_0 \in D$ é um \sub{ponto de mínimo absoluto} de $f$ se
	$f(x_0) \leq f(x)$, para todo $x \in D$;
	\item $x_0 \in D$ é um \sub{ponto de máximo local} de $f$ se
	existe $r>0$ tal que $f(x_0) \geq f(x)$, para todo $x \in D \inter \paren{x_0 - r , x_0+r}$;
	\item $x_0 \in D$ é um \sub{ponto de mínimo local} de $f$ se
	existe $r>0$ tal que $f(x_0) \leq f(x)$, para todo $x \in D \inter \paren{x_0 - r ,
	x_0+r}$.
\end{enumerate}
\end{definicao}

\end{frame}



%------------------------------------------------------------------------------------------------------------


\begin{frame}
\frametitle{Exemplo} 
\begin{exemplo}
A função $h : \left( -1 ; 6 \right] \to \R$, cujo gráfico é esboçado
abaixo, é definida por $h(x) = \begin{cases}
																3x-x^2 & \ \text{ se } \ x\leq 2 \\
																\modu{x-4} +1 & \ \text{ se } \ 2 < x \leq 5 \\
																2 & \ \text{ se } \ x > 5 \\
																\end{cases}.$

\begin{center}
\includegraphics[width=3.8cm]{figures/funch.jpg}
\end{center}

Determine os intervalos de monotonicidade e os extremos de $h$.
\end{exemplo}
\end{frame}

%------------------------------------------------------------------------------------------------------------
\section{Atividade Online}
\begin{frame}
\frametitle{Atividade Online} 

\href{https://pt.khanacademy.org/math/algebra/algebra-functions/positive-negative-increasing-decreasing-intervals/e/increasing-decreasing-intervals-of-functions}
{{\tt Atividade 12 - Intervalos Crescentes e Decrescentes}}

\href{https://pt.khanacademy.org/math/algebra/algebra-functions/maximum-and-minimum-points/e/recognize-maxima-and-minima}
{{\tt Atividade 13 - Mínimos e Máximos Relativos}}

\href{https://pt.khanacademy.org/math/algebra/algebra-functions/maximum-and-minimum-points/e/recognize-absolute-maxima-and-minima}
{{\tt Atividade 14 - Mínimos e Máximos Absolutos}}

Veja o desempenho na Missão Álgebra I.


\end{frame}


%------------------------------------------------------------------------------------------------------------

\section{Exercícios}
\begin{frame}
\frametitle{Exercícios} 
\Ex{Considere a função $g: [0 ; 5] \to \R$ definida por: $$g(x) =
																\begin{cases}
																4x-x^2 & \ \text{ se } \ x< 3 \\
																x-2 & \ \text{ se } \  x \geq 3 \\
																\end{cases}.$$
Determine as soluções de:
\begin{enumerate}[(a)]
	\item $g(x) = -1$;
	\item $g(x) = 0$;
	\item $g(x) = 3$;
	\item $g(x) = 4$;
	\item $g(x) < 3$;
	\item $g(x) \geq 3$.
\end{enumerate} }
\end{frame}


%------------------------------------------------------------------------------------------------------------

\begin{frame}
\frametitle{Exercícios} 

\Ex{Sejam $f: \R \to \R $. Determine se as afirmações abaixo são
verdadeiras ou falsas, justificando suas respostas. As funções que
forem usadas como contraexemplo podem ser exibidas somente com o
esboço de seu gráfico.
\begin{enumerate}[(a)]
	\item Se $f$ é limitada superiormente, então $f$ tem pelo menos um máximo absoluto;
	\item Se $f$ é limitada superiormente, então $f$ tem pelo menos um máximo local;
	\item Se $f$ tem um máximo local, então $f$ tem um máximo absoluto;
	\item Todo máximo local de $f$ é máximo absoluto;
	\item Todo máximo absoluto de $f$ é máximo local;
	\item Se $x_0$ é o ponto de extremo local de $f$, então é ponto de
	extremo local de $f^2$, onde $(f^2)(x) = f(x) \cdot f(x)$;
	\item Se $x_0$ é o ponto de extremo local de $f^2$, então é ponto de
	extremo local de $f$.
\end{enumerate}}
\end{frame}


%------------------------------------------------------------------------------------------------------------

\begin{frame}
\frametitle{Exercícios} 

\Ex{Sejam $f: \R \to \R $ e $g: \R \to \R$. Determine se as
afirmações abaixo são verdadeiras ou falsas, justificando suas
respostas. As funções que forem usadas como contraexemplo podem ser
exibidas somente com o esboço de seu gráfico.
\begin{enumerate}[(a)]
	\item Se $f$ e $g$ são crescentes, então a composta $f \circ g$ é uma função crescente;
	\item Se $f$ e $g$ são crescentes, então o produto $f\cdot g$ é
	uma função crescente, onde $(f \cdot g)(x) = f(x) \cdot g(x)$;
	\item Se $f$ é crescente em $A \contido \R$ e em $B \contido \R$, então $f$ é crescente em $A \uniao B \contido \R$.
\end{enumerate}}

\Ex{Mostre que a função inversa de uma função crescente é também uma
função crescente. E a função inversa de uma função decrescente é
decrescente.}

\end{frame}

%------------------------------------------------------------------------------------------------------------

\section{Bibliografia}

\frame{
	\frametitle{Bibliografia}
	\begin{thebibliography}{99}
		\bibitem {label1}
		IEZZI, Gelson; et al.
		\newblock \emph{Fundamentos de Matemática Elementar. Vol. 1 - Conjuntos e Funções}.
		\newblock São Paulo: Editora Atual.
	\end{thebibliography}
}

\end{document}
