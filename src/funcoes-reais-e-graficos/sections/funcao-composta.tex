\section{Funções Compostas}
\begin{frame}
\frametitle{Composição de Funções} 

\begin{definicao}
Sejam $f: X \to Y$ e $g: U \to V$ duas funções, com $Y \contido U$.\\
A \sub{função composta de $g$ com $f$} é a função denotada por $g
\circ f$, com domínio em $X$ e contradomínio em $V$, que a cada
elemento $x \in X$ faz corresponder o elemento $v = \paren{g \circ
f}(x) = g(f(x)) \in V$. Isto é:
$$\begin{array}{cccccc}
g \circ f : & X & \to     & Y \contido U & \to & V \\
		 &  x & \mapsto & f(x) & \mapsto & g(f(x))
\end{array}.$$
\end{definicao}

\end{frame}

%------------------------------------------------------------------------------------------------------------

\begin{frame}
\frametitle{Composição de Funções} 

\begin{exemplo}
Seja $f: X \to Y$ uma função. Mostre que $f \circ \I{X} = f$ e $\I{Y}
\circ f = f$.
\end{exemplo}\pause

\begin{exemplo}
Qual função resulta da composição $p \circ q$?
\end{exemplo}
\end{frame}



%------------------------------------------------------------------------------------------------------------

\begin{frame}
\frametitle{Composição de Funções} 


\begin{proposicao}[Associatividade da composição de funções]
	Considere $f: X \to Y$, $g: U \to V$ e $h: A \to B$ funções, com $B \contido U$ e $V \contido X$. Vale a seguinte igualdade:
	$$f \composta (g \composta h) = (f \composta g) \composta h.$$
\end{proposicao}

\end{frame}

%------------------------------------------------------------------------------------------------------------