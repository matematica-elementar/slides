
\section{Exercícios}
\begin{frame}
\frametitle{Exercícios} 

\Ex{Em cada um dos itens abaixo, defina uma função com a lei de
formação dada (indicando domínio e contradomínio). Verifique se é
injetiva, sobrejetiva ou bijetiva, a função
\begin{enumerate}[(a)]
	\item Que a cada ponto do plano cartesiano associa a distância desse ponto à origem do plano;
	\item Que a cada dois números naturais associa seu mdc;
%  \item Que a cada vetor do plano associa seu módulo;
%  \item Que a cada matriz $2 \times 2$ associa sua matriz
%  transposta;
%  \item Que a cada matriz $2 \times 2$ associa seu determinante;
	\item Que a cada polinômio (não nulo) com coeficientes reais
	associa seu grau;
	\item Que a cada figura plana fechada e limitada associa a sua
	área;
	\item Que a cada subconjunto de $\R$ associa seu complementar;
	\item Que a cada subconjunto finito de $\N$ associa seu número de
	elementos;
	\item Que a cada subconjunto não vazio de $\N$ associa seu menor
	elemento.
%	\item Que a cada função $f: \R \to \R$ associa seu valor no ponto $x_0 = 0$.
\end{enumerate}}



\end{frame}


%------------------------------------------------------------------------------------------------------------

\begin{frame}
\frametitle{Exercícios} 

\Ex{Considere a  função $f:  \N^\ast  \to      \Z $ tal que $$f(n) =
\begin{dcases} \dfrac {-n} 2, &\text{ se $n$ é par} \\ \dfrac {n-1} 2, &\text{ se $n$ é
ímpar} \end{dcases}.$$ Mostre que $f$ é bijetiva. %O que você pode concluir com esse resultado?
}

%\Ex{Mostre que a função inversa de $f: X \to Y$, caso exista, é única, isto é, se existem $g_1 : Y \to X$ e $g_2 : Y \to X$ satisfazendo a Definição \ref{funinv}, então $g_1 = g_2$.\\
%\emph{Dica: } Lembre-se que duas funções são iguais se, e só se, possuem mesmos domínios, contradomínios e seus valores são iguais em todos os elementos do domínio. Assim, procure mostrar que $g_1 (y) = g_2 (y)$, para todo $y \in Y$.}


\Ex{
	Considere a função $f : (0,1) \to \mathbb R$ tal que
$$f(x) =
\begin{dcases} 
  \dfrac 1 x - 2, & \text{ se $x \leq \dfrac 1 2$} \\
  2 - \dfrac 1 {1 - x} , & \text{ se $x > \dfrac 1 2$}
\end{dcases}.$$
Mostre que $f$ é bijetiva.
}

\end{frame}

%------------------------------------------------------------------------------------------------------------

\begin{frame}
\frametitle{Exercícios} 

\Ex{
    Considere a função $f: \R^\ast \to \R^\ast_+$ tal que $f(x) = \dfrac{1}{1+x^2}$. Responda as seguintes perguntas apresentando as respectivas justificativas.
    \begin{enumerate}[a)]
        \item $f$ é injetiva?
	\item $f$ é sobrejetiva?
    \end{enumerate}
}

\Ex{
	Considere as funções reais $f: X \to Y$ e $g: Y \to Z$. Demonstre, ou refute com um contraexemplo, as afirmações abaixo:
\begin{enumerate}[a)]
\item Se $f$ e $g$ são injetivas, então $(g \circ f)$ é injetiva;
\item Se $(g \circ f)$ é injetiva então $f$ e $g$ são injetivas;
\item Se $f$ e $g$ são sobrejetivas, então $(g \circ f)$ é sobrejetiva;
\item Se $(g \circ f)$ é sobrejetiva então $f$ e g são sobrejetivas.
\end{enumerate}
}

\end{frame}

%------------------------------------------------------------------------------------------------------------

\begin{frame}
\frametitle{Exercícios} 

\Ex{Seja $f: X \to Y$ uma função e seja $A$ um subconjunto de $X$.
Define-se $$f(A) = \set{f(x) \tq x\in A} \contido Y.$$ Se $A$ e $B$
são subconjuntos de $X$:
\begin{enumerate}[(a)]
	\item Mostre que $f(A \uniao B) = f(A) \uniao f(B)$;
	\item Mostre que $f(A \inter B) \contido f(A) \inter f(B)$;
	\item É possível afirmar que $f(A \inter B) = f(A) \inter f(B)$ para
	todos $A, B \contido X$? Justifique.
	\item Determine que condições deve satisfazer $f$ para que a
	afirmação feita no item (c) seja verdadeira.
\end{enumerate}
}

\end{frame}



%------------------------------------------------------------------------------------------------------------

\begin{frame}
\frametitle{Exercícios} 

\Ex{Seja $f: X \to Y$ uma função. Dado $y \in Y$, definimos a
\sub{contraimagem} ou \sub{imagem inversa} de $y$ como sendo o
seguinte subconjunto de $X$: $$ f^{-1}(y) = \set{x \in X \tq
f(x)=y}.$$
\begin{enumerate}[(a)]
	\item Se $f$ é injetiva e $y$ é um elemento qualquer de $Y$, o que
	se pode afirmar sobre a imagem inversa $f^{-1}(y)$?
	\item Se $f$ é sobrejetiva e $y$ é um elemento qualquer de $Y$, o que
	se pode afirmar sobre a imagem inversa $f^{-1}(y)$?
	\item Se $f$ é bijetiva e $y$ é um elemento qualquer de $Y$, o que
	se pode afirmar sobre a imagem inversa $f^{-1}(y)$?
\end{enumerate} }





\end{frame}



%------------------------------------------------------------------------------------------------------------

\begin{frame}
\frametitle{Exercícios} 

\Ex{Seja $f: X \to Y$ uma função. Dado $A \contido Y$, definimos a
\sub{contraimagem} ou \sub{imagem inversa} de $A$ como sendo o
seguinte subconjunto de $X$: $$ f^{-1}(A) = \set{x \in X \tq f(x)\in
A}.$$ Mostre que
\begin{enumerate}[(a)]
	\item $f^{-1} (A \uniao B) = f^{-1} (A) \uniao f^{-1} (B)$;
	\item $f^{-1} (A \inter B) = f^{-1} (A) \inter f^{-1} (B)$.
\end{enumerate}}




\end{frame}



%------------------------------------------------------------------------------------------------------------

\begin{frame}
\frametitle{Exercícios - Desafios} 

\Ex{Seja $f: X \to Y$ uma função. Mostre que, se existem $g_1 : Y
\to X$ e $g_2 : Y \to X$ tais que $f \circ g_1 = \I{Y}$ e $g_2 \circ
f = \I{X}$, então $g_1 = g_2$ (portanto, neste caso, $f$ será
invertível).}

\Ex{Seja $f: X \to Y$ uma função. Mostre que
\begin{enumerate}[(a)]
	\item $f(f^{-1}(B)) \contido B$, para todo $B \contido Y$;
	\item $f(f^{-1}(B)) = B$, para todo $B \contido Y$ se, e
	somente se, $f$ é sobrejetiva.
\end{enumerate}}

\Ex{Seja $f: X \to Y$ uma função. Mostre que
\begin{enumerate}[(a)]
	\item $f(f^{-1}(A)) \supset A$, para todo $A \contido X$;
	\item $f(f^{-1}(A)) = A$, para todo $A \contido X$ se, e
	somente se, $f$ é injetiva.
\end{enumerate}}

\Ex{Mostre que existe uma injeção $f: X \to Y$ se, e somente se,
existe uma sobrejeção $g: Y \to X$.}

\end{frame}
