\section{Fórmulas e Funções}
\begin{frame}
\frametitle{Fórmulas e Funções} 
É muito importante não pensar que uma função é uma fórmula.
Considere as funções
$$\begin{array}{cccc}
p_1 : & \R & \to     & \R \\
		 &  x & \mapsto & x^2
\end{array}
\text{\ \ \  e \ \ \ }
\begin{array}{cccc}
p_2 : & \R_+ & \to     & \R_+ \\
		 &  x & \mapsto &  x^2
\end{array}.$$
Essas funções são iguais? \\ \pause NÃO! Note que $p_2$ é bijetiva e
$p_1$ não é, mesmo tendo a mesma fórmula.

Além disso, funções podem ser definidas por mais de uma fórmula,
como na função $h :  \R  \to      \R$ tal que
		 $$h(x) =  \begin{cases}
						0, &  \ \text{ se } x \in \R \setminus \Q \\
						1, & \ \text{ se } x \in \Q
						\end{cases} .$$
\end{frame}

