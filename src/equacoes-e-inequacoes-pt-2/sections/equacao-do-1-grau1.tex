
\section{Equação do 1º grau}
\frame { \frametitle{Equação do 1º grau}
\begin{definicao}
Uma \sub{equação do primeiro grau} na variável $x$ é uma expressão
da forma $$ax+b=0,$$ onde $a,b \in \R$, $a \neq 0$ e $x$ é um número
real a ser encontrado.
\end{definicao}

}


%------------------------------------------------------------------------------------------------------------

\begin{frame}
\frametitle{Equação do 1º grau} 

\begin{proposicao}[Propriedades]
	Sejam $a, b, c \in \R$. Os seguintes valem:
	\begin{enumerate}[i.]
		\item $a=b \implica a+c = b+c$;
		\item $a=b \implica ac = bc$.
	\end{enumerate}
\end{proposicao}\pause


\begin{exemplo}
Resolva a equação $5x - 3 = 6$.
\end{exemplo}

\end{frame}



%------------------------------------------------------------------------------------------------------------

\begin{frame}
	\frametitle{Equação do 1º grau}

\begin{exemplo}
	Escreva em forma de expressões cada passo da brincadeira da
	Introdução e mostre porque ela sempre funciona.
	\begin{enumerate}
		\item Escolha um número;
		\item Multiplique esse número por 6;
		\item Some 12;
		\item Divida por 3;
		\item Subtraia o dobro do número que você escolheu;
		\item O resultado é igual a 4.
	\end{enumerate}
	\end{exemplo}

\end{frame}

%------------------------------------------------------------------------------------------------------------

\begin{frame}
\frametitle{Equação do 1º grau} 


\begin{exemplo}
	Analise as implicações abaixo:
	\begin{align*}
		x^2+1=0 & \implica (x^2+1)(x^2-1) = 0\vezes(x^2-1) \\
				& \implica x^4-1 = 0 \\
				& \implica x^4 = 1 \\
				& \implica x \pertence \conjunto{-1, 1}
	\end{align*}

	Isso quer dizer que o conjunto solução de $x^2 +1 = 0$ é $\conjunto{-1, 1}$?
\end{exemplo}\pause


\begin{block}{Observação}
Muito cuidado ao efetuar divisões em ambos os lados de uma equação
para não cometer o erro de dividir os lados por zero. Veja como ``podemos provar'' que $1 = 2$.
\end{block}

\end{frame}


%------------------------------------------------------------------------------------------------------------

