
\section{Atividade Online}
\begin{frame}
\frametitle{Atividade Online} 

\link{https://pt.khanacademy.org/math/algebra/quadratics/quadratics-square-root/e/quadratics-by-taking-square-roots-with-steps}
{Atividade Online 07 - Equações do segundo grau com cálculo de
raízes quadradas: com etapas}

\link{https://pt.khanacademy.org/math/algebra/quadratics/solving-quadratics-by-completing-the-square/e/completing_the_square_2}
{Atividade Online 08 - Método de completar quadrados}

Veja o desempenho na Missão Álgebra I -- Equações do segundo grau

\end{frame}
%------------------------------------------------------------------------------------------------------------

\begin{frame}
\frametitle{Equação do 2º grau} 

\begin{definicao}
Chamamos de \sub{discriminante} da equação do segundo grau a
expressão $b^2-4ac$ e denotamos pela letra grega maiúscula $\Delta$
(lê-se delta).
\end{definicao}

Em resumo:
\begin{itemize}
	\item Se $\Delta > 0$, existem duas soluções reais;
	\item Se $\Delta = 0 $, existe uma solução real ($x_1 = x_2 =
	-b/2a)$;
	\item Se $\Delta < 0$, não existe solução real.
\end{itemize}

\end{frame}



%------------------------------------------------------------------------------------------------------------

\begin{frame}
\frametitle{Equação do 2º grau} 

\begin{exemplo}
Sabendo que $x$ é um número real que satisfaz $$x = 1 + \frac 1 {1 +
\frac 1 x},$$ determine os valores possíveis de $x$.
\end{exemplo}
\pause
\begin{block}{Observação}
O número $\phi = \frac{\paren {1+\sqrt 5}}2$ é conhecido como razão
áurea, número de ouro, proporção divina, entre outras denominações.
Veja o episódio A Proporção Divina
\link{https://www.youtube.com/watch?v=mfL6-g5mQw4}{parte 01} e
\link{https://www.youtube.com/watch?v=xtsTXAwWF20&}{parte 02} do
programa português Isto É Matemática.
\end{block}

\end{frame}

%------------------------------------------------------------------------------------------------------------