\section{Sistemas de Equações do 1º grau}
\begin{frame}
    \frametitle{Sistemas de Equações do 1º grau} 
    
    
    \begin{definicao}[Equação do 1º grau em várias variáveis]
    Uma \sub{equação do primeiro grau nas varáveis $x_1$, $x_2$, \dots , $x_n$} é uma expressão da forma
    $$a_1x_1 + a_2x_2 + \cdots + a_nx_n +b = 0.$$
    \end{definicao}\pause

    Dizemos que os números $\paren{r_1, r_2, \dots , r_n}$ formam uma solução da equação, se substituindo $x_1$ por $r_1$, $x_2$ por $r_2$, \dots , $x_n$ por $r_n$, temos que a equação acima é satisfeita, isto é, $$a_1r_1 + a_2r_2 + \cdots + a_nr_n +b = 0.$$ \pause

    \begin{exemplo}
        $x + 3y - 2z +8 = 0$ é uma equação do primeiro grau nas variáveis $x$, $y$ e $z$.    Determine uma solução para essa equação.
    \end{exemplo}


\end{frame}
    
    
    
%------------------------------------------------------------------------------------------------------------

\begin{frame}
    \frametitle{Sistemas de Equações do 1º grau} 

    \begin{definicao}[Sistema de equações do 1º grau em várias variáveis]
        Um \sub{sistema de equações do primeiro grau em $n$ variáveis}  \sub{$x_1$, $x_2$, \dots , $x_n$} é um conjunto de $k$ equações do primeiro grau em ALGUMAS das variáveis $x_1$, $x_2$, \dots , $x_n$, isto é, tem-se o seguinte conjunto de equações
        $$\left\{
        \begin{array}{llll}
        a_{11}x_1 + a_{12}x_2 + \cdots + a_{1n}x_n +b_1 = 0 \\
        a_{21}x_1 + a_{22}x_2 + \cdots + a_{2n}x_n +b_2 = 0 \\
        (\dots) \\
        a_{k1}x_1 + a_{k2}x_2 + \cdots + a_{kn}x_n +b_k = 0 
        \end{array} \right.$$

        onde alguns dos elementos $a_{ij}$ $\paren{1 \leq i \leq k, \ 1 \leq j \leq n}$ podem ser zero. Porém, em cada uma das equações do sistema algum coeficiente $a_{ij}$ é diferente de zero e, além disso, cada variável $x_j$ aparece em alguma equação com coeficiente distinto de zero.
    \end{definicao}


\end{frame}
        
    
    %------------------------------------------------------------------------------------------------------------

    \begin{frame}
        \frametitle{Sistemas de Equações do 1º grau} 
    
        Dizemos que os números $\paren{r_1, r_2, \dots , r_n}$ formam uma solução do sistema de equações da definição anterior se  $\paren{r_1, r_2, \dots , r_n}$ é solução para todas as equações simultaneamente.

        Há três possibilidades quando se resolve um sistema de equações do primeiro grau:
        \begin{itemize}
            \item o sistema tem uma única solução;
            \item o sistema tem uma infinidade de soluções;
            \item o sistema não possui solução.
        \end{itemize}
    
    
    \end{frame}
        
        
        
%------------------------------------------------------------------------------------------------------------


\begin{frame}
    \frametitle{Sistemas de Equações do 1º grau} 
        
    \begin{exemplo}
        João possui 14 reais e deseja gastar esse dinheiro em chocolates e sanduíches para distribuir com seus 6 amigos, de modo que cada um fique exatamente com um chocolate ou um sanduíche. Sabendo que cada chocolate custa 2 reais e cada sanduíche custa 3 reais, quantos chocolates e sanduíches João deve comprar?
    \end{exemplo}\pause

    \begin{exemplo}
        Resolva o sistema nas variáveis $x$, $y$ e $z$ abaixo:
        $$\left\{
        \begin{array}{ll}
        x+y-z-1 = 0 \\
        x-y-1 = 0 
        \end{array} \right.$$
    \end{exemplo}
        
        
\end{frame}
            
            
            
%------------------------------------------------------------------------------------------------------------

\begin{frame}
    \frametitle{Sistemas de Equações do 1º grau} 
        
    
    \begin{exemplo}
        Resolva o sistema nas variáveis $x$, $y$ e $z$ abaixo:
        $$\left\{
        \begin{array}{lll}
        x+y+2z-1 = 0 \\
        x+z-2 = 0 \\
        y+z-3 = 0
        \end{array} \right.$$
    \end{exemplo}
        
        
\end{frame}
            
            
            
%------------------------------------------------------------------------------------------------------------