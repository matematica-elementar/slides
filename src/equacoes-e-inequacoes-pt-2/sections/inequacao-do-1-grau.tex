
\section{Inequação do 1º grau}
\begin{frame}
\frametitle{Inequação do 1º grau} 

\setcounter{teorema}{23}
\begin{definicao}
Uma \sub{inequação do primeiro grau} é uma relação de uma das formas
abaixo $$ax+b <0, \; \; ax+b>0,$$ $$ax+b \leq 0, \; \; ax+b \geq
0,$$ onde $a, b \in \R$ e $ a \neq 0$. Lemos os símbolos da seguinte
maneira: $<$ (menor que), $>$ (maior que), $\leq$ (menor ou igual
que) e $\geq$ (maior ou igual que).
\end{definicao}

O \sub{conjunto solução} de uma inequação do primeiro grau é o
conjunto $\mathcal{S}$ de números reais que satisfazem a inequação,
isto é, o conjunto de números que, quando substituídos na inequação,
tornam a desigualdade verdadeira.

\end{frame}


%------------------------------------------------------------------------------------------------------------
\begin{frame}
\frametitle{Inequação do 1º grau} 

\begin{proposicao}[Propriedades de inequações]
Sejam $a, b, c, d \in \R$; $n \in \N^*$. Valem:
\begin{enumerate}[i.]
	\item Invariância por adição de números reais: $a < b \implica a+c < b+c$;
	\item Invariância por multiplicação de números reais positivos:
	$a < b ; c>0 \implica a \cdot c < b \cdot c$;
	\item Mudança por multiplicação de números reais
	negativos: $a < b ; c<0 \implica a \cdot c > b \cdot c$;
	\item Se $a < b$, então $\frac 1 a > \frac 1 b$, para $a, b \neq 0$ e com mesmo sinal;
	\item Se $a \geq 0$, $b \geq 0$ e $c>0$, segue que $a < b \implica a^c < b^c$;
	\item Se $a< 0$, $b < 0$ e $n$ par, segue que $a < b \implica a^n > b^n$;
	\item Se $a<0$, $b < 0$ e $n$ ímpar, segue que $a < b \implica a^n <
	b^n$;
	\item Se $a< b$ e $c< d$, então $a+c < b+d$;
	\item Para $a, b, c, d \in \R_+$. Se $a< b$ e $c< d$, então $ac < bd$.
\end{enumerate}
O resultado é análogo para os tipos $>$, $\leq$ ou $\geq$.
\end{proposicao}


\end{frame}


%------------------------------------------------------------------------------------------------------------
\begin{frame}
\frametitle{Inequação do 1º grau} 

\begin{exemplo}
Qual o conjunto solução da inequação $8x - 4 \geq 0$?
\end{exemplo}\pause

\begin{exemplo}
Antes de fazer os cálculos, diga: qual dos números $a = 3456784
\cdot 3456786 + 3456785$ e $b = 3456785^2 - 3456788$ é maior?
\end{exemplo}

\end{frame}



%------------------------------------------------------------------------------------------------------------