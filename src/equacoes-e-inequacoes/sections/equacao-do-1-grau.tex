
\section{Equação do 1º grau}
\frame { \frametitle{Equação do 1º grau}
\begin{definicao}
Uma \sub{equação do primeiro grau} na variável $x$ é uma expressão
da forma $$ax+b=0,$$ onde $a,b \in \R$, $a \neq 0$ e $x$ é um número
real a ser encontrado.
\end{definicao}

\begin{proposicao}[Propriedades]
Sejam $a, b, c \in \R$. Os seguintes valem:
\begin{enumerate}[i.]
	\item $a=b \implies a+c = b+c$;
	\item $a=b \implies ac = bc$.
\end{enumerate}
\end{proposicao}
}

%------------------------------------------------------------------------------------------------------------

\begin{frame}
\frametitle{Equação do 1º grau} %\framesubtitle{Exemplos}
\begin{exemplo}
Resolva a equação $5x - 3 = 6$.
\end{exemplo}

\begin{exemplo}
Escreva em forma de expressões cada passo da brincadeira da
Introdução:
\begin{enumerate}
	\item Escolha um número;
	\item Multiplique esse número por 6;
	\item Some 12;
	\item Divida por 3;
	\item Subtraia o dobro do número que você escolheu;
	\item O resultado é igual a 4.
\end{enumerate}
\end{exemplo}

\end{frame}



%------------------------------------------------------------------------------------------------------------

\begin{frame}
\frametitle{Equação do 1º grau} %\framesubtitle{Exemplos}
\begin{block}{Observação}
Muito cuidado ao efetuar divisões em ambos os lados de uma equação
para não cometer o erro de dividir os lados por zero. Já vimos em
sala uma prova obviamente falsa que $1=2$, você lembra? Tente
fazê-la.
\end{block}

\end{frame}


%------------------------------------------------------------------------------------------------------------

