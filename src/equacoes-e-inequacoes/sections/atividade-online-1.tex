
\section{Atividade Online}
\begin{frame}
\frametitle{Atividade Online} %\framesubtitle{Exemplos}

\link{https://pt.khanacademy.org/math/cc-sixth-grade-math/cc-6th-equations-and-inequalities/cc-6th-super-yoga/e/model-with-one-step-equations-and-solve}
{Atividade Online 06 - Modelo com equações de primeiro grau e
resolução}



Veja o desempenho na Missão 7º ano -- Introdução às equações e
inequações

\end{frame}
%------------------------------------------------------------------------------------------------------------


\begin{frame}
\frametitle{Equação do 1º grau} %\framesubtitle{Exemplos}


\begin{Exem}
Se $x$ representa um dígito na base 10 na equação $$x11 + 11x + 1x1
= 777,$$ qual o valor de $x$?
\end{Exem}
\end{frame}



%------------------------------------------------------------------------------------------------------------

\begin{frame}
\frametitle{Equação do 1º grau} %\framesubtitle{Exemplos}

\begin{Exem}
Determine se é possível completar o preenchimento do tabuleiro
abaixo com os números naturais de 1 a 9, sem repetição, de modo que
a soma de qualquer linha seja igual à de qualquer coluna ou
diagonal.

\begin{center}
\begin{tabular}{|c|c|c|}
	\hline
	% after \\: \hline or \cline{col1-col2} \cline{col3-col4} ...
	1 &   & 6 \\ \hline
		&   &   \\ \hline
		& 9 &   \\
	\hline
\end{tabular}
\end{center}

\end{Exem}

Os tabuleiros preenchidos com essas propriedades são conhecidos como
\sub{quadrados mágicos}.
\end{frame}



%------------------------------------------------------------------------------------------------------------

\begin{frame}
\frametitle{Equação do 1º grau} %\framesubtitle{Exemplos}

\begin{Exem}
Imagine que você possui um fio de cobre extremamente longo, mas tão
longo que você consegue dar a volta na Terra com ele. Para
simplificar, considere que a Terra é uma bola redonda e que seu raio
é de exatamente 6.378.000 metros.

O fio com seus milhões de metros está ajustado à Terra, ficando bem
colado ao chão ao longo do Equador. Digamos, agora, que você
acrescente 1 metro ao fio e o molde de modo que ele forme um círculo
enorme, cujo raio é um pouco maior que o raio da Terra e tenha o
mesmo centro. Você acha que essa folga será de que tamanho?
\end{Exem}

\pause Já sabemos que a folga obtida aumentando o fio independe do
raio em consideração. Além desse problema, veja outras curiosidades
sobre o número $\pi$ no vídeo \link
{https://www.youtube.com/watch?v=evfc6bv6_lM}{O Pi existe} e tente
calculá-o em casa usando algum objeto redondo.


\end{frame}

%------------------------------------------------------------------------------------------------------------