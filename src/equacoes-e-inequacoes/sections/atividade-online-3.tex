
\section{Atividade Online}
\begin{frame}
\frametitle{Atividade Online} %\framesubtitle{Exemplos}

\link{https://pt.khanacademy.org/math/algebra/quadratics/solving-quadratics-using-the-quadratic-formula/e/quadratic_equation}
{Atividade Online 09 - Fórmula de Bhaskara}


Veja o desempenho na Missão Álgebra I -- Equações do segundo grau

\end{frame}
%------------------------------------------------------------------------------------------------------------

\begin{frame}
\frametitle{Equação do 2º grau} %\framesubtitle{Exemplos}

\begin{teorema}
Os números $\alpha$ e $\beta$ são as raízes da equação do segundo
grau $$ax^2+ bx+c=0$$ se, e somente se, $$\alpha + \beta = \frac
{-b} a \; \text{ e } \; \alpha \beta = \frac c a.$$
\end{teorema}

\end{frame}


%------------------------------------------------------------------------------------------------------------