
\section{Desigualdades clássicas}
\begin{frame}
\frametitle{Desigualdades clássicas} 

Para iniciar, apresentamos algumas desigualdades simples mas
famosas, válidas para quaisquer $a,b \in \R$:
\begin{itemize}
	\item $\modu a \geq 0$;
	\item $a^2 \geq 0$;
	\item $\modu {a+b} \leq \modu a + \modu b$ (desigualdade triangular).
\end{itemize}



\end{frame}
%------------------------------------------------------------------------------------------------------------
\begin{frame}
\frametitle{Desigualdades clássicas} 

\begin{teorema}
Para quaisquer $x, y \in \R$ vale
\begin{equation}
		xy \leq \frac {x^2 +y^2} 2.
\end{equation}
Além disso, a igualdade acontece se, e somente se, $x=y$.
\end{teorema}

Vejamos no quadro um experimento geométrico relacionado a essa
desigualdade.
\end{frame}
%------------------------------------------------------------------------------------------------------------
\begin{frame}
\frametitle{Desigualdades clássicas} 
\begin{teorema}
Para quaisquer $a, b \in \R_+$ vale
\begin{equation}
		\sqrt{ab} \leq \frac {a +b} 2.
\end{equation}
Além disso, a igualdade acontece se, e somente se, $a=b$.
\end{teorema}

\begin{teorema}[Desigualdade das médias aritmética e geométrica]
Para quaisquer $a_1, a_2, \dots , a_n \in \R_+$ vale
\begin{equation}
		\sqrt[n]{a_1\dots a_n} \leq \frac {a_1 + \dots + a_n} n.
\end{equation}
\end{teorema}

\end{frame}
%------------------------------------------------------------------------------------------------------------
\begin{frame}
\frametitle{Desigualdades clássicas} 
\begin{teorema}[Desigualdade das médias harmônica e geométrica]
Para quaisquer $a_1, a_2, \dots , a_n \in \R_+^*$ vale
\begin{equation}
		\frac n {\frac 1 {a_1} + \dots + \frac 1 {a_n}}  \leq \sqrt[n]{a_1\dots a_n}  .
\end{equation}
\end{teorema}

\begin{teorema}[Desigualdade de Cauchy-Schwarz]
Sejam $x_1, \dots , x_n, y_1, \dots y_n \in \R$, então vale:
\begin{equation}
		\modu{x_1y_1 + \dots + x_ny_n} \leq \sqrt{x^2_1+ \dots + x^2_n}
		\cdot \sqrt{y^2_1+ \dots + y^2_n}.
\end{equation}
Além disso, a igualdade só ocorre se existir um número real $\alpha$
tal que $x_1 = \alpha y_1$, ..., $x_n = \alpha y_n$.
\end{teorema}

\end{frame}
%------------------------------------------------------------------------------------------------------------
\begin{frame}
\frametitle{Aplicações} 
\begin{exemplo}
Duas torres são amarradas por uma corda $APB$ que vai do topo $A$ da
primeira torre para um ponto $P$ no chão, entre as torres, e então
até o topo $B$ da segunda torre. Qual a posição do ponto $P$ que nos
dá o comprimento mínimo da corda a ser utilizada?
\end{exemplo}
\end{frame}
%------------------------------------------------------------------------------------------------------------
\begin{frame}
\frametitle{Aplicações} 
\begin{exemplo}
Prove que, num triângulo retângulo, a altura relativa à hipotenusa é
sempre menor ou igual que a metade da hipotenusa. Prove, ainda, que
a igualdade só ocorre quando o triângulo retângulo é isósceles.
\end{exemplo}
\end{frame}
%------------------------------------------------------------------------------------------------------------
\begin{frame}
\frametitle{Aplicações} 
\begin{exemplo}
Prove que, entre todos os triângulos retângulos de catetos $a$ e
$b$, e com hipotenusa $c$ fixada, o que tem maior soma dos catetos
$S = a+b$ é o triângulo isósceles.
\end{exemplo}

\end{frame}
%------------------------------------------------------------------------------------------------------------
