\section{Plano Cartesiano}
\begin{frame}
\frametitle{O Plano Cartesiano}

\setcounter{teorema}{18}
\begin{definicao}
    Um \sub{sistema de coordenadas} (cartesianas) no plano $\Pi$ consiste num par de eixos perpendiculares $OX$ e $OY$ contidos nesse plano, com a mesma origem $O$. $OX$ chama-se o eixo das \sub{abcissas} e $OY$ é o eixo das \sub{ordenadas}. O sistema é indicado com a notação $OXY$. Um plano munido de um sistema de coordenadas cartesianas é chamado de \sub{cartesiano}.
\end{definicao}\pause

Fixado um sistema de coordenadas cartesianas $OXY$ num plano $\Pi$, cada ponto $P$ do plano possuirá um par ordenado $(x, y) \in \R^2$ associado e vice-versa. Dizemos que $x$ e $y$ são as \sub{coordenadas} do ponto $P$ e escrevemos $P = (x,y)$, onde $x$ é a \sub{abcissa} e $y$ é a \sub{ordenada} de $P$.\pause

\begin{exemplo}
    Represente os pontos $P_1 = (1,3)$, $P_2=(-2,0)$ e $P_3 = (0, -4)$ em um plano cartesiano.
\end{exemplo}


\end{frame}
%-------------------------------------------------------------------------

\begin{frame}
    \frametitle{O Plano Cartesiano}

    \begin{definicao}
        Os eixos ortogonais $OX$ e $OY$ decompõem o plano cartesiano em quatro regiões, cada uma das quais se chama um \sub{quadrante}. Dado um ponto $P= (x,y)$, dizemos que $P$ está no:
        \begin{itemize}
            \item \sub{primeiro quadrante}, se $x \geq 0$ e $y\geq 0$;
            \item \sub{segundo quadrante}, se $x \leq 0$ e $y\geq 0$;
            \item \sub{terceiro quadrante}, se $x \leq 0$ e $y\leq 0$;
            \item \sub{quarto quadrante}, se $x \geq 0$ e $y\leq 0$.
        \end{itemize}
    \end{definicao}\pause

    \begin{exemplo}
        Os pontos $P_1 = (1,3)$, $P_2=(-2,0)$ e $P_3 = (0, -4)$ estão em quais quadrantes?
    \end{exemplo}

\end{frame}
%-------------------------------------------------------------------------  

\begin{frame}
    \frametitle{O Plano Cartesiano}

    \begin{proposicao}[Distância entre dois pontos]
        Dados os pontos $P=(x,y)$ e $Q=(u,v)$, a distância entre $P$ e $Q$, $d(P,Q)$, é 
        $$d(P, Q) = \sqrt{(x-u)^2+(y-v)^2}.$$
    \end{proposicao}\pause

    \begin{exemplo}
        Dados os pontos $P_1 = (1,3)$, $P_2=(-2,0)$ e $P_3 = (0, -4)$, calcule $d(P_1,P_2)$ e $d(P_2, P_3)$.
    \end{exemplo}\pause

    \begin{exemplo}
        Qual a equação que determina os pontos $(x,y)$ pertencentes a uma circunferência centrada na origem $O$ e de raio $r$?
    \end{exemplo}

\end{frame}
%-------------------------------------------------------------------------  

\begin{frame}
    \frametitle{O Plano Cartesiano}

    \begin{exemplo}
        Dados os pontos distintos $A=(a,b)$ e $C=(c,d)$, as equações
        $$\begin{cases}
            x = a + t(c-a)\\
            y = b + t(d-b)
        \end{cases},$$
        onde $t\in \R$, chamam-se as \sub{equações paramétricas} da reta que passa pelos pontos $A$ e $C$.\\
        Quais as equações paramétricas da reta que passa pelos pontos $A=(-1,3)$ e $B=(0,2)$?
    \end{exemplo}\pause

    \begin{exemplo}
        Sejam $a,b, c \in \R$ tais que $a$ e $b$ não são ambos nulos. O conjunto de pontos $P= (x,y)$ cujas coordenadas satisfazem a equação $ax +by = c$ é uma reta.\\
        Esboce a reta determinada pela equação $4x+2y = -1$.
    \end{exemplo}

\end{frame}
%-------------------------------------------------------------------------  