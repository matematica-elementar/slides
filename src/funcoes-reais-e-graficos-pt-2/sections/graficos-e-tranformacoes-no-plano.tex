\section{Gráficos e Transformações no Plano}
\begin{frame}
\frametitle{Translações de gráficos} 

\begin{exemplo}
Compare os gráficos das funções reais $f, g , h: \R \to \R$ tais que
$f(x) = \sen x$, \\ $g(x) = f(x) + 1 = \sen x +1$ , \\ $h(x)=
f(x+\frac {\pi} 2)= \sen (x+ \frac {\pi} 2)$.
\end{exemplo}\pause

Dessa forma, se a função real $g$ é tal que $g(x) = f(x+b) +a$,
então o gráfico de $g$ pode ser obtido, do gráfico de $f$, através
de uma translação horizontal determinada pelo parâmetro $b$, e uma
translação vertical determinada pelo parâmetro $a$. \pause
\begin{itemize}
	\item O translado vertical será:
				\begin{itemize}
					\item No sentido positivo do eixo $y$ (para cima), se
					$a>0$;
					\item No sentido negativo do eixo $y$ (para baixo), se
					$a<0$.
				\end{itemize} \pause
	\item O translado horizontal será:
				\begin{itemize}
					\item No sentido positivo do eixo $x$ (para a direita), se $b<0$;
					\item No sentido negativo do eixo $x$ (para a esquerda), se $b>0$.
				\end{itemize}
\end{itemize}


\end{frame}

%------------------------------------------------------------------------------------------------------------

\begin{frame}
\frametitle{Dilatações de gráficos} 

\begin{exemplo}
Compare os gráficos das funções reais $f, g , h: \R \to \R$ tais que
$f(x) = \sen x$, \\ $g(x) = \frac 1 2 \cdot f(x)  = \frac 1 2 \cdot \sen x $, \\
$h(x)= f(2 \cdot x)= \sen (2 \cdot x)$.
\end{exemplo}\pause

\begin{exemplo}
Compare os gráficos das funções reais $f, g , h: \R \to \R$ tais que
$f(x) = \sen x$, \\ $g(x) = -1 \cdot f(x)  = -1 \cdot \sen x $ , \\
$h(x)= f(-1 \cdot x)= \sen (-1 \cdot x)$.
\end{exemplo}

\end{frame}

%------------------------------------------------------------------------------------------------------------

\begin{frame}
\frametitle{Dilatações de gráficos} 

Dessa forma, se a função real $g$ é tal que $g(x) = c \cdot f(d
\cdot x)$, então o gráfico de $g$ pode ser obtido, do gráfico de
$f$, através de uma dilatação horizontal determinada pelo parâmetro
$d$, e uma dilatação vertical determinada pelo parâmetro $c$. Se o
parâmetro for negativo, haverá, também, uma reflexão. \pause
\begin{itemize}
	\item A dilatação vertical será:
				\begin{itemize}
					\item Um esticamento se $c>1$;
					\item Um encolhimento se $0<c<1$;
					\item Um esticamento composto com reflexão em relação ao eixo $x$ se $c<-1$;
					\item Um encolhimento composto com reflexão em relação ao eixo $x$ se
					$-1<c<0$.
				\end{itemize} \pause
	\item A dilatação horizontal será:
				\begin{itemize}
					\item Um encolhimento se $d>1$;
					\item Um esticamento se $0<d<1$;
					\item Um encolhimento composto com reflexão em relação ao eixo $y$ se $d<-1$;
					\item Um esticamento composto com reflexão em relação ao eixo $y$ se
					$-1<d<0$.
				\end{itemize}
\end{itemize}


\end{frame}

%------------------------------------------------------------------------------------------------------------