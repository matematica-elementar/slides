\section{Exercícios}
\begin{frame}
\frametitle{Exercícios} 

    \setcounter{exercicios}{11}

	\begin{exercicio}
		Considere os pontos $A = (x_1, y_1)$ e $B = (x_2, y_2)$ distintos e pertencentes a um plano cartesiano. Responda o que se pede:
		\begin{enumerate}[a)]
			\item Qual as equações paramétricas da reta que passa por $A$ e $B$?
			\item Mostre que o ponto $M = \paren{\dfrac{x_1+x_2} 2 , \dfrac{y_1+y_2} 2 }$ pertence à reta que passa por $A$ e $B$;
			\item Mostre que  $d(A, M) = d(M,B)$ e conclua que $M$ é o ponto médio do segmento $AB$.
		\end{enumerate}
	\end{exercicio}

	\begin{exercicio}
		Mostre que $f : (- \infty ; -4] \to \R$, tal que $f(x) = -x^2 - 8x -12$, é uma função crescente.
	\end{exercicio}

	\begin{exercicio}
		Seja a função $f:[3;5]\to\reals$ tal que $f(x)=-x^2+4x-3$.
		\begin{enumerate}[a)]
		  \item Mostre que $f$ é decrescente.
		  \item $f$ possui máximo absoluto? Se sim, ocorre em qual ponto?
		  \item $f$ possui mínimo absoluto? Se sim, ocorre em qual ponto?
		\end{enumerate}
	  \end{exercicio}

	\end{frame}


	%------------------------------------------------------------------------------------------------------------
	
	\begin{frame}
	\frametitle{Exercícios} 

	  \begin{exercicio}
		  Considere a função $f: \reals_- \to \reals^\ast_+$ tal que $f(x) = \dfrac{1}{1+x^2}$. Responda as seguintes perguntas apresentando as respectivas justificativas.
		  \begin{enumerate}[a)]
		  \item $f$ é monótona? Se sim, de que tipo? Se não, $f$ possui algum intervalo de monotonicidade?
		  \item $f$ possui máximo absoluto?
		  \item $f$ possui mínimo absoluto?
		  \item $f$ é limitada?
		  \end{enumerate}
	  \end{exercicio}
	  
	  \begin{exercicio}
		Considere a função real $f$ tal que $f(x) = -x^2 +2x +8$.
	  \begin{enumerate}[a)]
	  \item Mostre que $f$ é crescente no intervalo $( - \infty , 1]$;
	  \item Mostre que $f$ é decrescente no intervalo $[1, + \infty )$;
	  \item Use os itens anteriores para concluir que $1 \in \mathbb R$ é um ponto de máximo absoluto de $f$.
	  \end{enumerate}  
	  \end{exercicio}

	\end{frame}


	%------------------------------------------------------------------------------------------------------------
	
	\begin{frame}
	\frametitle{Exercícios} 

	\Ex{
		Considere as funções $f: \R \to \R_+$ tal que $f(x) = x^2+3$ e $g: (-\infty ; 5] \to \R$ tal que $g(x) = \sqrt{x^2 - 10x +27}$. Faça o que se pede:
            \begin{enumerate}[a)]
                \item Calcule $(f \circ g)$ e $(g \circ f)$. Caso não seja possível, justifique;
                \item  Verifique se alguma das funções compostas que você calculou no primeiro item é monótona (crescente ou decrescente);
                \item  Verifique se alguma das funções compostas que você calculou no primeiro item possui máximo ou mínimo absoluto (escolha só um).
            \end{enumerate}
	}

\end{frame}


%------------------------------------------------------------------------------------------------------------

\begin{frame}
\frametitle{Exercícios} 

    \Ex{Sejam $f: \R \to \R $. Determine se as afirmações abaixo são
verdadeiras ou falsas, justificando suas respostas. As funções que
forem usadas como contraexemplo podem ser exibidas somente com o
esboço de seu gráfico.
\begin{enumerate}[(a)]
	\item Se $f$ é limitada superiormente, então $f$ tem pelo menos um máximo absoluto;
	\item Se $f$ é limitada superiormente, então $f$ tem pelo menos um máximo local;
	\item Se $f$ tem um máximo local, então $f$ tem um máximo absoluto;
	\item Todo máximo local de $f$ é máximo absoluto;
	\item Todo máximo absoluto de $f$ é máximo local;
	\item Se $x_0$ é o ponto de extremo local de $f$, então é ponto de
	extremo local de $f^2$, onde $(f^2)(x) = f(x) \cdot f(x)$;
	\item Se $x_0$ é o ponto de extremo local de $f^2$, então é ponto de
	extremo local de $f$.
\end{enumerate}}
\end{frame}


%------------------------------------------------------------------------------------------------------------

\begin{frame}
\frametitle{Exercícios} 

\begin{exercicio}
	Seja $f: \N \to \R $ e $g: \R \to \N$. Determine se as afirmações abaixo são
	verdadeiras ou falsas, justificando suas respostas. As funções que
	forem usadas como contraexemplo podem ser exibidas somente com o
	esboço de seu gráfico.
	\begin{enumerate}[a)]
	  \item A função $g$ pode ser ilimitada inferiormente;
	  \item $f$ é limitada superiormente ou $f$ é limitada inferiormente.
	\end{enumerate}
	\end{exercicio}

\Ex{Sejam $f: \R \to \R $ e $g: \R \to \R$. Determine se as
afirmações abaixo são verdadeiras ou falsas, justificando suas
respostas. As funções que forem usadas como contraexemplo podem ser
exibidas somente com o esboço de seu gráfico.
\begin{enumerate}[(a)]
	\item Se $f$ e $g$ são crescentes, então a composta $f \circ g$ é uma função crescente;
	\item Se $f$ e $g$ são crescentes, então o produto $f\cdot g$ é
	uma função crescente, onde $(f \cdot g)(x) = f(x) \cdot g(x)$;
	\item Se $f$ é crescente em $A \contido \R$ e em $B \contido \R$, então $f$ é crescente em $A \uniao B \contido \R$.
\end{enumerate}}

\end{frame}


%------------------------------------------------------------------------------------------------------------

\begin{frame}
\frametitle{Exercícios} 

\begin{exercicio}
	Seja $f$ uma função real. 
	\begin{enumerate}[a)]
		\item Suponha que $f$ é constante. Mostre que $f$ é não crescente e não decrescente;
		\item Suponha que $f$ é não crescente e não decrescente. Mostre que $f$ é constante.
	\end{enumerate}
%	Mostre que uma função real é constante se, e somente se, é não decrescente e não crescente.
  \end{exercicio}
  
  \begin{exercicio}
	Sejam $f : \reals \to \reals$ e $A$ e $B$ intervalos reais tais que $A \inter B$ é um intervalo não
	degenerado, ou seja, que possui pelo menos dois números. Mostre que, se $f$ é crescente
	em $A$ e em $B$, então $f$ é crescente em $A\inter B$.
  \end{exercicio}

\Ex{Mostre que a função inversa de uma função crescente é também uma
função crescente. E a função inversa de uma função decrescente é
decrescente.}

\end{frame}


%------------------------------------------------------------------------------------------------------------

\begin{frame}
\frametitle{Exercícios} 

\Ex{
	Dizemos que uma função $f: \R \to \R$ é \emph{par} quando se tem $f(-t) = f(t)$ para todo $t \in \R$. Se for o caso de $f(-t) = -f(t)$ para todo $t \in \R$, dizemos que $f$ é \emph{ímpar}.

  Considere a função real $f: \R \to \R$. Demonstre, ou refute com um contraexemplo, as afirmações abaixo:
  \begin{enumerate}[a)]
    \item Se $f$ é par e $x_0 \in \R$ é um ponto de máximo absoluto, então $-x_0 \in \R$ é também um ponto de máximo absoluto;
    \item Se $f$ é ímpar e $x_0 \in \R$ é um ponto de mínimo absoluto, então $-x_0 \in \R$ é um ponto de máximo absoluto;
	\item Se $f$ é par e limitada superiormente, então $f$ é limitada inferiormente;
    \item Se $f$ é ímpar e limitada superiormente, então $f$ é limitada inferiormente.
  \end{enumerate}
}

\end{frame}

%------------------------------------------------------------------------------------------------------------