\begin{frame}
    \section{Gráficos de Função Real}
    \frametitle{Gráfico de Função Real} 
    
    \begin{definicao}
    O \sub{gráfico} de uma função real é o seguinte subconjunto do plano
    cartesiano $\R^2$: $$G(f) = \set{(x,y) \in \R^2 \tq x \in D , y =
    f(x)}.$$
    \end{definicao}
    Em outras palavras, o gráfico de uma função $f$ é o lugar geométrico
    dos pontos cujas coordenadas satisfazem sua lei de associação.
    
    \end{frame}
    
    
    
    %------------------------------------------------------------------------------------------------------------
    
    \begin{frame}
    \frametitle{Gráfico de Função Real} 
    
    \begin{exemplo}
    Esboce o gráfico da função real
    $$\begin{array}{cccl}
    f : & \R^\ast & \to     & \R \\
             &  x & \mapsto & \begin{cases}
                                                    +1,  &  \ \text{ se } x >0 \\
                                                    -1, &  \ \text{ se } x <0
                                                    \end{cases}
    \end{array}.$$
    \end{exemplo}
    
    \end{frame}
    
    
    %------------------------------------------------------------------------------------------------------------

    \begin{frame}
        \frametitle{Gráfico de Função Real} 
        
        \begin{definicao}
            Chamamos de \sub{função piso} (também chamada de \sub{chão} ou \sub{solo}) a função real que associa a cada número real $x$ ao maior inteiro que é menor ou igual a $x$. Denotamos este número por $\lfloor x \rfloor$.\\
            Chamamos de \sub{função teto} a função real que associa a cada número real $x$ ao menor inteiro que é maior ou igual a $x$. Denotamos este número por $\lceil x \rceil$.
        \end{definicao} \pause

        \begin{exemplo}
            Calcule $\lfloor 7{,}5 \rfloor$, $\lfloor -2{,}2 \rfloor$, $\lceil 4{,}1 \rceil$ e $\lceil -3{,}9 \rceil$.\\
        Esboce o gráfico das funções piso e teto.
        \end{exemplo}
        
        \end{frame}
        
        
        %------------------------------------------------------------------------------------------------------------