\documentclass[10pt]{beamer}
%
%   Arquivo de Configuração dos Slides
%


%
%   Pacotes utilizados
%

% Codificação dos caracteres em formato universal.
\usepackage[utf8]{inputenc}
\usepackage[T1]{fontenc}

% Traduz o texto gerados pelo LaTeX para português. ex.: Capítulo, Seção, Conteúdo.
\usepackage[brazil]{babel}

% Pacotes para ambientes matemáticos
\usepackage{amsmath}
\usepackage{amsthm}
\usepackage{amssymb}

% Diversas funções para o uso das aspas.
\usepackage{csquotes}

% Outros pacotes
\usepackage{hyperref}
\usepackage{tikz}
\usepackage{yfonts}
\usepackage{colortbl}
\usepackage{ragged2e}
\usepackage{helvet}
\usepackage{verbatim}


%
%   Tema
%

% Copyright 2007 by Till Tantau
%
% This file may be distributed and/or modified
%
% 1. under the LaTeX Project Public License and/or
% 2. under the GNU Public License.
%
% See the file doc/licenses/LICENSE for more details.


% Common packages


\usepackage{times}
 \mode<article> {
	\usepackage{times}
	\usepackage{mathptmx}
	\usepackage[left=1.5cm,right=6cm,top=1.5cm,bottom=3cm]{geometry}
}

\usepackage{hyperref}
\usepackage[T1]{fontenc}
\usepackage{amsmath,amssymb}
\usepackage{tikz}
\usepackage{colortbl}
\usepackage{yfonts}
\usepackage{colortbl}
\usepackage{translator} % comment this, if not available
\usepackage{ragged2e} % justifying
% Or whatever. Note that the encoding and the font should match. If T1
% does not look nice, try deleting the line with the fontenc.
\usepackage{helvet}
\usepackage{verbatim}


%\usepackage{lipsum}
%\usepackage{enumitem}


\usetheme[
%%% options passed to the outer theme
%    hidetitle,           % hide the (short) title in the sidebar
%    hideauthor,          % hide the (short) author in the sidebar
%    hideinstitute,       % hide the (short) institute in the bottom of the sidebar
%    shownavsym,          % show the navigation symbols
%    width=2cm,           % width of the sidebar (default is 2 cm)
%    hideothersubsections,% hide all subsections but the subsections in the current section
%    hideallsubsections,  % hide all subsections
		right               % right of left position of sidebar (default is right)
%%% options passed to the color theme
%    lightheaderbg,       % use a light header background
	]{AAUsidebar}

% If you want to change the colors of the various elements in the theme, edit and uncomment the following lines
% Change the bar and sidebar colors:
%\setbeamercolor{AAUsidebar}{fg=red!20,bg=red}
%\setbeamercolor{sidebar}{bg=red!20}
% Change the color of the structural elements:
%\setbeamercolor{structure}{fg=red}
% Change the frame title text color:
%\setbeamercolor{frametitle}{fg=blue}
% Change the normal text color background:
%\setbeamercolor{normal text}{bg=gray!10}
% Highlight the text in the sidebar
\usecolortheme{rose,sidebartab}
% ... and you can of course change a lot more - see the beamer user manual.

% colored hyperlinks
\newcommand{\chref}[2]{%
	\href{#1}{{\usebeamercolor[bg]{AAUsidebar}#2}}%
}



% specify a logo on the titlepage (you can specify additional logos an include them in
% institute command below
\pgfdeclareimage[height=1cm]{titlepagelogo}{theme/figures/ufrn2} % placed on the title page
\pgfdeclareimage[height=1cm]{titlepagelogo2}{theme/figures/imd} % placed on the title page
\titlegraphic{% is placed on the bottom of the title page
	\pgfuseimage{titlepagelogo}
	\hspace{1cm}\pgfuseimage{titlepagelogo2}
}


% Article version layout settings

\mode<article>

\makeatletter
\def\@listI{\leftmargin\leftmargini
	\parsep 0pt
	\topsep 5\p@   \@plus3\p@ \@minus5\p@
	\itemsep0pt}
\let\@listi=\@listI


\setbeamertemplate{frametitle}{\paragraph*{\insertframetitle\
		\ \small\insertframesubtitle}\ \par
}
\setbeamertemplate{frame end}{%
	\marginpar{\scriptsize\hbox to 1cm{\sffamily%
			\hfill\strut\insertshortlecture.\insertframenumber}\hrule height .2pt}}
\setlength{\marginparwidth}{1cm}
\setlength{\marginparsep}{4.5cm}

\def\@maketitle{\makechapter}

\def\makechapter{
	\newpage
	\null
	\vskip 2em%
	{%
		\parindent=0pt
		\raggedright
		\sffamily
		\vskip8pt
		\includegraphics[width=\linewidth]{theme/figures/imd.png}\par\vskip2em
		{\fontsize{36pt}{36pt}\selectfont Aula \insertshortlecture \par\vskip2pt}
		{\fontsize{24pt}{28pt}\selectfont \color{blue!50!black} \@title\par\vskip4pt}
		%{\Large\selectfont \color{blue!50!black} \insertsubtitle\par}
		\vskip10pt

		\normalsize\selectfont [Notas de Aula]
		Disciplina: \emph{\lecturename \ (\semestre)} \par\vskip1.5em
		\nomedoautor\hskip1em Email: \ \emaildoautor
	}
	\par
	\vskip 1.5em%
}

\let\origstartsection=\@startsection
\def\@startsection#1#2#3#4#5#6{%
	\origstartsection{#1}{#2}{#3}{#4}{#5}{#6\normalfont\sffamily\color{blue!50!black}\selectfont}}

\makeatother

\mode
<all>




% Typesetting Listings

\usepackage{listings}
\lstset{language=Java}

\alt<presentation>
{\lstset{%
	basicstyle=\footnotesize\ttfamily,
	commentstyle=\slshape\color{green!50!black},
	keywordstyle=\bfseries\color{blue!50!black},
	identifierstyle=\color{blue},
	stringstyle=\color{orange},
	escapechar=\#,
	emphstyle=\color{red}}
}
{
	\lstset{%
		basicstyle=\ttfamily,
		keywordstyle=\bfseries,
		commentstyle=\itshape,
		escapechar=\#,
		emphstyle=\bfseries\color{red}
	}
}



% Common theorem-like environments
%\usepackage{amsthm}

\setbeamertemplate{theorems}[numbered]

%
%	New useful definitions:
%

\newbox\mytempbox
\newdimen\mytempdimen

\newcommand\includegraphicscopyright[3][]{%
	\leavevmode\vbox{\vskip3pt\raggedright\setbox\mytempbox=\hbox{\includegraphics[#1]{#2}}%
		\mytempdimen=\wd\mytempbox\box\mytempbox\par\vskip1pt%
		\fontsize{3}{3.5}\selectfont{\color{black!25}{\vbox{\hsize=\mytempdimen#3}}}\vskip3pt%
}}

\newenvironment{colortabular}[1]{\medskip\rowcolors[]{1}{blue!20}{blue!10}\tabular{#1}\rowcolor{blue!40}}{\endtabular\medskip}

\def\equad{\leavevmode\hbox{}\quad}

\newenvironment{greencolortabular}[1]
{\medskip\rowcolors[]{1}{green!50!black!20}{green!50!black!10}%
	\tabular{#1}\rowcolor{green!50!black!40}}%
{\endtabular\medskip}

%\setbeamertemplate{theorem begin}{{ \inserttheoremheadfont
%\inserttheoremname \inserttheoremnumber
%\ifx\inserttheoremaddition\empty\else\ (\inserttheoremaddition)\fi%
%\inserttheorempunctuation }} \setbeamertemplate{theorem end}{}

\newcommand{\vu}{\vec{u}}
\newcommand{\vv}{\vec{v}}
\newcommand{\vi}{\vec{i}}
\newcommand{\vj}{\vec{j}}
\newcommand{\vk}{\vec{k}}
\newcommand{\vw}{\vec{w}}
\newcommand\segmento[2]{\overline{#1#2}}
\def\colc#1{\left[#1\right]}



%
%   Macros
%

\usepackage{macros/macros}


%
%   Ambientes
%

\theoremstyle{plain}
\newtheorem{teorema}{Teorema}

\theoremstyle{definition}
\newtheorem{definicao}[teorema]{Definição}
%\newtheorem{exercicio}{Exercício}

\theoremstyle{remark}
\newtheorem{obs}[teorema]{Observação}
\newtheorem{observacao}[teorema]{Observação}
\newtheorem{corolario}[teorema]{Corolário}
\newtheorem{exemplo}[teorema]{Exemplo}
\newtheorem{lema}[teorema]{Lema}
\newtheorem{proposicao}[teorema]{Proposição}

\newcounter{exercicios}
\newenvironment{exercicio}{\stepcounter{exercicios} \textbf{\arabic{exercicios}}.}{}

% compatibilidade
\newcommand{\Ex}[1]{\begin{exercicio}#1\end{exercicio}}

%
%   Definições e comandos auxiliares do preâmbulo
%

\newcommand{\capitulo}[1]{\lecture[#1]{Capítulo}}
\newcommand{\aula}[1]{\subtitle{#1}}
\newcommand{\autor}{Igor Oliveira}
\newcommand{\email}{\href{mailto:matematicaelementar@imd.ufrn.br}{\texttt{matematicaelementar@imd.ufrn.br}}}
\newcommand{\disciplina}{Matemática Elementar}
\newcommand{\codigo}{IMD1001}

\title{\disciplina}
\date{\today}
\author[\autor]
{
    \autor\\
    \email
}

\def\lecturename{\codigo

\disciplina}

\institute[
	UFRN\\
	Natal-RN
]
{
	Instituto Metrópole Digital\\
	Universidade Federal do Rio Grande do Norte\\
	Natal-RN

}

% compatibilidade
\newcommand{\vu}{\vec{u}}
\newcommand{\vv}{\vec{v}}
\newcommand{\vi}{\vec{i}}
\newcommand{\vj}{\vec{j}}
\newcommand{\vk}{\vec{k}}
\newcommand{\vw}{\vec{w}}
\newcommand{\segmento}[2]{\overline{#1#2}}
\def\colc#1{\left[#1\right]}
\newcommand{\negacao}{\sim}

\justifying


\aula{Matrizes e Sistemas Lineares}
\capitulo{3}


\begin{document}	

	%
	%	Capa
	%

	{\backgroundimage\begin{frame}[plain]
		\titlepage
	\end{frame}}


	%
	%	Sumário
	%

	\begin{frame}
		\frametitle{Índice}
		\tableofcontents
	\end{frame}


	%
	%	Seções
	%

	\section{Apresentação}


\begin{frame}
    \frametitle{Apresentação da Aula}
    
    \begin{block}{Motivação}
        Matrizes são, fundamentalmente, tabelas numéricas sobre as quais se definem certas operações algébricas, útil para armazenar vários dados em um só elemento. Além disso, as matrizes se aplicam ao estudo dos sistemas lineares (conforme veremos neste capítulo), bem como desempenham um papel decisivo no estudo das transformações lineares, as quais são justamente as funções estudadas na Álgebra Linear.
    \end{block}
\end{frame}

	\section{Matrizes}


\begin{frame}
    \frametitle{Definição de Matriz}
    
    \begin{definicao}
        Sejam $m, n \pertence \N^\ast$. Uma \sub{matriz (real)} do \sub{tipo $m \times n$} (lê-se: $m$ por $n$) é uma \aspas{tabela} disposta em $m$ linhas e $n$ colunas. Denotamos os números reais que formam a $i$-ésima linha de uma matriz $A$ por $a_{i1}, a_{i2}, \dots , a_{in}$ e sua $j$-ésima coluna por $a_{1j}, a_{2j}, \dots , a_{mj}$. Assim, escrevemos:
        \begin{displaymath} A = 
            \begin{bmatrix}
                a_{11} & \dots & a_{1n} \\
                \vdots & \ddots & \vdots \\
                a_{m1} & \dots & a_{mn}
            \end{bmatrix}.
        \end{displaymath}
        Chamamos de \sub{entradas}, de uma matriz $A$, os reais $a_{ij}$ que a compõem.
        Poderemos indicar uma matriz $A$ do tipo $m \times n$ com entradas $a_{ij}$ por
        $$ A = \paren{a_{ij}}_{m \times n}$$
        ou, simplesmente, $A = \paren{a_{ij}}$.
    \end{definicao}
\end{frame}

%------------------------------------------------------------------------

\begin{frame}
    \frametitle{Definição de Matriz}

    \begin{exemplo}
        As matrizes $A = \paren{a_{ij}}_{3 \times 2}$, em que $a_{ij}=i+j$ e, $B = \paren{b_{ij}}_{ 2 \times 4}$, em que $b_{ij} = i^j$, são:
        \begin{displaymath} A = 
            \begin{bmatrix}
                2 & 3 \\
                3 & 4 \\
                4 & 5
            \end{bmatrix}
            \space \text{ e } \space B=
            \begin{bmatrix}
                1 & 1 & 1 & 1 \\
                2 & 4 & 8 & 16
            \end{bmatrix}.
        \end{displaymath}
    \end{exemplo}

\end{frame}

%------------------------------------------------------------------------

\begin{frame}
    \frametitle{Definição de Matriz}

    \begin{definicao}
        Uma matriz $A$ do tipo $n \times n$, é dita \sub{quadrada de ordem $n$}. O conjunto formado por suas entradas $a_{ii}$ é chamado de \sub{diagonal} de $A$. O conjunto formado pelas entradas $a_{ij}$ tais que $i+j = n+1$ é chamado de \sub{diagonal secundária} de $A$.
        
        Uma matriz do tipo $m \times n$ cujas entradas são todas iguais a zero chama-se \sub{nula} e será denotada por $0_{m \times n}$.
    \end{definicao} \pause

    \begin{exemplo}
        Dada a matriz $A$ quadrada de ordem $3$ abaixo, sua diagonal é formada pelos números $2$, $4$ e $1$. A diagonal secundária é formada por $1$, $4$ e $4$. 
        \begin{displaymath} A = 
            \begin{bmatrix}
                2 & 3 & 1\\
                3 & 4 & 0\\
                4 & 5 & 1
            \end{bmatrix}.
        \end{displaymath}
    \end{exemplo}
\end{frame}
	\section{Atividade Online}
\begin{frame}
\frametitle{Atividade Online} 

\link{https://pt.khanacademy.org/math/precalculus/x9e81a4f98389efdf:matrices/x9e81a4f98389efdf:model-situations-with-matrices/e/use-matrices-to-represent-data}
{Atividade Online 17 - Use Matrizes para Representar \\ Dados}


\end{frame}
	\section{Operações com Matrizes}


\begin{frame}
    \frametitle{Operações com Matrizes}
    
    \begin{definicao}[Produto por escalar e adição de matrizes]
        Dadas matrizes de mesmo tipo, $ A = \paren{a_{ij}}_{m \times n}$ e $ B = \paren{b_{ij}}_{m \times n}$, e $\lambda \pertence \R$, definimos o \sub{produto por escalar $\lambda A$} e a \sub{adição $A+B$} por:
        \begin{displaymath} \lambda A = 
            \begin{bmatrix}
                \lambda a_{11} & \dots & \lambda a_{1n} \\
                \vdots & \ddots & \vdots \\
                \lambda a_{m1} & \dots & \lambda a_{mn}
            \end{bmatrix}
        \end{displaymath}
        e
        \begin{displaymath}    A+B=
            \begin{bmatrix}
                a_{11} + b_{11} & \dots & a_{1n} + b_{1n} \\
                \vdots & \ddots & \vdots \\
                a_{m1} + b_{m1} & \dots & a_{mn} + b_{mn}
            \end{bmatrix}.
        \end{displaymath}
    \end{definicao}
\end{frame}

%------------------------------------------------------------------------

\begin{frame}
    \frametitle{Operações com Matrizes}
    
    \begin{exemplo}
        Calcule: 
        \begin{displaymath} A = 3 \cdot 
            \begin{bmatrix}
                2 & 3 \\
                3 & 4 \\
                4 & 5
            \end{bmatrix}
        \end{displaymath}
        e
        \begin{displaymath}   B = 
            \begin{bmatrix}
                1 & 1 & 1 & 1 \\
                2 & 4 & 8 & 16
            \end{bmatrix} + 
            \begin{bmatrix}
                1 & 2 & 3 & 4 \\
                1 & 4 & 9 & 16
            \end{bmatrix}.
        \end{displaymath}
    \end{exemplo}
\end{frame}

%------------------------------------------------------------------------

\begin{frame}
	\frametitle{Operações com Matrizes} 
	
	\begin{proposicao}[Propriedades do produto por escalar e da adição]
		\label{propopmatriz1}
		Sejam $A$, $B$ e $C$ matrizes de mesmo tipo $m \times n$, e $\lambda , \mu \in \R$. Tem-se:
		\begin{enumerate}[i.]
			\item \sub{Comutatividade da adição}: $A+B = B+A$;
			\item \sub{Associatividade da adição}: $A + (B+C) = (A+B)+C$;
			\item \sub{Elemento neutro da adição}: $A + 0_{m \times n} = A$;
			\item \sub{Existência do oposto aditivo}: $A + (-A) = 0_{m \times n}$;
			\item \sub{Associatividade da multiplicação por escalar}: $\lambda (\mu A) = (\lambda \mu) A$;
            \item \sub{Elemento neutro da multiplicação por escalar}: $1\cdot A = A$;
			\item \sub{Distributividade, de uma em relação à outra}: $\lambda(A+B) = \lambda A + \lambda B$ e $(\lambda +\mu)A = \lambda A + \mu A$.
		\end{enumerate}
	\end{proposicao} 

\end{frame}


%------------------------------------------------------------------------

\begin{frame}
    \frametitle{Operações com Matrizes}
    
    \begin{definicao}[Multiplicação de matrizes]
        Dadas matrizes $ A = \paren{a_{ij}}_{m \times n}$ e $ B = \paren{b_{ij}}_{n \times p}$, onde o número de colunas de $A$ coincide com o número de linhas de $B$. Definimos o \sub{produto $ AB$} como a matriz $P = \paren{p_{ij}}_{m \times p}$, cujas entradas são:
        $$ p_{ij} = \sum_{k=1}^n a_{ik}b_{kj} = a_{i1}b_{1j}+ a_{i2}b_{2j}+ \dots + a_{in}b_{nj}.$$
    \end{definicao} \pause

    \begin{exemplo}\label{exprodmatriz}
        Calcule: 
        \begin{displaymath} A =
            \begin{bmatrix}
                2 & 3 \\
                3 & 4 \\
                4 & 5
            \end{bmatrix}  
            \begin{bmatrix}
                0 & -1 & 3 \\
                3 & 2 & 1 
            \end{bmatrix} \space \text{ e } \space B =
            \begin{bmatrix}
                2 & 3 \\
                4 & 5
            \end{bmatrix}  
            \begin{bmatrix}
                0 & -1 \\
                3 & 2 
            \end{bmatrix} .
        \end{displaymath}
    \end{exemplo}
\end{frame}

%------------------------------------------------------------------------

\begin{frame}
	\frametitle{Operações com Matrizes} 
	
	\begin{proposicao}[Propriedades do produto de matrizes]
		\label{propopmatriz2}
		Sejam as matrizes $A=(a_{ij})_{m \times n}$, $B=(b_{ij})_{n \times p}$, $C=(c_{ij})_{p \times q}$, $D=(d_{ij})_{n \times p}$ e $E=(e_{ij})_{m \times n}$. Tem-se:
		\begin{enumerate}[i.]
			\item \sub{Associatividade}: $A  \paren{B  C}= \paren{A  B} C$;
			\item \sub{Distributividade à esquerda, em relação a soma}: $A\paren{B + D} = AB + AD$;
			\item \sub{Distributividade à direita, em relação a soma}: $\paren{A + E}B = AB + EB$.
		\end{enumerate}
	\end{proposicao} 
\end{frame}

%------------------------------------------------------------------------

\begin{frame}
	\frametitle{Operações com Matrizes} 


	\begin{observacao}
		Dadas duas matrizes $A$ e $B$ quadradas e de mesma ordem, os dois produtos $AB$ e $BA$ estão bem definidos. No entanto, de modo geral, eles NÃO SÃO IGUAIS, isto é, O PRODUTO DE MATRIZES QUADRADAS DE MESMA ORDEM NÃO É COMUTATIVO. Quando, excepcionalmente, ocorre a igualdade $AB = BA$, dizemos que $A$ e $B$ \sub{comutam}.
	\end{observacao}\pause

    \begin{exemplo}
        Comprove a observação anterior comparando o produto das matrizes quadradas do Exemplo \ref{exprodmatriz} com o produto abaixo: 
        \begin{displaymath} C = 
            \begin{bmatrix}
                0 & -1 \\
                3 & 2
            \end{bmatrix}  
            \begin{bmatrix}
                2 & 3 \\
                4 & 5 
            \end{bmatrix} .
        \end{displaymath}
    \end{exemplo}
\end{frame}

%------------------------------------------------------------------------
	\section{Atividade Online}
\begin{frame}
\frametitle{Atividade Online} 

\link{https://pt.khanacademy.org/math/precalculus/x9e81a4f98389efdf:matrices/x9e81a4f98389efdf:multiplying-matrices-by-scalars/e/scalar_matrix_multiplication}
{Atividade Online 18 - Multiplicação de Matrizes por \\ Números Escalares}

\link{https://pt.khanacademy.org/math/precalculus/x9e81a4f98389efdf:matrices/x9e81a4f98389efdf:adding-and-subtracting-matrices/e/matrix_addition_and_subtraction}
{Atividade Online 19 - Some e Subtraia Matrizes}

\link{https://pt.khanacademy.org/math/precalculus/x9e81a4f98389efdf:matrices/x9e81a4f98389efdf:using-matrices-to-manipulate-data/e/use-matrices-to-manipulate-data}
{Atividade Online 20 - Use Matrizes para Manipular Dados}

\link{https://pt.khanacademy.org/math/precalculus/x9e81a4f98389efdf:matrices/x9e81a4f98389efdf:multiplying-matrices-by-matrices/e/multiplying_a_matrix_by_a_matrix}
{Atividade Online 21 - Multiplique Matrizes}


\end{frame}
	\section{Sistemas Lineares}


\begin{frame}
    \frametitle{Sistemas Lineares e Matrizes}

    Um sistema de $m$ equações lineares nas variáveis $x_1, x_2, \dots, x_n$ pode ser representado pelas equações
    $$\left\{
        \begin{array}{llll}
        a_{11}x_1 + a_{12}x_2 + \cdots + a_{1n}x_n =b_1 \\
        a_{21}x_1 + a_{22}x_2 + \cdots + a_{2n}x_n =b_2 \\
        (\dots) \\
        a_{m1}x_1 + a_{m2}x_2 + \cdots + a_{mn}x_n =b_m  
        \end{array} \right. .$$
    
    Tal sistema é equivalente à equação matricial $AX = B$, onde

        \begin{displaymath} A = 
            \begin{bmatrix}
                a_{11} & a_{12} & \dots & a_{1n} \\
                a_{21} & a_{22} & \dots & a_{2n} \\
                \vdots & \vdots & \ddots & \vdots \\
                a_{m1} & a_{m2} & \dots & a_{mn}
            \end{bmatrix}
        ,    \ \ 
        X=
            \begin{bmatrix}
                x_1   \\
                x_2 \\
                \vdots \\
                x_n
            \end{bmatrix}
        \ \text{ e } \  B = 
            \begin{bmatrix}
                b_1   \\
                b_2 \\
                \vdots \\
                b_m 
            \end{bmatrix}.
        \end{displaymath}

    Nesse caso, dizemos que $A$ é a \sub{matriz do sistema}. Quando $B = 0_{m \times 1}$, o sistema é dito \sub{homogêneo}. Observe que todo sistema homogêneo admite a solução $X = 0_{n \times 1}$, dita \sub{trivial}.
\end{frame}

%------------------------------------------------------------------------

\begin{frame}
    \frametitle{Sistemas Lineares e Matrizes}

    \begin{definicao}[Sistemas lineares equivalentes]
        Dois sistemas lineares são ditos \sub{equivalentes} quando têm o mesmo conjunto solução.
    \end{definicao}\pause

    \begin{definicao}[Matriz aumentada de um sistema linear]
        Dado um sistema linear $AX=B$, define-se a sua \sub{matriz aumentada} $(A|B)$, como sendo a matriz obtida ``posicionando'' a matriz $B$ à direita da matriz $A$, isto é,
        \begin{displaymath}(A|B)=
            \begin{bmatrix}
                a_{11} & \dots & a_{1n} & b_1\\
                \vdots & \ddots & \vdots & \vdots \\
                a_{m1} & \dots & a_{mn} & b_m
            \end{bmatrix}.
        \end{displaymath}
    \end{definicao}
    
    
\end{frame}

%------------------------------------------------------------------------



\begin{frame}
    \frametitle{Sistemas Lineares e Matrizes}

    \begin{exemplo}
        Qual a matriz aumentada do sistema linear abaixo?
        $$\left\{
        \begin{array}{rrrrrrr}
        x &+& y &-& z& = &1 \\
        x& -& y& & &=& 1 
        \end{array} \right.$$
    \end{exemplo}

\end{frame}

%------------------------------------------------------------------------
	\section{Atividade Online}
\begin{frame}
\frametitle{Atividade Online} 

\link{https://pt.khanacademy.org/math/algebra-home/alg-matrices/alg-representing-systems-with-matrices/e/represent-systems-with-matrices}
{Atividade Online 22 - Represente Sistemas Lineares com Matrizes}


\end{frame}
	\section{Operações Elementares}

\begin{frame}
    \frametitle{Operações Elementares sobre Matrizes}

    Algumas operações sobre as linhas de uma matriz são ditas \sub{elementares}. São elas:
    \begin{itemize}
        \item $(l_i \sselogico l_j)$: Troca de posição entre duas linhas  $l_i$ e $l_j$;
        \item $(l_i \to \lambda l_i)$: Multiplicação de uma linha $l_i$ por um escalar $\lambda \neq 0$;
        \item $(l_j \to l_j + \lambda l_i)$: Substituição de uma linha $l_j$ por $l_j +\lambda l_i$, sendo $\lambda \neq 0$.\pause
    \end{itemize}
    
    \begin{definicao}[Equivalência por linhas]
        Diz-se que uma matriz $B$ é \sub{linha equivalente} a uma matriz $A$ quando $B$ é obtida de $A$ efetuando-se nesta uma sequência de operações elementares.
    \end{definicao}\pause

    \begin{proposicao}
        Dois sistemas lineares $AX=B$ e $A'X=B'$ são equivalentes se suas matrizes aumentadas $(A|B)$ e $(A'|B')$ são linha equivalentes.
    \end{proposicao}

\end{frame}

%------------------------------------------------------------------------

\begin{frame}
    \frametitle{Operações Elementares sobre Matrizes}

    \begin{definicao}[Matriz escalonada]
        Diz-se que uma matriz $A= (a_{ij})_{m \times n}$ é \sub{escalonada} quando cumpre as seguintes condições:
        \begin{itemize}
            \item O primeiro elemento não-nulo de uma linha está à esquerda do primeiro elemento não-nulo da linha subsequente;
            \item As linhas nulas, caso existam, estão abaixo das demais.
        \end{itemize}
    \end{definicao}\pause

    Para resolvermos um sistema linear, fazemos o \sub{escalonamento} da matriz aumentada do sistema a fim de obtermos uma matriz escalonada que seja linha equivalente à matriz aumentada do sistema. Esse procedimento é chamado de \sub{Método de Gauss} ou \sub{Eliminação Gaussiana}.
    

    
\end{frame}

%------------------------------------------------------------------------

\begin{frame}
    \frametitle{Operações Elementares sobre Matrizes}

    \begin{exemplo}
        Encontre o conjunto solução para o sistema

        \[\left\{
        \begin{array}{rrrrrrr}
        x &-&2y &+ &3z &= &1 \\
        2x &+& y &-&z &= &2 \\
        4x &-&3y &+&5z &= &4  
        \end{array} \right..\]
    \end{exemplo} \pause
    
    \begin{exemplo}
        Encontre o conjunto solução para o sistema

        $$\left\{
        \begin{array}{rrrrrrrrr}
        2x &-&y & & &+ &4t &= &9 \\
        x &+& y &-&z & + & 2t&= &7 \\
        -x &+& 2y &+&z & - & t&= &3 \\
         & &4y &-&z &+ & 3t&= &13  
        \end{array} \right. .$$
    \end{exemplo}
    

\end{frame}

%------------------------------------------------------------------------


	\section{Atividade Online}
\begin{frame}
\frametitle{Atividade Online} 

\link{https://pt.khanacademy.org/math/algebra-home/alg-matrices/alg-elementary-matrix-row-operations/e/perform-elementary-matrix-row-operations}
{Atividade Online 23 - Operações sobre Linhas de uma \\ Matriz}


\end{frame}
	\section{Matrizes Quadradas}


\begin{frame}
    \frametitle{Tipos de Matrizes Quadradas}
    
    \begin{definicao}[Matrizes triangulares e diagonais]
        Uma matriz quadrada $A=(a_{ij})_{n\times n}$ é dita \sub{triangular}, quando ocorre uma das possibilidades:
        
        $$a_{ij} = 0 \ \ \forall i>j \ \  \text{ ou } \ \  a_{ij} = 0 \ \  \forall i<j.$$

        No primeiro caso, ela é dita \sub{triangular superior} e, no segundo, \sub{triangular inferior}. Uma matriz que é triangular superior e inferior é dita \sub{diagonal}.
    \end{definicao}\pause

    \begin{exemplo}
        As matrizes $A$, $B$ e $C$ abaixo são, respectivamente, triangular superior, inferior e diagonal.
        
        \begin{displaymath} A = 
            \begin{bmatrix}
                3 & 0 & 2 \\
                0 & 0 & 1 \\
                0 & 0 & 1
            \end{bmatrix}
        ,    
        B=
            \begin{bmatrix}
                1 & 0 & 0 \\
                7 & 2 & 0 \\
                -2 & 0 & 0
            \end{bmatrix}
        \text{ e } C = 
            \begin{bmatrix}
                1 & 0  \\
                0 & 2  
            \end{bmatrix}.
        \end{displaymath}
    \end{exemplo}
\end{frame}

%------------------------------------------------------------------------

\begin{frame}
    \frametitle{Tipos de Matrizes Quadradas}
    
    \begin{definicao}[Matriz Identidade]
        Uma matriz diagonal $n\times n$ cujas entradas não obrigatoriamente nulas são todas iguais a $1$ é chamada de \sub{matriz identidade} de ordem $n$, a qual denota-se por $I_n$ ou, simplesmente, por $I$.
    \end{definicao}\pause
    A matriz identidade de ordem $n$ é o elemento neutro da multiplicação de matrizes quadradas, pois, para toda matriz quadrada $A$ de ordem $n$, vale a igualdade: $$AI = IA = A.$$

    Além disso, dada uma matriz $B_{n \times m}$, vale também: $$BI_m = I_nB = B.$$

\end{frame}

%------------------------------------------------------------------------

\begin{frame}
    \frametitle{Tipos de Matrizes Quadradas}
    
    \begin{definicao}[Matrizes invertíveis]
        Diz-se que uma matriz quadrada $A$ é  \sub{invertível}, quando existe uma matriz quadrada $B$, de mesma ordem que $A$, tal que $$AB = BA = I.$$
    \end{definicao}\pause

    Prova-se que $A$, quando invertível, possui uma única matriz inversa. Tal matriz é dita a \sub{inversa} de $A$, e denotada por $A^{-1}$.\pause


    \begin{exemplo}
        Verifique que as matrizes abaixo são invertíveis multiplicando-as.
        
        \begin{displaymath} A = 
            \begin{bmatrix}
                1 & 0  \\
                0 & \dfrac 1 2
            \end{bmatrix}
        \text{ e } B = 
            \begin{bmatrix}
                1 & 0  \\
                0 & 2  
            \end{bmatrix}.
        \end{displaymath}
    \end{exemplo}
\end{frame}

%------------------------------------------------------------------------

\begin{frame}
    \frametitle{Tipos de Matrizes Quadradas}

    Para calcular a inversa de uma matriz $A$, escalona-se a matriz aumentada $(A|I)$ a fim de se obter uma matriz equivalente por linhas do tipo $(I|B)$. Se for possível tal procedimento, então $B = A^{-1}$. Caso contrário, $A$ não é invertível.
    
    \begin{exemplo}\label{exinversa}
        Calcule a inversa das matrizes abaixo
        
        \begin{displaymath} A = 
            \begin{bmatrix}
                1 & 1  \\
                0 & 2
            \end{bmatrix}
        \text{ e } B = 
            \begin{bmatrix}
                1 & 2 & 1 \\
                0 & 1 & 1 \\
                2 & 4 & 2 
            \end{bmatrix}.
        \end{displaymath}
    \end{exemplo}\pause

    \begin{exemplo}
        Qual a solução do sistema abaixo?
        $$\left\{
        \begin{array}{rrrrr}
        x &+& y & = &2 \\
        & & 2y& =& 4 
        \end{array} \right.$$
    \end{exemplo}
\end{frame}

%------------------------------------------------------------------------
	\section{Atividade Online}
\begin{frame}
\frametitle{Atividade Online} 

\link{https://pt.khanacademy.org/math/algebra-home/alg-matrices/alg-intro-to-matrix-inverses/e/determine-inverse-matrices}
{Atividade Online 24 - Determine as Matrizes Inversas}

\link{https://pt.khanacademy.org/math/algebra-home/alg-matrices/alg-practice-finding-inverses-of-2x2-matrices/e/matrix_inverse_2x2}
{Atividade Online 25 - Encontre a Inversa de uma Matriz 2x2}

\link{https://pt.khanacademy.org/math/algebra-home/alg-matrices/alg-determinants-and-inverses-of-large-matrices/e/matrix_inverse_3x3}
{Atividade Online 26 - Matriz Inversa de uma Matriz 3x3}

\link{https://pt.khanacademy.org/math/precalculus/x9e81a4f98389efdf:matrices/x9e81a4f98389efdf:solving-equations-with-inverse-matrices/e/use-matrices-to-solve-systems-of-equations}
{Atividade Online 27 - Use Matrizes para Resolver \\ Sistemas de Equações}


\end{frame}
	\section{Determinante}


\begin{frame}
    \frametitle{Determinante de Matrizes $2 \times 2$ e $3 \times 3$}
    
    \begin{definicao}[Determinante de matrizes $2 \times 2$ e $3 \times 3$]
        Dadas as matrizes  $ A = \paren{a_{ij}}_{2 \times 2}$ e $ B = \paren{b_{ij}}_{3 \times 3}$, definimos o \sub{determinante} de $A$ e $B$, denotados, respectivamente, por $\det A$ e $\det B$ como sendo
        \begin{displaymath} \det A = 
            \begin{vmatrix}
                 a_{11} & a_{12} \\
                 a_{21} &  a_{22}
            \end{vmatrix} = a_{11}a_{22} - a_{12}a_{21}
        \end{displaymath}
        e
        \begin{align*}    \det B =&
            \begin{vmatrix}
                b_{11} & b_{12} & b_{13} \\
                b_{21} & b_{22} & b_{23} \\
                b_{31} & b_{32} & b_{33}
            \end{vmatrix} \\
            =& \paren{b_{11}b_{22}b_{33} + b_{12}b_{23}b_{31} + b_{13}b_{21}b_{32}} \\ 
            & - \paren{b_{13}b_{22}b_{31} + b_{11}b_{23}b_{32} + b_{12}b_{21}b_{33}}.
        \end{align*}
    \end{definicao}
\end{frame}

%------------------------------------------------------------------------

\begin{frame}
    \frametitle{Determinante de Matrizes $2 \times 2$ e $3 \times 3$}
    
    \begin{exemplo}
        Calcule o determinante das matrizes $A$ e $B$ do Exemplo \ref{exinversa}, ou seja,
        
        \begin{displaymath}  
            \begin{vmatrix}
                1 & 1  \\
                0 & 2
            \end{vmatrix}
        \text{ e }  
            \begin{vmatrix}
                1 & 2 & 1 \\
                0 & 1 & 1 \\
                2 & 4 & 2 
            \end{vmatrix}.
        \end{displaymath}
    \end{exemplo}

\end{frame}

%------------------------------------------------------------------------


	\section{Atividade Online}
\begin{frame}
\frametitle{Atividade Online} 

\link{https://pt.khanacademy.org/math/algebra-home/alg-matrices/alg-determinant-of-2x2-matrix/e/matrix_determinant}
{Atividade Online 28 - Determinante de uma Matriz 2x2}

\link{https://pt.khanacademy.org/math/algebra-home/alg-matrices/alg-determinants-and-inverses-of-large-matrices/e/matrix_determinant_3x3}
{Atividade Online 29 - Determinante de uma Matriz 3x3}


\end{frame}
	\section{Determinante}

\begin{frame}
    \frametitle{Determinante de Matrizes de Ordem $n$}

    Antes de apresentar uma definição para o determinante de uma matriz quadrada qualquer, vejamos alguns resultados mais simples sobre o determinante. \pause

    \begin{proposicao}\label{prop-equiv-sl-mat}
        Dada uma matriz $A_{n \times n}$, são equivalentes as afirmações abaixo:
        \begin{enumerate}[i)]
            \item Para toda matriz $B_{n \times 1}$, o sistema linear $AX=B$ admite uma única solução;
            \item $A$ é invertível;
            \item $\det A \neq 0$.
        \end{enumerate}
    \end{proposicao}\pause

 
    
    \begin{exemplo}
        Considere a matriz $C_{3 \times 1}$. O que a  Proposição \ref{prop-equiv-sl-mat} pode te garantir acerca de um sistema linear $BX = C$ onde $B$ é definida no Exemplo \ref{exinversa}?
    \end{exemplo}
\end{frame}

%------------------------------------------------------------------------

\begin{frame}
    \frametitle{Determinante de Matrizes de Ordem $n$}

    
    \begin{proposicao}[Propriedades]
        Considere as matrizes $A_{n \times n}$ e $B_{n \times n}$, são válidas as afirmações abaixo:
        \begin{enumerate}[i)]
            \item $\det (AB) = \det A \cdot \det B$;
            \item Se $A$ é invertível, então $\det (A^{-1}) = \dfrac 1 {\det A}$.
        \end{enumerate}
    \end{proposicao}\pause

 
    
    \begin{exemplo}
        Qual o determinante das matrizes $C$ e $D$ abaixo?
        \begin{displaymath} C = 
            \begin{bmatrix}
                1 & -\dfrac {1} 2 \\
                0 & \dfrac 1 2
            \end{bmatrix}
        \text{ e } D = 
            \begin{bmatrix}
                1 & 1 & 2 \\
                0 & 1 & 1 \\
                2 & 2 & 4 
            \end{bmatrix}.
        \end{displaymath}
    \end{exemplo}
\end{frame}

%------------------------------------------------------------------------

\begin{frame}
    \frametitle{Determinante de Matrizes de Ordem $n$}

    
    \begin{proposicao}[Determinante e Operações Elementares]
        Seja $A_{n \times n}$ uma matriz.
        \begin{enumerate}[i)]
            \item Se $B$ é a matriz que resulta quando duas linhas de $A$ são permutadas, então $\det A = - \det B$;
            \item Se $B$ é a matriz que resulta quando uma única linha de $A$ é multiplicada por um escalar $\lambda \neq 0$, então $\det A = \dfrac 1 \lambda \det B$;
            \item Se $B$ é a matriz que resulta quando um múltiplo não-nulo de uma linha de $A$ é somado a uma outra linha de $A$, então $\det A = \det B$.
        \end{enumerate}
        O resultado é análogo quando as operações elementares são feitas sobre as colunas de $A$.
    \end{proposicao}
\end{frame}

%------------------------------------------------------------------------

\begin{frame}
    \frametitle{Determinante de Matrizes de Ordem $n$}

    
    \begin{definicao}[Permutação]
        Dado $n \pertence \N^{\ast}$, uma \sub{permutação} do conjunto $\conjunto{1, 2, \dots , n}$ é um rearranjo dos elementos desse conjunto  em alguma ordem, sem omissão ou repetição. 
        
        Dizemos que uma permutação $\sigma$ é \sub{par} quando o rearranjo pode ser obtido por um número par de trocas de elementos a partir da ordem crescente. Caso contrário, a permutação é dita \sub{ímpar}.

        Definimos o \sub{sinal} da permutação $\sigma$, denotado por $\sgn  \sigma$, como sendo igual a $1$ se  $\sigma$ for par e igual a $-1$ se $\sigma$ for ímpar.
    \end{definicao}\pause

 
    
    \begin{exemplo}
        $\sigma_1 = \paren{3, 1, 2}$ e $\sigma_2 = \paren{1,3,2}$ são permutações do conjunto $\conjunto{1, 2, 3}$. Qual o sinal de $\sigma_1$ e $\sigma_2$?
    \end{exemplo}
\end{frame}

%------------------------------------------------------------------------

\begin{frame}
    \frametitle{Determinante de Matrizes de Ordem $n$}

    
    \begin{definicao}[Produto Elementar]
        Dado uma matriz $A_{n \times n}$ e $\sigma = (\sigma_1, \sigma_2, \dots , \sigma_n)$ uma permutação de $\conjunto{1, 2, \dots , n}$. Dizemos que $$a_{1\sigma_1} \cdot a_{2\sigma_2} \dots  a_{n\sigma_n}$$ é um \sub{produto elementar} de $A$. Em outras palavras, é o produto de $n$ entradas de $A$ sem que haja mais de uma entrada de alguma linha ou coluna.
        Dizemos também que $$\sgn \sigma \cdot a_{1\sigma_1}\cdot a_{2\sigma_2} \dots a_{n\sigma_n}$$ é um \sub{produto elementar com sinal} de $A$.
    \end{definicao}\pause

 
    
    \begin{definicao}[Determinante]
        Seja $A_{n \times n}$. O \sub{determinante} de $A$ é o somatório de todos os seus produtos elementares com sinal.
    \end{definicao}
\end{frame}

%------------------------------------------------------------------------

\begin{frame}
    \frametitle{Determinante de Matrizes de Ordem $n$}

    
    \begin{proposicao}[Determinante de matrizes triangulares]\label{det-mat-triang}
        Seja $A_{n \times n}$ uma matriz triangular. Então
        $$ \det A = a_{11} a_{22} \dots a_{nn}.$$
    \end{proposicao}\pause

    \begin{exemplo}
        Calcule o determinante de 
        \begin{displaymath} A = 
            \begin{bmatrix}
                2 & 0 & 1 & 0 & 3 \\
                3 & 1 & 2 & 0 & 4 \\
                \frac 4 3 & \frac 2 3 & 1 & \frac 1 3 & \frac 5 3 \\
                1 & 0 & 0 & 0 & 2 \\
                -8 & -4 & -6 & -2 & -9 
            \end{bmatrix}.
        \end{displaymath}
    \end{exemplo}


\end{frame}

%------------------------------------------------------------------------

\begin{frame}
    \frametitle{Determinante de Matrizes de Ordem $n$}

    \begin{definicao}[Matriz menor e cofator]
        Dada uma matriz $A = (a_{ij})_{n \times n}$, definimos a \sub{matriz menor $ij$} de $A$, denotada por $A_{ij}$, como sendo a matriz obtida a partir de $A$ excluindo-se a linha $i$ e a coluna $j$. Definimos também o \sub{cofator} de $a_{ij}$, como sendo 
        $$ C_{ij} = (-1)^{i+j} \cdot \det A_{ij}. $$
    \end{definicao}\pause

    
    \begin{proposicao}[Determinante a partir de cofatores]
        Seja $A_{n \times n}$ uma matriz. Fixada uma linha $i$ ou uma coluna $j$ de $A$, teremos, respectivamente:
        $$\det A = a_{i1}C_{i1} + a_{i2}C_{i2} + \dots + a_{in}C_{in}$$
        ou
        $$\det A = a_{1j}C_{1j} + a_{2j}C_{2j} + \dots + a_{nj}C_{nj}.$$
    \end{proposicao}
\end{frame}

%------------------------------------------------------------------------

\begin{frame}
    \frametitle{Determinante de Matrizes de Ordem $n$}

    
    \begin{exemplo}
        Calcule
        \begin{displaymath}  
            \begin{vmatrix}
                1 & 2 & 4 & 1 \\
                0 & 1 & 2 & 1 \\
                2 & 2 & 7 & 4\\
                2 & 4 & 8 & 2
            \end{vmatrix}.
        \end{displaymath}
    \end{exemplo}
\end{frame}

%------------------------------------------------------------------------


%	\section{Inclusão}


\begin{frame}
\frametitle{A Relação de Inclusão} %\framesubtitle{Exemplos}

\begin{definicao}
Sejam $A$ e $B$ conjuntos. Se todo elemento de $A$ for também
elemento de $B$, diz-se que $A$ é um \sub{subconjunto} de $B$, que
$A$ \sub{está contido} em $B$, ou que $A$ é \sub{parte} de $B$. Para
indicar esse fato, usa-se a notação $A \subset B$.

\end{definicao}

Quando $A$ não é um subconjunto de $B$, escreve-se $A \not\subset
B$. Em outras palavras, existe pelo menos um elemento $a$ tal que $a
\in A$ e $a \notin B$.
\bigskip

Quando $A \subset B$, dizemos que $B$ \sub{contém} $A$ e escrevemos
$B \supset A$.


\end{frame}


%------------------------------------------------------------------------------------------------------------

\begin{frame}
\frametitle{A Relação de Inclusão} %\framesubtitle{Exemplos}

\begin{exemplo}
Sejam $T$ o conjunto de todos os triângulos e $P$ o conjunto dos
polígonos do plano. Todo triângulo é um polígono, logo $ T \subset
P$.
\end{exemplo}

\begin{exemplo}
Na Geometria, uma reta, um plano e o espaço são conjuntos. Seus
elementos são pontos.

Quando dizemos que uma reta $r$ está no plano $\Pi$, estamos
afirmando que $r$ está contida em $\Pi$ ou, equivalentemente, que
$r$ é um subconjunto de $\Pi$, pois todos os pontos que pertencem a
$r$ pertencem também a $\Pi$.

Nesse caso, deve-se escrever $ r \subset \Pi$. Porém, não é correto
dizer que $r$ pertence a $\Pi$, nem escrever $r \in \Pi$. Os
elementos do conjunto $\Pi$ são pontos e não retas.
\end{exemplo}

\end{frame}


%------------------------------------------------------------------------------------------------------------

\begin{frame}
\frametitle{A Relação de Inclusão} %\framesubtitle{Exemplos}

\begin{proposicao}[Inclusão universal do $\emptyset$]
Para todo conjunto $A$, vale $\emptyset \subset A$.
\end{proposicao}

\begin{definicao}
Dizemos que $A \neq \emptyset$ é um \sub{subconjunto próprio} de $B$
quando $A \subset B$  e $A \neq B$.
\end{definicao}



\end{frame}


%------------------------------------------------------------------------------------------------------------

\begin{frame}
\frametitle{A Relação de Inclusão} %\framesubtitle{Exemplos}
\begin{proposicao}[Propriedades da inclusão]
Sejam $A$, $B$ e $C$ conjuntos. Tem-se:
\begin{enumerate}[i.]
	\item \sub{Reflexividade}: $A \subset A$;
	\item \sub{Antissimetria}: Se $A \subset B$ e $B \subset A$,
	então $A = B$;
	\item \sub{Transitividade}: Se $A \subset B$ e $B \subset C$,
	então $A \subset C$.
\end{enumerate}
\end{proposicao}

Demonstração no quadro.


\end{frame}

%------------------------------------------------------------------------------------------------------------

\begin{frame}
\frametitle{A Relação de Inclusão} %\framesubtitle{Exemplos}

\begin{definicao}
Dado um conjunto $A$, chamamos de \sub{conjunto das partes} de $A$ o conjunto formado por todos
os seus subconjuntos, e denotamo-lo $\mathcal{P}(A)$.
\end{definicao}

\begin{exemplo}
Dado $A = \set {1, 2, 3}$, determine $\mathcal{P}(A)$.
\end{exemplo}



\end{frame}
%	\section{União e Interseção}


\begin{frame}
	\frametitle{União e Interseção de Conjuntos} 

	\begin{definicao}[União e Interseção]
		Dados os conjuntos $A$ e $B$:
		\begin{enumerate}[i.]
			\item A \sub{união} $A \uniao B$ é o conjunto formado pelos elementos que pertencem a pelo menos um dos conjuntos $A$ e $B$;
			\item A \sub{interseção} $A \inter B$ é o conjunto formado por elementos que pertencem a ambos $A$ e $B$.
		\end{enumerate}
	\end{definicao}\pause

	\begin{exemplo}
		Sejam $A = \conjunto{1, 2, 3}$ e $B = \conjunto{2, 5}$. Determine $A \uniao B$ e $A \inter B$.
	\end{exemplo}\pause

	\begin{definicao}[Conjuntos disjuntos]
		Sejam  $A$ e $B$ conjuntos. Dizemos que $A$ e $B$ são \sub{conjuntos disjuntos} quando $A \inter B = \vazio$.
	\end{definicao}
\end{frame}


\begin{frame}
	\frametitle{União e Interseção de Conjuntos}
	Algumas propriedades das operações de união e interseção de conjuntos dizem respeito a um conjunto chamado de \emph{conjunto universo}, que denotaremos por $\U$. Esse conjunto deve ser fixado a fim de que fique claro quais são os possíveis objetos que podem ser elementos dos conjuntos a serem abordados. Uma vez fixado $\universo$, todos os elementos considerados pertencerão a $\U$ e todos os conjuntos serão subconjuntos de $\U$. 

\begin{exemplo}
	Na geometria plana, $\universo$ é o plano onde os elementos são pontos, e todos os conjuntos são constituídos por pontos desse plano. As retas servem como exemplos desses conjuntos; portanto, são subconjuntos de $\U$ (não elementos! Conforme vimos no Exemplo \ref{exem:subconjuntosdoplano}).
\end{exemplo}
\end{frame}

\begin{frame}
	\frametitle{União e Interseção de Conjuntos} 
	
	\begin{proposicao}[Propriedades da união e interseção]
		\label{propuniaoint}
		Sejam $A$, $B$ e $C$ conjuntos. Tem-se:
		\begin{enumerate}[i.]
			\item $A \contido \paren{A \uniao B}$ e $\paren{A \inter B} \contido A$;
			\item \sub{União/interseção com o universo}: $A \uniao \U = \U$ e $A \inter \U = A$;
			\item \sub{União/interseção com o vazio}: $A \uniao \vazio = A$ e $A \inter \vazio = \vazio$;
			\item \sub{Comutatividade}: $A \uniao B = B \uniao A$ e $A \inter B = B \inter A$;
			\item \sub{Associatividade}: $(A \uniao B) \uniao C = A \uniao (B \uniao C)$ e $(A \inter B) \inter C = A \inter (B \inter C)$;
			\item \sub{Distributividade, de uma em relação à outra}: $A \inter
			(B \uniao C) = (A \inter B) \uniao (A \inter C)$ e $A \uniao (B \inter C) = (A \uniao B) \inter (A \uniao C)$.
		\end{enumerate}
	\end{proposicao} \pause



	\begin{exemplo}
		Sejam $A = \conjunto{1, 2, 3}$, $B = \conjunto{2, 5}$ e $C = \conjunto{3, 4}$. Calcule $A \inter \paren{B \uniao C}$, $(A \inter B) \uniao (A \inter C)$ e $(A \inter B) \uniao  C$.
	\end{exemplo}
\end{frame}

%	\section{Complementar}


\begin{frame}
\frametitle{O Complementar de um Conjunto} %\framesubtitle{Exemplos}

A noção de complementar de um conjunto só faz sentido quando fixamos
um \sub{conjunto universo}, que denotaremos por $\U$. Uma vez fixado
$\U$, todos os elementos considerados pertencerão a $\U$ e todos os
conjuntos serão subconjuntos de $\U$. Por exemplo, na geometria
plana, $\U$ é o plano.

\begin{definicao}
Dado um conjunto $A$ (isto é, um subconjunto de $\U$), chama-se
\sub{complementar} de $A$ ao conjunto $A^C$ formado pelos elementos
de $\U$ que não pertencem a $A$.
\end{definicao}

\begin{exemplo}
Seja $\U$ o conjunto dos triângulos. Qual o complementar do conjunto
dos triângulos escalenos?
\end{exemplo}

\end{frame}
%------------------------------------------------------------------------------------------------------------
\begin{frame}
\frametitle{O Complementar de um Conjunto} %\framesubtitle{Exemplos}
\begin{proposicao}[Propriedades do complementar] \label{prop-comp}
Fixado um conjunto universo $\U$, sejam $A$ e $B$ conjuntos. Tem-se:
\begin{enumerate}[i.]
	\item $\U^C = \emptyset$ e $\emptyset^C = \U$;
	\item $\left( A^C \right)^C = A$ (Todo conjunto é complementar do seu complementar);
	\item Se $A \subset B$ então $B^C \subset A^C$ (se um conjunto
	está contido em outro, seu complementar contém o complementar desse outro).
\end{enumerate}
\end{proposicao}

Demonstração no quadro.


\end{frame}

%------------------------------------------------------------------------------------------------------------
\begin{frame}
\frametitle{O Complementar de um Conjunto} %\framesubtitle{Exemplos}

\begin{definicao}
A \sub{diferença} entre dois conjuntos $A$ e $B$ é definida por:
$$ B \setminus A = \set {x \tq x \in B \text { e } x \notin A}.$$
\end{definicao}

\begin{itemize}
	\item Em geral, não temos $B \setminus A = A \backslash B$. Pense em um contraexemplo a essa
	igualdade.
	\item Note que $A^C = \U \setminus A$.
\end{itemize}

\end{frame}
%	
\section{Lógica}
\begin{frame}
\frametitle{Conjuntos e Lógica} %\framesubtitle{Exemplos}

Em toda essa seção, considere $P$ e $Q$ propriedades aplicáveis aos
elementos de $\U$. Considere também $A = \set {x \tq x \text{ possui
} P}$ e $B= \set {x \tq x \text{ possui } Q}$.

\begin{itemize}
	\item \sub{Inclusão e implicação}: $A \subset B$ é equivalente a
	$P \implies Q$.
	\item \sub{Igualdade e bi-implicação}: $A=B$ é equivalente a $P
	\iff Q$.
\end{itemize}
\end{frame}
%------------------------------------------------------------------------------------------------------------
\begin{frame}
\frametitle{Conjuntos e Lógica} %\framesubtitle{Exemplos}

\begin{exemplo}
Analise as implicações abaixo:
\begin{align*}
x^2+1=0 & \implies \left(x^2 +1 \right) \left( x^2-1 \right) = 0
\cdot \left( x^2-1 \right) \\
& \implies x^4 - 1 = 0 \\
& \implies x^4 = 1 \\
& \implies x \in \set {-1, 1}
\end{align*}

Isso quer dizer que o conjunto solução de $x^2 +1 = 0$ é $\set{-1,
1}$?
\end{exemplo}

\end{frame}
%------------------------------------------------------------------------------------------------------------
\begin{frame}
\frametitle{Conjuntos e Lógica} %\framesubtitle{Exemplos}

\begin{itemize}
	\item \sub{Complementar e negação}: $A^C$ é equivalente a $\sim P$;
	\item Podemos combinar os itens (ii) e (iii) da Proposição
	\ref{prop-comp} (Propriedades do complementar) e obter que $$P
	\implies Q \text{ se, e somente se, }\sim Q \implies \sim P.$$
	Chamamos $\sim Q \implies \sim P$ de \sub{contrapositiva} de $P
	\implies Q$.
	\item Chamamos $Q \implies P$ de \sub{recíproca} de $P \implies
	Q$ e $P \land \sim Q$ de \sub{negação} de $P \implies Q$.
\end{itemize}
%Exemplos no Exercício \ref{exrec}.
\end{frame}
%------------------------------------------------------------------------------------------------------------
\begin{frame}
\frametitle{Conjuntos e Lógica} %\framesubtitle{Exemplos}

\begin{exemplo}
Observe as afirmações abaixo:
\begin{itemize}
	\item Todo número primo maior do que 2 é ímpar;
	\item Todo número par maior do que 2 é composto.
\end{itemize}

Essas afirmações dizem exatamente a mesma coisa, ou seja, exprimem a
mesma ideia, só que com diferentes termos. Podemos reescrevê-las na
forma de implicações vendo claramente que uma é a contrapositiva da
outra, todas sob a hipótese que  $n \in \N$, $n>2$:
\begin{align*}
n \text{ primo } & \implies n \text{ ímpar } \\
\sim \left( n \text{ ímpar } \right) & \implies \sim \left( n \text{
primo } \right) \\
n \text{ par } & \implies n \text{ composto }
\end{align*}
\end{exemplo}

\end{frame}
%------------------------------------------------------------------------------------------------------------

\begin{frame}
\frametitle{Conjuntos e Lógica} %\framesubtitle{Exemplos}

\begin{itemize}
	\item \sub{União e disjunção}: $A \cup B$ é equivalente a $P \lor
	Q$ ($P$ ou $Q$).
	\item \sub{Interseção e conjunção}: $A \cap B$ é equivalente a $P
	\land Q$ ($P$ e $Q$).
\end{itemize}

\begin{observacao}
O conectivo lógico \sub{ou} tem significado diferente do usado
normalmente no português. Na linguagem coloquial, usamos $P$
\sub{ou} $Q$ sem permitir que sejam as duas coisas ao mesmo tempo.
Analisem a seguinte história:

Um obstetra que também era matemático acabara de realizar um parto
quando o pai perguntou: ``é menino ou menina, doutor?''. E ele
respondeu: ``sim''.
\end{observacao}
\end{frame}
%------------------------------------------------------------------------------------------------------------

\begin{frame}
\frametitle{Conjuntos e Lógica} \framesubtitle{Resumo}
\begin{center}
\begin{tabular}{|c|c|}
	\hline
	% after \\: \hline or \cline{col1-col2} \cline{col3-col4} ...
	$A=B$ & $P \iff Q$ \\ \hline
	$A \subset B$ & $P \implies Q$ \\ \hline
	$A^C$ & $\sim P$ \\ \hline
	$A \cup B$ & $P \lor Q$ \\ \hline
	$A \cap B$ & $P \land Q$ \\
	\hline
\end{tabular}
\end{center}
\end{frame}
%------------------------------------------------------------------------------------------------------------

\begin{frame}
\frametitle{Conjuntos e Lógica} %\framesubtitle{Resumo}
Problema: A
polícia prende quatro homens, um dos quais cometeu um furto. Eles fazem
as seguintes declarações:
\begin{itemize}
	\item Arnaldo: Bernaldo fez o furto.
	\item Bernaldo: Cernaldo fez o furto.
	\item Dernaldo: eu não fiz o furto.
	\item Cernaldo: Bernaldo mente ao dizer que eu fiz o furto.
\end{itemize}
Se sabemos que só uma destas declarações é a verdadeira, quem é
culpado pelo furto?

\end{frame}

%	
\section{Conjuntos Numéricos}
\frame { \frametitle{Naturais}
\begin{definicao}
Ao conjunto $\N = \set {0, 1, 2, \dots , n, n+1, \dots}$ damos o
nome de \sub{conjunto dos números naturais}.
\end{definicao}\pause
\begin{itemize}
\item Denotamos $\N \setminus \set 0 = \set {1, 2, \dots , n, n+1,
\dots}$ por $\N ^\ast$.

\item Usamos o conjunto dos números naturais para contar coisas, como
casas, animais, etc.
\end{itemize}
}

%------------------------------------------------------------------------------------------------------------


\begin{frame}
\frametitle{Inteiros} 
\begin{definicao}
Ao conjunto $\Z =\set {\dots , -m -1, -m, \dots, -1, 0, 1,  \dots ,
n, n+1, \dots}$ damos o nome de \sub{conjunto dos números inteiros}.
\end{definicao}\pause

\begin{block}{Notação}
$\Z^\ast = \Z \setminus \set 0$; \\
$\Z_+ = \N$ (Inteiros não negativos); \\
$\Z^\ast_+ =\N ^\ast$ (Inteiros positivos); \\
$\Z_- =\set {\dots , -m -1, -m, \dots, -1, 0}$ (Inteiros não
positivos); \\
$\Z_-^\ast =\Z_- \setminus \set 0$ (Inteiros negativos).
\end{block}
\end{frame}



%------------------------------------------------------------------------------------------------------------
\begin{frame}
\frametitle{Racionais} 
\begin{definicao}
Ao conjunto $\Q = \set{\frac p q \tq p, q \in \Z \text{ e } q \neq
0}$ damos o nome de \sub{conjunto dos números racionais}.
\end{definicao}\pause

A representação decimal de um número racional é finita ou é uma
dízima periódica (infinita).
\begin{exemplo}
Reescreva as frações $\dfrac{12}{30}$ e $\dfrac 3 9 $ em forma de número decimal. Além disso, reescreva os números $0{,}6$; $1{,}37$; $0{,}222\dots$; $0{,}313131 \dots$ e $1{,}123123123 \dots$ em forma de fração irredutível, ou seja, já simplificada.
\end{exemplo}
\end{frame}



%------------------------------------------------------------------------------------------------------------

%	\section{Atividade Online}


\begin{frame}
    \frametitle{Atividade Online}

    \link{https://pt.khanacademy.org/math/pt-8-ano/numeros-8ano/pt-dzimas-peridicas/e/converting_repeating_decimals_to_fractions_2}{Atividade 02 - Conversão de Dízimas Periódicas \\ Compostas em Frações}

\end{frame}
%	\section{Conjuntos Numéricos}
\begin{frame}
    \frametitle{Irracionais} 
    \begin{definicao}
    O \sub{conjunto dos números irracionais} é constituído por todos os
    números que possuem uma representação decimal infinita e não
    periódica.
    \end{definicao}\pause
    
    \begin{exemplo}
    $\sqrt 2$, $e$ e $\pi$ são números irracionais.
    \end{exemplo}
    
    Você sabia que existem infinitos ``maiores'' que outros? Qual
    conjunto você diria que tem mais elementos: racionais ou
    irracionais?
    \end{frame}
    
    %------------------------------------------------------------------------------------------------------------
    
    \begin{frame}
    \frametitle{Problema} 
    
    O Grande Hotel Georg Cantor tinha uma infinidade de quartos,
    numerados consecutivamente, um para cada número natural. Todos eram
    igualmente confortáveis. Num fim de semana prolongado, o hotel
    estava com seus quartos todos ocupados, quando chega um visitante. A
    recepcionista vai logo dizendo: \\
    -Sinto muito, mas não há vagas. \\
    Ouvindo isto, o gerente interveio: \\
    -Podemos abrigar o cavalheiro sim, senhora. \\
    E ordena:\pause \\ 
    Transfira o hóspede do quarto 1 para o quarto 2, passe o do quarto 2
    para o quarto 3 e assim por diante. Quem estiver no quarto $n$, mude
    para o quarto $n+1$. Isto manterá todos alojados e deixará
    disponível o quarto 1 para o recém chegado. \pause Logo depois chegou um
    ônibus com 30 passageiros, todos querendo hospedagem. Como deve
    proceder a recepcionista para acomodar todos? \pause
    \\ Horas depois, chegou um trem com uma infinidade de
    passageiros. Como proceder para acomodá-los?
    
    
    \end{frame}
    
    %------------------------------------------------------------------------------------------------------------
%	
\section{Atividade Online}
\begin{frame}
\frametitle{Atividade Online} 

\href{https://pt.khanacademy.org/math/algebra/x2f8bb11595b61c86:irrational-numbers/x2f8bb11595b61c86:irrational-numbers-intro/e/recognizing-rational-and-irrational-numbers}
{{\tt Atividade 03 - Classifique Números: Racionais e \\ Irracionais}}

\href{https://pt.khanacademy.org/math/algebra/x2f8bb11595b61c86:irrational-numbers/x2f8bb11595b61c86:sums-and-products-of-rational-and-irrational-numbers/e/recognizing-rational-and-irrational-expressions}
{{\tt Atividade 04 - Expressões Racionais Versus Irracionais}}




\end{frame}


%	\section{Conjuntos Numéricos}
\begin{frame}
    \frametitle{Reais} 
    \begin{definicao}
    À reunião de $\Q$ com o conjunto dos números irracionais, nomeamos
    de \sub{conjunto dos números reais}. Denotamos por $\R$.
    \end{definicao}\pause
    
    \begin{itemize}
    \item $\R \setminus \Q = \set {x \tq x \text{ é irracional}}$;
    \item Usamos os números reais para medir algo. A cada número real
    está associado um ponto na reta graduada e vice-versa;
    \item Entre dois números reais distintos sempre há pelo menos um número racional e um
    irracional.
    \href{https://pt.khanacademy.org/math/algebra/rational-and-irrational-numbers/proofs-concerning-irrational-numbers/v/proof-that-there-is-an-irrational-number-between-any-two-rational-numbers}
    {{\tt Este vídeo}} do Khan Academy mostra que entre dois racionais
    distintos sempre há pelo menos um número irracional;
    \item A igualdade $0,999\dots = 1 $ é verdadeira?
    \end{itemize}
    \end{frame}
    
    
    
    %------------------------------------------------------------------------------------------------------------
    \begin{frame}
        \frametitle{Intervalos Reais} 
        \begin{definicao}
            Sejam $a, b \in \R$ tais que $a < b$. Definimos o \sub{intervalo aberto de $a$ a $b$}, denotado por $(a, b)$, como sendo o seguinte subconjunto de $\R$:
            $$ (a,b) = \{ x \in \R ; \  a < x < b \}.$$
            Definimos o \sub{intervalo fechado de $a$ a $b$}, denotado por $[a, b]$, como sendo o seguinte subconjunto de $\R$:
            $$ [a,b] = \{ x \in \R ; \  a \leq x \leq b \}.$$


        \end{definicao}
        \end{frame}

        %------------------------------------------------------------------------------------------------------------
    \begin{frame}
        \frametitle{Intervalos Reais} 
        Além dos intervalos da definição anterior, nas mesmas condições, temos os seguintes:
        
        \begin{itemize}
        \item $ (a,b] = \{ x \in \R ; \  a < x \leq b \}$;
        \item $ [a,b) = \{ x \in \R ; \  a \leq x < b \}$; \pause
        \item $ (a,+\infty) = \{ x \in \R ; \   x > a \}$;
        \item $ [a,+\infty) = \{ x \in \R ; \   x \geq a \}$;
        \item $ (-\infty,a) = \{ x \in \R ; \   x < a \}$;
        \item $ (-\infty,a] = \{ x \in \R ; \   x \leq a \}$;
        \item $ (-\infty, +\infty) = \R$.
        \end{itemize}
        \end{frame}
        
        
        
        %------------------------------------------------------------------------------------------------------------
    \begin{frame}
    \frametitle{Complexos} 
    \begin{definicao}
    Chamamos $i = \sqrt {-1}$ de \sub{número imaginário}, e ao conjunto
    $\C = \set{ a+bi \tq a,b \in \R}$ damos o nome de \sub{conjunto dos
    números complexos}.
    \end{definicao}\pause
    
    Seja $a+bi \in \C$. Nomeamos o número $a-bi$ de \sub{conjugado} de
    $a+bi$.
    
    Temos a seguinte cadeia de inclusões próprias: $\N \contidoproprio \Z
    \contidoproprio \Q \contidoproprio \R \contidoproprio \C$.
    \end{frame}
    
    
    
    %------------------------------------------------------------------------------------------------------------
%	
\section{Operações}
\begin{frame}
\frametitle{Operações} %\framesubtitle{Exemplos}
Definimos duas operações básicas com os elementos dos conjuntos
numéricos: a adição e a multiplicação. A subtração e a divisão
provêm da adição e da multiplicação, respectivamente.
\begin{itemize}
	\item Adição
		\begin{itemize}
		\item Subtração: é a soma de números negativos;
		\end{itemize}
	\item Multiplicação
		\begin{itemize}
		\item Divisão: é a multiplicação de números da forma $\frac 1
		q$.
		\end{itemize}
\end{itemize}

Você está bem treinado nas operações com frações? Dê uma treinada
\href{https://pt.khanacademy.org/math/arithmetic-home/arith-review-fractions}
{{\tt aqui}} no Khan Academy!
\end{frame}



%------------------------------------------------------------------------------------------------------------
%	
\section{Potenciação}
\begin{frame}
\frametitle{Potenciação} 
\begin{definicao}
A \sub{potência} $n \in \N^\ast$ de um número real $a$ é definida
como sendo a multiplicação de $a$ por ele mesmo $n$ vezes, ou seja:
$$a^n = \underbrace{a \cdot a  \dots  a}_{n \text{
vezes}}.$$
\end{definicao}\pause

\begin{definicao}
Quando $a \neq 0$, $a^0 = 1$. $0^0$ é uma indeterminação; \\
$a^{-n} = \frac{1}{a^n}$; \\
$a^{1/n} = \sqrt[n] a$, para $n> 0$.
\end{definicao}

É importante ressaltar que é comum definir $0^0 =1$ dependendo da
abordagem que se quer com as potências. Saiba mais
\href{https://pt.wikipedia.org/wiki/Zero_elevado_a_zero}{{\tt
aqui}}.
\end{frame}



%------------------------------------------------------------------------------------------------------------
\begin{frame}
\frametitle{Potenciação} 
\begin{proposicao}[Propriedades]
Sejam $a, b, n, m \in \R$ a menos que se diga o contrário.
\begin{enumerate}[i.]
	\item $a^m \cdot a^n = a^{m+n}$;
	\item $\frac {a^m}{a^n} = a^{m-n}$, $a \neq 0$;
	\item $\paren{a^m}^n = a^{m\cdot n}$;
	\item $a^{m^n} = a^{\overbrace{m \cdot m  \dots  m}^{n \text{
	vezes}}}$, $n \in \N^\ast$;
	\item $\paren{a \cdot b }^n= a^n \cdot b^n$;
	\item $\paren{\frac a b }^n = \frac {a^n} {b^n}$;
	\item $a^{m/n} = \sqrt[n]{a^m}$, $n \neq 0$.
\end{enumerate}
\end{proposicao}

\end{frame}



%------------------------------------------------------------------------------------------------------------
\begin{frame}
\frametitle{Potenciação} 
\begin{observacao}
Seja $a \in \R$. Temos que $\sqrt{a^2} = \modu a$. Mais geralmente,
$\sqrt[n] {a^n} = \modu a$ para $n$ par. \\
É errado dizer que $\sqrt 4 = \pm 2$. O correto é $\sqrt 4 = 2$,
mesmo que escrevas $\sqrt 4 = \sqrt{\paren {-2}^2}$. \\
Tal erro é comum, e o fator de confusão é que responder o conjunto
solução da equação $x^2=4$ não é equivalente a responder qual a raiz
de $4$, e sim responder quais números que elevados ao quadrado são
iguais a $4$.
\end{observacao}
\end{frame}



%------------------------------------------------------------------------------------------------------------


%	

\section{Atividade Online}
\begin{frame}
\frametitle{Atividade Online} 

\href{https://pt.khanacademy.org/math/algebra/rational-exponents-and-radicals/rational-exponents-and-the-properties-of-exponents/e/exponents_4}{{\tt Atividade 05 - Introdução às Propriedades da \\Potenciação  (Expoentes Racionais)}}

\href{https://pt.khanacademy.org/math/algebra2/x2ec2f6f830c9fb89:exp/x2ec2f6f830c9fb89:exp-properties/e/rational-exp-prop-challenge}{{\tt Atividade 06 - Propriedades da Potenciação (Expoentes \\ Racionais)}}


\href{https://pt.khanacademy.org/math/algebra/rational-exponents-and-radicals/alg1-simplify-square-roots/e/multiplying_radicals}
{{\tt Atividade 07 - Simplifique Raízes Quadradas (Variáveis)}}

\href{https://pt.khanacademy.org/math/algebra/rational-exponents-and-radicals/alg1-simplify-square-roots/e/adding_and_subtracting_radicals}
{{\tt Atividade 08 - Simplifique Expressões de Raiz Quadrada}}


\end{frame}

%------------------------------------------------------------------------------------------------------------

	
\section{Exercícios}
\begin{frame}
\frametitle{Exercícios} 

\Ex{ Determine, caso exista, a matriz $A$, tal que $AB = C$, em que
\begin{displaymath} B = 
	\begin{bmatrix}
		1 & -1 \\
		2 & 2 \\
		1 & 0
	\end{bmatrix}
	\space \text{ e } \space C=
	\begin{bmatrix}
		3 & 1  \\
		-1 & 4 
	\end{bmatrix}.
\end{displaymath}
}

\Ex{ Sejam $A$ e $B$ matrizes $m \times n$ e $n \times p$ respectivamente. A afirmação abaixo é sempre válida?

\begin{center}
	Se $AB=0_{m \times p}$, então $A = 0_{m \times n}$ ou $B = 0_{n \times p}$.
\end{center}
 }




\end{frame}





%------------------------------------------------------------------------------------------------------------

\begin{frame}
\frametitle{Exercícios} 

\Ex{
	Encontre uma matriz $A_{2 \times 2}$, não-nula, tal que $AA = 0_{2 \times 2}$.
}
	
\Ex{
	Determine as soluções dos seguintes sistemas lineares:
	\begin{enumerate}[a)]
		\item \[\left\{
			\begin{array}{rrrrrrr}
			x &-& 2y &-& 3z& = &0 \\
			3x& +& y&- &z &=& -1 
			\end{array} \right.;\]
		\item \[\left\{
			\begin{array}{rrrrrrr}
			x &+& y &+& z& = &2 \\
			2x& -& y&+ &3z &=& 9 \\
			x & + & 2y & - & z & = & -3 
			\end{array} \right.;\]
		\item \[\left\{
			\begin{array}{rrrrrrrrr}
			x_1 & - & 2x_2 & + & x_3& + & 2x_4 & = &1 \\
			x_1 & + & x_2 & - & x_3 & + & x_4 &=& 2 \\
			x_1 & + & 7x_2 & - & 5x_3 & - & x_4 &= & 3 
			\end{array} \right..\]
	\end{enumerate}
}


\end{frame}


%------------------------------------------------------------------------------------------------------------

\begin{frame}
\frametitle{Exercícios} 

\Ex{
	Seja 
	\begin{displaymath} A = 
		\begin{bmatrix}
			3 & -6 & 2 & -1 \\
			-2 & 4 & 1 & 3 \\
			0 & 0 & 1 & 1 \\
			1 & -2 & 1 & 0
		\end{bmatrix}.
	\end{displaymath}
	Para quais matrizes $B_{4 \times 1}$, o sistema $AX = B$ tem solução?
}

\Ex{
	Mostre que, se $A$ e $B$ são matrizes $n \times n$, ambas invertíveis, então $AB$ é invertível e vale a igualdade $(AB)^{-1} = B^{-1}A^{-1}$.
}

\Ex{
	Sejam $A$ e $B$ matrizes quadradas de ordem $n$. Mostre, através de um contra-exemplo, que a seguinte igualdade não é sempre válida:
	$$\det (A+B) = \det A + \det B.$$
}


\end{frame}


%------------------------------------------------------------------------------------------------------------

\begin{frame}
	\frametitle{Exercícios}

\Ex{
Calcule a matriz inversa, se existir, e o determinante das matrizes abaixo:
\begin{enumerate}[a)]
	\item \begin{displaymath}  
		\begin{bmatrix}
			0 & 3 & 1  \\
			1 & 1 & 2  \\
			3 & 2 & 4 
		\end{bmatrix};
	\end{displaymath}
	\item \begin{displaymath}  
		\begin{bmatrix}
			3 & -6 & 9  \\
			-2 & 7 & -2  \\
			0 & 1 & 5 
		\end{bmatrix};
	\end{displaymath}
	\item \begin{displaymath}  
		\begin{bmatrix}
			4 & 0 & 0 & 1 & 0 \\
			3 & 3 & 3 & -1 & 0 \\
			1 & 2 & 4 & 2 & 3 \\
			9 & 4 & 6 & 2 & 3 \\
			2 & 2 & 4 & 2 & 3
		\end{bmatrix}.
	\end{displaymath}
\end{enumerate}
}
\end{frame}


%------------------------------------------------------------------------------------------------------------

\begin{frame}
	\frametitle{Exercícios} 

\Ex{
	Considere a matriz 
        \begin{displaymath} A = 
            \begin{bmatrix}
                2 & 1 & 0 \\
                1 & -2 & -1 \\
                3 & -5 & -3
            \end{bmatrix}.
        \end{displaymath}

        Calcule $A^{-1}$, caso $A$ seja invertível, e calcule as soluções do sistema linear $AX=B$, onde $B = 0_{3 \times 1}$.
}

\Ex{
	Considere a matriz 
        \begin{displaymath} A = 
            \begin{bmatrix}
                -1 & 11 & 4 \\
                -2 & 1 & 5 \\
                1 & 3 & -2
            \end{bmatrix}.
        \end{displaymath}

        Calcule $\det A$ e para quais matrizes $B_{3 \times 1}$ o sistema $AX=B$ tem solução.
}


\end{frame}


%------------------------------------------------------------------------------------------------------------

\begin{frame}
	\frametitle{Exercícios} 


\Ex{
	Considere a matriz 
        \begin{displaymath} A = 
            \begin{bmatrix}
			-2 & 3 & 1 & 2 & -1 \\
			-3 & -9 & -3 & -2 & 3 \\
			1 & 1 & 4 & -1 & -1 \\
			3 & 0 & -2 & -1 & 2 \\
			1 & 3 & 1 & 0 & -1
		\end{bmatrix}.
        \end{displaymath}

        Calcule $\det A$ e, caso $A$ seja invertível, $\det A^{-1}$. Além disso, para quais matrizes $B_{5 \times 1}$ o sistema linear $AX=B$ possui alguma solução?
}


\end{frame}


%------------------------------------------------------------------------------------------------------------

\begin{frame}
	\frametitle{Exercícios} 


\Ex{
	Considere as Matrizes 
        \begin{displaymath} A = 
		\begin{bmatrix}
			3 & 0 & 0 & 0 \\
			-2 & 4 & 0 & 0 \\
			0 & 0 & 1 & 0 \\
			2 & -2 & 2 & -2
		\end{bmatrix} \text{ e } 
        B = 
		\begin{bmatrix}
			-1 & -2 & 2 & -1 \\
			0 & 2 & 1 & 3 \\
			0 & 0 & 3 & 1 \\
			0 & 0 & 0 & 1
		\end{bmatrix}.
	\end{displaymath}
 Calcule:
        \begin{enumerate}[a)]
            \item  $A\cdot B$;
            \item  $\det (A \cdot B)$;
            \item  $\det (A + B)$;
            \item  A inversa da matriz $A + B$, caso exista;
            \item  A solução do sistema $AX = C$, onde \begin{displaymath} C = 
		\begin{bmatrix}
			 3 \\
			 2 \\
			 1 \\
			 0
		\end{bmatrix}.
	\end{displaymath}
        \end{enumerate}
}


\end{frame}


%------------------------------------------------------------------------------------------------------------

\begin{frame}
	\frametitle{Exercícios} 

	\Ex{
		Considere a matriz 
        \begin{displaymath} A = 
            \begin{bmatrix}
                3 & -7 & 2 & -1\\
                -2 & 7 & 1 & 3\\
                0 & 1 & 1 & 1\\
                1 & -2 & 1 & 0
            \end{bmatrix}.
        \end{displaymath}
        Responda, em qualquer ordem, as perguntas abaixo:
		\begin{enumerate}[a)]
			\item Qual o conjunto solução do sistema linear homogêneo $AX=B_{4 \times 1}$?
			\item A matriz $A$ é invertível? Se sim, qual a sua inversa?
			\item  Qual o determinante de $A$?
		\end{enumerate}
	}

\end{frame}


%------------------------------------------------------------------------------------------------------------

\begin{frame}
	\frametitle{Exercícios} 

	\Ex{
		Encontre uma matriz quadrada $A$, de ordem 2 e não-nula, tal que $AA = 0_{2 \times 2}$. Em seguida, SEM USAR A FÓRMULA da Definição \ref{detbasicos}, mostre que $\det A = 0$. Para finalizar, para quais matrizes $B_{2 \times 1}$ a solução do sistema $AX = B$ é única?
	}

	\Ex{
		Através do uso das operações elementares, mostre que:
		\begin{displaymath}  
			\begin{vmatrix}
				1 & 1 & 1  \\
				a & b & c  \\
				a^2 & b^2 & c^2 
			\end{vmatrix} = \paren{b-a} \paren{c-a} \paren{c-b}.
		\end{displaymath}
	}

	\Ex{
	Prove a Proposição \ref{det-mat-triang}, ou seja, que o determinante de uma matriz triangular é igual ao produto das entradas de sua diagonal.
	}
	
	\Ex{
		Seja $A$ uma matriz $n \times n$ com mais de $n^2-n$ entradas nulas. Mostre que $\det A = 0$. 
	}
	
	\Ex{
		Sejam $A$ e $B$ matrizes quadradas de ordem $n$ linha equivalentes. Mostre que, se $\det A = 0$, então $\det B = 0$. No caso de $\det A \neq 0$, sob que condições teremos $\det A = \det B$?
	}
	
	\end{frame}
	
	
	%------------------------------------------------------------------------------------------------------------


	\section{Bibliografia}


\begin{frame}
    \frametitle{Bibliografia}

    \begin{thebibliography}{99}
        \bibitem {label1}
        LIMA, Ronaldo F.
        \newblock \emph{Álgebra Linear Essencial}.
        \newblock Acesso em \link{https://www.ronaldofreiredelima.com}{www.ronaldofreiredelima.com}


       \bibitem {label2}
       BOLDRINI, José L. (et al). 
        \newblock \emph{Álgebra linear}.
        \newblock 3. ed. São Paulo, SP: Harbra, 1986.
    \end{thebibliography}
\end{frame}


\end{document}
