\section{Operações Elementares}

\begin{frame}
    \frametitle{Operações Elementares sobre Matrizes}

    Algumas operações sobre as linhas de uma matriz são ditas \sub{elementares}. São elas:
    \begin{itemize}
        \item $(l_i \sselogico l_j)$: Troca de posição entre duas linhas  $l_i$ e $l_j$;
        \item $(l_i \to \lambda l_i)$: Multiplicação de uma linha $l_i$ por um escalar $\lambda \neq 0$;
        \item $(l_j \to l_j + \lambda l_i)$: Substituição de uma linha $l_j$ por $l_j +\lambda l_i$, sendo $\lambda \neq 0$.\pause
    \end{itemize}
    
    \begin{definicao}[Equivalência por linhas]
        Diz-se que uma matriz $B$ é \sub{linha equivalente} a uma matriz $A$ quando $B$ é obtida de $A$ efetuando-se nesta uma sequência de operações elementares.
    \end{definicao}\pause

    \begin{proposicao}
        Dois sistemas lineares $AX=B$ e $A'X=B'$ são equivalentes se suas matrizes aumentadas $(A|B)$ e $(A'|B')$ são linha equivalentes.
    \end{proposicao}

\end{frame}

%------------------------------------------------------------------------

\begin{frame}
    \frametitle{Operações Elementares sobre Matrizes}

    \begin{definicao}[Matriz escalonada]
        Diz-se que uma matriz $A= (a_{ij})_{m \times n}$ é \sub{escalonada} quando cumpre as seguintes condições:
        \begin{itemize}
            \item O primeiro elemento não-nulo de uma linha está à esquerda do primeiro elemento não-nulo da linha subsequente;
            \item As linhas nulas, caso existam, estão abaixo das demais.
        \end{itemize}
    \end{definicao}\pause

    Para resolvermos um sistema linear, fazemos o \sub{escalonamento} da matriz aumentada do sistema a fim de obtermos uma matriz escalonada que seja linha equivalente à matriz aumentada do sistema. Esse procedimento é chamado de \sub{Método de Gauss} ou \sub{Eliminação Gaussiana}.
    

    
\end{frame}

%------------------------------------------------------------------------

\begin{frame}
    \frametitle{Operações Elementares sobre Matrizes}

    \begin{exemplo}
        Encontre o conjunto solução para o sistema

        \[\left\{
        \begin{array}{rrrrrrr}
        x &-&2y &+ &3z &= &1 \\
        2x &+& y &-&z &= &2 \\
        4x &-&3y &+&5z &= &4  
        \end{array} \right..\]
    \end{exemplo} \pause
    
    \begin{exemplo}
        Encontre o conjunto solução para o sistema

        $$\left\{
        \begin{array}{rrrrrrrrr}
        2x &-&y & & &+ &4t &= &9 \\
        x &+& y &-&z & + & 2t&= &7 \\
        -x &+& 2y &+&z & - & t&= &3 \\
         & &4y &-&z &+ & 3t&= &13  
        \end{array} \right. .$$
    \end{exemplo}
    

\end{frame}

%------------------------------------------------------------------------

