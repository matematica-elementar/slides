\section{Matrizes Quadradas}


\begin{frame}
    \frametitle{Tipos de Matrizes Quadradas}
    
    \begin{definicao}[Matrizes triangulares e diagonais]
        Uma matriz quadrada $A=(a_{ij})_{n\times n}$ é dita \sub{triangular}, quando ocorre uma das possibilidades:
        
        $$a_{ij} = 0 \ \ \forall i>j \ \  \text{ ou } \ \  a_{ij} = 0 \ \  \forall i<j.$$

        No primeiro caso, ela é dita \sub{triangular superior} e, no segundo, \sub{triangular inferior}. Uma matriz que é triangular superior e inferior é dita \sub{diagonal}.
    \end{definicao}\pause

    \begin{exemplo}
        As matrizes $A$, $B$ e $C$ abaixo são, respectivamente, triangular superior, inferior e diagonal.
        
        \begin{displaymath} A = 
            \begin{bmatrix}
                3 & 0 & 2 \\
                0 & 0 & 1 \\
                0 & 0 & 1
            \end{bmatrix}
        ,    
        B=
            \begin{bmatrix}
                1 & 0 & 0 \\
                7 & 2 & 0 \\
                -2 & 0 & 0
            \end{bmatrix}
        \text{ e } C = 
            \begin{bmatrix}
                1 & 0  \\
                0 & 2  
            \end{bmatrix}.
        \end{displaymath}
    \end{exemplo}
\end{frame}

%------------------------------------------------------------------------

\begin{frame}
    \frametitle{Tipos de Matrizes Quadradas}
    
    \begin{definicao}[Matriz Identidade]
        Uma matriz diagonal $n\times n$ cujas entradas não obrigatoriamente nulas são todas iguais a $1$ é chamada de \sub{matriz identidade} de ordem $n$, a qual denota-se por $I_n$ ou, simplesmente, por $I$.
    \end{definicao}\pause
    A matriz identidade de ordem $n$ é o elemento neutro da multiplicação de matrizes quadradas, pois, para toda matriz quadrada $A$ de ordem $n$, vale a igualdade: $$AI = IA = A.$$

    Além disso, dada uma matriz $B_{n \times m}$, vale também: $$BI_m = I_nB = B.$$

\end{frame}

%------------------------------------------------------------------------

\begin{frame}
    \frametitle{Tipos de Matrizes Quadradas}
    
    \begin{definicao}[Matrizes invertíveis]
        Diz-se que uma matriz quadrada $A$ é  \sub{invertível}, quando existe uma matriz quadrada $B$, de mesma ordem que $A$, tal que $$AB = BA = I.$$
    \end{definicao}\pause

    Prova-se que $A$, quando invertível, possui uma única matriz inversa. Tal matriz é dita a \sub{inversa} de $A$, e denotada por $A^{-1}$.\pause


    \begin{exemplo}
        Verifique que as matrizes abaixo são invertíveis multiplicando-as.
        
        \begin{displaymath} A = 
            \begin{bmatrix}
                1 & 0  \\
                0 & \dfrac 1 2
            \end{bmatrix}
        \text{ e } B = 
            \begin{bmatrix}
                1 & 0  \\
                0 & 2  
            \end{bmatrix}.
        \end{displaymath}
    \end{exemplo}
\end{frame}

%------------------------------------------------------------------------

\begin{frame}
    \frametitle{Tipos de Matrizes Quadradas}

    Para calcular a inversa de uma matriz $A$, escalona-se a matriz aumentada $(A|I)$ a fim de se obter uma matriz equivalente por linhas do tipo $(I|B)$. Se for possível tal procedimento, então $B = A^{-1}$. Caso contrário, $A$ não é invertível.
    
    \begin{exemplo}\label{exinversa}
        Calcule a inversa das matrizes abaixo
        
        \begin{displaymath} A = 
            \begin{bmatrix}
                1 & 1  \\
                0 & 2
            \end{bmatrix}
        \text{ e } B = 
            \begin{bmatrix}
                1 & 2 & 1 \\
                0 & 1 & 1 \\
                2 & 4 & 2 
            \end{bmatrix}.
        \end{displaymath}
    \end{exemplo}\pause

    \begin{exemplo}
        Qual a solução do sistema abaixo?
        $$\left\{
        \begin{array}{rrrrr}
        x &+& y & = &2 \\
        & & 2y& =& 4 
        \end{array} \right.$$
    \end{exemplo}
\end{frame}

%------------------------------------------------------------------------