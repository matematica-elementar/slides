\section{Matrizes}


\begin{frame}
    \frametitle{Definição de Matriz}
    
    \begin{definicao}
        Sejam $m, n \pertence \N^\ast$. Uma \sub{matriz (real)} do \sub{tipo $m \times n$} (lê-se: $m$ por $n$) é uma \aspas{tabela} disposta em $m$ linhas e $n$ colunas. Denotamos os números reais que formam a $i$-ésima linha de uma matriz $A$ por $a_{i1}, a_{i2}, \dots , a_{in}$ e sua $j$-ésima coluna por $a_{1j}, a_{2j}, \dots , a_{mj}$. Assim, escrevemos:
        \begin{displaymath} A = 
            \begin{bmatrix}
                a_{11} & \dots & a_{1n} \\
                \vdots & \ddots & \vdots \\
                a_{m1} & \dots & a_{mn}
            \end{bmatrix}.
        \end{displaymath}
        Chamamos de \sub{entradas}, de uma matriz $A$, os reais $a_{ij}$ que a compõem.
        Poderemos indicar uma matriz $A$ do tipo $m \times n$ com entradas $a_{ij}$ por
        $$ A = \paren{a_{ij}}_{m \times n}$$
        ou, simplesmente, $A = \paren{a_{ij}}$.
    \end{definicao}
\end{frame}

%------------------------------------------------------------------------

\begin{frame}
    \frametitle{Definição de Matriz}

    \begin{exemplo}
        As matrizes $A = \paren{a_{ij}}_{3 \times 2}$, em que $a_{ij}=i+j$ e, $B = \paren{b_{ij}}_{ 2 \times 4}$, em que $b_{ij} = i^j$, são:
        \begin{displaymath} A = 
            \begin{bmatrix}
                2 & 3 \\
                3 & 4 \\
                4 & 5
            \end{bmatrix}
            \space \text{ e } \space B=
            \begin{bmatrix}
                1 & 1 & 1 & 1 \\
                2 & 4 & 8 & 16
            \end{bmatrix}.
        \end{displaymath}
    \end{exemplo}

\end{frame}

%------------------------------------------------------------------------

\begin{frame}
    \frametitle{Definição de Matriz}

    \begin{definicao}
        Uma matriz $A$ do tipo $n \times n$, é dita \sub{quadrada de ordem $n$}. O conjunto formado por suas entradas $a_{ii}$ é chamado de \sub{diagonal} de $A$. O conjunto formado pelas entradas $a_{ij}$ tais que $i+j = n+1$ é chamado de \sub{diagonal secundária} de $A$.
        
        Uma matriz do tipo $m \times n$ cujas entradas são todas iguais a zero chama-se \sub{nula} e será denotada por $0_{m \times n}$.
    \end{definicao} \pause

    \begin{exemplo}
        Dada a matriz $A$ quadrada de ordem $3$ abaixo, sua diagonal é formada pelos números $2$, $4$ e $1$. A diagonal secundária é formada por $1$, $4$ e $4$. 
        \begin{displaymath} A = 
            \begin{bmatrix}
                2 & 3 & 1\\
                3 & 4 & 0\\
                4 & 5 & 1
            \end{bmatrix}.
        \end{displaymath}
    \end{exemplo}
\end{frame}