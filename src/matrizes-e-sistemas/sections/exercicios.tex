
\section{Exercícios}
\begin{frame}
\frametitle{Exercícios} 

\Ex{ Determine, caso exista, a matriz $A$, tal que $AB = C$, em que
\begin{displaymath} B = 
	\begin{bmatrix}
		1 & -1 \\
		2 & 2 \\
		1 & 0
	\end{bmatrix}
	\space \text{ e } \space C=
	\begin{bmatrix}
		3 & 1  \\
		-1 & 4 
	\end{bmatrix}.
\end{displaymath}
}

\Ex{ Sejam $A$ e $B$ matrizes $m \times n$ e $n \times p$ respectivamente. A afirmação abaixo é sempre válida?

\begin{center}
	Se $AB=0_{m \times p}$, então $A = 0_{m \times n}$ ou $B = 0_{n \times p}$.
\end{center}
 }




\end{frame}





%------------------------------------------------------------------------------------------------------------

\begin{frame}
\frametitle{Exercícios} 

\Ex{
	Encontre uma matriz $A_{2 \times 2}$, não-nula, tal que $AA = 0_{2 \times 2}$.
}
	
\Ex{
	Determine as soluções dos seguintes sistemas lineares:
	\begin{enumerate}[a)]
		\item \[\left\{
			\begin{array}{rrrrrrr}
			x &-& 2y &-& 3z& = &0 \\
			3x& +& y&- &z &=& -1 
			\end{array} \right.;\]
		\item \[\left\{
			\begin{array}{rrrrrrr}
			x &+& y &+& z& = &2 \\
			2x& -& y&+ &3z &=& 9 \\
			x & + & 2y & - & z & = & -3 
			\end{array} \right.;\]
		\item \[\left\{
			\begin{array}{rrrrrrrrr}
			x_1 & - & 2x_2 & + & x_3& + & 2x_4 & = &1 \\
			x_1 & + & x_2 & - & x_3 & + & x_4 &=& 2 \\
			x_1 & + & 7x_2 & - & 5x_3 & - & x_4 &= & 3 
			\end{array} \right..\]
	\end{enumerate}
}


\end{frame}


%------------------------------------------------------------------------------------------------------------

\begin{frame}
\frametitle{Exercícios} 

\Ex{
	Seja 
	\begin{displaymath} A = 
		\begin{bmatrix}
			3 & -6 & 2 & -1 \\
			-2 & 4 & 1 & 3 \\
			0 & 0 & 1 & 1 \\
			1 & -2 & 1 & 0
		\end{bmatrix}.
	\end{displaymath}
	Para quais matrizes $B_{4 \times 1}$, o sistema $AX = B$ tem solução?
}

\Ex{
	Mostre que, se $A$ e $B$ são matrizes $n \times n$, ambas invertíveis, então $AB$ é invertível e vale a igualdade $(AB)^{-1} = B^{-1}A^{-1}$.
}

\Ex{
	Sejam $A$ e $B$ matrizes quadradas de ordem $n$. Mostre, através de um contra-exemplo, que a seguinte igualdade não é sempre válida:
	$$\det (A+B) = \det A + \det B.$$
}


\end{frame}


%------------------------------------------------------------------------------------------------------------

\begin{frame}
	\frametitle{Exercícios}

\Ex{
Calcule a matriz inversa, se existir, e o determinante das matrizes abaixo:
\begin{enumerate}[a)]
	\item \begin{displaymath}  
		\begin{bmatrix}
			0 & 3 & 1  \\
			1 & 1 & 2  \\
			3 & 2 & 4 
		\end{bmatrix};
	\end{displaymath}
	\item \begin{displaymath}  
		\begin{bmatrix}
			3 & -6 & 9  \\
			-2 & 7 & -2  \\
			0 & 1 & 5 
		\end{bmatrix};
	\end{displaymath}
	\item \begin{displaymath}  
		\begin{bmatrix}
			4 & 0 & 0 & 1 & 0 \\
			3 & 3 & 3 & -1 & 0 \\
			1 & 2 & 4 & 2 & 3 \\
			9 & 4 & 6 & 2 & 3 \\
			2 & 2 & 4 & 2 & 3
		\end{bmatrix}.
	\end{displaymath}
\end{enumerate}
}
\end{frame}


%------------------------------------------------------------------------------------------------------------

\begin{frame}
	\frametitle{Exercícios} 

\Ex{
	Considere a matriz 
        \begin{displaymath} A = 
            \begin{bmatrix}
                2 & 1 & 0 \\
                1 & -2 & -1 \\
                3 & -5 & -3
            \end{bmatrix}.
        \end{displaymath}

        Calcule $A^{-1}$, caso $A$ seja invertível, e calcule as soluções do sistema linear $AX=B$, onde $B = 0_{3 \times 1}$.
}

\Ex{
	Considere a matriz 
        \begin{displaymath} A = 
            \begin{bmatrix}
                -1 & 11 & 4 \\
                -2 & 1 & 5 \\
                1 & 3 & -2
            \end{bmatrix}.
        \end{displaymath}

        Calcule $\det A$ e para quais matrizes $B_{3 \times 1}$ o sistema $AX=B$ tem solução.
}


\end{frame}


%------------------------------------------------------------------------------------------------------------

\begin{frame}
	\frametitle{Exercícios} 


\Ex{
	Considere a matriz 
        \begin{displaymath} A = 
            \begin{bmatrix}
			-2 & 3 & 1 & 2 & -1 \\
			-3 & -9 & -3 & -2 & 3 \\
			1 & 1 & 4 & -1 & -1 \\
			3 & 0 & -2 & -1 & 2 \\
			1 & 3 & 1 & 0 & -1
		\end{bmatrix}.
        \end{displaymath}

        Calcule $\det A$ e, caso $A$ seja invertível, $\det A^{-1}$. Além disso, para quais matrizes $B_{5 \times 1}$ o sistema linear $AX=B$ possui alguma solução?
}


\end{frame}


%------------------------------------------------------------------------------------------------------------

\begin{frame}
	\frametitle{Exercícios} 


\Ex{
	Considere as Matrizes 
        \begin{displaymath} A = 
		\begin{bmatrix}
			3 & 0 & 0 & 0 \\
			-2 & 4 & 0 & 0 \\
			0 & 0 & 1 & 0 \\
			2 & -2 & 2 & -2
		\end{bmatrix} \text{ e } 
        B = 
		\begin{bmatrix}
			-1 & -2 & 2 & -1 \\
			0 & 2 & 1 & 3 \\
			0 & 0 & 3 & 1 \\
			0 & 0 & 0 & 1
		\end{bmatrix}.
	\end{displaymath}
 Calcule:
        \begin{enumerate}[a)]
            \item  $A\cdot B$;
            \item  $\det (A \cdot B)$;
            \item  $\det (A + B)$;
            \item  A inversa da matriz $A + B$, caso exista;
            \item  A solução do sistema $AX = C$, onde \begin{displaymath} C = 
		\begin{bmatrix}
			 3 \\
			 2 \\
			 1 \\
			 0
		\end{bmatrix}.
	\end{displaymath}
        \end{enumerate}
}


\end{frame}


%------------------------------------------------------------------------------------------------------------

\begin{frame}
	\frametitle{Exercícios} 

	\Ex{
		Considere a matriz 
        \begin{displaymath} A = 
            \begin{bmatrix}
                3 & -7 & 2 & -1\\
                -2 & 7 & 1 & 3\\
                0 & 1 & 1 & 1\\
                1 & -2 & 1 & 0
            \end{bmatrix}.
        \end{displaymath}
        Responda, em qualquer ordem, as perguntas abaixo:
		\begin{enumerate}[a)]
			\item Qual o conjunto solução do sistema linear homogêneo $AX=B_{4 \times 1}$?
			\item A matriz $A$ é invertível? Se sim, qual a sua inversa?
			\item  Qual o determinante de $A$?
		\end{enumerate}
	}

\end{frame}


%------------------------------------------------------------------------------------------------------------

\begin{frame}
	\frametitle{Exercícios} 

	\Ex{
		Encontre uma matriz quadrada $A$, de ordem 2 e não-nula, tal que $AA = 0_{2 \times 2}$. Em seguida, SEM USAR A FÓRMULA da Definição \ref{detbasicos}, mostre que $\det A = 0$. Para finalizar, para quais matrizes $B_{2 \times 1}$ a solução do sistema $AX = B$ é única?
	}

	\Ex{
		Através do uso das operações elementares, mostre que:
		\begin{displaymath}  
			\begin{vmatrix}
				1 & 1 & 1  \\
				a & b & c  \\
				a^2 & b^2 & c^2 
			\end{vmatrix} = \paren{b-a} \paren{c-a} \paren{c-b}.
		\end{displaymath}
	}

	\Ex{
	Prove a Proposição \ref{det-mat-triang}, ou seja, que o determinante de uma matriz triangular é igual ao produto das entradas de sua diagonal.
	}
	
	\Ex{
		Seja $A$ uma matriz $n \times n$ com mais de $n^2-n$ entradas nulas. Mostre que $\det A = 0$. 
	}
	
	\Ex{
		Sejam $A$ e $B$ matrizes quadradas de ordem $n$ linha equivalentes. Mostre que, se $\det A = 0$, então $\det B = 0$. No caso de $\det A \neq 0$, sob que condições teremos $\det A = \det B$?
	}
	
	\end{frame}
	
	
	%------------------------------------------------------------------------------------------------------------

