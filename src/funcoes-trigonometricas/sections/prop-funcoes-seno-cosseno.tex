\section{Prop. das Funções Seno e Cosseno}
\begin{frame}
\frametitle{Propriedades} 


Considere as seguintes definições acerca de funções reais.

\begin{definicao}
Uma função $f: \R \to \R$ chama-se \sub{periódica} quando existe $T
\in \R^\ast$ tal que $f(t + T) = f(t)$ para todo $t \in \R$. Ao
menor número $T>0$ que faz a propriedade anterior ser satisfeita,
damos o nome de \sub{período} da função $f$.
\end{definicao}\pause
Como uma volta completa no círculo trigonométrico tem $2 \pi$ de
comprimento, é fácil ver que a função seno e cosseno são periódicas
de período $2\pi$.
\end{frame}
%------------------------------------------------------------------------------------------------------------

\begin{frame}
\frametitle{Propriedades} 

\begin{definicao}
Uma função $f: \R \to \R$ é \sub{par} quando se tem $f(-t) = f(t)$
para todo $t\in \R$. Se for o caso de $f(-t) = - f(t)$ para todo $t
\in \R$, dizemos que $f$ é \sub{ímpar}.
\end{definicao}\pause

\begin{proposicao}
A função seno é ímpar e a função cosseno é par.
\end{proposicao}




\end{frame}
%------------------------------------------------------------------------------------------------------------

\begin{frame}
\frametitle{Propriedades} 

\begin{exemplo}
    Calcule os senos e cossenos dos arcos $\dfrac {5 \pi} 6$, $ \dfrac {4 \pi} 3$ e $\dfrac {11 \pi} 6$.
\end{exemplo}\pause

Segue imediatamente da definição das funções trigonométricas que a
relação fundamental $$ \sen^2 t + \cos^2 t = 1$$ vale para todo $t
\in \R$. \\
Além disso, valem as seguintes igualdades para todo $t \in \R$:
\begin{center}
\begin{tabular}{ c c }
		%\hline
		% after \\: \hline or \cline{col1-col2} \cline{col3-col4} ...
		$\cos \paren{t+ \pi} = -\cos t$, & $\sen \paren{t+ \pi} = -\sen t$, \\
		$\cos \paren{t+ \frac {\pi} 2} = -\sen t$, & $\sen \paren{t+ \frac {\pi} 2} = \cos t$, \\
		$\cos \paren{ \frac {\pi} 2 -t} = \sen t$, & $\sen \paren{ \frac {\pi} 2 -t} = \cos t$, \\
		$\cos \paren{ \pi -t} = -\cos t$, & $\sen \paren{ \pi -t} = \sen t$. \\
		%\hline
	\end{tabular}
\end{center}

\end{frame}

%------------------------------------------------------------------------------------------------------------