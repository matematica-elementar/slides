\section{Lei dos Cossenos e Lei dos Senos}
\begin{frame}
\frametitle{Lei dos Cossenos} 

\begin{teorema}[Lei dos Cossenos]
Seja $ABC$ um triângulo com $a = \segmento BC$, $b = \segmento AC$ e $c = \segmento
AB$. Então
$$b^2 = a^2 + c^2 - 2 ac \cdot \cos \widehat B.$$
\end{teorema}
A Lei dos Cossenos é uma generalização do Teorema de Pitágoras. Note
que, se $\widehat B$ é um ângulo reto, então $\cos \widehat B = 0$ e
$b$ será a hipotenusa do triângulo.
\end{frame}

%------------------------------------------------------------------------------------------------------------



\begin{frame}
\frametitle{Lei dos Senos} 

\begin{teorema}[Lei dos Senos]
Seja $ABC$ um triângulo com $a = \segmento BC$, $b = \segmento AC$ e $c = \segmento
AB$. Então
$$\frac a {\sen \widehat A} = \frac b {\sen \widehat B} = \frac c {\sen \widehat C}$$
\end{teorema}
A Lei dos Senos nos diz que, em todo triângulo, a razão entre um
lado e o seno do ângulo oposto é constante.\pause

As leis dos cossenos e dos senos permitem obter os seis elementos de
um triângulo quando são dados três deles, desde que um seja lado,
conforme os casos clássicos de congruência de triângulos.


\end{frame}



%------------------------------------------------------------------------------------------------------------