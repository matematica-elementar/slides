
\section{Definição de Função}
\begin{frame} \frametitle{O que É uma Função?}
\begin{definicao}
Sejam $X$ e $Y$ dois conjuntos quaisquer.\\
Uma \sub{função} é uma relação $f: X \to Y$ que, a cada elemento $x
\in X$, associa um e somente um elemento $y \in Y$.\\
Nesse caso:
\begin{enumerate}[(i)]
	\item Os conjuntos $X$ e $Y$ são chamados \sub{domínio} e
	\sub{contradomínio} de $f$, respectivamente;
	\item O conjunto $f\paren X = \set{y \in Y \tq \exists x \in X , f \paren x =
	y} \contido Y$ é chamado \sub{imagem} de $f$;
	\item Dado $x \in X$, o (único) elemento $y = f(x) \in Y$
	correspondente é chamado \sub{imagem} de $x$.
\end{enumerate}
\end{definicao}

\end{frame}

%------------------------------------------------------------------------------------------------------------

\begin{frame} \frametitle{O que É uma Função?} 

Dessa forma, uma função é um terno constituído por: \sub{domínio},
\sub{contradomínio} e \sub{lei de associação} (dos elementos do
domínio com os do contradomínio). Precisamos desses três elementos
para que uma função seja bem definida. Poderíamos definir
função da seguinte forma: \\
Para que uma relação $f: X \to Y$ seja
uma função, ela deve satisfazer a duas condições fundamentais:
\begin{enumerate}[(I)]
	\item Estar bem definida em todo elemento do domínio (existência);
	\item Não fazer corresponder mais de um elemento do contradomínio
	a cada elemento do domínio (unicidade).
\end{enumerate}

\end{frame}

%------------------------------------------------------------------------------------------------------------

\begin{frame}
\frametitle{O que É uma Função?} 
\begin{exemplo}
Considere as funções
$$\begin{array}{cccc}
p : & \R & \to     & \R_+ \\
		 &  x & \mapsto & x^2
\end{array}
\text{\ \ \  e \ \ \ }
\begin{array}{cccc}
q : & \R_+ & \to     & \R \\
		 &  x & \mapsto & \sqrt x
\end{array}.$$
Qual o domínio, contradomínio e a lei de associação de $p$ e $q$?
\end{exemplo}
\pause

\begin{exemplo}
Seja $\I{X} : X \to X $ uma função tal que $\I{X} \paren x = x$ para
todo $x \in X$. Chamamos $\I{X}$ de \sub{função identidade do
conjunto $X$}.
\end{exemplo}

\end{frame}


%------------------------------------------------------------------------------------------------------------
