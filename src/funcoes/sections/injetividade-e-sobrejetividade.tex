\section{Injetividade e Sobrejetividade}
\begin{frame}
\frametitle{Funções Injetivas, Sobrejetivas e Bijetivas} 


\begin{definicao}
Considere uma função $f: X \to Y$.
\begin{enumerate}[(i)]
	\item $f$ é \sub{sobrejetiva} se, para todo $y \in Y$, existe $x
	\in X$ tal que $f(x) = y$;
	\item $f$ é \sub{injetiva} se $x_1, x_2 \in X, x_1 \neq x_2
	\implica f(x_1) \neq f(x_2)$;
	\item $f$ é \sub{bijetiva} se é sobrejetiva e injetiva.
\end{enumerate}
\end{definicao}\pause
Há, ainda, formas alternativas de enunciar as definições acima:
\begin{itemize}
	\item $f$ é \sub{sobrejetiva} se, e somente se, $f(X) = Y$;
	\item $f$ é \sub{injetiva} se, e somente se, $x_1, x_2 \in X, f(x_1) = f(x_2)
	\implica x_1 = x_2 $;
	\item $f$ é \sub{injetiva} se, e somente se, para todo $y \in
	f(X)$, existe um único $x \in X$ tal que $f(x) = y$;
	\item $f$ é \sub{bijetiva} se, e somente se, para todo $y \in Y$,
	existe um único $x \in X$ tal que $f(x) = y$.
\end{itemize}
\end{frame}



%------------------------------------------------------------------------------------------------------------

\begin{frame}
\frametitle{Funções Injetivas, Sobrejetivas e Bijetivas} 

\begin{exemplo}\label{pqbij}
As funções $p$ e $q$ são sobrejetivas, injetivas ou bijetivas?
\end{exemplo}
\end{frame}



%------------------------------------------------------------------------------------------------------------

\begin{frame}
\frametitle{Funções Injetivas, Sobrejetivas e Bijetivas} 

\begin{teorema}\label{teoinv}
Uma função $f: X \to Y$ é invertível se, e somente se, é bijetiva.
\end{teorema}\pause

\begin{exemplo}
Decorre do Teorema \ref{teoinv} e do Exemplo \ref{pqbij} que as
funções $p$ e $q$ não são invertíveis.
\end{exemplo}

\end{frame}

