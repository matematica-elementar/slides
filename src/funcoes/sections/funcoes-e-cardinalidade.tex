\section{Funções e Cardinalidade}
\begin{frame}
\frametitle{Funções e Cardinalidade} 

\begin{definicao}
Dois conjuntos $X$ e $Y$ são ditos \sub{cardinalmente equivalentes}
(ou \sub{equipotentes}) se existe uma bijeção $f : X \to Y$.
\end{definicao}

\begin{definicao}
Dizemos que um conjunto $X$ é \sub{enumerável} se $X$ é um conjunto
cardinalmente equivalente ao conjunto $\N$ ou a algum dos seus
subconjuntos.
\end{definicao}


\end{frame}



%------------------------------------------------------------------------------------------------------------

\begin{frame}
\frametitle{Funções e Cardinalidade} 

\begin{teorema}
Se existe uma injeção $f: X \to Y$, então existe uma bijeção entre
$X$ e um subconjunto $Y' \contido Y$, isto é, $X$ é cardinalmente
equivalente a um subconjunto de $Y$.
\end{teorema} \pause

\begin{teorema}
Se existe uma sobrejeção $f : X \to Y$, então existe uma bijeção
entre $Y$ e um subconjunto $X' \contido X$, isto é, $Y$ é
cardinalmente equivalente a um subconjunto de $X$.
\end{teorema}

\end{frame}

%------------------------------------------------------------------------------------------------------------

\begin{frame}
\frametitle{Funções e Cardinalidade} 

\begin{exemplo}
O conjunto $\Q$ é enumerável.
\end{exemplo}

\end{frame}

