\section{Função Inversa}
\begin{frame}
\frametitle{Função Inversa} 

\begin{definicao}\label{funinv}
Uma função $f: X \to Y$ é \sub{invertível} se existe uma função $g:
Y \to X$ tal que
\begin{enumerate}[(i)]
	\item $f \circ g = \I{Y}$;
	\item $g \circ f = \I{X}$.
\end{enumerate}
Nesse caso, a função $g$ é dita \sub{função inversa} de $f$ e
denotada por $g = f^{-1}$.
\end{definicao}

\end{frame}



%------------------------------------------------------------------------------------------------------------


\begin{frame}
\frametitle{Função Inversa} 

\begin{exemplo}
A função $q$ é inversa de $p$?
\end{exemplo}\pause
 Esse exemplo ilustra a importância de verificarmos as duas
 condições para que tenhamos uma função inversa.

\end{frame}
