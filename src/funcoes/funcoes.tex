\documentclass[10pt]{beamer}
%
%   Arquivo de Configuração dos Slides
%


%
%   Pacotes utilizados
%

% Codificação dos caracteres em formato universal.
\usepackage[utf8]{inputenc}
\usepackage[T1]{fontenc}

% Traduz o texto gerados pelo LaTeX para português. ex.: Capítulo, Seção, Conteúdo.
\usepackage[brazil]{babel}

% Pacotes para ambientes matemáticos
\usepackage{amsmath}
\usepackage{amsthm}
\usepackage{amssymb}

% Diversas funções para o uso das aspas.
\usepackage{csquotes}

% Outros pacotes
\usepackage{hyperref}
\usepackage{tikz}
\usepackage{yfonts}
\usepackage{colortbl}
\usepackage{ragged2e}
\usepackage{helvet}
\usepackage{verbatim}


%
%   Tema
%

% Copyright 2007 by Till Tantau
%
% This file may be distributed and/or modified
%
% 1. under the LaTeX Project Public License and/or
% 2. under the GNU Public License.
%
% See the file doc/licenses/LICENSE for more details.


% Common packages


\usepackage{times}
 \mode<article> {
	\usepackage{times}
	\usepackage{mathptmx}
	\usepackage[left=1.5cm,right=6cm,top=1.5cm,bottom=3cm]{geometry}
}

\usepackage{hyperref}
\usepackage[T1]{fontenc}
\usepackage{amsmath,amssymb}
\usepackage{tikz}
\usepackage{colortbl}
\usepackage{yfonts}
\usepackage{colortbl}
\usepackage{translator} % comment this, if not available
\usepackage{ragged2e} % justifying
% Or whatever. Note that the encoding and the font should match. If T1
% does not look nice, try deleting the line with the fontenc.
\usepackage{helvet}
\usepackage{verbatim}


%\usepackage{lipsum}
%\usepackage{enumitem}


\usetheme[
%%% options passed to the outer theme
%    hidetitle,           % hide the (short) title in the sidebar
%    hideauthor,          % hide the (short) author in the sidebar
%    hideinstitute,       % hide the (short) institute in the bottom of the sidebar
%    shownavsym,          % show the navigation symbols
%    width=2cm,           % width of the sidebar (default is 2 cm)
%    hideothersubsections,% hide all subsections but the subsections in the current section
%    hideallsubsections,  % hide all subsections
		right               % right of left position of sidebar (default is right)
%%% options passed to the color theme
%    lightheaderbg,       % use a light header background
	]{AAUsidebar}

% If you want to change the colors of the various elements in the theme, edit and uncomment the following lines
% Change the bar and sidebar colors:
%\setbeamercolor{AAUsidebar}{fg=red!20,bg=red}
%\setbeamercolor{sidebar}{bg=red!20}
% Change the color of the structural elements:
%\setbeamercolor{structure}{fg=red}
% Change the frame title text color:
%\setbeamercolor{frametitle}{fg=blue}
% Change the normal text color background:
%\setbeamercolor{normal text}{bg=gray!10}
% Highlight the text in the sidebar
\usecolortheme{rose,sidebartab}
% ... and you can of course change a lot more - see the beamer user manual.

% colored hyperlinks
\newcommand{\chref}[2]{%
	\href{#1}{{\usebeamercolor[bg]{AAUsidebar}#2}}%
}



% specify a logo on the titlepage (you can specify additional logos an include them in
% institute command below
\pgfdeclareimage[height=1cm]{titlepagelogo}{theme/figures/ufrn2} % placed on the title page
\pgfdeclareimage[height=1cm]{titlepagelogo2}{theme/figures/imd} % placed on the title page
\titlegraphic{% is placed on the bottom of the title page
	\pgfuseimage{titlepagelogo}
	\hspace{1cm}\pgfuseimage{titlepagelogo2}
}


% Article version layout settings

\mode<article>

\makeatletter
\def\@listI{\leftmargin\leftmargini
	\parsep 0pt
	\topsep 5\p@   \@plus3\p@ \@minus5\p@
	\itemsep0pt}
\let\@listi=\@listI


\setbeamertemplate{frametitle}{\paragraph*{\insertframetitle\
		\ \small\insertframesubtitle}\ \par
}
\setbeamertemplate{frame end}{%
	\marginpar{\scriptsize\hbox to 1cm{\sffamily%
			\hfill\strut\insertshortlecture.\insertframenumber}\hrule height .2pt}}
\setlength{\marginparwidth}{1cm}
\setlength{\marginparsep}{4.5cm}

\def\@maketitle{\makechapter}

\def\makechapter{
	\newpage
	\null
	\vskip 2em%
	{%
		\parindent=0pt
		\raggedright
		\sffamily
		\vskip8pt
		\includegraphics[width=\linewidth]{theme/figures/imd.png}\par\vskip2em
		{\fontsize{36pt}{36pt}\selectfont Aula \insertshortlecture \par\vskip2pt}
		{\fontsize{24pt}{28pt}\selectfont \color{blue!50!black} \@title\par\vskip4pt}
		%{\Large\selectfont \color{blue!50!black} \insertsubtitle\par}
		\vskip10pt

		\normalsize\selectfont [Notas de Aula]
		Disciplina: \emph{\lecturename \ (\semestre)} \par\vskip1.5em
		\nomedoautor\hskip1em Email: \ \emaildoautor
	}
	\par
	\vskip 1.5em%
}

\let\origstartsection=\@startsection
\def\@startsection#1#2#3#4#5#6{%
	\origstartsection{#1}{#2}{#3}{#4}{#5}{#6\normalfont\sffamily\color{blue!50!black}\selectfont}}

\makeatother

\mode
<all>




% Typesetting Listings

\usepackage{listings}
\lstset{language=Java}

\alt<presentation>
{\lstset{%
	basicstyle=\footnotesize\ttfamily,
	commentstyle=\slshape\color{green!50!black},
	keywordstyle=\bfseries\color{blue!50!black},
	identifierstyle=\color{blue},
	stringstyle=\color{orange},
	escapechar=\#,
	emphstyle=\color{red}}
}
{
	\lstset{%
		basicstyle=\ttfamily,
		keywordstyle=\bfseries,
		commentstyle=\itshape,
		escapechar=\#,
		emphstyle=\bfseries\color{red}
	}
}



% Common theorem-like environments
%\usepackage{amsthm}

\setbeamertemplate{theorems}[numbered]

%
%	New useful definitions:
%

\newbox\mytempbox
\newdimen\mytempdimen

\newcommand\includegraphicscopyright[3][]{%
	\leavevmode\vbox{\vskip3pt\raggedright\setbox\mytempbox=\hbox{\includegraphics[#1]{#2}}%
		\mytempdimen=\wd\mytempbox\box\mytempbox\par\vskip1pt%
		\fontsize{3}{3.5}\selectfont{\color{black!25}{\vbox{\hsize=\mytempdimen#3}}}\vskip3pt%
}}

\newenvironment{colortabular}[1]{\medskip\rowcolors[]{1}{blue!20}{blue!10}\tabular{#1}\rowcolor{blue!40}}{\endtabular\medskip}

\def\equad{\leavevmode\hbox{}\quad}

\newenvironment{greencolortabular}[1]
{\medskip\rowcolors[]{1}{green!50!black!20}{green!50!black!10}%
	\tabular{#1}\rowcolor{green!50!black!40}}%
{\endtabular\medskip}

%\setbeamertemplate{theorem begin}{{ \inserttheoremheadfont
%\inserttheoremname \inserttheoremnumber
%\ifx\inserttheoremaddition\empty\else\ (\inserttheoremaddition)\fi%
%\inserttheorempunctuation }} \setbeamertemplate{theorem end}{}

\newcommand{\vu}{\vec{u}}
\newcommand{\vv}{\vec{v}}
\newcommand{\vi}{\vec{i}}
\newcommand{\vj}{\vec{j}}
\newcommand{\vk}{\vec{k}}
\newcommand{\vw}{\vec{w}}
\newcommand\segmento[2]{\overline{#1#2}}
\def\colc#1{\left[#1\right]}



%
%   Macros
%

\usepackage{macros/macros}


%
%   Ambientes
%

\theoremstyle{plain}
\newtheorem{teorema}{Teorema}

\theoremstyle{definition}
\newtheorem{definicao}[teorema]{Definição}
%\newtheorem{exercicio}{Exercício}

\theoremstyle{remark}
\newtheorem{obs}[teorema]{Observação}
\newtheorem{observacao}[teorema]{Observação}
\newtheorem{corolario}[teorema]{Corolário}
\newtheorem{exemplo}[teorema]{Exemplo}
\newtheorem{lema}[teorema]{Lema}
\newtheorem{proposicao}[teorema]{Proposição}

\newcounter{exercicios}
\newenvironment{exercicio}{\stepcounter{exercicios} \textbf{\arabic{exercicios}}.}{}

% compatibilidade
\newcommand{\Ex}[1]{\begin{exercicio}#1\end{exercicio}}

%
%   Definições e comandos auxiliares do preâmbulo
%

\newcommand{\capitulo}[1]{\lecture[#1]{Capítulo}}
\newcommand{\aula}[1]{\subtitle{#1}}
\newcommand{\autor}{Igor Oliveira}
\newcommand{\email}{\href{mailto:matematicaelementar@imd.ufrn.br}{\texttt{matematicaelementar@imd.ufrn.br}}}
\newcommand{\disciplina}{Matemática Elementar}
\newcommand{\codigo}{IMD1001}

\title{\disciplina}
\date{\today}
\author[\autor]
{
    \autor\\
    \email
}

\def\lecturename{\codigo

\disciplina}

\institute[
	UFRN\\
	Natal-RN
]
{
	Instituto Metrópole Digital\\
	Universidade Federal do Rio Grande do Norte\\
	Natal-RN

}

% compatibilidade
\newcommand{\vu}{\vec{u}}
\newcommand{\vv}{\vec{v}}
\newcommand{\vi}{\vec{i}}
\newcommand{\vj}{\vec{j}}
\newcommand{\vk}{\vec{k}}
\newcommand{\vw}{\vec{w}}
\newcommand{\segmento}[2]{\overline{#1#2}}
\def\colc#1{\left[#1\right]}
\newcommand{\negacao}{\sim}

\justifying


\aula{Funções}
\capitulo{5}


\begin{document}	

	%
	%	Capa
	%

	{\backgroundimage\begin{frame}[plain]
		\titlepage
	\end{frame}}


	%
	%	Sumário
	%

	\begin{frame}
		\frametitle{Índice}
		\tableofcontents
	\end{frame}


	%
	%	Seções
	%

\section{Introdução}

\begin{frame}  \frametitle{Apresentação da Aula}

Considere as funções
$$\begin{array}{cccc}
p : & \R & \to     & \R_+ \\
		 &  x & \mapsto & x^2
\end{array}
\text{\ \ \  e \ \ \ }
\begin{array}{cccc}
q : & \R_+ & \to     & \R \\
		 &  x & \mapsto & \sqrt x
\end{array}.$$
As funções $p$ e $q$ são inversas uma da outra? \\ \pause Elas são
bijetivas? \\ \pause

Caso você precise de um conceito mais básico de função, sugiro ver o
material de
\href{https://pt.khanacademy.org/math/algebra/algebra-functions}
{{\tt funções}} no Khan Academy.


\end{frame}

%------------------------------------------------------------------------------------------------------------

\section{Definição de Função}
\begin{frame} \frametitle{O que É uma Função?}
\begin{definicao}
Sejam $X$ e $Y$ dois conjuntos quaisquer.\\
Uma \sub{função} é uma relação $f: X \to Y$ que, a cada elemento $x
\in X$, associa um e somente um elemento $y \in Y$.\\
Nesse caso:
\begin{enumerate}[(i)]
	\item Os conjuntos $X$ e $Y$ são chamados \sub{domínio} e
	\sub{contradomínio} de $f$, respectivamente;
	\item O conjunto $f\paren X = \set{y \in Y \tq \exists x \in X , f \paren x =
	y} \contido Y$ é chamado \sub{imagem} de $f$;
	\item Dado $x \in X$, o (único) elemento $y = f(x) \in Y$
	correspondente é chamado \sub{imagem} de $x$.
\end{enumerate}
\end{definicao}

\end{frame}

%------------------------------------------------------------------------------------------------------------

\begin{frame} \frametitle{O que É uma Função?} 

Dessa forma, uma função é um terno constituído por: \sub{domínio},
\sub{contradomínio} e \sub{lei de associação} (dos elementos do
domínio com os do contradomínio). Precisamos desses três elementos
para que uma função seja bem definida. Poderíamos definir
função da seguinte forma: \\
Para que uma relação $f: X \to Y$ seja
uma função, ela deve satisfazer a duas condições fundamentais:
\begin{enumerate}[(I)]
	\item Estar bem definida em todo elemento do domínio (existência);
	\item Não fazer corresponder mais de um elemento do contradomínio
	a cada elemento do domínio (unicidade).
\end{enumerate}

\end{frame}

%------------------------------------------------------------------------------------------------------------

\begin{frame}
\frametitle{O que É uma Função?} 
\begin{exemplo}
Considere as funções
$$\begin{array}{cccc}
p : & \R & \to     & \R_+ \\
		 &  x & \mapsto & x^2
\end{array}
\text{\ \ \  e \ \ \ }
\begin{array}{cccc}
q : & \R_+ & \to     & \R \\
		 &  x & \mapsto & \sqrt x
\end{array}.$$
Qual o domínio, contradomínio e a lei de associação de $p$ e $q$?
\end{exemplo}
\pause

\begin{exemplo}
Seja $\I{X} : X \to X $ uma função tal que $\I{X} \paren x = x$ para
todo $x \in X$. Chamamos $\I{X}$ de \sub{função identidade do
conjunto $X$}.
\end{exemplo}

\end{frame}


%------------------------------------------------------------------------------------------------------------
\section{Funções Compostas}
\begin{frame}
\frametitle{Composição de Funções} 

\begin{definicao}
Sejam $f: X \to Y$ e $g: U \to V$ duas funções, com $Y \contido U$.\\
A \sub{função composta de $g$ com $f$} é a função denotada por $g
\circ f$, com domínio em $X$ e contradomínio em $V$, que a cada
elemento $x \in X$ faz corresponder o elemento $v = \paren{g \circ
f}(x) = g(f(x)) \in V$. Isto é:
$$\begin{array}{cccccc}
g \circ f : & X & \to     & Y \contido U & \to & V \\
		 &  x & \mapsto & f(x) & \mapsto & g(f(x))
\end{array}.$$
\end{definicao}

\end{frame}

%------------------------------------------------------------------------------------------------------------

\begin{frame}
\frametitle{Composição de Funções} 

\begin{exemplo}
Seja $f: X \to Y$ uma função. Mostre que $f \circ \I{X} = f$ e $\I{Y}
\circ f = f$.
\end{exemplo}\pause

\begin{exemplo}
Qual função resulta da composição $p \circ q$?
\end{exemplo}
\end{frame}

%------------------------------------------------------------------------------------------------------------
\section{Atividade Online}
\begin{frame}
\frametitle{Atividade Online} 

\href{https://pt.khanacademy.org/math/algebra2/manipulating-functions/function-composition/e/compose-functions}
{{\tt Atividade 05 - Encontre Funções Compostas}}

\href{https://pt.khanacademy.org/math/algebra2/manipulating-functions/combining-and-composing-modeling-functions/e/modeling-with-composite-functions}
{{\tt Atividade 06 - Modele com Funções Compostas}}


Veja o desempenho na Missão Álgebra II.


\end{frame}

%------------------------------------------------------------------------------------------------------------


\section{Função Inversa}
\begin{frame}
\frametitle{Função Inversa} 

\begin{definicao}\label{funinv}
Uma função $f: X \to Y$ é \sub{invertível} se existe uma função $g:
Y \to X$ tal que
\begin{enumerate}[(i)]
	\item $f \circ g = \I{Y}$;
	\item $g \circ f = \I{X}$.
\end{enumerate}
Nesse caso, a função $g$ é dita \sub{função inversa} de $f$ e
denotada por $g = f^{-1}$.
\end{definicao}

\end{frame}



%------------------------------------------------------------------------------------------------------------


\begin{frame}
\frametitle{Função Inversa} 

\begin{exemplo}
A função $q$ é inversa de $p$?
\end{exemplo}\pause
 Esse exemplo ilustra a importância de verificarmos as duas
 condições para que tenhamos uma função inversa.

\end{frame}

%------------------------------------------------------------------------------------------------------------
\section{Atividade Online}
\begin{frame}
\frametitle{Atividade Online} 

\href{https://pt.khanacademy.org/math/algebra2/manipulating-functions/verifying-that-functions-are-inverses/e/inverses_of_functions}
{{\tt Atividade 07 - Verifique Funções Inversas}}


Veja o desempenho na Missão Álgebra II.


\end{frame}
%------------------------------------------------------------------------------------------------------------
\section{Injetividade e Sobrejetividade}
\begin{frame}
\frametitle{Funções Injetivas, Sobrejetivas e Bijetivas} 


\begin{definicao}
Considere uma função $f: X \to Y$.
\begin{enumerate}[(i)]
	\item $f$ é \sub{sobrejetiva} se, para todo $y \in Y$, existe $x
	\in X$ tal que $f(x) = y$;
	\item $f$ é \sub{injetiva} se $x_1, x_2 \in X, x_1 \neq x_2
	\implica f(x_1) \neq f(x_2)$;
	\item $f$ é \sub{bijetiva} se é sobrejetiva e injetiva.
\end{enumerate}
\end{definicao}\pause
Há, ainda, formas alternativas de enunciar as definições acima:
\begin{itemize}
	\item $f$ é \sub{sobrejetiva} se, e somente se, $f(X) = Y$;
	\item $f$ é \sub{injetiva} se, e somente se, $x_1, x_2 \in X, f(x_1) = f(x_2)
	\implica x_1 = x_2 $;
	\item $f$ é \sub{injetiva} se, e somente se, para todo $y \in
	f(X)$, existe um único $x \in X$ tal que $f(x) = y$;
	\item $f$ é \sub{bijetiva} se, e somente se, para todo $y \in Y$,
	existe um único $x \in X$ tal que $f(x) = y$.
\end{itemize}
\end{frame}



%------------------------------------------------------------------------------------------------------------

\begin{frame}
\frametitle{Funções Injetivas, Sobrejetivas e Bijetivas} 

\begin{exemplo}\label{pqbij}
As funções $p$ e $q$ são sobrejetivas, injetivas ou bijetivas?
\end{exemplo}
\end{frame}



%------------------------------------------------------------------------------------------------------------

\begin{frame}
\frametitle{Funções Injetivas, Sobrejetivas e Bijetivas} 

\begin{teorema}\label{teoinv}
Uma função $f: X \to Y$ é invertível se, e somente se, é bijetiva.
\end{teorema}\pause

\begin{exemplo}
Decorre do Teorema \ref{teoinv} e do Exemplo \ref{pqbij} que as
funções $p$ e $q$ não são invertíveis.
\end{exemplo}

\end{frame}



%------------------------------------------------------------------------------------------------------------
\section{Fórmulas e Funções}
\begin{frame}
\frametitle{Fórmulas e Funções} 
É muito importante não pensar que uma função é uma fórmula.
Considere as funções
$$\begin{array}{cccc}
p_1 : & \R & \to     & \R \\
		 &  x & \mapsto & x^2
\end{array}
\text{\ \ \  e \ \ \ }
\begin{array}{cccc}
p_2 : & \R_+ & \to     & \R_+ \\
		 &  x & \mapsto &  x^2
\end{array}.$$
Essas funções são iguais? \\ \pause NÃO! Note que $p_2$ é bijetiva e
$p_1$ não é, mesmo tendo a mesma fórmula.

Além disso, funções podem ser definidas por mais de uma fórmula,
como na função $h :  \R  \to      \R$ tal que
		 $$h(x) =  \begin{cases}
						0, &  \ \text{ se } x \in \R \setminus \Q \\
						1, & \ \text{ se } x \in \Q
						\end{cases} .$$
\end{frame}



%------------------------------------------------------------------------------------------------------------
\section{Funções e Cardinalidade}
\begin{frame}
\frametitle{Funções e Cardinalidade} 

\begin{definicao}
Dois conjuntos $X$ e $Y$ são ditos \sub{cardinalmente equivalentes}
(ou \sub{equipotentes}) se existe uma bijeção $f : X \to Y$.
\end{definicao}

\begin{definicao}
Dizemos que um conjunto $X$ é \sub{enumerável} se $X$ é um conjunto
cardinalmente equivalente ao conjunto $\N$ ou a algum dos seus
subconjuntos.
\end{definicao}


\end{frame}



%------------------------------------------------------------------------------------------------------------

\begin{frame}
\frametitle{Funções e Cardinalidade} 

\begin{teorema}
Se existe uma injeção $f: X \to Y$, então existe uma bijeção entre
$X$ e um subconjunto $Y' \contido Y$, isto é, $X$ é cardinalmente
equivalente a um subconjunto de $Y$.
\end{teorema} \pause

\begin{teorema}
Se existe uma sobrejeção $f : X \to Y$, então existe uma bijeção
entre $Y$ e um subconjunto $X' \contido X$, isto é, $Y$ é
cardinalmente equivalente a um subconjunto de $X$.
\end{teorema}

\end{frame}

%------------------------------------------------------------------------------------------------------------

\begin{frame}
\frametitle{Funções e Cardinalidade} 

\begin{exemplo}
O conjunto $\Q$ é enumerável.
\end{exemplo}

\end{frame}


%------------------------------------------------------------------------------------------------------------

\section{Atividade Online}
\begin{frame}
\frametitle{Atividade Online} 

\href{https://pt.khanacademy.org/math/algebra2/manipulating-functions/invertible-functions/e/inverse-domain-range}
{{\tt Atividade 08 - Determine se uma Função É Inversível}}

\href{https://pt.khanacademy.org/math/algebra2/manipulating-functions/invertible-functions/e/restrict-the-domains-of-functions}
{{\tt Atividade 09 - Restrinja os Domínios de Funções para Torná-las
Inversíveis}}


Veja o desempenho na Missão Álgebra II.
\end{frame}

%------------------------------------------------------------------------------------------------------------

\section{Exercícios}
\begin{frame}
\frametitle{Exercícios} 

\Ex{Em cada um dos itens abaixo, defina uma função com a lei de
formação dada (indicando domínio e contradomínio). Verifique se é
injetiva, sobrejetiva ou bijetiva, a função
\begin{enumerate}[(a)]
	\item Que a cada dois números naturais associa seu mdc;
%  \item Que a cada vetor do plano associa seu módulo;
%  \item Que a cada matriz $2 \times 2$ associa sua matriz
%  transposta;
%  \item Que a cada matriz $2 \times 2$ associa seu determinante;
	\item Que a cada polinômio (não nulo) com coeficientes reais
	associa seu grau;
	\item Que a cada figura plana fechada e limitada associa a sua
	área;
	\item Que a cada subconjunto de $\R$ associa seu complementar;
	\item Que a cada subconjunto finito de $\N$ associa seu número de
	elementos;
	\item Que a cada subconjunto não vazio de $\N$ associa seu menor
	elemento;
	\item Que a cada função $f: \R \to \R$ associa seu valor no ponto
	$x_0 = 0$.
\end{enumerate}}



\end{frame}


%------------------------------------------------------------------------------------------------------------

\begin{frame}
\frametitle{Exercícios} 

\Ex{Considere a  função $f:  \N^*  \to      \Z $ tal que $$f(n) =
\begin{cases} \frac {-n} 2, \text{ se $n$ é par} \\ \frac {n-1} 2, \text{ se $n$ é
ímpar} \end{cases}.$$ Mostre que $f$ é bijetiva. O que você pode
concluir com esse resultado?}

\Ex{Mostre que a função inversa de $f: X \to Y$, caso exista, é
única, isto é, se existem $g_1 : Y \to X$ e $g_2 : Y \to X$
satisfazendo a Definição \ref{funinv}, então $g_1 = g_2$.\\
\emph{Dica: } Lembre-se que duas funções são iguais se, e só se,
possuem mesmos domínios, contradomínios e seus valores são iguais em
todos os elementos do domínio. Assim, procure mostrar que $g_1 (y) =
g_2 (y)$, para todo $y \in Y$.}







\end{frame}

%------------------------------------------------------------------------------------------------------------

\begin{frame}
\frametitle{Exercícios} 

\Ex{Seja $f: X \to Y$ uma função e seja $A$ um subconjunto de $X$.
Define-se $$f(A) = \set{f(x) \tq x\in A} \contido Y.$$ Se $A$ e $B$
são subconjuntos de $X$:
\begin{enumerate}[(a)]
	\item Mostre que $f(A \uniao B) = f(A) \uniao f(B)$;
	\item Mostre que $f(A \inter B) \contido f(A) \inter f(B)$;
	\item É possível afirmar que $f(A \inter B) = f(A) \inter f(B)$ para
	todos $A, B \contido X$? Justifique.
	\item Determine que condições deve satisfazer $f$ para que a
	afirmação feita no item (c) seja verdadeira.
\end{enumerate}
}

\end{frame}



%------------------------------------------------------------------------------------------------------------

\begin{frame}
\frametitle{Exercícios} 

\Ex{Seja $f: X \to Y$ uma função. Dado $y \in Y$, definimos a
\sub{contraimagem} ou \sub{imagem inversa} de $y$ como sendo o
seguinte subconjunto de $X$: $$ f^{-1}(y) = \set{x \in X \tq
f(x)=y}.$$
\begin{enumerate}[(a)]
	\item Se $f$ é injetiva e $y$ é um elemento qualquer de $Y$, o que
	se pode afirmar sobre a imagem inversa $f^{-1}(y)$?
	\item Se $f$ é sobrejetiva e $y$ é um elemento qualquer de $Y$, o que
	se pode afirmar sobre a imagem inversa $f^{-1}(y)$?
	\item Se $f$ é bijetiva e $y$ é um elemento qualquer de $Y$, o que
	se pode afirmar sobre a imagem inversa $f^{-1}(y)$?
\end{enumerate} }





\end{frame}



%------------------------------------------------------------------------------------------------------------

\begin{frame}
\frametitle{Exercícios} 

\Ex{Seja $f: X \to Y$ uma função. Dado $A \contido Y$, definimos a
\sub{contraimagem} ou \sub{imagem inversa} de $A$ como sendo o
seguinte subconjunto de $X$: $$ f^{-1}(A) = \set{x \in X \tq f(x)\in
A}.$$ Mostre que
\begin{enumerate}[(a)]
	\item $f^{-1} (A \uniao B) = f^{-1} (A) \uniao f^{-1} (B)$;
	\item $f^{-1} (A \inter B) = f^{-1} (A) \inter f^{-1} (B)$.
\end{enumerate}}




\end{frame}



%------------------------------------------------------------------------------------------------------------

\begin{frame}
\frametitle{Exercícios - Desafios} 

\Ex{Seja $f: X \to Y$ uma função. Mostre que, se existem $g_1 : Y
\to X$ e $g_2 : Y \to X$ tais que $f \circ g_1 = \I{Y}$ e $g_2 \circ
f = \I{X}$, então $g_1 = g_2$ (portanto, neste caso, $f$ será
invertível).}

\Ex{Seja $f: X \to Y$ uma função. Mostre que
\begin{enumerate}[(a)]
	\item $f(f^{-1}(B)) \contido B$, para todo $B \contido Y$;
	\item $f(f^{-1}(B)) = B$, para todo $B \contido Y$ se, e
	somente se, $f$ é sobrejetiva.
\end{enumerate}}

\Ex{Seja $f: X \to Y$ uma função. Mostre que
\begin{enumerate}[(a)]
	\item $f(f^{-1}(A)) \supset A$, para todo $A \contido X$;
	\item $f(f^{-1}(A)) = A$, para todo $A \contido X$ se, e
	somente se, $f$ é injetiva.
\end{enumerate}}

\Ex{Mostre que existe uma injeção $f: X \to Y$ se, e somente se,
existe uma sobrejeção $g: Y \to X$.}

\end{frame}

%------------------------------------------------------------------------------------------------------------

\section{Bibliografia}

\frame{
	\frametitle{Bibliografia}
	\begin{thebibliography}{99}
		\bibitem {label1}
		IEZZI, Gelson; et al.
		\newblock \emph{Fundamentos de Matemática Elementar. Vol. 1 - Conjuntos e Funções}.
		\newblock São Paulo: Editora Atual.
	\end{thebibliography}
}

\end{document}
