
\section{Conjuntos Numéricos}
\frame { \frametitle{Naturais}
\begin{Def}
Ao conjunto $\N = \set {0, 1, 2, \dots , n, n+1, \dots}$ damos o
nome de \sub{conjunto dos números naturais}.
\end{Def}
\begin{itemize}
\item Denotamos $\N \setminus \set 0 = \set {1, 2, \dots , n, n+1,
\dots}$ por $\N ^\ast$.

\item Usamos o conjunto dos números naturais para contar coisas, como
casas, animais, etc.
\end{itemize}
}

%------------------------------------------------------------------------------------------------------------


\begin{frame}
\frametitle{Inteiros} %\framesubtitle{Exemplos}
\begin{Def}
Ao conjunto $\Z =\set {\dots , -m -1, -m, \dots, -1, 0, 1,  \dots ,
n, n+1, \dots}$ damos o nome de \sub{conjunto dos números inteiros}.
\end{Def}

\begin{block}{Notação}
$\Z^\ast = \Z \setminus \set 0$; \\
$\Z_+ = \N$ (Inteiros não negativos); \\
$\Z^\ast_+ =\N ^\ast$ (Inteiros positivos); \\
$\Z_- =\set {\dots , -m -1, -m, \dots, -1, 0}$ (Inteiros não
positivos); \\
$\Z_-^\ast =\Z_- \setminus \set 0$ (Inteiros negativos).
\end{block}
\end{frame}



%------------------------------------------------------------------------------------------------------------
\begin{frame}
\frametitle{Racionais} %\framesubtitle{Exemplos}
\begin{Def}
Ao conjunto $\Q = \set{\frac p q \tq p, q \in \Z \text{ e } q \neq
0}$ damos o nome de \sub{conjunto dos números racionais}.
\end{Def}

A representação decimal de um número racional é finita ou é uma
dízima periódica (infinita).
\begin{block}{Exercício}
Reescreva os números $0,6$; $1,37$; $0,222\dots$; $0,313131 \dots$ e
$1,123123123 \dots$ em forma de fração irredutível, ou seja, já
simplificada.
\end{block}
\end{frame}



%------------------------------------------------------------------------------------------------------------
\begin{frame}
\frametitle{Irracionais} %\framesubtitle{Exemplos}
\begin{Def}
O \sub{conjunto dos números irracionais} é constituído por todos os
números que possuem uma representação decimal infinita e não
periódica.
\end{Def}

\begin{Exem}
$\sqrt 2$, $e$ e $\pi$ são números irracionais. Provemos que $\sqrt
2 \notin \Q$.
\end{Exem}

Você sabia que existem infinitos ``maiores'' que outros? Qual
conjunto você diria que tem mais elementos: racionais ou
irracionais?
\end{frame}

%------------------------------------------------------------------------------------------------------------

\begin{frame}
\frametitle{Problema} %\framesubtitle{Exemplos}

O Grande Hotel Georg Cantor tinha uma infinidade de quartos,
numerados consecutivamente, um para cada número natural. Todos eram
igualmente confortáveis. Num fim de semana prolongado, o hotel
estava com seus quartos todos ocupados, quando chega um visitante. A
recepcionista vai logo dizendo: \\
-Sinto muito, mas não há vagas. \\
Ouvindo isto, o gerente interveio: \\
-Podemos abrigar o cavalheiro sim, senhora. \\
E ordena: \\
Transfira o hóspede do quarto 1 para o quarto 2, passe o do quarto 2
para o quarto 3 e assim por diante. Quem estiver no quarto $n$, mude
para o quarto $n+1$. Isto manterá todos alojados e deixará
disponível o quarto 1 para o recém chegado. Logo depois chegou um
ônibus com 30 passageiros, todos querendo hospedagem. Como deve
proceder a recepcionista para acomodar todos?
\\ Horas depois, chegou um trem com uma infinidade de
passageiros. Como proceder para acomodá-los?


\end{frame}

%------------------------------------------------------------------------------------------------------------