
\section{Potenciação}
\begin{frame}
\frametitle{Potenciação} %\framesubtitle{Exemplos}
\begin{definicao}
A \sub{potência} $n \in \N^\ast$ de um número real $a$ é definida
como sendo a multiplicação de $a$ por ele mesmo $n$ vezes, ou seja:
$$a^n = \underbrace{a \cdot a  \dots  a}_{n \text{
vezes}}.$$
\end{definicao}

\begin{definicao}
Quando $a \neq 0$, $a^0 = 1$. $0^0$ é uma indeterminação; \\
$a^{-n} = \frac{1}{a^n}$; \\
$a^{1/n} = \sqrt[n] a$, para $n> 0$.
\end{definicao}

É importante ressaltar que é comum definir $0^0 =1$ dependendo da
abordagem que se quer com as potências. Saiba mais
\href{https://pt.wikipedia.org/wiki/Zero_elevado_a_zero}{{\tt
aqui}}.
\end{frame}



%------------------------------------------------------------------------------------------------------------
\begin{frame}
\frametitle{Potenciação} %\framesubtitle{Exemplos}
\begin{proposicao}[Propriedades]
Sejam $a, b, n, m \in \R$ a menos que se diga o contrário.
\begin{enumerate}[i.]
	\item $a^m \cdot a^n = a^{m+n}$;
	\item $\frac {a^m}{a^n} = a^{m-n}$, $a \neq 0$;
	\item $\paren{a^m}^n = a^{m\cdot n}$;
	\item $a^{m^n} = a^{\overbrace{m \cdot m  \dots  m}^{n \text{
	vezes}}}$, $n \in \N^\ast$;
	\item $\paren{a \cdot b }^n= a^n \cdot b^n$;
	\item $\paren{\frac a b }^n = \frac {a^n} {b^n}$;
	\item $a^{m/n} = \sqrt[n]{a^m}$, $n \neq 0$.
\end{enumerate}
\end{proposicao}

\end{frame}



%------------------------------------------------------------------------------------------------------------
\begin{frame}
\frametitle{Potenciação} %\framesubtitle{Exemplos}
\begin{observacao}
Seja $a \in \R$. Temos que $\sqrt{a^2} = \modu a$. Mais geralmente,
$\sqrt[n] {a^n} = \modu a$ para $n$ par. \\
É errado dizer que $\sqrt 4 = \pm 2$. O correto é $\sqrt 4 = 2$,
mesmo que escrevas $\sqrt 4 = \sqrt{\paren {-2}^2}$. \\
Tal erro é comum, e o fator de confusão é que responder o conjunto
solução da equação $x^2=4$ não é equivalente a responder qual a raiz
de $4$, e sim responder quais números que elevados ao quadrado são
iguais a $4$.
\end{observacao}
\end{frame}



%------------------------------------------------------------------------------------------------------------

