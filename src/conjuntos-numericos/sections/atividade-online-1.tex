
\section{Atividade Online}
\begin{frame}
\frametitle{Atividade Online} 

\href{https://pt.khanacademy.org/math/algebra/rational-and-irrational-numbers/modal/e/recognizing-rational-and-irrational-numbers}
{{\tt Atividade 02 - Classifique números: racionais e irracionais}}

\href{https://pt.khanacademy.org/math/algebra/rational-and-irrational-numbers/modal/e/recognizing-rational-and-irrational-expressions}
{{\tt Atividade 03 - Expressões racionais versus irracionais}}

Veja o desempenho na Missão Álgebra I - Números Racionais e
Irracionais


\end{frame}

%------------------------------------------------------------------------------------------------------------
\begin{frame}
\frametitle{Reais} 
\begin{definicao}
À reunião de $\Q$ com o conjunto dos números irracionais, nomeamos
de \sub{conjunto dos números reais}. Denotamos por $\R$.
\end{definicao}

\begin{itemize}
\item $\R \setminus \Q = \set {x \tq x \text{ é irracional}}$;
\item Usamos os números reais para medir algo. A cada número real
está associado um ponto na reta graduada e vice-versa.
\item Entre dois números reais distintos sempre há pelo menos um número racional e um
irracional.
\href{https://pt.khanacademy.org/math/algebra/rational-and-irrational-numbers/proofs-concerning-irrational-numbers/v/proof-that-there-is-an-irrational-number-between-any-two-rational-numbers}
{{\tt Este vídeo}} do Khan Academy mostra que entre dois racionais
distintos sempre há pelo menos um número irracional.
\item A igualdade $0,999\dots = 1 $ é verdadeira?
\end{itemize}
\end{frame}



%------------------------------------------------------------------------------------------------------------
\begin{frame}
\frametitle{Complexos} 
\begin{definicao}
Chamamos $i = \sqrt {-1}$ de \sub{número imaginário}, e ao conjunto
$\C = \set{ a+bi \tq a,b \in \R}$ damos o nome de \sub{conjunto dos
números complexos}.
\end{definicao}

Seja $a+bi \in \C$. Nomeamos o número $a-bi$ de \sub{conjugado} de
$a+bi$.

Temos a seguinte cadeia de inclusões próprias: $\N \contido \Z
\contido \Q \contido \R \contido \C$.
\end{frame}



%------------------------------------------------------------------------------------------------------------