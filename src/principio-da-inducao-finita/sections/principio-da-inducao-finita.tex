
\section{Princípio da Indução Finita}
\begin{frame} \frametitle{Formulação Matemática}
\begin{Teo}[Princípio da Indução Finita]
Considere $n_0$ um inteiro não negativo. Suponhamos que, para cada
inteiro $n \geq n_0$, seja dada uma proposição $p \paren n$. Suponha
que se pode verificar as seguintes propriedades:
\begin{enumerate}[(a)]
	\item $p \paren{n_0}$ é verdadeira;
	\item Se $p \paren n$ é verdadeira, então $p \paren {n+1}$ também
	é verdadeira, para todo $n \geq n_0$.
\end{enumerate}
Então, $p \paren n$ é verdadeira para qualquer $n \geq n_0$.
\end{Teo}

A afirmação (a) é chamada de \sub{base da indução}, e a (b), de
\sub{passo indutivo}. O fato de que $p \paren n$ é verdadeira no
item (b) é chamado de \sub{hipótese de indução}.

\end{frame}

%------------------------------------------------------------------------------------------------------------

\begin{frame}
\frametitle{Demonstrando Identidades} %\framesubtitle{Exemplos}
\begin{Exem}
Demonstre que, para qualquer $n \in \N^\ast$, é válida a igualdade:
$$2+ 4+ \dots + 2n = n \paren {n+1}.$$
\end{Exem} \pause

\begin{Exem}
Demonstre que, para qualquer $n \in \N^\ast$, é válida a igualdade:
$$1+3+\dots +\paren {2n-1} = n^2.$$
\end{Exem}

\end{frame}



%------------------------------------------------------------------------------------------------------------

\begin{frame}
\frametitle{Demonstrando Desigualdades} %\framesubtitle{Exemplos}
\begin{Exem}
Mostre que, para todo número $n \in \N^\ast$, $n>3$, vale que
$$2^n < n!.$$
\end{Exem}\pause

\begin{Exem}
Prove  que, para todo $n \in \N^\ast$,
$$\underbrace{\sqrt{2+\sqrt{2+\sqrt{2+ \dots + \sqrt 2}}}}_{n  \text{ - radicais}} < 2.$$
\end{Exem}

\end{frame}


%------------------------------------------------------------------------------------------------------------

\begin{frame}
\frametitle{Indução na Geometria} %\framesubtitle{Exemplos}
\begin{Exem}
Seja $n \in \N$ tal que $n\geq 3$. Mostre  que podemos cobrir os
$n^2$ pontos no reticulado a seguir traçando $2n-2$ segmentos de
reta sem tirar o lápis do papel.
$$\underbrace{\begin{array}{ccccc}
								\bullet & \bullet & \bullet & \bullet & \bullet \\
								\bullet & \bullet & \bullet & \bullet & \bullet \\
								\bullet & \bullet & \bullet & \bullet & \bullet \\
								\bullet & \bullet & \bullet & \bullet & \bullet \\
								\bullet & \bullet & \bullet & \bullet & \bullet
							\end{array}
}_{n \times n \text{ - pontos}}$$
\end{Exem}


\end{frame}

%------------------------------------------------------------------------------------------------------------

\begin{frame}
\frametitle{A Moeda Falsa} %\framesubtitle{Exemplos}

\begin{Exem}
Um rei muito rico possui $3^n$ moedas de ouro. No entanto, uma dessas
moedas é falsa, e seu peso é menor que o peso das demais. Com uma
balança de 2 pratos e sem nenhum peso, mostre que é possível
encontrar a moeda falsa com apenas $n$ pesagens.
\end{Exem}

\end{frame}

%------------------------------------------------------------------------------------------------------------


\begin{frame}
\frametitle{Desigualdade das médias aritmética e geométrica} %\framesubtitle{Exemplos}

\begin{Teo}[Desigualdade das médias aritmética e geométrica]
Para quaisquer $a_1, a_2, \dots , a_n \in \R_+$ vale
\begin{equation}
		\sqrt[n]{a_1\dots a_n} \leq \frac {a_1 + \dots + a_n} n.
\end{equation}
\end{Teo}

\end{frame}

