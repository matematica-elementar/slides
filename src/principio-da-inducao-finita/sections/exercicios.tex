
\section{Exercícios}
\begin{frame}
\frametitle{Exercícios} 

\Ex{Demonstre, por indução, que para qualquer $n \in \N^\ast$ é
válida a igualdade:
$$1+2+ 3+ \dots + n = \frac{n \paren {n+1}} 2.$$ }

\Ex{Demonstre, por indução, que para qualquer $n \in \N^\ast$ é
válida a igualdade:
$$1^2 +2^2 + 3^2+ \dots + n^2 = \frac{n \paren {n+1} \paren{2n+1}} 6.$$ }

\Ex{Prove que $3^{n-1} < 2^{n^2}$ para todo $n \in \N^\ast$.
 }

\Ex{Mostre, por indução, que $$\paren{\frac {n+1}{n}}^n \leq n,$$
para todo $n \in \N^\ast$ tal que $n \geq 3$. \\ \emph{Dica: }
Mostre que $\frac{k+2}{k+1} \leq \frac{k+1} k$ para todo $k \in
\N^\ast$. Depois, eleve tudo à potência $k+1$.}
\end{frame}





%------------------------------------------------------------------------------------------------------------

\begin{frame}
\frametitle{Exercícios} 

\Ex{Prove que $$\frac 1 {\sqrt 1} +\frac 1 {\sqrt 2} +\frac 1 {\sqrt
3} + \dots + \frac 1 {\sqrt n} \geq \sqrt n,$$ para todo $ n \in
\N^\ast$.}


\Ex{Um subconjunto do plano é \emph{convexo} se o segmento ligando
quaisquer dois de seus pontos está totalmente nele contido. Os
exemplos mais simples de conjuntos convexos são o próprio plano e
qualquer semi-plano. Mostre que, para qualquer $n \in \N^\ast$, a
interseção $n$ de conjuntos convexos é ainda um conjunto convexo.}

\end{frame}

%------------------------------------------------------------------------------------------------------------
