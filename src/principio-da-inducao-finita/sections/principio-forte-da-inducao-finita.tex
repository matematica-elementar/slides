%------------------------------------------------------------------------------------------------------------
\section{Princípio Forte da Indução Finita}
\begin{frame}
\frametitle{Formulação Matemática} %\framesubtitle{Exemplos}


\begin{Teo}[Princípio Forte da Indução Finita]
Considere $n_0$ um inteiro não negativo. Suponhamos que, para cada
inteiro $n \geq n_0$, seja dada uma proposição $p \paren n$. Suponha
que se pode verificar as seguintes propriedades:
\begin{enumerate}[(a)]
	\item $p \paren{n_0}$ é verdadeira;
	\item Se para cada inteiro não negativo $k$, com $n_0 \leq k \leq n$, temos que
	 $p \paren k$ é verdadeira, então $p \paren {n+1}$ também
	é verdadeira.
\end{enumerate}
Então, $p \paren n$ é verdadeira para qualquer $n \geq n_0$.
\end{Teo}
\end{frame}



%------------------------------------------------------------------------------------------------------------

\begin{frame}
\frametitle{Teorema Fundamental da Aritmética} %\framesubtitle{Exemplos}

\begin{Teo}[Teorema Fundamental da Aritmética]
Todo número natural $N$ maior que 1 pode ser escrito como um produto
\begin{equation}\label{fatpri}
N = p_1 \cdot p_2 \cdot p_3 \dots p_m,
\end{equation}
onde $m \geq 1$ é um número natural e os $p_i$, $1 \leq i \leq m$,
são números primos. Além disso, a fatoração exibida na Equação \ref{fatpri} é única
se exigirmos que $p_1 \leq p_2 \leq \dots \leq p_m$.

\end{Teo}

\end{frame}
