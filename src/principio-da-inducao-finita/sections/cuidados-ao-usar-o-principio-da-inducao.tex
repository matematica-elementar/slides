

%------------------------------------------------------------------------------------------------------------
\section{Cuidados ao Usar o Princípio da Indução}
\begin{frame}
\frametitle{Cuidados ao Usar o Princípio da Indução} 

\begin{exemplo}
Critique a seguinte argumentação: Quer-se provar que todo número
natural é pequeno. Evidentemente, 1 é um número pequeno. Além disso,
se $n$ for pequeno, $n+1$ também o será, pois não se torna grande um
número pequeno simplesmente somando-lhe uma unidade. Logo, por
indução, todo número natural é pequeno.
\end{exemplo} \pause

\begin{exemplo}
\emph{\textbf{Afirmação:} Em um conjunto qualquer de $n$ bolas,
todas as bolas possuem a mesma cor.}

Analise a seguinte demonstração por indução para a afirmação
anterior e aponte o problema da demonstração, já que a afirmação é
claramente falsa.
\end{exemplo}
\end{frame}



%------------------------------------------------------------------------------------------------------------

\begin{frame}
\frametitle{Cuidados ao Usar o Princípio da Indução} 

\emph{\textbf{Afirmação:} Em um conjunto qualquer de $n$ bolas,
todas as bolas possuem a mesma cor.}

\sub{Dem.}: Para $n=1$, nossa proposição é verdadeira pois em
qualquer conjunto com uma bola, todas as bolas têm a mesma cor, pois
só existe uma bola. Assuma por hipótese de indução que a afirmação é
verdadeira para $n$ e provemos que a afirmação é verdadeira para
$n+1$. Ora, seja $A = \set {b_1, \dots, b_n, b_{n+1}}$ o conjunto
com $n+1$ bolas referido. Considere os subconjuntos $B$ e $C$ de $A$
com $n$ elementos, construídos como:
$$B = \set {b_1, b_2, \dots, b_n} \text{ e } C= \set{ b_2, \dots,
b_{n+1}}.$$ De fato ambos os conjuntos têm $n$ elementos. Assim, as
bolas $b_1, b_2, \dots , b_n$ têm a mesma cor. Do mesmo modo, as
bolas do conjunto $C$ também têm a mesma cor. Em particular, as
bolas $b_n$ e $b_{n+1}$ têm a mesma cor (ambas estão em $C$). Assim,
todas as $n+1$ bolas têm a mesma cor.


\end{frame}
