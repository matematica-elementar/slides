\documentclass[brazil, notheorems, 10pt]{beamer}
%
%   Arquivo de Configuração dos Slides
%


%
%   Pacotes utilizados
%

% Codificação dos caracteres em formato universal.
\usepackage[utf8]{inputenc}
\usepackage[T1]{fontenc}

% Traduz o texto gerados pelo LaTeX para português. ex.: Capítulo, Seção, Conteúdo.
\usepackage[brazil]{babel}

% Pacotes para ambientes matemáticos
\usepackage{amsmath}
\usepackage{amsthm}
\usepackage{amssymb}

% Diversas funções para o uso das aspas.
\usepackage{csquotes}

% Outros pacotes
\usepackage{hyperref}
\usepackage{tikz}
\usepackage{yfonts}
\usepackage{colortbl}
\usepackage{ragged2e}
\usepackage{helvet}
\usepackage{verbatim}


%
%   Tema
%

% Copyright 2007 by Till Tantau
%
% This file may be distributed and/or modified
%
% 1. under the LaTeX Project Public License and/or
% 2. under the GNU Public License.
%
% See the file doc/licenses/LICENSE for more details.


% Common packages


\usepackage{times}
 \mode<article> {
	\usepackage{times}
	\usepackage{mathptmx}
	\usepackage[left=1.5cm,right=6cm,top=1.5cm,bottom=3cm]{geometry}
}

\usepackage{hyperref}
\usepackage[T1]{fontenc}
\usepackage{amsmath,amssymb}
\usepackage{tikz}
\usepackage{colortbl}
\usepackage{yfonts}
\usepackage{colortbl}
\usepackage{translator} % comment this, if not available
\usepackage{ragged2e} % justifying
% Or whatever. Note that the encoding and the font should match. If T1
% does not look nice, try deleting the line with the fontenc.
\usepackage{helvet}
\usepackage{verbatim}


%\usepackage{lipsum}
%\usepackage{enumitem}


\usetheme[
%%% options passed to the outer theme
%    hidetitle,           % hide the (short) title in the sidebar
%    hideauthor,          % hide the (short) author in the sidebar
%    hideinstitute,       % hide the (short) institute in the bottom of the sidebar
%    shownavsym,          % show the navigation symbols
%    width=2cm,           % width of the sidebar (default is 2 cm)
%    hideothersubsections,% hide all subsections but the subsections in the current section
%    hideallsubsections,  % hide all subsections
		right               % right of left position of sidebar (default is right)
%%% options passed to the color theme
%    lightheaderbg,       % use a light header background
	]{AAUsidebar}

% If you want to change the colors of the various elements in the theme, edit and uncomment the following lines
% Change the bar and sidebar colors:
%\setbeamercolor{AAUsidebar}{fg=red!20,bg=red}
%\setbeamercolor{sidebar}{bg=red!20}
% Change the color of the structural elements:
%\setbeamercolor{structure}{fg=red}
% Change the frame title text color:
%\setbeamercolor{frametitle}{fg=blue}
% Change the normal text color background:
%\setbeamercolor{normal text}{bg=gray!10}
% Highlight the text in the sidebar
\usecolortheme{rose,sidebartab}
% ... and you can of course change a lot more - see the beamer user manual.

% colored hyperlinks
\newcommand{\chref}[2]{%
	\href{#1}{{\usebeamercolor[bg]{AAUsidebar}#2}}%
}



% specify a logo on the titlepage (you can specify additional logos an include them in
% institute command below
\pgfdeclareimage[height=1cm]{titlepagelogo}{theme/figures/ufrn2} % placed on the title page
\pgfdeclareimage[height=1cm]{titlepagelogo2}{theme/figures/imd} % placed on the title page
\titlegraphic{% is placed on the bottom of the title page
	\pgfuseimage{titlepagelogo}
	\hspace{1cm}\pgfuseimage{titlepagelogo2}
}


% Article version layout settings

\mode<article>

\makeatletter
\def\@listI{\leftmargin\leftmargini
	\parsep 0pt
	\topsep 5\p@   \@plus3\p@ \@minus5\p@
	\itemsep0pt}
\let\@listi=\@listI


\setbeamertemplate{frametitle}{\paragraph*{\insertframetitle\
		\ \small\insertframesubtitle}\ \par
}
\setbeamertemplate{frame end}{%
	\marginpar{\scriptsize\hbox to 1cm{\sffamily%
			\hfill\strut\insertshortlecture.\insertframenumber}\hrule height .2pt}}
\setlength{\marginparwidth}{1cm}
\setlength{\marginparsep}{4.5cm}

\def\@maketitle{\makechapter}

\def\makechapter{
	\newpage
	\null
	\vskip 2em%
	{%
		\parindent=0pt
		\raggedright
		\sffamily
		\vskip8pt
		\includegraphics[width=\linewidth]{theme/figures/imd.png}\par\vskip2em
		{\fontsize{36pt}{36pt}\selectfont Aula \insertshortlecture \par\vskip2pt}
		{\fontsize{24pt}{28pt}\selectfont \color{blue!50!black} \@title\par\vskip4pt}
		%{\Large\selectfont \color{blue!50!black} \insertsubtitle\par}
		\vskip10pt

		\normalsize\selectfont [Notas de Aula]
		Disciplina: \emph{\lecturename \ (\semestre)} \par\vskip1.5em
		\nomedoautor\hskip1em Email: \ \emaildoautor
	}
	\par
	\vskip 1.5em%
}

\let\origstartsection=\@startsection
\def\@startsection#1#2#3#4#5#6{%
	\origstartsection{#1}{#2}{#3}{#4}{#5}{#6\normalfont\sffamily\color{blue!50!black}\selectfont}}

\makeatother

\mode
<all>




% Typesetting Listings

\usepackage{listings}
\lstset{language=Java}

\alt<presentation>
{\lstset{%
	basicstyle=\footnotesize\ttfamily,
	commentstyle=\slshape\color{green!50!black},
	keywordstyle=\bfseries\color{blue!50!black},
	identifierstyle=\color{blue},
	stringstyle=\color{orange},
	escapechar=\#,
	emphstyle=\color{red}}
}
{
	\lstset{%
		basicstyle=\ttfamily,
		keywordstyle=\bfseries,
		commentstyle=\itshape,
		escapechar=\#,
		emphstyle=\bfseries\color{red}
	}
}



% Common theorem-like environments
%\usepackage{amsthm}

\setbeamertemplate{theorems}[numbered]

%
%	New useful definitions:
%

\newbox\mytempbox
\newdimen\mytempdimen

\newcommand\includegraphicscopyright[3][]{%
	\leavevmode\vbox{\vskip3pt\raggedright\setbox\mytempbox=\hbox{\includegraphics[#1]{#2}}%
		\mytempdimen=\wd\mytempbox\box\mytempbox\par\vskip1pt%
		\fontsize{3}{3.5}\selectfont{\color{black!25}{\vbox{\hsize=\mytempdimen#3}}}\vskip3pt%
}}

\newenvironment{colortabular}[1]{\medskip\rowcolors[]{1}{blue!20}{blue!10}\tabular{#1}\rowcolor{blue!40}}{\endtabular\medskip}

\def\equad{\leavevmode\hbox{}\quad}

\newenvironment{greencolortabular}[1]
{\medskip\rowcolors[]{1}{green!50!black!20}{green!50!black!10}%
	\tabular{#1}\rowcolor{green!50!black!40}}%
{\endtabular\medskip}

%\setbeamertemplate{theorem begin}{{ \inserttheoremheadfont
%\inserttheoremname \inserttheoremnumber
%\ifx\inserttheoremaddition\empty\else\ (\inserttheoremaddition)\fi%
%\inserttheorempunctuation }} \setbeamertemplate{theorem end}{}

\newcommand{\vu}{\vec{u}}
\newcommand{\vv}{\vec{v}}
\newcommand{\vi}{\vec{i}}
\newcommand{\vj}{\vec{j}}
\newcommand{\vk}{\vec{k}}
\newcommand{\vw}{\vec{w}}
\newcommand\segmento[2]{\overline{#1#2}}
\def\colc#1{\left[#1\right]}



%
%   Macros
%

\usepackage{macros/macros}


%
%   Ambientes
%

\theoremstyle{plain}
\newtheorem{teorema}{Teorema}

\theoremstyle{definition}
\newtheorem{definicao}[teorema]{Definição}
%\newtheorem{exercicio}{Exercício}

\theoremstyle{remark}
\newtheorem{obs}[teorema]{Observação}
\newtheorem{observacao}[teorema]{Observação}
\newtheorem{corolario}[teorema]{Corolário}
\newtheorem{exemplo}[teorema]{Exemplo}
\newtheorem{lema}[teorema]{Lema}
\newtheorem{proposicao}[teorema]{Proposição}

\newcounter{exercicios}
\newenvironment{exercicio}{\stepcounter{exercicios} \textbf{\arabic{exercicios}}.}{}

% compatibilidade
\newcommand{\Ex}[1]{\begin{exercicio}#1\end{exercicio}}

%
%   Definições e comandos auxiliares do preâmbulo
%

\newcommand{\capitulo}[1]{\lecture[#1]{Capítulo}}
\newcommand{\aula}[1]{\subtitle{#1}}
\newcommand{\autor}{Igor Oliveira}
\newcommand{\email}{\href{mailto:matematicaelementar@imd.ufrn.br}{\texttt{matematicaelementar@imd.ufrn.br}}}
\newcommand{\disciplina}{Matemática Elementar}
\newcommand{\codigo}{IMD1001}

\title{\disciplina}
\date{\today}
\author[\autor]
{
    \autor\\
    \email
}

\def\lecturename{\codigo

\disciplina}

\institute[
	UFRN\\
	Natal-RN
]
{
	Instituto Metrópole Digital\\
	Universidade Federal do Rio Grande do Norte\\
	Natal-RN

}

% compatibilidade
\newcommand{\vu}{\vec{u}}
\newcommand{\vv}{\vec{v}}
\newcommand{\vi}{\vec{i}}
\newcommand{\vj}{\vec{j}}
\newcommand{\vk}{\vec{k}}
\newcommand{\vw}{\vec{w}}
\newcommand{\segmento}[2]{\overline{#1#2}}
\def\colc#1{\left[#1\right]}
\newcommand{\negacao}{\sim}

\justifying


%lecture[number of class]{type}
\lecture[4]{Capítulo}

\def\lecturename{IMD1001


Matemática Elementar}
\def\semestre{2018.2}
\def\nomedoautor{Igor Oliveira}
\def\emaildoautor{\href{mailto:igoroliveira@imd.ufrn.br}{{\tt igoroliveira@imd.ufrn.br}}}

\title[\lecturename]% optional, use only with long paper titles
{Matemática Elementar}

\subtitle{Princípio da Indução Finita}  % could also be a conference name

\date{\today}

\author[Igor Oliveira] % optional, use only with lots of authors
{
	\nomedoautor\\
	\emaildoautor
}
% - Give the names in the same order as they appear in the paper.
% - Use the \inst{?} command only if the authors have different
%   affiliation. See the beamer manual for an example

\institute[
%  {\includegraphics[scale=0.2]{aau_segl}}\\ %insert a company, department or university logo
	UFRN\\
	Natal-RN
] % optional - is placed in the bottom of the sidebar on every slide
{% is placed on the title page
	Instituto Metrópole Digital\\
	Universidade Federal do Rio Grande do Norte\\
	Natal-RN

	%there must be an empty line above this line - otherwise some unwanted space is added between the university and the country (I do not know why;( )
}


\begin{document}

\mode<article>
{
\begin{frame} % the plain option removes the sidebar and header from the title page
	\maketitle
\end{frame}

% the titlepage
\begin{frame}[plain,noframenumbering] % the plain option removes the sidebar and header from the title page
	\titlepage
\end{frame}
}

\mode<presentation>
{
% the titlepage
{\imagemfundo
\begin{frame}[plain,noframenumbering] % the plain option removes the sidebar and header from the title page
	\titlepage
\end{frame}}
}

% TOC
\begin{frame}{Índice}{}
\tableofcontents
\end{frame}
%%%%%%%%%%%%%%%%

\section{Introdução}

\begin{frame}  \frametitle{Apresentação da Aula}

Considere as funções
$$\begin{array}{cccc}
p : & \R & \to     & \R_+ \\
		 &  x & \mapsto & x^2
\end{array}
\text{\ \ \  e \ \ \ }
\begin{array}{cccc}
q : & \R_+ & \to     & \R \\
		 &  x & \mapsto & \sqrt x
\end{array}.$$
As funções $p$ e $q$ são inversas uma da outra? \\ \pause Elas são
bijetivas? \\ \pause
Quais outras informações podemos dizer acerca dessas funções?



\end{frame}


\section{Princípio da Indução Finita}
\begin{frame} \frametitle{Formulação Matemática}
\begin{teorema}[Princípio da Indução Finita]
Considere $n_0$ um inteiro não negativo. Suponhamos que, para cada
inteiro $n \geq n_0$, seja dada uma proposição $p \paren n$. Suponha
que se pode verificar as seguintes propriedades:
\begin{enumerate}[(a)]
	\item $p \paren{n_0}$ é verdadeira;
	\item Se $p \paren n$ é verdadeira, então $p \paren {n+1}$ também
	é verdadeira, para todo $n \geq n_0$.
\end{enumerate}
Então, $p \paren n$ é verdadeira para qualquer $n \geq n_0$.
\end{teorema}

A afirmação (a) é chamada de \sub{base da indução}, e a (b), de
\sub{passo indutivo}. O fato de que $p \paren n$ é verdadeira no
item (b) é chamado de \sub{hipótese de indução}.

\end{frame}

%------------------------------------------------------------------------------------------------------------

\begin{frame}
\frametitle{Demonstrando Identidades} 
\begin{exemplo}
Demonstre que, para qualquer $n \in \N^\ast$, é válida a igualdade:
$$2+ 4+ \dots + 2n = n \paren {n+1}.$$
\end{exemplo} \pause

\begin{exemplo}
Demonstre que, para qualquer $n \in \N^\ast$, é válida a igualdade:
$$1+3+\dots +\paren {2n-1} = n^2.$$
\end{exemplo}

\end{frame}



%------------------------------------------------------------------------------------------------------------

\begin{frame}
\frametitle{Demonstrando Desigualdades} 
\begin{exemplo}
Mostre que, para todo número $n \in \N^\ast$, $n>3$, vale que
$$2^n < n!.$$
\end{exemplo}\pause

\begin{exemplo}
Prove  que, para todo $n \in \N^\ast$,
$$\underbrace{\sqrt{2+\sqrt{2+\sqrt{2+ \dots + \sqrt 2}}}}_{n  \text{ - radicais}} < 2.$$
\end{exemplo}

\end{frame}


%------------------------------------------------------------------------------------------------------------

\begin{frame}
\frametitle{Indução na Geometria} 
\begin{exemplo}
Seja $n \in \N$ tal que $n\geq 3$. Mostre  que podemos cobrir os
$n^2$ pontos no reticulado a seguir traçando $2n-2$ segmentos de
reta sem tirar o lápis do papel.
$$\underbrace{\begin{array}{ccccc}
								\bullet & \bullet & \bullet & \bullet & \bullet \\
								\bullet & \bullet & \bullet & \bullet & \bullet \\
								\bullet & \bullet & \bullet & \bullet & \bullet \\
								\bullet & \bullet & \bullet & \bullet & \bullet \\
								\bullet & \bullet & \bullet & \bullet & \bullet
							\end{array}
}_{n \times n \text{ - pontos}}$$
\end{exemplo}


\end{frame}

%------------------------------------------------------------------------------------------------------------

\begin{frame}
\frametitle{A Moeda Falsa} 

\begin{exemplo}
Um rei muito rico possui $3^n$ moedas de ouro. No entanto, uma dessas
moedas é falsa, e seu peso é menor que o peso das demais. Com uma
balança de 2 pratos e sem nenhum peso, mostre que é possível
encontrar a moeda falsa com apenas $n$ pesagens.
\end{exemplo}

\end{frame}

%------------------------------------------------------------------------------------------------------------


\begin{frame}
\frametitle{Desigualdade das médias aritmética e geométrica} 

\begin{teorema}[Desigualdade das médias aritmética e geométrica]
Para quaisquer $a_1, a_2, \dots , a_n \in \R_+$ vale
\begin{equation}
		\sqrt[n]{a_1\dots a_n} \leq \frac {a_1 + \dots + a_n} n.
\end{equation}
\end{teorema}

\end{frame}


%------------------------------------------------------------------------------------------------------------
\section{Princípio Forte da Indução Finita}
\begin{frame}
\frametitle{Formulação Matemática} %\framesubtitle{Exemplos}


\begin{Teo}[Princípio Forte da Indução Finita]
Considere $n_0$ um inteiro não negativo. Suponhamos que, para cada
inteiro $n \geq n_0$, seja dada uma proposição $p \paren n$. Suponha
que se pode verificar as seguintes propriedades:
\begin{enumerate}[(a)]
	\item $p \paren{n_0}$ é verdadeira;
	\item Se para cada inteiro não negativo $k$, com $n_0 \leq k \leq n$, temos que
	 $p \paren k$ é verdadeira, então $p \paren {n+1}$ também
	é verdadeira.
\end{enumerate}
Então, $p \paren n$ é verdadeira para qualquer $n \geq n_0$.
\end{Teo}
\end{frame}



%------------------------------------------------------------------------------------------------------------

\begin{frame}
\frametitle{Teorema Fundamental da Aritmética} %\framesubtitle{Exemplos}

\begin{Teo}[Teorema Fundamental da Aritmética]
Todo número natural $N$ maior que 1 pode ser escrito como um produto
\begin{equation}\label{fatpri}
N = p_1 \cdot p_2 \cdot p_3 \dots p_m,
\end{equation}
onde $m \geq 1$ é um número natural e os $p_i$, $1 \leq i \leq m$,
são números primos. Além disso, a fatoração exibida na Equação \ref{fatpri} é única
se exigirmos que $p_1 \leq p_2 \leq \dots \leq p_m$.

\end{Teo}

\end{frame}



%------------------------------------------------------------------------------------------------------------
\section{Cuidados ao Usar o Princípio da Indução}
\begin{frame}
\frametitle{Cuidados ao Usar o Princípio da Indução} %\framesubtitle{Exemplos}

\begin{exemplo}
Critique a seguinte argumentação: Quer-se provar que todo número
natural é pequeno. Evidentemente, 1 é um número pequeno. Além disso,
se $n$ for pequeno, $n+1$ também o será, pois não se torna grande um
número pequeno simplesmente somando-lhe uma unidade. Logo, por
indução, todo número natural é pequeno.
\end{exemplo} \pause

\begin{exemplo}
\emph{\textbf{Afirmação:} Em um conjunto qualquer de $n$ bolas,
todas as bolas possuem a mesma cor.}

Analise a seguinte demonstração por indução para a afirmação
anterior e aponte o problema da demonstração, já que a afirmação é
claramente falsa.
\end{exemplo}
\end{frame}



%------------------------------------------------------------------------------------------------------------

\begin{frame}
\frametitle{Cuidados ao Usar o Princípio da Indução} %\framesubtitle{Exemplos}

\emph{\textbf{Afirmação:} Em um conjunto qualquer de $n$ bolas,
todas as bolas possuem a mesma cor.}

\sub{Dem.}: Para $n=1$, nossa proposição é verdadeira pois em
qualquer conjunto com uma bola, todas as bolas têm a mesma cor, pois
só existe uma bola. Assuma por hipótese de indução que a afirmação é
verdadeira para $n$ e provemos que a afirmação é verdadeira para
$n+1$. Ora, seja $A = \set {b_1, \dots, b_n, b_{n+1}}$ o conjunto
com $n+1$ bolas referido. Considere os subconjuntos $B$ e $C$ de $A$
com $n$ elementos, construídos como:
$$B = \set {b_1, b_2, \dots, b_n} \text{ e } C= \set{ b_2, \dots,
b_{n+1}}.$$ De fato ambos os conjuntos têm $n$ elementos. Assim, as
bolas $b_1, b_2, \dots , b_n$ têm a mesma cor. Do mesmo modo, as
bolas do conjunto $C$ também têm a mesma cor. Em particular, as
bolas $b_n$ e $b_{n+1}$ têm a mesma cor (ambas estão em $C$). Assim,
todas as $n+1$ bolas têm a mesma cor.


\end{frame}

\section{Exercícios}
\begin{frame}
\frametitle{Exercícios} 

    \setcounter{exercicios}{12}

	\begin{exercicio}
		Considere os pontos $A = (x_1, y_1)$ e $B = (x_2, y_2)$ distintos e pertencentes a um plano cartesiano. Responda o que se pede:
		\begin{enumerate}[a)]
			\item Qual as equações paramétricas da reta que passa por $A$ e $B$?
			\item Mostre que o ponto $M = \paren{\dfrac{x_1+x_2} 2 , \dfrac{y_1+y_2} 2 }$ pertence à reta que passa por $A$ e $B$;
			\item Mostre que  $d(A, M) = d(M,B)$ e conclua que $M$ é o ponto médio do segmento $AB$.
		\end{enumerate}
	\end{exercicio}

	\begin{exercicio}
		Mostre que $f : (- \infty ; -4] \to \R$, tal que $f(x) = -x^2 - 8x -12$, é uma função crescente.
	\end{exercicio}

	\begin{exercicio}
		Seja a função $f:[3;5]\to\reals$ tal que $f(x)=-x^2+4x-3$.
		\begin{enumerate}[a)]
		  \item Mostre que $f$ é decrescente.
		  \item $f$ possui máximo absoluto? Se sim, ocorre em qual ponto?
		  \item $f$ possui mínimo absoluto? Se sim, ocorre em qual ponto?
		\end{enumerate}
	  \end{exercicio}

	\end{frame}


	%------------------------------------------------------------------------------------------------------------
	
	\begin{frame}
	\frametitle{Exercícios} 

	  \begin{exercicio}
		  Considere a função $f: \reals_- \to \reals^\ast_+$ tal que $f(x) = \dfrac{1}{1+x^2}$. Responda as seguintes perguntas apresentando as respectivas justificativas.
		  \begin{enumerate}[a)]
		  \item $f$ é monótona? Se sim, de que tipo? Se não, $f$ possui algum intervalo de monotonicidade?
		  \item $f$ possui máximo absoluto?
		  \item $f$ possui mínimo absoluto?
		  \item $f$ é limitada?
		  \end{enumerate}
	  \end{exercicio}
	  
	  \begin{exercicio}
		Considere a função real $f$ tal que $f(x) = -x^2 +2x +8$.
	  \begin{enumerate}[a)]
	  \item Mostre que $f$ é crescente no intervalo $( - \infty , 1]$;
	  \item Mostre que $f$ é decrescente no intervalo $[1, + \infty )$;
	  \item Use os itens anteriores para concluir que $1 \in \mathbb R$ é um ponto de máximo absoluto de $f$.
	  \end{enumerate}  
	  \end{exercicio}

	\end{frame}


	%------------------------------------------------------------------------------------------------------------
	
	\begin{frame}
	\frametitle{Exercícios} 

	\Ex{
	Considere a função $f: (- \infty , 2] \to \R$ tal que $f(x) = |x - 2| + 3$.
	\begin{enumerate}[a)]
		\item  $f$ é monótona de que tipo?
		\item  Qual dos extremos absolutos $f$ não possui?
	\end{enumerate}
}

	\Ex{
		Considere as funções $f: \R \to \R_+$ tal que $f(x) = x^2+3$ e $g: (-\infty ; 5] \to \R$ tal que $g(x) = \sqrt{x^2 - 10x +27}$. Faça o que se pede:
            \begin{enumerate}[a)]
                \item Calcule $(f \circ g)$ e $(g \circ f)$. Caso não seja possível, justifique;
                \item  Verifique se alguma das funções compostas que você calculou no primeiro item é monótona (crescente ou decrescente);
                \item  Verifique se alguma das funções compostas que você calculou no primeiro item possui máximo ou mínimo absoluto (escolha só um).
            \end{enumerate}
	}

	\Ex{
		Verifique os exercícios do capítulo que tem as leis de formação das funções iguais. Considere como se fosse um só exercício e tente refazer cada item usando outra informação dada ou pedida pelo exercício.
	}

\end{frame}


%------------------------------------------------------------------------------------------------------------

\begin{frame}
\frametitle{Exercícios} 

    \Ex{Sejam $f: \R \to \R $. Determine se as afirmações abaixo são
verdadeiras ou falsas, justificando suas respostas. As funções que
forem usadas como contraexemplo podem ser exibidas somente com o
esboço de seu gráfico.
\begin{enumerate}[(a)]
	\item Se $f$ é limitada superiormente, então $f$ tem pelo menos um máximo absoluto;
	\item Se $f$ é limitada superiormente, então $f$ tem pelo menos um máximo local;
	\item Se $f$ tem um máximo local, então $f$ tem um máximo absoluto;
	\item Todo máximo local de $f$ é máximo absoluto;
	\item Todo máximo absoluto de $f$ é máximo local;
	\item Se $x_0$ é o ponto de extremo local de $f$, então é ponto de
	extremo local de $f^2$, onde $(f^2)(x) = f(x) \cdot f(x)$;
	\item Se $x_0$ é o ponto de extremo local de $f^2$, então é ponto de
	extremo local de $f$.
\end{enumerate}}
\end{frame}


%------------------------------------------------------------------------------------------------------------

\begin{frame}
\frametitle{Exercícios} 

\begin{exercicio}
	Seja $f: \N \to \R $ e $g: \R \to \N$. Determine se as afirmações abaixo são
	verdadeiras ou falsas, justificando suas respostas. As funções que
	forem usadas como contraexemplo podem ser exibidas somente com o
	esboço de seu gráfico.
	\begin{enumerate}[a)]
	  \item A função $g$ pode ser ilimitada inferiormente;
	  \item $f$ é limitada superiormente ou $f$ é limitada inferiormente.
	\end{enumerate}
	\end{exercicio}

\Ex{Sejam $f: \R \to \R $ e $g: \R \to \R$. Determine se as
afirmações abaixo são verdadeiras ou falsas, justificando suas
respostas. As funções que forem usadas como contraexemplo podem ser
exibidas somente com o esboço de seu gráfico.
\begin{enumerate}[(a)]
	\item Se $f$ e $g$ são crescentes, então a composta $f \circ g$ é uma função crescente;
	\item Se $f$ e $g$ são crescentes, então o produto $f\cdot g$ é
	uma função crescente, onde $(f \cdot g)(x) = f(x) \cdot g(x)$;
	\item Se $f$ é crescente em $A \contido \R$ e em $B \contido \R$, então $f$ é crescente em $A \uniao B \contido \R$.
\end{enumerate}}

\end{frame}


%------------------------------------------------------------------------------------------------------------

\begin{frame}
\frametitle{Exercícios} 

\begin{exercicio}
	Seja $f$ uma função real. 
	\begin{enumerate}[a)]
		\item Suponha que $f$ é constante. Mostre que $f$ é não crescente e não decrescente;
		\item Suponha que $f$ é não crescente e não decrescente. Mostre que $f$ é constante.
	\end{enumerate}
%	Mostre que uma função real é constante se, e somente se, é não decrescente e não crescente.
  \end{exercicio}
  
  \begin{exercicio}
	Sejam $f : \reals \to \reals$ e $A$ e $B$ intervalos reais tais que $A \inter B$ é um intervalo não
	degenerado, ou seja, que possui pelo menos dois números. Mostre que, se $f$ é crescente
	em $A$ e em $B$, então $f$ é crescente em $A\inter B$.
  \end{exercicio}

\Ex{Mostre que a função inversa de uma função crescente é também uma
função crescente. E a função inversa de uma função decrescente é
decrescente.}

\end{frame}


%------------------------------------------------------------------------------------------------------------

\begin{frame}
\frametitle{Exercícios} 

\Ex{
	Dizemos que uma função $f: \R \to \R$ é \emph{par} quando se tem $f(-t) = f(t)$ para todo $t \in \R$. Se for o caso de $f(-t) = -f(t)$ para todo $t \in \R$, dizemos que $f$ é \emph{ímpar}.

  Considere a função real $f: \R \to \R$. Demonstre, ou refute com um contraexemplo, as afirmações abaixo:
  \begin{enumerate}[a)]
    \item Se $f$ é par e $x_0 \in \R$ é um ponto de máximo absoluto, então $-x_0 \in \R$ é também um ponto de máximo absoluto;
    \item Se $f$ é ímpar e $x_0 \in \R$ é um ponto de mínimo absoluto, então $-x_0 \in \R$ é um ponto de máximo absoluto;
	\item Se $f$ é par e limitada superiormente, então $f$ é limitada inferiormente;
    \item Se $f$ é ímpar e limitada superiormente, então $f$ é limitada inferiormente.
  \end{enumerate}
}

\end{frame}

%------------------------------------------------------------------------------------------------------------

\section{Bibliografia}

\frame{
\frametitle{Bibliografia}

\begin{thebibliography}{99}

\bibitem {label1}
OLIVEIRA, Krerley I M; FERNÁNDEZ, Adán J C.
\newblock \emph{Iniciação à Matemática: um Curso com Problemas e Soluções}.
\newblock 2. ed. Rio de Janeiro: SBM, 2010.

\bibitem {label2}
OLIVEIRA, Krerley I M; FERNÁNDEZ, Adán J C.
\newblock \emph{Estágio dos Alunos Bolsistas - OBMEP 2005 - 4. Equações, Inequações e Desigualdades}.
\newblock Rio de Janeiro: SBM, 2006.

\end{thebibliography}
}


\end{document}
