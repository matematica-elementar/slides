\section{O número $e$}
\begin{frame}
\frametitle{O número $e$} 

\begin{definicao}
Definimos o número $e$ como sendo o número cujos valores aproximados
por falta são os números racionais da forma $$
\paren{1+ \frac 1 n}^n , n\in \N^\ast.$$ Em outras palavras, quanto
maior for $n \in \N^\ast$, melhor a aproximação de $\paren{1+ \frac
1 n}^n$ para $e$, e ela se dá na medida que desejarmos.
\end{definicao}

\end{frame}

%------------------------------------------------------------------------------------------------------------



\begin{frame}
\frametitle{O número $e$} 

O número $e$ é irracional. Um valor aproximado dessa importante
constante é $e = 2{,}718281828459$.

Muito usado como base das funções exponenciais e logarítmicas,
principalmente no estudo dessas funções no Cálculo Infinitesimal, o
logaritmo na base $e$ recebe uma notação e nomenclatura especial.
Denotamos $$\log_e x = \ln x$$ e o chamamos de \sub{logaritmo
natural}.


\end{frame}