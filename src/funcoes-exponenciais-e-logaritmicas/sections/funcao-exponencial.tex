\section{Função Exponencial}
\begin{frame} \frametitle{Definição}
\begin{definicao}
Seja $a$ um número real positivo diferente de 1. Chamamos de
\sub{função exponencial} uma função $f: \R \to \R_+^\ast$ com lei de
formação $f(x) =
 a^x$. O número $a$ é chamado de \sub{base} da função exponencial.
\end{definicao}\pause



\begin{definicao}
Dizemos que uma função $f: \R \to \R$ é de \sub{tipo exponencial}
quando $f(x) =b\cdot a^x$, onde $a,b \in\R$, $b$ é não nulo e $a$ é
positivo e diferente de 1.
\end{definicao}





\end{frame}

%------------------------------------------------------------------------------------------------------------

\begin{frame}
\frametitle{Propriedades} 

\begin{proposicao}[Propriedades Fundamentais da Função Exponencial]
Seja $f: \R \to \R_+^\ast$ uma função exponencial de base $a$.
Então, para quaisquer $x, y \in \R$ valem:
\begin{enumerate}[(i)]
	\item  $a^{x+y} = a^x\cdot a^y$, ou seja, $f(x+y) = f(x)\cdot f(y)$;
	\item $a^1 = a$, ou seja, $f(1) = a$;
	\item $x<y \implica \begin{cases} a^x < a^y, \ \ \ \text{ quando } \ \ \ a>1 \\
																		a^y < a^x, \ \ \ \text{ quando } \ \ \ 0<a<1
											 \end{cases}.$
\end{enumerate}
\end{proposicao}




\end{frame}

%------------------------------------------------------------------------------------------------------------

\begin{frame}
\frametitle{Propriedades} 

Devido a essas propriedades, podemos concluir os seguintes
resultados acerca de uma função exponencial $f: \R \to \R_+^\ast$:
\begin{itemize}
	\item $f^{-1}(0) = \emptyset$, ou seja, $f$ não pode assumir o valor
	zero;
	\item $f(x)>0$, para todo $x \in \R$;
	\item Ao escolhermos o conjunto $\R_+^\ast$ como contradomínio de $f$, obtemos
	a sobrejetividade da função;
	\item $f$ é ilimitada superiormente;
	\item O gráfico de $f$ é uma linha contínua;
	\item $f$ é bijetiva e crescente se $a>1$, ou decrescente se
	$0<a<1$.
\end{itemize}
\end{frame}

