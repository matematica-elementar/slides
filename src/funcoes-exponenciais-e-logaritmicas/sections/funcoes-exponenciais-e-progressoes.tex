\section{Funções Exponenciais e Progressões}
\begin{frame}
\frametitle{Funções Exponenciais e Progressões} 

\begin{proposicao}
Seja  $f: \R \to \R$. Se $f$ é uma função do tipo exponencial e
$\paren{x_1, x_2, \dots , x_i, \dots}$ é uma PA, então a sequência
formada pelos pontos $y_i = f(x_i)$, $i \in \N^{\ast}$ é uma PG.
Reciprocamente, se $f$ for monótona injetiva e transformar qualquer
PA $\paren{x_1, x_2, \dots , x_i, \dots}$ numa PG com termo geral
$y_i = f(x_i)$, $i \in \N^{\ast}$ então $f$ é uma função real tal
que $f(x) = b \cdot a^x$ com $b = f(0)$ e $a = \frac {f(1)} {f(0)}$.
\end{proposicao}

\end{frame}

