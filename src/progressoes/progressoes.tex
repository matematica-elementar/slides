\documentclass[10pt]{beamer}
%
%   Arquivo de Configuração dos Slides
%


%
%   Pacotes utilizados
%

% Codificação dos caracteres em formato universal.
\usepackage[utf8]{inputenc}
\usepackage[T1]{fontenc}

% Traduz o texto gerados pelo LaTeX para português. ex.: Capítulo, Seção, Conteúdo.
\usepackage[brazil]{babel}

% Pacotes para ambientes matemáticos
\usepackage{amsmath}
\usepackage{amsthm}
\usepackage{amssymb}

% Diversas funções para o uso das aspas.
\usepackage{csquotes}

% Outros pacotes
\usepackage{hyperref}
\usepackage{tikz}
\usepackage{yfonts}
\usepackage{colortbl}
\usepackage{ragged2e}
\usepackage{helvet}
\usepackage{verbatim}


%
%   Tema
%

% Copyright 2007 by Till Tantau
%
% This file may be distributed and/or modified
%
% 1. under the LaTeX Project Public License and/or
% 2. under the GNU Public License.
%
% See the file doc/licenses/LICENSE for more details.


% Common packages


\usepackage{times}
 \mode<article> {
	\usepackage{times}
	\usepackage{mathptmx}
	\usepackage[left=1.5cm,right=6cm,top=1.5cm,bottom=3cm]{geometry}
}

\usepackage{hyperref}
\usepackage[T1]{fontenc}
\usepackage{amsmath,amssymb}
\usepackage{tikz}
\usepackage{colortbl}
\usepackage{yfonts}
\usepackage{colortbl}
\usepackage{translator} % comment this, if not available
\usepackage{ragged2e} % justifying
% Or whatever. Note that the encoding and the font should match. If T1
% does not look nice, try deleting the line with the fontenc.
\usepackage{helvet}
\usepackage{verbatim}


%\usepackage{lipsum}
%\usepackage{enumitem}


\usetheme[
%%% options passed to the outer theme
%    hidetitle,           % hide the (short) title in the sidebar
%    hideauthor,          % hide the (short) author in the sidebar
%    hideinstitute,       % hide the (short) institute in the bottom of the sidebar
%    shownavsym,          % show the navigation symbols
%    width=2cm,           % width of the sidebar (default is 2 cm)
%    hideothersubsections,% hide all subsections but the subsections in the current section
%    hideallsubsections,  % hide all subsections
		right               % right of left position of sidebar (default is right)
%%% options passed to the color theme
%    lightheaderbg,       % use a light header background
	]{AAUsidebar}

% If you want to change the colors of the various elements in the theme, edit and uncomment the following lines
% Change the bar and sidebar colors:
%\setbeamercolor{AAUsidebar}{fg=red!20,bg=red}
%\setbeamercolor{sidebar}{bg=red!20}
% Change the color of the structural elements:
%\setbeamercolor{structure}{fg=red}
% Change the frame title text color:
%\setbeamercolor{frametitle}{fg=blue}
% Change the normal text color background:
%\setbeamercolor{normal text}{bg=gray!10}
% Highlight the text in the sidebar
\usecolortheme{rose,sidebartab}
% ... and you can of course change a lot more - see the beamer user manual.

% colored hyperlinks
\newcommand{\chref}[2]{%
	\href{#1}{{\usebeamercolor[bg]{AAUsidebar}#2}}%
}



% specify a logo on the titlepage (you can specify additional logos an include them in
% institute command below
\pgfdeclareimage[height=1cm]{titlepagelogo}{theme/figures/ufrn2} % placed on the title page
\pgfdeclareimage[height=1cm]{titlepagelogo2}{theme/figures/imd} % placed on the title page
\titlegraphic{% is placed on the bottom of the title page
	\pgfuseimage{titlepagelogo}
	\hspace{1cm}\pgfuseimage{titlepagelogo2}
}


% Article version layout settings

\mode<article>

\makeatletter
\def\@listI{\leftmargin\leftmargini
	\parsep 0pt
	\topsep 5\p@   \@plus3\p@ \@minus5\p@
	\itemsep0pt}
\let\@listi=\@listI


\setbeamertemplate{frametitle}{\paragraph*{\insertframetitle\
		\ \small\insertframesubtitle}\ \par
}
\setbeamertemplate{frame end}{%
	\marginpar{\scriptsize\hbox to 1cm{\sffamily%
			\hfill\strut\insertshortlecture.\insertframenumber}\hrule height .2pt}}
\setlength{\marginparwidth}{1cm}
\setlength{\marginparsep}{4.5cm}

\def\@maketitle{\makechapter}

\def\makechapter{
	\newpage
	\null
	\vskip 2em%
	{%
		\parindent=0pt
		\raggedright
		\sffamily
		\vskip8pt
		\includegraphics[width=\linewidth]{theme/figures/imd.png}\par\vskip2em
		{\fontsize{36pt}{36pt}\selectfont Aula \insertshortlecture \par\vskip2pt}
		{\fontsize{24pt}{28pt}\selectfont \color{blue!50!black} \@title\par\vskip4pt}
		%{\Large\selectfont \color{blue!50!black} \insertsubtitle\par}
		\vskip10pt

		\normalsize\selectfont [Notas de Aula]
		Disciplina: \emph{\lecturename \ (\semestre)} \par\vskip1.5em
		\nomedoautor\hskip1em Email: \ \emaildoautor
	}
	\par
	\vskip 1.5em%
}

\let\origstartsection=\@startsection
\def\@startsection#1#2#3#4#5#6{%
	\origstartsection{#1}{#2}{#3}{#4}{#5}{#6\normalfont\sffamily\color{blue!50!black}\selectfont}}

\makeatother

\mode
<all>




% Typesetting Listings

\usepackage{listings}
\lstset{language=Java}

\alt<presentation>
{\lstset{%
	basicstyle=\footnotesize\ttfamily,
	commentstyle=\slshape\color{green!50!black},
	keywordstyle=\bfseries\color{blue!50!black},
	identifierstyle=\color{blue},
	stringstyle=\color{orange},
	escapechar=\#,
	emphstyle=\color{red}}
}
{
	\lstset{%
		basicstyle=\ttfamily,
		keywordstyle=\bfseries,
		commentstyle=\itshape,
		escapechar=\#,
		emphstyle=\bfseries\color{red}
	}
}



% Common theorem-like environments
%\usepackage{amsthm}

\setbeamertemplate{theorems}[numbered]

%
%	New useful definitions:
%

\newbox\mytempbox
\newdimen\mytempdimen

\newcommand\includegraphicscopyright[3][]{%
	\leavevmode\vbox{\vskip3pt\raggedright\setbox\mytempbox=\hbox{\includegraphics[#1]{#2}}%
		\mytempdimen=\wd\mytempbox\box\mytempbox\par\vskip1pt%
		\fontsize{3}{3.5}\selectfont{\color{black!25}{\vbox{\hsize=\mytempdimen#3}}}\vskip3pt%
}}

\newenvironment{colortabular}[1]{\medskip\rowcolors[]{1}{blue!20}{blue!10}\tabular{#1}\rowcolor{blue!40}}{\endtabular\medskip}

\def\equad{\leavevmode\hbox{}\quad}

\newenvironment{greencolortabular}[1]
{\medskip\rowcolors[]{1}{green!50!black!20}{green!50!black!10}%
	\tabular{#1}\rowcolor{green!50!black!40}}%
{\endtabular\medskip}

%\setbeamertemplate{theorem begin}{{ \inserttheoremheadfont
%\inserttheoremname \inserttheoremnumber
%\ifx\inserttheoremaddition\empty\else\ (\inserttheoremaddition)\fi%
%\inserttheorempunctuation }} \setbeamertemplate{theorem end}{}

\newcommand{\vu}{\vec{u}}
\newcommand{\vv}{\vec{v}}
\newcommand{\vi}{\vec{i}}
\newcommand{\vj}{\vec{j}}
\newcommand{\vk}{\vec{k}}
\newcommand{\vw}{\vec{w}}
\newcommand\segmento[2]{\overline{#1#2}}
\def\colc#1{\left[#1\right]}



%
%   Macros
%

\usepackage{macros/macros}


%
%   Ambientes
%

\theoremstyle{plain}
\newtheorem{teorema}{Teorema}

\theoremstyle{definition}
\newtheorem{definicao}[teorema]{Definição}
%\newtheorem{exercicio}{Exercício}

\theoremstyle{remark}
\newtheorem{obs}[teorema]{Observação}
\newtheorem{observacao}[teorema]{Observação}
\newtheorem{corolario}[teorema]{Corolário}
\newtheorem{exemplo}[teorema]{Exemplo}
\newtheorem{lema}[teorema]{Lema}
\newtheorem{proposicao}[teorema]{Proposição}

\newcounter{exercicios}
\newenvironment{exercicio}{\stepcounter{exercicios} \textbf{\arabic{exercicios}}.}{}

% compatibilidade
\newcommand{\Ex}[1]{\begin{exercicio}#1\end{exercicio}}

%
%   Definições e comandos auxiliares do preâmbulo
%

\newcommand{\capitulo}[1]{\lecture[#1]{Capítulo}}
\newcommand{\aula}[1]{\subtitle{#1}}
\newcommand{\autor}{Igor Oliveira}
\newcommand{\email}{\href{mailto:matematicaelementar@imd.ufrn.br}{\texttt{matematicaelementar@imd.ufrn.br}}}
\newcommand{\disciplina}{Matemática Elementar}
\newcommand{\codigo}{IMD1001}

\title{\disciplina}
\date{\today}
\author[\autor]
{
    \autor\\
    \email
}

\def\lecturename{\codigo

\disciplina}

\institute[
	UFRN\\
	Natal-RN
]
{
	Instituto Metrópole Digital\\
	Universidade Federal do Rio Grande do Norte\\
	Natal-RN

}

% compatibilidade
\newcommand{\vu}{\vec{u}}
\newcommand{\vv}{\vec{v}}
\newcommand{\vi}{\vec{i}}
\newcommand{\vj}{\vec{j}}
\newcommand{\vk}{\vec{k}}
\newcommand{\vw}{\vec{w}}
\newcommand{\segmento}[2]{\overline{#1#2}}
\def\colc#1{\left[#1\right]}
\newcommand{\negacao}{\sim}

\justifying


\aula{Progressões}
\capitulo{6}


\begin{document}	

	%
	%	Capa
	%

	{\backgroundimage\begin{frame}[plain]
		\titlepage
	\end{frame}}


	%
	%	Sumário
	%

	\begin{frame}
		\frametitle{Índice}
		\tableofcontents
	\end{frame}


	%
	%	Seções
	%
	
\section{Progressão Aritmética}
\begin{frame} \frametitle{Progressão Aritmética}
\begin{definicao}
Uma \sub{progressão aritmética} (ou simplesmente \sub{PA}) é uma
sequência na qual a diferença entre um termo e seu anterior (exceto
quando o termo em questão é o primeiro) é constante. Essa diferença
constante é chamada de \sub{razão} da progressão e representada pela
letra $r$.
\end{definicao}


De maneira recursiva, o $n$-ésimo, $n>1$, termo de uma PA é escrito
como
$$a_n = a_{n-1} + r.$$


\end{frame}

%------------------------------------------------------------------------------------------------------------

\begin{frame}
\frametitle{Progressão Aritmética} %\framesubtitle{Exemplos}
\begin{exemplo}
Uma fábrica de automóveis produziu 400 veículos em janeiro e aumenta
mensalmente sua produção em 30 veículos. Quantos veículos foram
produzidos em junho?
\end{exemplo} \pause

Em uma PA $\paren{ a_1 , a_2 , a_3 , \dots }$, para avançar um termo
basta somar a razão; para avançar dois termos, basta somar duas
vezes a razão, e assim por diante. Dessa forma, teremos $a_{13} = a_5
+8r$, $a_4 = a_{17} - 13r$, e, mais geralmente, $$a_i = a_j +
\paren{i-j}r.$$
Em particular:
$$a_n = a_1 + \paren{n-1}r.$$

\end{frame}



%------------------------------------------------------------------------------------------------------------

\begin{frame}
\frametitle{Progressão Aritmética} %\framesubtitle{Exemplos}
\begin{exemplo}
Em uma PA, o quinto termo vale 30, e o vigésimo termo vale 50. Quanto
vale o oitavo termo dessa progressão?
\end{exemplo}\pause

\begin{exemplo}
Qual a razão da PA que se obtém inserindo 10 termos entre os números
3 e 25?
\end{exemplo}\pause

\begin{exemplo}
O cometa Halley visita a Terra a cada 76 anos. Sua última passagem
por aqui foi em 1986. Quantas vezes ele visitou a Terra desde o
nascimento de Cristo? Em que ano foi sua primeira passagem na Era
Cristã?
\end{exemplo}

\end{frame}


%------------------------------------------------------------------------------------------------------------
\section{Atividade Online}
\begin{frame}
\frametitle{Atividade Online} %\framesubtitle{Exemplos}

\href{https://pt.khanacademy.org/math/algebra/sequences/introduction-to-arithmetic-squences/e/arithmetic_sequences_2}
{{\tt Atividade 01 - Use Fórmulas de Progressão Aritmética}}

\href{https://pt.khanacademy.org/math/algebra/sequences/constructing-arithmetic-sequences/e/explicit-and-recursive-formulas-of-arithmetic-sequences}
{{\tt Atividade 02 - Conversão das Formas Recursiva e Explícita de
Progressões Aritméticas}}


Veja o desempenho na Missão Álgebra I.


\end{frame}

%------------------------------------------------------------------------------------------------------------
\begin{frame}
\frametitle{Progressão Aritmética} %\framesubtitle{Exemplos}

\begin{exemplo}
O preço de um carro novo é de R\$ $30000{,}00$ e diminui R\$ $1000{,}00$
a cada ano de uso. Qual será o preço com 4 anos de uso?
\end{exemplo}\pause

\begin{exemplo}
Determine 4 números em uma PA crescente tais que sua soma é 8 e a
soma de seus quadrados é 36.
\end{exemplo}


\end{frame}

%------------------------------------------------------------------------------------------------------------



\begin{frame}
\frametitle{Progressão Aritmética} %\framesubtitle{Exemplos}

\begin{definicao}
Uma PA de razão $r \neq 0$ é chamada de \sub{progressão aritmética
de primeira ordem}. Se $r=0$, chamamos de \sub{progressão aritmética
estacionária}.
\end{definicao}

Essas definições são motivadas pelo fato de que, em uma PA, o termo
geral é dado por um polinômio em $n$, $$a_n = a_1 + \paren{n-1} r =
r \cdot n + \paren{a_1 - r}.$$

Assim, se $r\neq 0$, esse polinômio é de grau 1. Note que, se $r=0$,
a PA é constante. \pause

A recíproca desse resultado também é válida, ou seja, se uma
sequência tiver seu termo de ordem $n$ $\paren{a_n}$ definido por um
polinômio em $n$ de grau menor ou igual a 1, então essa sequência
será uma PA.

\end{frame}

%------------------------------------------------------------------------------------------------------------

\section{Somatório dos $n$ primeiros termos de uma PA}
\begin{frame}
\frametitle{Soma dos $n$ primeiros termos de uma PA} %\framesubtitle{Exemplos}

\begin{proposicao}
A soma dos $n$ primeiros termos da PA $\paren{a_1, a_2 , a_3,
\dots}$ é $$S_n = \frac {\paren{a_1 + a_n}n}{2}.$$
\end{proposicao}\pause

\begin{corolario}
Nas condições da Proposição anterior, temos: $$S_n = \frac r 2 \cdot
n^2 + \paren{a_1 - \frac r 2}n.$$
\end{corolario}

Tem-se que todo polinômio de segundo grau em $n$ que não possua o
termo independente é o somatório de alguma PA.

De fato, tendo $P \paren n = an^2 + bn$, basta tomar $r = 2a$ e $a_1
= a+b$. Verifique!

\end{frame}
%------------------------------------------------------------------------------------------------------------
\section{Progressão Geométrica}
\begin{frame}
\frametitle{Progressão Geométrica} %\framesubtitle{Exemplos}


\begin{exemplo}
Uma pessoa, começando com R\$ $64{,}00$, faz seis apostas
consecutivas, em cada uma das quais arrisca perder ou ganhar a
metade do que possui na ocasião. Se ela ganha três e perde três
dessas apostas, pode-se afirmar que ela:
\begin{enumerate}[a)]
	\item Ganha dinheiro;
	\item Não ganha nem perde dinheiro;
	\item Perde R\$ $27{,}00$;
	\item Perde R\$ $37{,}00$;
	\item Ganha ou perde dinheiro, dependendo da ordem em que
	ocorreram suas vitórias e derrotas.
\end{enumerate}
\end{exemplo}
\end{frame}



%------------------------------------------------------------------------------------------------------------

\begin{frame}
\frametitle{Progressão Geométrica} %\framesubtitle{Exemplos}

\begin{exemplo}
A população de um país é, hoje, igual a $P_0$ e cresce $2 \%$ ao ano.
Qual será a população desse país daqui a $n$ anos?
\end{exemplo}\pause

\begin{exemplo}
A torcida de certo clube é, hoje, igual a $T_0$ e decresce $5\%$ ao
ano. Qual será a torcida desse clube daqui a $n$ anos?
\end{exemplo}


\end{frame}



%------------------------------------------------------------------------------------------------------------
\begin{frame}
\frametitle{Progressão Geométrica} %\framesubtitle{Exemplos}

Note que, nos exemplos anteriores, se uma grandeza teve taxa de
crescimento igual a $i$, então cada valor da grandeza é igual a
$\paren{1+i}$ vezes o valor anterior.

\begin{definicao}
Uma \sub{progressão geométrica} (ou, simplesmente, PG) é uma sequência
na qual a taxa de crescimento $i$ de cada termo para o seguinte é
sempre a mesma.
\end{definicao}


\end{frame}



%------------------------------------------------------------------------------------------------------------

\begin{frame}
\frametitle{Progressão Geométrica} %\framesubtitle{Exemplos}

\begin{exemplo}
A sequência $\paren{1, 2, 4, 8, 16, 32, \dots }$ é um exemplo de uma
PG. Aqui, a taxa de crescimento de cada termo para o seguinte é de
$100 \% $, o que faz com que cada termo seja igual a $200 \% $ do
termo anterior.
\end{exemplo}

\begin{exemplo}
A sequência $\paren{1000, 800, 640, 512 , \dots }$ é um exemplo de
uma PG. Aqui, cada termo é $80 \% $ do termo anterior. A taxa de
crescimento de cada termo para o seguinte é de $ -20 \% $.
\end{exemplo}

\end{frame}

%------------------------------------------------------------------------------------------------------------
\section{Fórmulas de uma PG}
\begin{frame}
\frametitle{Progressão Geométrica} %\framesubtitle{Exemplos}

É claro que, numa PG, cada termo é igual ao anterior multiplicado por
$1 + i$, onde $i$ é a taxa de crescimento dos termos. Chamamos $1+i$
de \sub{razão da progressão} e a representamos por $q$. Assim, para
$n \geq 2$, $$a_n = a_{n-1} \cdot q.$$

Portanto, uma progressão geométrica é uma sequência na qual é
constante o quociente da divisão de cada termo pelo termo anterior
(exceto quando o termo anterior é o primeiro). \pause


Em uma PG $\paren{a_1, a_2, a_3 , \dots}$, para avançar um termo
basta multiplicar pela razão; para avançar dois termos, basta
multiplicar duas vezes pela razão, e assim por diante. Assim, $$a_i
= a_j \cdot q^{i-j}.$$ Em particular,
$$a_n = a_1\cdot q^{n-1}.$$

\end{frame}

%------------------------------------------------------------------------------------------------------------

\begin{frame}
\frametitle{Progressão Geométrica} %\framesubtitle{Exemplos}

\begin{exemplo}
Em uma PG, o quinto termo vale 5 e o oitavo termo vale 135. Quanto
vale o sétimo termo dessa progressão?
\end{exemplo}

\begin{exemplo}
Qual é a razão da PG que se obtém inserindo 3 termos entre os
números 30 e 480?
\end{exemplo}

\end{frame}

%------------------------------------------------------------------------------------------------------------

\section{Atividade Online}
\begin{frame}
\frametitle{Atividade Online} %\framesubtitle{Exemplos}

\href{https://pt.khanacademy.org/math/algebra/sequences/introduction-to-geometric-sequences/e/geometric_sequences_2}
{{\tt Atividade 03 - Use Fórmulas de Progressão Geométrica}}

\href{https://pt.khanacademy.org/math/algebra/sequences/constructing-geometric-sequences/e/explicit-and-recursive-formulas-of-geometric-sequences}
{{\tt Atividade 04 - Conversão das Formas Recursiva e Explícita de
Progressões Geométricas (no Khan aparece errado como Aritméticas)}}


Veja o desempenho na Missão Álgebra I.
\end{frame}

%------------------------------------------------------------------------------------------------------------
\section{Somatório dos $n$ primeiros termos de uma PG}
\begin{frame}
\frametitle{Soma dos $n$ primeiros termos de uma PG} %\framesubtitle{Exemplos}

\begin{proposicao}
A soma dos $n$ primeiros termos de uma PG $\paren {a_n}$ de razão $q
\neq 1$ é $$S_n = a_1 \frac{1-q^n}{1-q}.$$
\end{proposicao}

\end{frame}
%------------------------------------------------------------------------------------------------------------
\begin{frame}
\frametitle{Soma dos $n$ primeiros termos de uma PG} %\framesubtitle{Exemplos}

\begin{exemplo}
Diz a lenda que o inventor do xadrez pediu como recompensa 1 grão de
trigo pela primeira casa, 2 grãos pela segunda, 4 pela terceira e
assim por diante, sempre dobrando a quantidade a cada nova casa.
Sabendo que o tabuleiro de xadrez tem 64 casas, qual o número de
grãos pedido pelo inventor do jogo?
\end{exemplo}\pause

Resposta: 18446744073709551615.

\end{frame}
%------------------------------------------------------------------------------------------------------------

\section{Exercícios}
\begin{frame}
\frametitle{Exercícios} %\framesubtitle{Exemplos}

\Ex{Formam-se $n$ triângulos com palitos, conforme a figura abaixo.
\begin{center}
	\includegraphicscopyright[width=1.5cm]{figures/1-Triangle.png}{\textbf{$n=1$}}
	\includegraphicscopyright[width=2.15cm]{figures/2-Triangle.png}{\textbf{$n=2$}}
	\includegraphicscopyright[width=2.8cm]{figures/3-Triangle.png}{\textbf{$n=3$}}
\end{center}
Qual o número de palitos usados para construir $n$ triângulos? }

\Ex{A soma dos ângulos internos de um pentágono convexo é igual a
540º e estes ângulos estão em PA. Determine a mediana dos valores
dos ângulos.}

\Ex{Se $3-x$, $-x$, $\sqrt{9-x}$, $\dots$ é uma PA, determine $x$ e
calcule o quinto termo.}

\Ex{Calcule a soma dos termos da PA 2, 5, 8, 11, $\dots$ desde o 25º
termo até o 41º termo, inclusive.}

\end{frame}





%------------------------------------------------------------------------------------------------------------

\begin{frame}
\frametitle{Exercícios} %\framesubtitle{Exemplos}

\Ex{Determine o maior valor inteiro que pode ter a razão de uma PA
que admita os números 32, 227 e 942 como termos da progressão.}



\Ex{Em um quadrado mágico (relembre a definição na Aula 03),
chamamos de constante mágica o valor da soma de quaisquer uma das
linhas, colunas ou diagonais. Calcule a constante mágica de um
quadrado mágico $n \times n$.}

%\Ex{Podem os números $\sqrt 2$, $\sqrt 3$ e $\sqrt 5 $ pertencer a
%uma mesma PA?}

\Ex{Suprimindo um dos elementos do conjunto $\set{1, 2, \dots , n}$,
a média aritmética dos elementos restantes é $16,1$. Determine o
valor de $n$ e qual foi o elemento suprimido.}

%\Ex{Um bem, cujo valor hoje é de R\$ 8000,00, desvaloriza-se de tal
%forma que seu valor daqui a 4 anos será de R\$ 2000,00. Supondo
%constante a desvalorização anual, qual será o valor do bem daqui a 3
%anos? }
\end{frame}

%------------------------------------------------------------------------------------------------------------

\begin{frame}
\frametitle{Exercícios} %\framesubtitle{Exemplos}

\Ex{O gordinho jaguatirica e Júnio vão jogar um  jogo com as
seguintes regras:
\begin{itemize}
	\item Na primeira jogada, o gordinho jaguatirica escolhe um número no
	conjunto $A = \set{1, 2, 3, 4, 5, 6, 7}$ e diz esse número;
	\item Os jogadores jogam alternadamente;
	\item Cada jogador ao jogar escolhe um elemento de $A$, soma-o ao
	número DITO pelo jogador anterior e DIZ a soma;
	\item O vencedor é aquele que disser 63.
\end{itemize}
Pode o gordinho jaguatirica ou Júnio ter uma estratégia vencedora?
Se sim, quem pode e qual é essa estratégia?
 }


%\Ex{Refaça o exercício anterior para o caso do vencedor ser quem
%disser 64.}

%\Ex{Refaça o penúltimo exercício para o caso de $A = \set{3, 4, 5,
%6}$.}

\Ex{Mostre que Júnio pode ter uma estratégia que impeça o gordinho
vencer o jogo se a condição de vitória for 62 e $A =\set{3, 4, 5, 6,
7}$. }


\end{frame}



%------------------------------------------------------------------------------------------------------------

\begin{frame}
\frametitle{Exercícios} %\framesubtitle{Exemplos}

\Ex{Descontos sucessivos de 10\% e 20\% equivalem a um desconto
total de quanto?}


%\Ex{Um aumento de 10\% seguido de um desconto de 20\% equivale a um
%desconto único de quanto?}

%\Ex{Aumentando sua velocidade em 60\%, de quanto você diminui o
%tempo de viagem?}

\Ex{Um decrescimento mensal de 5\% gera um decrescimento anual de
quanto?}

%\Ex{O período de um pêndulo simples é diretamente proporcional à
%raiz quadrada do seu comprimento. De quanto devemos aumentar o
%comprimento para aumentar o período em 20\%?}

\Ex{Mantida constante a temperatura, a pressão de um gás perfeito é
inversamente proporcional a seu volume. De quanto aumenta a pressão
quando reduzimos em 20\% o volume?}

%\Ex{Se a base de um retângulo aumenta em 10\% e a altura diminui em
%10\%, quanto aumenta a sua área?}

\end{frame}



%------------------------------------------------------------------------------------------------------------

\begin{frame}
\frametitle{Exercícios} %\framesubtitle{Exemplos}

%\Ex{Um carro novo custa R\$ 18000,00 e, com 4 anos de uso, vale R\$
%12000,00. Supondo que o valor decresça a uma taxa anual constante,
%determine quanto vale o carro com 1 ano de uso.}

\Ex{Considere um triângulo retângulo tal que seus lados formam uma
PG crescente. Determine a razão dessa progressão.}

\Ex{Qual o quarto termo da PG $\sqrt 2 , \sqrt[3] 2 , \sqrt[6] 2 ,
\dots$?}

\Ex{Determine três números em PG, tais que a soma desses números é
19 e a soma de seus quadrados é 133.}

\Ex{A soma de três números em PG é 19. Subtraindo-se 1 do primeiro,
eles passam a formar uma PA. Calcule-os.}

\Ex{Quatro números que estão em sequência são tais que, os três
primeiros formam uma PA de razão 6, os três últimos uma PG e o
primeiro número é igual ao quarto. Determine-os.}


\end{frame}



%------------------------------------------------------------------------------------------------------------

\begin{frame}
\frametitle{Exercícios} %\framesubtitle{Exemplos}

\Ex{Será formada uma pilha de folhas de estanho que têm, cada uma,
espessura de $0,1$mm. Inicialmente, adiciona-se uma folha na pilha.
Nas operações seguintes, é inserida uma quantidade de folhas igual
ao número de folhas que já estavam na pilha no momento da inserção.
Após serem realizadas 33 adições de folhas, a altura da pilha será,
aproximadamente:
\begin{enumerate}[a.]
	\item A altura de um poste de luz;
	\item A altura de um prédio de 40 andares;
	\item O comprimento da praia de Copacabana;
	\item A distância Rio-São Paulo;
	\item O comprimento do equador terrestre.
\end{enumerate}
}

\Ex{Larga-se uma bola de uma altura de 5m. Após cada choque com o
solo, ela recupera apenas $\frac 4 9 $ da altura anterior. Qual a
distância total percorrida pela bola até o décimo choque com o solo?
E até o centésimo?}

\end{frame}



%------------------------------------------------------------------------------------------------------------

\begin{frame}
\frametitle{Exercícios} %\framesubtitle{Exemplos}

\Ex{Começando com um segmento de tamanho 1, divide-se esse segmento
em três partes iguais e retiramos o interior da parte central,
obtendo dois segmentos de comprimento $\frac 1 3$. Repetimos agora
essa operação com cada um desses segmentos e assim por diante. Sendo
$S_n$ a soma dos comprimentos dos intervalos que restaram depois de
$n$ dessas operações, determine o valor de $S_n$.}

\end{frame}

%------------------------------------------------------------------------------------------------------------

\section{Bibliografia}

\frame{
\frametitle{Bibliografia}
	\begin{thebibliography}{99}
		\bibitem {label1}
		LIMA, Elon L; CARVALHO, Paulo César P; Wagner, Eduardo; MORGADO,
		Augusto C.
		\newblock \emph{A Matemática do Ensino Médio. Vol. 2}.
		\newblock 5. ed. Rio de Janeiro: SBM, 2004.
	\end{thebibliography}
}

\end{document}
