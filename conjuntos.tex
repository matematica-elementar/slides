\documentclass[brazil, notheorems, 10pt]{beamer}
%
%   Arquivo de Configuração dos Slides
%


%
%   Pacotes utilizados
%

% Codificação dos caracteres em formato universal.
\usepackage[utf8]{inputenc}
\usepackage[T1]{fontenc}

% Traduz o texto gerados pelo LaTeX para português. ex.: Capítulo, Seção, Conteúdo.
\usepackage[brazil]{babel}

% Pacotes para ambientes matemáticos
\usepackage{amsmath}
\usepackage{amsthm}
\usepackage{amssymb}

% Diversas funções para o uso das aspas.
\usepackage{csquotes}

% Outros pacotes
\usepackage{hyperref}
\usepackage{tikz}
\usepackage{yfonts}
\usepackage{colortbl}
\usepackage{ragged2e}
\usepackage{helvet}
\usepackage{verbatim}


%
%   Tema
%

% Copyright 2007 by Till Tantau
%
% This file may be distributed and/or modified
%
% 1. under the LaTeX Project Public License and/or
% 2. under the GNU Public License.
%
% See the file doc/licenses/LICENSE for more details.


% Common packages


\usepackage{times}
 \mode<article> {
	\usepackage{times}
	\usepackage{mathptmx}
	\usepackage[left=1.5cm,right=6cm,top=1.5cm,bottom=3cm]{geometry}
}

\usepackage{hyperref}
\usepackage[T1]{fontenc}
\usepackage{amsmath,amssymb}
\usepackage{tikz}
\usepackage{colortbl}
\usepackage{yfonts}
\usepackage{colortbl}
\usepackage{translator} % comment this, if not available
\usepackage{ragged2e} % justifying
% Or whatever. Note that the encoding and the font should match. If T1
% does not look nice, try deleting the line with the fontenc.
\usepackage{helvet}
\usepackage{verbatim}


%\usepackage{lipsum}
%\usepackage{enumitem}


\usetheme[
%%% options passed to the outer theme
%    hidetitle,           % hide the (short) title in the sidebar
%    hideauthor,          % hide the (short) author in the sidebar
%    hideinstitute,       % hide the (short) institute in the bottom of the sidebar
%    shownavsym,          % show the navigation symbols
%    width=2cm,           % width of the sidebar (default is 2 cm)
%    hideothersubsections,% hide all subsections but the subsections in the current section
%    hideallsubsections,  % hide all subsections
		right               % right of left position of sidebar (default is right)
%%% options passed to the color theme
%    lightheaderbg,       % use a light header background
	]{AAUsidebar}

% If you want to change the colors of the various elements in the theme, edit and uncomment the following lines
% Change the bar and sidebar colors:
%\setbeamercolor{AAUsidebar}{fg=red!20,bg=red}
%\setbeamercolor{sidebar}{bg=red!20}
% Change the color of the structural elements:
%\setbeamercolor{structure}{fg=red}
% Change the frame title text color:
%\setbeamercolor{frametitle}{fg=blue}
% Change the normal text color background:
%\setbeamercolor{normal text}{bg=gray!10}
% Highlight the text in the sidebar
\usecolortheme{rose,sidebartab}
% ... and you can of course change a lot more - see the beamer user manual.

% colored hyperlinks
\newcommand{\chref}[2]{%
	\href{#1}{{\usebeamercolor[bg]{AAUsidebar}#2}}%
}



% specify a logo on the titlepage (you can specify additional logos an include them in
% institute command below
\pgfdeclareimage[height=1cm]{titlepagelogo}{theme/figures/ufrn2} % placed on the title page
\pgfdeclareimage[height=1cm]{titlepagelogo2}{theme/figures/imd} % placed on the title page
\titlegraphic{% is placed on the bottom of the title page
	\pgfuseimage{titlepagelogo}
	\hspace{1cm}\pgfuseimage{titlepagelogo2}
}


% Article version layout settings

\mode<article>

\makeatletter
\def\@listI{\leftmargin\leftmargini
	\parsep 0pt
	\topsep 5\p@   \@plus3\p@ \@minus5\p@
	\itemsep0pt}
\let\@listi=\@listI


\setbeamertemplate{frametitle}{\paragraph*{\insertframetitle\
		\ \small\insertframesubtitle}\ \par
}
\setbeamertemplate{frame end}{%
	\marginpar{\scriptsize\hbox to 1cm{\sffamily%
			\hfill\strut\insertshortlecture.\insertframenumber}\hrule height .2pt}}
\setlength{\marginparwidth}{1cm}
\setlength{\marginparsep}{4.5cm}

\def\@maketitle{\makechapter}

\def\makechapter{
	\newpage
	\null
	\vskip 2em%
	{%
		\parindent=0pt
		\raggedright
		\sffamily
		\vskip8pt
		\includegraphics[width=\linewidth]{theme/figures/imd.png}\par\vskip2em
		{\fontsize{36pt}{36pt}\selectfont Aula \insertshortlecture \par\vskip2pt}
		{\fontsize{24pt}{28pt}\selectfont \color{blue!50!black} \@title\par\vskip4pt}
		%{\Large\selectfont \color{blue!50!black} \insertsubtitle\par}
		\vskip10pt

		\normalsize\selectfont [Notas de Aula]
		Disciplina: \emph{\lecturename \ (\semestre)} \par\vskip1.5em
		\nomedoautor\hskip1em Email: \ \emaildoautor
	}
	\par
	\vskip 1.5em%
}

\let\origstartsection=\@startsection
\def\@startsection#1#2#3#4#5#6{%
	\origstartsection{#1}{#2}{#3}{#4}{#5}{#6\normalfont\sffamily\color{blue!50!black}\selectfont}}

\makeatother

\mode
<all>




% Typesetting Listings

\usepackage{listings}
\lstset{language=Java}

\alt<presentation>
{\lstset{%
	basicstyle=\footnotesize\ttfamily,
	commentstyle=\slshape\color{green!50!black},
	keywordstyle=\bfseries\color{blue!50!black},
	identifierstyle=\color{blue},
	stringstyle=\color{orange},
	escapechar=\#,
	emphstyle=\color{red}}
}
{
	\lstset{%
		basicstyle=\ttfamily,
		keywordstyle=\bfseries,
		commentstyle=\itshape,
		escapechar=\#,
		emphstyle=\bfseries\color{red}
	}
}



% Common theorem-like environments
%\usepackage{amsthm}

\setbeamertemplate{theorems}[numbered]

%
%	New useful definitions:
%

\newbox\mytempbox
\newdimen\mytempdimen

\newcommand\includegraphicscopyright[3][]{%
	\leavevmode\vbox{\vskip3pt\raggedright\setbox\mytempbox=\hbox{\includegraphics[#1]{#2}}%
		\mytempdimen=\wd\mytempbox\box\mytempbox\par\vskip1pt%
		\fontsize{3}{3.5}\selectfont{\color{black!25}{\vbox{\hsize=\mytempdimen#3}}}\vskip3pt%
}}

\newenvironment{colortabular}[1]{\medskip\rowcolors[]{1}{blue!20}{blue!10}\tabular{#1}\rowcolor{blue!40}}{\endtabular\medskip}

\def\equad{\leavevmode\hbox{}\quad}

\newenvironment{greencolortabular}[1]
{\medskip\rowcolors[]{1}{green!50!black!20}{green!50!black!10}%
	\tabular{#1}\rowcolor{green!50!black!40}}%
{\endtabular\medskip}

%\setbeamertemplate{theorem begin}{{ \inserttheoremheadfont
%\inserttheoremname \inserttheoremnumber
%\ifx\inserttheoremaddition\empty\else\ (\inserttheoremaddition)\fi%
%\inserttheorempunctuation }} \setbeamertemplate{theorem end}{}

\newcommand{\vu}{\vec{u}}
\newcommand{\vv}{\vec{v}}
\newcommand{\vi}{\vec{i}}
\newcommand{\vj}{\vec{j}}
\newcommand{\vk}{\vec{k}}
\newcommand{\vw}{\vec{w}}
\newcommand\segmento[2]{\overline{#1#2}}
\def\colc#1{\left[#1\right]}



%
%   Macros
%

\usepackage{macros/macros}


%
%   Ambientes
%

\theoremstyle{plain}
\newtheorem{teorema}{Teorema}

\theoremstyle{definition}
\newtheorem{definicao}[teorema]{Definição}
%\newtheorem{exercicio}{Exercício}

\theoremstyle{remark}
\newtheorem{obs}[teorema]{Observação}
\newtheorem{observacao}[teorema]{Observação}
\newtheorem{corolario}[teorema]{Corolário}
\newtheorem{exemplo}[teorema]{Exemplo}
\newtheorem{lema}[teorema]{Lema}
\newtheorem{proposicao}[teorema]{Proposição}

\newcounter{exercicios}
\newenvironment{exercicio}{\stepcounter{exercicios} \textbf{\arabic{exercicios}}.}{}

% compatibilidade
\newcommand{\Ex}[1]{\begin{exercicio}#1\end{exercicio}}

%
%   Definições e comandos auxiliares do preâmbulo
%

\newcommand{\capitulo}[1]{\lecture[#1]{Capítulo}}
\newcommand{\aula}[1]{\subtitle{#1}}
\newcommand{\autor}{Igor Oliveira}
\newcommand{\email}{\href{mailto:matematicaelementar@imd.ufrn.br}{\texttt{matematicaelementar@imd.ufrn.br}}}
\newcommand{\disciplina}{Matemática Elementar}
\newcommand{\codigo}{IMD1001}

\title{\disciplina}
\date{\today}
\author[\autor]
{
    \autor\\
    \email
}

\def\lecturename{\codigo

\disciplina}

\institute[
	UFRN\\
	Natal-RN
]
{
	Instituto Metrópole Digital\\
	Universidade Federal do Rio Grande do Norte\\
	Natal-RN

}

% compatibilidade
\newcommand{\vu}{\vec{u}}
\newcommand{\vv}{\vec{v}}
\newcommand{\vi}{\vec{i}}
\newcommand{\vj}{\vec{j}}
\newcommand{\vk}{\vec{k}}
\newcommand{\vw}{\vec{w}}
\newcommand{\segmento}[2]{\overline{#1#2}}
\def\colc#1{\left[#1\right]}
\newcommand{\negacao}{\sim}

\justifying



%lecture[number of class]{type}  
\lecture[1]{Capítulo}

\def\lecturename{IMD1001


Matemática Elementar}
\def\semestre{2018.1}
\def\nomedoautor{Igor Oliveira}
\def\emaildoautor{\href{mailto:matematicaelementar@imd.ufrn.br}{{\tt igoroliveira@imd.ufrn.br}}}

\title[\lecturename]% optional, use only with long paper titles
{Matemática Elementar}

\subtitle{Conjuntos}  % could also be a conference name

\date{\today}

\author[Igor Oliveira] % optional, use only with lots of authors
{
	\nomedoautor\\
	\emaildoautor
}
% - Give the names in the same order as they appear in the paper.
% - Use the \inst{?} command only if the authors have different
%   affiliation. See the beamer manual for an example

\institute[
%  {\includegraphics[scale=0.2]{aau_segl}}\\ %insert a company, department or university logo
	UFRN\\
	Natal-RN
] % optional - is placed in the bottom of the sidebar on every slide
{% is placed on the title page
	Instituto Metrópole Digital\\
	Universidade Federal do Rio Grande do Norte\\
	Natal-RN

	%there must be an empty line above this line - otherwise some unwanted space is added between the university and the country (I do not know why;( )
}


\begin{document}

\mode<article>
{
\begin{frame} % the plain option removes the sidebar and header from the title page
	\maketitle
\end{frame}

% the titlepage
\begin{frame}[plain,noframenumbering] % the plain option removes the sidebar and header from the title page
	\titlepage
\end{frame}
}

\mode<presentation>
{
% the titlepage
{\imagemfundo
\begin{frame}[plain,noframenumbering] % the plain option removes the sidebar and header from the title page
	\titlepage
\end{frame}}
}

% TOC
\begin{frame}{Índice}{}
\tableofcontents
\end{frame}
%%%%%%%%%%%%%%%%

\section{Apresentação}

\frame
{
\frametitle{Apresentação da Aula}
\begin{block}{Motivação}
\justifying
		Praticamente toda a matemática atual é formulada na linguagem de
		conjuntos mesmo sendo a mais simples das ideias matemáticas. Portanto, o bom entendimento de como trabalhar com
		conjuntos é fundamental.
 \end{block}
}


%------------------------------------------------------------------------------------------------------------

\begin{comment}
\section{Objetivos}
\frame
{
\frametitle{Objetivos}

\begin{itemize}
\item<1-> Conhecer os \alert{sinais anal�gicos} e digitais
\item<2-> Fazer uma comparação entre sistemas anal�gicos e digitais
\item<3-> Conhecer os elementos el�tricos passivos : resistor, capacitor e indutor
\item<4-> Conhecer as Leis de Kirchhoff
\item<5-> Conhecer os transistores e os diodos
\item<6-> Analisar as principais aplicações de sistemas digitais.
\end{itemize}

}
\end{comment}

%------------------------------------------------------------------------------------------------------------

\section{Introdução}
\frame { \frametitle{A Noção de Conjunto}
\begin{itemize}
\item<1-> Um conjunto é \sub{definido} por seus elementos (e nada
mais). Isso nos traz imediatamente que dois conjuntos são
\sub{iguais} se, e somente se, possuem os mesmos elementos.

\item<2-> Dados um conjunto $A$ e um objeto qualquer $b$, há somente uma
pergunta cabível para nós: $b$ é um elemento do conjunto $A$? Tal
pergunta só admite \sub{sim} ou \sub{não} como resposta. Isso se dá
porque, na Matemática, qualquer afirmação é \sub{verdadeira} ou é
\sub{falsa}, sem possibilidade de uma terceira opção ou de ser as
duas coisas ao mesmo tempo.
\begin{itemize}
\item<3-> O item anterior faz parecer que a Matemática é infalível se
utilizada corretamente, mas ela não é. Gödel provou que todo sistema
formal que inclua a aritmética é falho no sentido de que vai possuir verdades que não podem
ser provadas -- os chamados paradoxos.  Antes de assistir ao vídeo
\href{https://youtu.be/UI1xR_AECrU}{{\tt Este vídeo está mentindo}},
reflita se você vai acreditar nele ou não.
\end{itemize}
\end{itemize}
}

%------------------------------------------------------------------------------------------------------------


\frame { \frametitle{A Noção de Conjunto} %\framesubtitle{Exemplos}


\begin{Exem}
 Temos $V = \left\{a, e, i, o, u \right\}$ como sendo
o conjunto das vogais.

\end{Exem}


\begin{Exem}
 O conjunto $PP$ dos números primos pares pode ser representado
por $PP = \left\{ x \tq x \text{ é primo e par} \right\} =
\left\{ 2 \right\}$. Nunca escreva $PP = \left\{
\text{números primos pares} \right\}$.
\end{Exem}



Quando um elemento pertence a um determinado conjunto,
	usamos o símbolo $\in$, e, quando não pertence, usamos $\notin$.
		\begin{Exem}
			Considere $PP$ e $V$ conforme definido anteriormente. Temos que $e \in V$ e $3 \notin PP$.
		\end{Exem}

}

%------------------------------------------------------------------------------------------------------------


\frame { \frametitle{A Noção de Conjunto} %\framesubtitle{Exemplos}


\begin{Exem}
 Considere o conjunto $A = \set{\set{1, 2}, \set{2}, 1 }$. Observe que existem elementos em $A$ que são conjuntos. Além disso:
 \begin{enumerate}
	 \item $\set{1, 2} \in A$;
	 \item $\set{2} \in A$;
	 \item $1 \in A$;
	 \item $2 \notin A$.
 \end{enumerate}

\end{Exem}







}

%------------------------------------------------------------------------------------------------------------



\frame { \frametitle{A Noção de Conjunto}


\begin{Def}
O conjunto que não possui elementos é chamado de \sub{conjunto
vazio} e é representado por $\emptyset$.
\end{Def}

\begin{Exem}
 Quais outros conjuntos você conhece? Que tal pensar sobre o conjunto
$ A = \set{ x \tq x \notin A} $?

\end{Exem}

}

%------------------------------------------------------------------------------------------------------------

\section{Inclusão}
\begin{frame}
\frametitle{A Relação de Inclusão} %\framesubtitle{Exemplos}

\begin{Def}
Sejam $A$ e $B$ conjuntos. Se todo elemento de $A$ for também
elemento de $B$, diz-se que $A$ é um \sub{subconjunto} de $B$, que
$A$ \sub{está contido} em $B$, ou que $A$ é \sub{parte} de $B$. Para
indicar esse fato, usa-se a notação $A \subset B$.

\end{Def}

Quando $A$ não é um subconjunto de $B$, escreve-se $A \not\subset
B$. Em outras palavras, existe pelo menos um elemento $a$ tal que $a
\in A$ e $a \notin B$.
\bigskip

Quando $A \subset B$, dizemos que $B$ \sub{contém} $A$ e escrevemos
$B \supset A$.


\end{frame}


%------------------------------------------------------------------------------------------------------------

\begin{frame}
\frametitle{A Relação de Inclusão} %\framesubtitle{Exemplos}

\begin{Exem}
Sejam $T$ o conjunto de todos os triângulos e $P$ o conjunto dos
polígonos do plano. Todo triângulo é um polígono, logo $ T \subset
P$.
\end{Exem}

\begin{Exem}
Na Geometria, uma reta, um plano e o espaço são conjuntos. Seus
elementos são pontos.

Quando dizemos que uma reta $r$ está no plano $\Pi$, estamos
afirmando que $r$ está contida em $\Pi$ ou, equivalentemente, que
$r$ é um subconjunto de $\Pi$, pois todos os pontos que pertencem a
$r$ pertencem também a $\Pi$.

Nesse caso, deve-se escrever $ r \subset \Pi$. Porém, não é correto
dizer que $r$ pertence a $\Pi$, nem escrever $r \in \Pi$. Os
elementos do conjunto $\Pi$ são pontos e não retas.
\end{Exem}

\end{frame}


%------------------------------------------------------------------------------------------------------------

\begin{frame}
\frametitle{A Relação de Inclusão} %\framesubtitle{Exemplos}

\begin{Prop}[Inclusão universal do $\emptyset$]
Para todo conjunto $A$, vale $\emptyset \subset A$.
\end{Prop}

\begin{Def}
Dizemos que $A \neq \emptyset$ é um \sub{subconjunto próprio} de $B$
quando $A \subset B$  e $A \neq B$.
\end{Def}



\end{frame}


%------------------------------------------------------------------------------------------------------------

\begin{frame}
\frametitle{A Relação de Inclusão} %\framesubtitle{Exemplos}
\begin{Prop}[Propriedades da inclusão]
Sejam $A$, $B$ e $C$ conjuntos. Tem-se:
\begin{enumerate}[i.]
	\item \sub{Reflexividade}: $A \subset A$;
	\item \sub{Antissimetria}: Se $A \subset B$ e $B \subset A$,
	então $A = B$;
	\item \sub{Transitividade}: Se $A \subset B$ e $B \subset C$,
	então $A \subset C$.
\end{enumerate}
\end{Prop}

Demonstração no quadro.


\end{frame}

%------------------------------------------------------------------------------------------------------------

\begin{frame}
\frametitle{A Relação de Inclusão} %\framesubtitle{Exemplos}

\begin{Def}
Dado um conjunto $A$, chamamos de \sub{conjunto das partes} de $A$ o conjunto formado por todos
os seus subconjuntos, e denotamo-lo $\mathcal{P}(A)$.
\end{Def}

\begin{Exem}
Dado $A = \set {1, 2, 3}$, determine $\mathcal{P}(A)$.
\end{Exem}



\end{frame}
%------------------------------------------------------------------------------------------------------------
\section{Complementar}
\begin{frame}
\frametitle{O Complementar de um Conjunto} %\framesubtitle{Exemplos}

A noção de complementar de um conjunto só faz sentido quando fixamos
um \sub{conjunto universo}, que denotaremos por $\U$. Uma vez fixado
$\U$, todos os elementos considerados pertencerão a $\U$ e todos os
conjuntos serão subconjuntos de $\U$. Por exemplo, na geometria
plana, $\U$ é o plano.

\begin{Def}
Dado um conjunto $A$ (isto é, um subconjunto de $\U$), chama-se
\sub{complementar} de $A$ ao conjunto $A^C$ formado pelos elementos
de $\U$ que não pertencem a $A$.
\end{Def}

\begin{Exem}
Seja $\U$ o conjunto dos triângulos. Qual o complementar do conjunto
dos triângulos escalenos?
\end{Exem}

\end{frame}
%------------------------------------------------------------------------------------------------------------
\begin{frame}
\frametitle{O Complementar de um Conjunto} %\framesubtitle{Exemplos}
\begin{Prop}[Propriedades do complementar] \label{prop-comp}
Fixado um conjunto universo $\U$, sejam $A$ e $B$ conjuntos. Tem-se:
\begin{enumerate}[i.]
	\item $\U^C = \emptyset$ e $\emptyset^C = \U$;
	\item $\left( A^C \right)^C = A$ (Todo conjunto é complementar do seu complementar);
	\item Se $A \subset B$ então $B^C \subset A^C$ (se um conjunto
	está contido em outro, seu complementar contém o complementar desse outro).
\end{enumerate}
\end{Prop}

Demonstração no quadro.


\end{frame}

%------------------------------------------------------------------------------------------------------------
\begin{frame}
\frametitle{O Complementar de um Conjunto} %\framesubtitle{Exemplos}

\begin{Def}
A \sub{diferença} entre dois conjuntos $A$ e $B$ é definida por:
$$ B \setminus A = \set {x \tq x \in B \text { e } x \notin A}.$$
\end{Def}

\begin{itemize}
	\item Em geral, não temos $B \setminus A = A \backslash B$. Pense em um contraexemplo a essa
	igualdade.
	\item Note que $A^C = \U \setminus A$.
\end{itemize}

\end{frame}

%------------------------------------------------------------------------------------------------------------
\section{União e Interseção}
\begin{frame}
\frametitle{União e Interseção de Conjuntos} %\framesubtitle{Exemplos}

\begin{Def}
Dados os conjuntos $A$ e $B$:
\begin{enumerate}[i.]
	\item A \sub{união} $A \cup B$ é o conjunto formado pelos
	elementos que pertencem a pelo menos um dos conjuntos $A$ e $B$;
	\item A \sub{interseção} $A \cap B$ é o conjunto formado por elementos que pertencem a ambos $A$ e
	$B$.
\end{enumerate}
\end{Def}

\begin{Exem}
Sejam $A = \set{1, 2, 3}$ e $ B = \set{2,5}$. Determine $A \cup B$,
$A \cap B$, $A \setminus B$ e $B \setminus A$.
\end{Exem}

\end{frame}

%------------------------------------------------------------------------------------------------------------

\begin{frame}
\frametitle{União e Interseção de Conjuntos} %\framesubtitle{Exemplos}
\begin{Prop}[Propriedades da união e interseção] \label{propuniaoint}
Sejam $A$, $B$ e $C$ conjuntos. Tem-se:
\begin{enumerate}[i.]
	\item $A \subset \paren{A \cup B}$ e $\paren{A \cap B} \subset A$;
	\item \sub{União/interseção com o universo}: $\U \cup A = \U$ e $A \cap \U = A$;
	\item \sub{Comutatividade}: $A \cup B = B \cup A$ e $A \cap B = B \cap A$;
	\item \sub{Associatividade}: $\left(A \cup B \right) \cup C = A
	\cup \left( B \cup C \right)$ e $\left(A \cap B \right) \cap C = A
	\cap \left( B \cap C \right)$;

	\item \sub{Distributividade, de uma em relação à outra}: $A \cap
	\left( B \cup C \right) = \left(A \cap B \right) \cup \left( A \cap C
	\right)$ e $A \cup \left( B \cap C \right) = \left(A \cup B \right) \cap
	\left( A \cup C  \right)$;

	\item \sub{Leis de DeMorgan}: $\left( A \cup B \right)^C = A^C \cap
	B^C$ e $\left(A \cap B \right)^C = A^C \cup B^C$.

	\end{enumerate}
\end{Prop}

Demonstração no quadro.


\end{frame}

%------------------------------------------------------------------------------------------------------------
\section{Atividade Online}
\begin{frame}
\frametitle{Atividade Online} %\framesubtitle{Exemplos}

\href{https://pt.khanacademy.org/math/statistics-probability/probability-library/basic-set-ops/e/basic_set_notation}
{{\tt Atividade 01 - Notação Básica de Conjunto}}

Veja o desempenho na Missão O Mundo da Matemática - Probabilidade


\end{frame}

%------------------------------------------------------------------------------------------------------------
\section{Lógica}
\begin{frame}
\frametitle{Conjuntos e Lógica} %\framesubtitle{Exemplos}

Em toda essa seção, considere $P$ e $Q$ propriedades aplicáveis aos
elementos de $\U$. Considere também $A = \set {x \tq x \text{ possui
} P}$ e $B= \set {x \tq x \text{ possui } Q}$.

\begin{itemize}
	\item \sub{Inclusão e implicação}: $A \subset B$ é equivalente a
	$P \implies Q$.
	\item \sub{Igualdade e bi-implicação}: $A=B$ é equivalente a $P
	\iff Q$.
\end{itemize}
\end{frame}
%------------------------------------------------------------------------------------------------------------
\begin{frame}
\frametitle{Conjuntos e Lógica} %\framesubtitle{Exemplos}

\begin{Exem}
Analise as implicações abaixo:
\begin{align*}
x^2+1=0 & \implies \left(x^2 +1 \right) \left( x^2-1 \right) = 0
\cdot \left( x^2-1 \right) \\
& \implies x^4 - 1 = 0 \\
& \implies x^4 = 1 \\
& \implies x \in \set {-1, 1}
\end{align*}

Isso quer dizer que o conjunto solução de $x^2 +1 = 0$ é $\set{-1,
1}$?
\end{Exem}

\end{frame}
%------------------------------------------------------------------------------------------------------------
\begin{frame}
\frametitle{Conjuntos e Lógica} %\framesubtitle{Exemplos}

\begin{itemize}
	\item \sub{Complementar e negação}: $A^C$ é equivalente a $\sim P$;
	\item Podemos combinar os itens (ii) e (iii) da Proposição
	\ref{prop-comp} (Propriedades do complementar) e obter que $$P
	\implies Q \text{ se, e somente se, }\sim Q \implies \sim P.$$
	Chamamos $\sim Q \implies \sim P$ de \sub{contrapositiva} de $P
	\implies Q$.
	\item Chamamos $Q \implies P$ de \sub{recíproca} de $P \implies
	Q$ e $P \land \sim Q$ de \sub{negação} de $P \implies Q$.
\end{itemize}
Exemplos no Exercício \ref{exrec}.
\end{frame}
%------------------------------------------------------------------------------------------------------------
\begin{frame}
\frametitle{Conjuntos e Lógica} %\framesubtitle{Exemplos}

\begin{Exem}
Observe as afirmações abaixo:
\begin{itemize}
	\item Todo número primo maior do que 2 é ímpar;
	\item Todo número par maior do que 2 é composto.
\end{itemize}

Essas afirmações dizem exatamente a mesma coisa, ou seja, exprimem a
mesma ideia, só que com diferentes termos. Podemos reescrevê-las na
forma de implicações vendo claramente que uma é a contrapositiva da
outra, todas sob a hipótese que  $n \in \N$, $n>2$:
\begin{align*}
n \text{ primo } & \implies n \text{ ímpar } \\
\sim \left( n \text{ ímpar } \right) & \implies \sim \left( n \text{
primo } \right) \\
n \text{ par } & \implies n \text{ composto }
\end{align*}
\end{Exem}

\end{frame}
%------------------------------------------------------------------------------------------------------------

\begin{frame}
\frametitle{Conjuntos e Lógica} %\framesubtitle{Exemplos}

\begin{itemize}
	\item \sub{União e disjunção}: $A \cup B$ é equivalente a $P \lor
	Q$ ($P$ ou $Q$).
	\item \sub{Interseção e conjunção}: $A \cap B$ é equivalente a $P
	\land Q$ ($P$ e $Q$).
\end{itemize}

\begin{Obs}
O conectivo lógico \sub{ou} tem significado diferente do usado
normalmente no português. Na linguagem coloquial, usamos $P$
\sub{ou} $Q$ sem permitir que sejam as duas coisas ao mesmo tempo.
Analisem a seguinte história:

Um obstetra que também era matemático acabara de realizar um parto
quando o pai perguntou: ``é menino ou menina, doutor?''. E ele
respondeu: ``sim''.
\end{Obs}
\end{frame}
%------------------------------------------------------------------------------------------------------------

\begin{frame}
\frametitle{Conjuntos e Lógica} \framesubtitle{Resumo}
\begin{center}
\begin{tabular}{|c|c|}
	\hline
	% after \\: \hline or \cline{col1-col2} \cline{col3-col4} ...
	$A=B$ & $P \iff Q$ \\ \hline
	$A \subset B$ & $P \implies Q$ \\ \hline
	$A^C$ & $\sim P$ \\ \hline
	$A \cup B$ & $P \lor Q$ \\ \hline
	$A \cap B$ & $P \land Q$ \\
	\hline
\end{tabular}
\end{center}
\end{frame}
%------------------------------------------------------------------------------------------------------------

\begin{frame}
\frametitle{Conjuntos e Lógica} %\framesubtitle{Resumo}
Problema: A
polícia prende quatro homens, um dos quais cometeu um furto. Eles fazem
as seguintes declarações:
\begin{itemize}
	\item Arnaldo: Bernaldo fez o furto.
	\item Bernaldo: Cernaldo fez o furto.
	\item Dernaldo: eu não fiz o furto.
	\item Cernaldo: Bernaldo mente ao dizer que eu fiz o furto.
\end{itemize}
Se sabemos que só uma destas declarações é a verdadeira, quem é
culpado pelo furto?

\end{frame}

\section{Exercícios}

\begin{frame}
\frametitle{Exercícios}
\Ex{
	De que outras formas podemos representar o conjunto vazio utilizando as duas notações de definição de conjuntos que conhecemos?
}


\Ex{Decida quais das afirmações a seguir estão corretas. Justifique suas respostas.
		\begin{enumerate}[a.]
			\item $\emptyset \in \emptyset$;
			\item $\emptyset \subset \emptyset$;
			\item $\emptyset \in \set{\emptyset}$;
			\item $\emptyset \subset \set{\emptyset}$.
		\end{enumerate}}

\Ex{
%\label{exe:} @TODO linkar com as propriedades
Complete as demonstrações em \nameref{prop:uniao-e-intersecao} que não foram feitas em sala de aula.
}
\end{frame}

\begin{frame}
\frametitle{Exercícios}


\Ex{%\label{exe:}
Demonstre que os seguintes itens são equivalentes:
	\begin{enumerate}[a.]
		\item $A \cup B = B$;
		\item $A \subset B$;
		\item $A \cap B = A$;
	\end{enumerate}
	\textit{Dica}: Para tanto, é preciso provar \textbf{a. $\iff$ b.} e \textbf{b. $\iff$ c.}.
		Outra maneira é provar \textbf{a. $\implies$ b.}, \textbf{b. $\implies$ c.} e por fim, \textbf{c. $\implies$ a.}.
}
\end{frame}

\begin{frame}
\frametitle{Exercícios}
\Ex{O diagrama de Venn para os conjuntos $X$, $Y$, $Z$ decompõe o
plano em oito regiões. Numere essas regiões e exprima cada um dos
conjuntos abaixo como reunião de algumas dessas regiões. (Por
exemplo: $X \cap Y = 1 \cup 2$.)
\begin{enumerate}[a.]
	\item $\left(X^C \cup Y \right)^C$;
	\item $\left(X^C \cup Y \right) \cup Z^C$;
	\item $\left(X^C \cap Y \right) \cup \left(X \cap Z^C \right)$;
	\item $\left(X \cup Y \right)^C \cap Z$.
\end{enumerate}
}

\Ex{Exprimindo cada membro como reunião de regiões numeradas, prove
as igualdades:
\begin{enumerate}[a.]
	\item $\left(X \cup Y \right)\cap Z = \left(X \cap Z \right) \cup \left(Y \cap Z
	\right)$;
	\item $X \cup \left(Y \cap Z \right)^C = X \cup Y^C \cup Z^C$.
\end{enumerate}
}
\end{frame}

\begin{frame}
\frametitle{Exercícios}
\Ex{Sejam $A$, $B$ e $C$ conjuntos. Determine uma condição necessária e
suficiente para que se tenha 
	$$A \cup \left( B \cap C \right) = \left(A \cup B \right) \cap C$$
}

\Ex{Recorde a definição da diferença entre conjuntos:
	$$B \setminus A = \set {x \tq x \in B \text { e } x \notin A}.$$
	Mostre que
		\begin{enumerate}[a.]
			\item $B \setminus A = B \cap A^C$;
			\item $\prn{B \setminus A}^C = B^C\cup A$;
			\item $B \setminus A = \emptyset$ se, e somente se, $B \subset
			A$;
			\item $B \setminus A = B$ se, e somente se, $A \cap B =
			\emptyset$;
			\item Vale a igualdade $B \setminus A = A \setminus B$ se, e
			somente se, $A = B$;
			\item Determine uma condição necessária e suficiente para que
			se tenha $$A \setminus \left(B \setminus C \right) = \left(A
			\setminus B \right) \setminus C.$$
			%\sub{Dica}: Use o diagrama de Venn para enxergar tal condição
			%necessária e suficiente antes de demonstrar a igualdade.
		\end{enumerate}
}
\end{frame}

\begin{frame}
\frametitle{Exercícios}
	\Ex{Dê exemplos de implicações, envolvendo conteúdos de ensino
	médio, que sejam: verdadeiras com recíproca verdadeira;
	verdadeiras com recíproca falsa; falsas, com recíproca verdadeira;
	falsas, com recíproca falsa.
}

\Ex{Considere $P$, $Q$ e $R$ condições aplicáveis aos elementos
de um conjunto universo $\U$, e $A$, $B$ e $C$ os subconjuntos de
$\U$ dos elementos que satisfazem $P$, $Q$ e $R$, respectivamente.
Expresse, em termos de implicações entre $P$, $Q$ e $R$, as
seguintes relações entre os conjuntos $A$, $B$ e $C$.
\begin{enumerate}[a.]
\item $A \cap B^C \subset C$;
\item $A^C \cup B^C \subset C$;
\item $A^C \cup B \subset C^C$;
\item $A^C \subset B^C \cup C$;
\item $A \subset B^C \cup C^C$.
\end{enumerate}
}
\end{frame}

\begin{frame}
\frametitle{Exercícios}
 \Ex{Considere as seguintes (aparentes) equivalências lógicas:
\begin{align*}
x=1 & \iff x^2 -2x +1 = 0 \\
& \iff x^2 -2 \cdot 1 +1 =0 \\
& \iff x^2 - 1 =0 \\
& \iff x = \pm 1
\end{align*}
Conclusão (?): $x=1 \iff x= \pm 1$. Onde está o erro?
} 
\end{frame}

\begin{frame}
\frametitle{Exercícios}
\Ex{\label{exe:escrever-reciprocas}
Escreva as recíprocas, contrapositivas e negações
matemáticas das seguintes afirmações:
\begin{enumerate}[a.]
	\item Todos os gatos têm rabo; $\left(G \implies R \right)$\\
	\sub{Recíproca:} Se têm rabo então é gato; $\left(R \implies G \right)$\\
	\sub{Contrapositiva:} Se não tem rabo então não é gato; $\left(\sim R \implies \sim G \right)$\\
	\sub{Negação:} Existe um gato que não tem rabo. $\left(G \land \sim R \right)$
	\item Sempre que chove, eu saio de guarda-chuva ou fico em casa;
	\item Todas as bolas de ping pong são redondas e brancas;
	\item Sempre que é terça-feira e o dia do mês é um número primo,
	eu vou ao cinema;
	\item Todas as camisas amarelas ou vermelhas têm manga comprida;
	\item Todas as coisas quadradas ou redondas são amarelas e
	vermelhas.
\end{enumerate}
}  
\end{frame}

\begin{frame}
\frametitle{Exercícios}
\Ex{Considere os conjuntos: $F$ composto por todos os filósofos;
$M$ por todos os matemáticos; $C$ por todos os cientistas; e $P$ por
todos os professores.
\begin{enumerate}[a.]
	\item Exprima cada uma das afirmativas abaixo usando a linguagem
	de conjuntos: \\
	(i) Todos os matemáticos são cientistas; (ii) Alguns matemáticos
	são professores; (iii) Alguns cientistas são filósofos; (iv) Todos
	os filósofos são cientistas ou professores; (v) Nem todo professor
	é cientista.
	\item Faça o mesmo com as afirmativas abaixo: \\
	(vi) Alguns matemáticos são filósofos; (vii) Nem todo filósofo é
	cientista; (viii) Alguns filósofos são professores; (ix) Se um
	filósofo não é matemático, ele é professor; (x) Alguns filósofos
	são matemáticos.
	\item Tomando as cinco primeiras afirmativas como hipóteses,
	verifique quais das afirmativas do segundo grupo são
	necessariamente verdadeiras.
\end{enumerate}
}
\end{frame}

\begin{frame}
\frametitle{Exercícios}
\Ex{Considere um grupo de 4 cartões, que possuem uma letra escrita
em um dos lados e um número do outro. Suponha que seja feita, sobre
esses cartões, a seguinte afirmação: \emph{Todo cartão com uma vogal
de um lado tem um número ímpar do outro}. Quais dos cartões abaixo
você precisaria virar para verificar se essa afirmativa é verdadeira
ou falsa?
\begin{center}
\begin{tabular}{|c|c|c|c|c|c|c|}
	\cline{1-1} \cline{3-3} \cline{5-5} \cline{7-7}
	% after \\: \hline or \cline{col1-col2} \cline{col3-col4} ...
	A & $\empty$ & 1 & $\empty$ & B & $\empty$ & 4 \\
	\cline{1-1} \cline{3-3} \cline{5-5} \cline{7-7}
\end{tabular}
\end{center}
}
\end{frame}

\begin{frame}
\frametitle{Exercícios}  
\Ex{  Numa mesa há cinco cartas:
	%
	\begin{center}
	\begin{tabular}{|c|c|c|c|c|c|c|c|c|}
		\cline{1-1} \cline{3-3} \cline{5-5} \cline{7-7} \cline{9-9}
		Q & $\empty$ & T & $\empty$ & 3 & $\empty$ & 4 & $\empty$ & 6 \\
		\cline{1-1} \cline{3-3} \cline{5-5} \cline{7-7} \cline{9-9}
	\end{tabular}
	\end{center}
	%
	Cada carta tem um número natural de um lado e uma letra de outro lado. 
	Nico afirma: ``Qualquer carta que tenha uma vogal tem um número par do outro lado''.
	Jorel provou que Nico mente virando somente uma das cartas. Qual das cinco cartas
	Jorel teve que virar para provar que Nico mentiu?
}  
\end{frame}













%------------------------------------------------------------------------------------------------------------

\section{Bibliografia}

\frame{
\frametitle{Bibliografia}

\begin{thebibliography}{99}

\bibitem {label1}
LIMA, Elon L.
\newblock {\em Números e Funções Reais}.
\newblock 1. ed. Rio de Janeiro: SBM, 2013.

\bibitem {label2}
LIMA, Elon L; CARVALHO, Paulo César P; Wagner, Eduardo; MORGADO,
Augusto C.
\newblock {\em A Matemática do Ensino Médio. Vol. 1}.
\newblock 9. ed. Rio de Janeiro: SBM, 2006.

\bibitem {label3}
OLIVEIRA, Krerley I M; FERNÁNDEZ, Adán J C.
\newblock {\em Iniciação à Matemática: um Curso com Problemas e Soluções}.
\newblock 2. ed. Rio de Janeiro: SBM, 2010.


\end{thebibliography}
}


%------------------------------------------------------------------------------------------------------------


%{\aauwavesbg
%\begin{frame}[plain,noframenumbering]
%  \finalpage{Thank you for using this theme!}
%\end{frame}}
%%%%%%%%%%%%%%%%%

\end{document}
